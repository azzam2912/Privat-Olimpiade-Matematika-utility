\documentclass{article}
\usepackage{amsmath}
\usepackage{graphicx}


\renewcommand{\baselinestretch}{1.5}
\addtolength{\oddsidemargin}{-1in}
\addtolength{\evensidemargin}{-1.5in}
\addtolength{\textwidth}{1.9in}

\addtolength{\topmargin}{-1in}
\addtolength{\textheight}{1.9in} 

\title{Sesi Pendek Soal tipe OSK}
\author{45 menit}

\date{Jumat, 29 Januari 2021}

\begin{document}
	\maketitle
	
	\begin{enumerate}
		\item
		Tentukan penjumlahan semua bilangan asli yang kurang dari 2016 dan relatif prima dengan 2016.
		
		\item 
		Budi mengerjakan tes yang terdir dari 20 soal pilihan ganda yang masing-masing memilki $1,2,3,...,20$ pilihan jawaban. Karena budi tidak belajar sama sekali, Budi memilih sembarang satu pilihan jawaban di setiap soal. Peluang Budi menjawab total sebanyak genap soal dengan benar adalah...
		
		\item
		Tentukan nilai minimum dari $$\frac{2}{x}+\frac{2}{y}+\frac{y|x-2|}{2x}+\frac{x|y-2|}{2y}$$dimana $x,y$ bilangan riil positif.
		
		\item
		Sebuah bilangan asli disebut "eskrim" jika dapat dinyatakan sebagai penjumlahan dari 2016 buah (boleh sama) bilangan asli komposit. Tentukan bilangan asli terbesar yang bukan merupakan bilangan eskrim.
		
		\item
		Misalkan titik $D$ terletak pada sisi $BC$ dari segitiga $ABC$ sehingga $AD$ adalah garis bagi sudut $A$. Misalkan $AD$ memotong lingkaran luar segitiga $ABC$ sekali lagi di titik $E$. Misalkan juga $F$ adalah titik pusat lingkaran dalam segitiga $ABC$. Jika diketahui $AE=70, AF=40, DF=20$, tentukan panjang $DE$.
		
		\item
		Jika $\phi(n)$ adalah banyaknya bilangan asli kurang dari $n$ yang relatif prima dengan $n$ dan $d(n)$ adalah banyaknya bilangan asli faktor dari $n$, tentukan jumlah dari semua bilangan asli $n$ yang memenuhi $$\phi(n)=2d(n)$$
		
		
	\end{enumerate}
		

\end{document}