\documentclass[11pt]{scrartcl}
\usepackage{graphicx}
\graphicspath{{./}}
\usepackage[sexy]{evan}
\usepackage[normalem]{ulem}
\usepackage{hyperref}
\usepackage{mathtools}
\hypersetup{
    colorlinks=true,
    linkcolor=blue,
    filecolor=magenta,      
    urlcolor=cyan,
    pdfpagemode=FullScreen,
    }

\renewcommand{\dangle}{\measuredangle}

\renewcommand{\baselinestretch}{1.5}

\addtolength{\oddsidemargin}{-0.4in}
\addtolength{\evensidemargin}{-0.4in}
\addtolength{\textwidth}{0.8in}
% \addtolength{\topmargin}{-0.2in}
% \addtolength{\textheight}{1in} 


\setlength{\parindent}{0pt}

\usepackage{pgfplots}
\pgfplotsset{compat=1.15}
\usepackage{mathrsfs}
\usetikzlibrary{arrows}

\title{Combinatorial Games dan Invarian - Soal Post Test}
\author{Azzam (IG: haxuv.world)}
\date{Sabtu, 27 Januari 2024}

\begin{document}
\maketitle

\textbf{Aturan umum:}
\begin{itemize}
    \item Tulis \textbf{nama lengkap} dan \textbf{asal sekolah} di pojok kiri atas halaman pertama.
    \item \textbf{Soal bertipe esai}. Sertakan argumentasi atau cara mendapatkan jawaban yang ditanyakan di soal.
    \item Waktu standar untuk mengerjakan semua soal berikut adalah 90-120 menit.
    \item Setiap soal bernilai bilangan bulat antara 0 sampai 10 (inklusif).
    \item Soal yang wajib dikerjakan untuk mendapatkan \textit{full points} adalah \textbf{TIGA SOAL}.
\end{itemize}


\section{Soal}
\begin{enumerate}
\end{enumerate}
\end{document}