\documentclass[11pt]{scrartcl}
\usepackage{graphicx}
\graphicspath{{./}}
\usepackage[sexy]{evan}
\usepackage[normalem]{ulem}
\usepackage{hyperref}
\usepackage{mathtools}
\hypersetup{
    colorlinks=true,
    linkcolor=blue,
    filecolor=magenta,      
    urlcolor=cyan,
    pdfpagemode=FullScreen,
    }

\renewcommand{\dangle}{\measuredangle}

\renewcommand{\baselinestretch}{1.5}

\addtolength{\oddsidemargin}{-0.4in}
\addtolength{\evensidemargin}{-0.4in}
\addtolength{\textwidth}{0.8in}
% \addtolength{\topmargin}{-0.2in}
% \addtolength{\textheight}{1in} 


\setlength{\parindent}{0pt}

\usepackage{pgfplots}
\pgfplotsset{compat=1.15}
\usepackage{mathrsfs}
\usetikzlibrary{arrows}

\title{PigeonHole Principle - Easier Version}
\author{Azzam (IG: haxuv.world)}
\date{\today}

\begin{document}
\maketitle
\section{PHP - PigeonHole Principle}
\begin{definition}[First PigeonHole Principle]
     If $m$ objects are put into $n$ boxes, then at least one box contains $\floor{\frac{m-1}{n}}$ or more objects.
\end{definition}

\textbf{Corollary}: Let $m_1, m_2, \dots, m_n$ be $n$ positive integers. If $m_1+m_2+\dots+m_n+1$ objects are put into $n$ boxes, then the first boxes contains at least $m_1+1$ objects, \dots, or the $n-th$ box contains at least $m_n+1$ objects.

\begin{definition}[Second PigeonHole Principle]
    If $m$ objects are put into $n$ boxes, then at leas one box contains $\floor{\frac{m}{n}}$ or less objects.
\end{definition}

\textbf{Corollary}: Let $m_1, m_2, \dots, m_n$ be $n$ positive integers. If $m_1+m_2+\dots+m_n-1$ objects are put into $n$ boxes, then the first boxes contains at least $m_1-1$ objects, \dots, or the $n-th$ box contains at least $m_n-1$ objects.

\subsection{Latihan Soal - Introductory}
\begin{enumerate}
\item Berapa banyak orang minimum yang harus hadir di suatu pesta sehingga dipastikan terdapat 3 orang yang lahir di bulan yang sama di pesta itu?

\item Misalkan Naruko memilih $k$ buah bilangan dari himpunan $\{1,2,3,\dots,2016\}$ secara acak. Berapakah nilai $k$ terkecil sehingga Naruko pasti bisa mendapatkan setidaknya sepasang bilangan (dari $k$ bilangan itu) yang jika dijumlahkan hasilnya 2017?

\item Suatu malam, sebelum show teater JKT48 terjadi pemadaman listrik di asrama member. Karena Freya sangat malas, ia ingin membawa banyak kaus kaki, agar saat akan show tinggal memilih kaus kakinya saja. Ia mengambil kaus kaki dari lemari di ruangan yang sangat gelap. Lemari itu berisi 100 buah kaus kaki merah, 80 kaus kaki hijau, 60 kaus kaki biru, dan 40 kaus kaki hitam. Freya mengambil banyak kaus kaki tapi tidak bisa tahu warnanya. Berapa banyak kaus kaki paling sedikit yang perlu diambil sehingga dijamin terdapat setidaknya 10 pasang kaus kaki (dengan setiap pasang kaus kaki harus berwarna sama) ?

\item (OSK 2011) Di lemari hanya ada 2 macam kaos kaki yaitu kaos kaki berwarna hitam dan putih. Ali, Budi dan Candra berangkat di malam hari saat mati lampu dan mereka mengambil kaos kaki secara acak di dalam lemari dalam kegelapan. Berapa kaos kaki minimal harus mereka ambil untuk memastikan bahwa akan ada tiga pasang kaos kaki yang bisa mereka pakai ? (Sepasang kaos kaki harus memiliki warna yang sama).
    
\item Tandai satu buah kartu dengan angka 1, dua buah kartu dengan angka 2, tiga buah kartu dengan angka satu hingga lima puluh buah kartu dengan angka 50. Semua kartu tersebut dimasukkan ke dalam kotak. Berapa buah kartu minimal yang harus diambil agar dapat dipastikan terdapat sekurang-kurangnya 10 buah kartu dengan tanda angka yang sama 

\item (OSK 2016) Anak laki-laki dan anak perempuan yang berjumlah 48 orang duduk melingkar secara acak. Banyaknya minimum anak perempuan sehingga pasti ada enam anak perempuan yang duduk berdekatan tanpa diselingi anak laki-laki adalah \dots
\end{enumerate}

\subsection{Latihan Soal - Medium}
\begin{enumerate}
    \item (Ramsey's Theorem) Jika ada 6 orang atau lebih, maka ada tiga orang yang saling mengenal atau ada tiga orang yang tidak saling mengenal.

    \item Misalkan $a_1, a_2, \dots, a_{10}$ adalah sepuluh bilangan bulat. Tunjukkan bahwa ada bilangan bulat $1 \le i,j \le 10$ sehingga $a_i+a_{i+1}+\dots+a_{j-1}+a_j$ habis terbagi 10.

    \item Dari bilangan bula $1,2,\dots 200$ pilih 101 bilangan. Perlihatkan bahwa dari bilangan yang dipilih, ada dua bilangan $a$ dan $b$ sehingga $a \mid b$.

    \item Buktikan bahwa subhimpunan $n+1$ anggota dari $A=\{1,2,\dots, 2n\}$ selalu memuat dua bilangan yang relatif prima.

    \item Buktikan bahwa ada bilangan berbentuk $$123456712345671234567\dots1234567$$
    yang habis dibagi oleh $7654321$.

    \item Seorang wibu sejati mempunyai waktu 11 minggu untuk mengikuti suatu turnamen marathon anime. Sebagai persiapan, ia berlatih dengan menghabiskan setidaknya 1 episode anime (1 episode sepanjang 2 jam) setiap hari, tetapi ia tidak ingin menghabiskan lebih dari 12 episode dalam satu minggu (ya, dia masih peduli pada kesehatannya). Buktikan bahwa ia pernah melakukan marathon episode anime sebanyak tepat 21 episode dalam beberapa hari berurutan.

    \item Buktikan diantara setiap lima bilangan real, ada dua diantaranya yaitu $x$ dan $y$ yang memenuhi $0 < x-y < 1+xy$.

    \item Misalkan terdapat lima bilangan bulat $a_1, a_2, \dots, a_5$. Misalkan $b_1,b_2,\dots,b_5$ merupakan permutasi dari $a_1,\dots,a_5$. Buktikan bahwa 
    $$A=(b_1-a_1)+(b_2-a_2)+\dots+(b_5-a_5)$$ adalah bilangan genap.

    \item Let a space figure $G$ consist of $n \ge 4$ vertices, no $4$ vertices of which are coplanar and there are $\left\lfloor \frac{n^2}{4} \right\rfloor + 1$ line segments connecting these vertices. Prove there exist two triangles with a common side in the figure $G$.
    
    \item Let $k$ be a given positive integer number. Find the smallest value of $n$ such that among any $n$ positive integer numbers there exist two numbers whose sum or difference is divisible by $2k$.
    
    \item Let $a$ be a positive real number and $n$ be a positive integer. Prove that there exist two positive integers $p$ and $q$ satisfying the following inequality: $\left|a-\frac{q}{p}\right| \le \frac{1}{np}$.
    
    \item Suppose that $n$ numbers are deleted in the set $S = \{1 ,2, \dots , 2005 \}$ such that among the remaining numbers, no number equals to the product of other two. Find the smallest positive integers $n$ with the above properties.
    
    \item Suppose that $a_1, a_2, \dots, a_n$ $(n \ge 4)$ are $n$ distinct integers in open interval $(0, 2n)$. Prove that there exists a non-empty subset $S$ of $\{a_1, a_2, \dots, a_n\}$ such that the sum of all elements in $S$ is divisible by $2n$.
    
    \item (IMO SL 1988) Let $49$ students solve $3$ problems and the score of each problem is one of these nonnegative integers $0, 1, 2, 3, 4, 5, 6, 7$. Prove that there exist two students $A$ and $B$ such that for each of the problems the scores of $A$ is not less than $B$.
    
    \item Let $n$ and $r$ be two positive integers. Find the smallest positive integer $m$ satisfying the following condition: For each classification of the set $\{1, 2, \dots, m \}$ into $r$ subsets $A_1, A_2, \dots, A_r$ where $A_i \cap A_j = \emptyset$, $i \neq j$, there exist two numbers $a$ and $b$ in a subset $A_i$ $(1 \le i \le r)$ such that $1 \le \frac{b}{a} \le 1 + \frac{1}{n}$.
\end{enumerate}

\section{Referensi}
\begin{itemize}
    \item Yao-Zhang, Combinatorial Problems in Mathematical Competitions.
    \item Wono Setya Budhi, Langkah Awal Menuju Olimpiade Matematika: Kombinatorik.
\end{itemize}
\end{document}