\documentclass[11pt]{scrartcl}
\usepackage{graphicx}
\graphicspath{{./}}
\usepackage[sexy]{evan}
\usepackage[normalem]{ulem}
\usepackage{hyperref}
\usepackage{mathtools}
\hypersetup{
    colorlinks=true,
    linkcolor=blue,
    filecolor=magenta,      
    urlcolor=cyan,
    pdfpagemode=FullScreen,
    }

\renewcommand{\dangle}{\measuredangle}

\renewcommand{\baselinestretch}{1.5}

\addtolength{\oddsidemargin}{-0.4in}
\addtolength{\evensidemargin}{-0.4in}
\addtolength{\textwidth}{0.8in}
% \addtolength{\topmargin}{-0.2in}
% \addtolength{\textheight}{1in} 


\setlength{\parindent}{0pt}

\usepackage{pgfplots}
\pgfplotsset{compat=1.15}
\usepackage{mathrsfs}
\usetikzlibrary{arrows}

\title{PigeonHole Principle and Mean Value Principle}
\author{Azzam (IG: haxuv.world)}
\date{Sabtu, 27 Januari 2024}

\begin{document}
\maketitle
\section{PHP - PigeonHole Principle}
\begin{definition}[First PigeonHole Principle]
     If $m$ objects are put into $n$ boxes, then at least one box contains $\floor{\frac{m-1}{n}}$ or more objects.
\end{definition}

\textbf{Corollary}: Let $m_1, m_2, \dots, m_n$ be $n$ positive integers. If $m_1+m_2+\dots+m_n+1$ objects are put into $n$ boxes, then the first boxes contains at least $m_1+1$ objects, \dots, or the $n-th$ box contains at least $m_n+1$ objects.

\begin{definition}[Second PigeonHole Principle]
    If $m$ objects are put into $n$ boxes, then at leas one box contains $\floor{\frac{m}{n}}$ or less objects.
\end{definition}

\textbf{Corollary}: Let $m_1, m_2, \dots, m_n$ be $n$ positive integers. If $m_1+m_2+\dots+m_n-1$ objects are put into $n$ boxes, then the first boxes contains at least $m_1-1$ objects, \dots, or the $n-th$ box contains at least $m_n-1$ objects.

\subsection{Contoh Soal}
\begin{enumerate}
    \item (Ramsey's Theorem) Jika ada 6 orang atau lebih, maka ada tiga orang yang saling mengenal atau ada tiga orang yang tidak saling mengenal.

    \item Misalkan $a_1, a_2, \dots, a_{10}$ adalah sepuluh bilangan bulat. Tunjukkan bahwa ada bilangan bulat $1 \le i,j \le 10$ sehingga $a_i+a_j+\dots+a_j$ habis terbagi 10.

    \item Dari bilangan bula $1,2,\dots 200$ pilih 101 bilangan. Perlihatkan bahwa dari bilangan yang dipilih, ada dua bilangan $a$ dan $b$ sehingga $a \mid b$.

    \item Buktikan bahwa subhimpunan $n+1$ anggota dari $A=\{1,2,\dots, 2n\}$ selalu memuat dua bilangan yang relatif prima.

    \item Buktikan bahwa ada bilangan berbentuk $$123456712345671234567\dots1234567$$
    yang habis dibagi oleh $7654321$.

    \item Seorang wibu sejati mempunyai waktu 11 minggu untuk mengikuti suatu turnamen marathon anime. Sebagai persiapan, ia berlatih dengan menghabiskan setidaknya 1 episode anime (1 episode sepanjang 2 jam) setiap hari, tetapi ia tidak ingin menghabiskan lebih dari 12 episode dalam satu minggu (ya, dia masih peduli pada kesehatannya). Buktikan bahwa ia pernah melakukan marathon episode anime sebanyak tepat 21 episode dalam beberapa hari berurutan.

    \item Buktikan diantara setiap lima bilangan real, ada dua diantaranya yaitu $x$ dan $y$ yang memenuhi $0 < x-y < 1+xy$.

    \item Misalkan terdapat lima bilangan bulat $a_1, a_2, \dots, a_5$. Misalkan $b_1,b_2,\dots,b_5$ merupakan permutasi dari $a_1,\dots,a_5$. Buktikan bahwa 
    $$A=(b_1-a_1)+(b_2-a_2)+\dots+(b_5-a_5)$$ adalah bilangan genap.

    \item Let a space figure $G$ consist of $n \ge 4$ vertices, no $4$ vertices of which are coplanar and there are $\left\lfloor \frac{n^2}{4} \right\rfloor + 1$ line segments connecting these vertices. Prove there exist two triangles with a common side in the figure $G$.
    
    \item Let $k$ be a given positive integer number. Find the smallest value of $n$ such that among any $n$ positive integer numbers there exist two numbers whose sum or difference is divisible by $2k$.
    
    \item Let $a$ be a positive real number and $n$ be a positive integer. Prove that there exist two positive integers $p$ and $q$ satisfying the following inequality: $\left|a-\frac{q}{p}\right| \le \frac{1}{np}$.
    
    \item Suppose that $n$ numbers are deleted in the set $S = \{1 ,2, \dots , 2005 \}$ such that among the remaining numbers, no number equals to the product of other two. Find the smallest positive integers $n$ with the above properties.
    
    \item Suppose that $a_1, a_2, \dots, a_n$ $(n \ge 4)$ are $n$ distinct integers in open interval $(0, 2n)$. Prove that there exists a non-empty subset $S$ of $\{a_1, a_2, \dots, a_n\}$ such that the sum of all elements in $S$ is divisible by $2n$.
    
    \item (IMO SL 1988) Let $49$ students solve $3$ problems and the score of each problem is one of these nonnegative integers $0, 1, 2, 3, 4, 5, 6, 7$. Prove that there exist two students $A$ and $B$ such that for each of the problems the scores of $A$ is not less than $B$.
    
    \item Let $n$ and $r$ be two positive integers. Find the smallest positive integer $m$ satisfying the following condition: For each classification of the set $\{1, 2, \dots, m \}$ into $r$ subsets $A_1, A_2, \dots, A_r$ where $A_i \cap A_j = \emptyset$, $i \neq j$, there exist two numbers $a$ and $b$ in a subset $A_i$ $(1 \le i \le r)$ such that $1 \le \frac{b}{a} \le 1 + \frac{1}{n}$.
\end{enumerate}

\section{MVP - Mean Value Principle}
\begin{definition}
    \begin{enumerate}
        \item Misalkan $a_1, a_2, \dots, a_n$ adalah $n$ bilangan real dan $A = \frac{1}{n}(a_1+a_2+\dots+a_n)$, maka setidaknya satu angka dari $a_1,a_2,\dots,a_n$ lebih dari atau sama dengan $A$ dan setidaknya salah satu bilangan tersebut kurang dari sama dengan $A$.
        
        \item  Misalkan $a_1, a_2, \dots, a_n$ adalah $n$ bilangan real dan $G = \sqrt[n]{a_1\cdot a_2\cdot\dots\cdot a_n}$, maka setidaknya satu angka dari $a_1,a_2,\dots,a_n$ lebih dari atau sama dengan $A$ dan setidaknya salah satu bilangan tersebut kurang dari sama dengan $A$.
    \end{enumerate}
\end{definition}

\subsection{Contoh Soal}
\begin{enumerate}
\item Suppose that $10$ distinct numbers $1, 2, 3, 4, 5, 6, 7,8, 9, 10$ are arranged in a circle with any order. Show that there exist $3$ successive numbers whose sum is not less than $18$.

\item Suppose that there are $n \ge 4$ distinct points in the plane and each pair of them is connected by a line segment. If among all lengths of these line segments, only $n + 1$ lengths equal $d$. Prove that there exists a point $P$ such that there are at least $3$ line segments meeting $P$ their lengths all equal $d$.

\end{enumerate}
\section{Referensi}
\begin{itemize}
    \item Yao-Zhang, Combinatorial Problems in Mathematical Competitions.
    \item Wono Setya Budhi, Langkah Awal Menuju Olimpiade Matematika: Kombinatorik.
\end{itemize}
\end{document}