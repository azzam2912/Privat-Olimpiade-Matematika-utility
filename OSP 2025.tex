\documentclass[12pt]{scrartcl}
\usepackage[hagavi]{azzam}

% You can set the title and author here, or it will use the default from azzam.sty

\title{Soal OSP SMA 2025}
\date{Selasa, 19 Agustus 2025}

\begin{document}
\maketitle

\section{Isian Singkat}

\begin{enumerate}
    \item Diberikan suatu dadu tidak standar dengan bilangan pada sisi-sisinya 1, 2, 5, 8, 13, 21, dan 34. Dadu tersebut dilemparkan dua kali. Banyaknya kemungkinan jumlah bilangan yang muncul merupakan suatu bilangan pada sisi dadu tersebut adalah...

    \item Misalkan $u_1, u_2, u_3, \dots$ adalah barisan geometri yang memenuhi persamaan 
    $$u_2 + u_4 + u_6 + u_8 = 31 \text{ dan } u_1+\dfrac{u_2}{u_1} = 149$$
    Nilai $u_1 + u_2 + u_3 + u_4 + \dots = \dots$

    \item Diberikan segitiga lancip $ABC$ dengan titik $P$ dan $Q$ pada sisi $BC$, titik $R$ pada sisi $AQ$ sehingga $$|PB|=|PQ|=|PR| \text{ dan } |QC|=|QR|.$$
    Diketahui bahwa $ACPR$ merupakan segiempat talibusur.\\
    Jika $\angle APR=54^\circ$, maka $\angle ABC = \dots$ \\
    \textbf{Catatan:} notasi $|XY|$ mengatakan panjang ruas garis $XY$.

    \item Misalkan bilangan asli $a, b, c, d$ memenuhi persamaan $$2^a+2^b+2^c=4^d.$$
    Jika $a+b+c+d \le 500$, maka nilai terbesar yang mungkin dari $d$ adalah\dots

    \item Misalkan $f$ suatu polinomial monik berderajat 5. Sehingga 
    $$f(1)=4, f(2)=7, f(3)=12, f(4)=19 \text{ dan }f(5)=28.$$
    Nilai $f(6)=\dots$ \\
    \textbf{Catatan:} Polinomial $P(x)$ berderajat $n$ disebut polinomial monik jika koefisien dari $x^n$ adalah 1.

    \item Banyaknya bilangan asli 8 digit yang hanya terdiri dari digit-digit 1 atau 2 serta tidak memuat 121 maupun 212 adalah\dots \\
    \textbf{Catatan:}
    \begin{itemize}
        \item Contoh bilangan 5 digit yang memenuhi syarat tersebut adalah 12211 dan 22222.
        \item Contoh bilangan 5 digit yang tidak memenuhi syarat adalah 11211 dan 21222.
    \end{itemize}

    \item Diberikan segiempat konveks $ABCD$ dengan luas 288, $AC$ tegak lurus $BD$, dan $AB$ tidak sejajar $CD$. Misalkan $P$ suatu titik di dalam segiempat $ABCD$. Selanjutnya, misalkan $Q$ dan $R$ berturut-turut merupakan proyeksi titik $P$ pada sisi $AC$ dan $BD$. Jika $|AQ|:|CQ|=5:3$ dan $|BR|:|DR|=7:2$, maka selisih luas segitiga $ABP$ dengan luas segitiga $CDP$ adalah \dots \\
    \textbf{Catatan:} Segiempat konveks adalah segiempat yang memenuhi:
    \begin{itemize}
        \item Perpotongan kedua diagonalnya terletak didalam segiempat.
        \item Keempat sudut dalam dari segiempat tersebut kurang dari $180^\circ$.
    \end{itemize}

    \item Banyaknya bilangan asli $(a,b)$ dimana $1 \le a,b \le 19^2$ sehingga $$a^4+b^3\text{ habis dibagi }19^2$$ adalah\dots
\end{enumerate}

\newpage
\section{Esai}
\begin{enumerate}
    \item Tentukan banyaknya bilangan asli $n \ge 2$ sedemikian sehingga terdapat $n$ bilangan bulat berurutan yang jumlahnya 2025.

    \item Misalkan $S$ adalah himpunan semua tripel bilangan real positif $(a,b,c)$ yang memenuhi $a+b+c = ab+bc+ca$.
    \begin{enumerate}
        \item Buktikan bahwa ketaksamaan
        \[ \min\{a+b,b+c,c+a\} > 1 \]
        berlaku untuk setiap $(a,b,c) \in S$.
        \item Apakah terdapat tripel $(a,b,c) \in S$ sehingga
        \[ \min\{a+b,b+c,c+a\} < 1+\dfrac{1}{20^{25}} \]
    \end{enumerate}
    \textbf{Catatan}: Notasi $\min\{x,y,z\}$ menyatakan bilangan terkecil di antara $x,y,z$.

    \item Pada segitiga $ABC$, misalkan $D$ titik tengah ruas garis $AB$ dan $E$ titik pada sisi $BC$. Misalkan garis yang melalui $E$ dan sejajar $AB$ memotong garis bagi $\angle ACB$ di titik $P$. Misalkan juga $I$ titik pusat lingkaran dalam segitiga $ABC$ dan $J$ titik pusat lingkaran singgung luar dari segitiga $ABC$ yang menyinggung sisi $CA$ (\textbf{bukan} perpanjangan sisi $CA$). Garis $DJ$ memotong sisi $CA$ di titik $F$.

    \begin{enumerate}[(a)]
        \item Buktikan bahwa garis $IF$ sejajar dengan $AB$.
        \item Buktikan bahwa garis $AP, BJ,$ dan $EF$ berpotongan di satu titik.
    \end{enumerate}
    
    \item Diberikan suatu segitiga pada bidang $xy$ dengan ketiga titik sudutnya \textbf{bukan} merupakan titik latis dan ketiga sisinya \textbf{tidak} melalui titik latis. Diketahui juga bahwa segitiga tersebut memuat paling sedikit 10 titik latis di bagian dalamnya. Buktikan bahwa terdapat 4 titik latis di bagian dalam segitiga tersebut yang terletak pada satu garis.
    
    \textbf{Catatan:} Pada bidang $xy$, titik latis adalah titik berbentuk $(a, b)$ dengan $a$ dan $b$ bilangan bulat.
\end{enumerate}

\end{document}