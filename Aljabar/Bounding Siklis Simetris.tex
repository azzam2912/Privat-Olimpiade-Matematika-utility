\documentclass[11pt]{scrartcl}
\usepackage{graphicx}
\graphicspath{{./}}
\usepackage[sexy]{evan}
\usepackage[normalem]{ulem}
\usepackage{hyperref}
\usepackage{mathtools}
\hypersetup{
    colorlinks=true,
    linkcolor=blue,
    filecolor=magenta,      
    urlcolor=cyan,
    pdfpagemode=FullScreen,
    }

\renewcommand{\dangle}{\measuredangle}

\renewcommand{\baselinestretch}{1.5}

\addtolength{\oddsidemargin}{-0.4in}
\addtolength{\evensidemargin}{-0.4in}
\addtolength{\textwidth}{0.8in}
% \addtolength{\topmargin}{-0.2in}
% \addtolength{\textheight}{1in} 


\setlength{\parindent}{0pt}

\usepackage{pgfplots}
\pgfplotsset{compat=1.15}
\usepackage{mathrsfs}
\usetikzlibrary{arrows}

\title{Mengeksploitasi Sifat Simetris dan Siklis Sistem Persamaan dengan Ketaksamaan}
\author{Azzam (IG: haxuv.world)}
\date{\today}

\begin{document}
\maketitle

Pernahkah anda menebak solusi suatu soal sistem persamaan yang berbentuk siklis atau simetris dan sangat yakin bahwa jawaban tersebut adalah satu-satunya solusi, namun anda kesulitan membuktikannya dengan substitusi, eliminasi, atau manipulasi aljabar lainnya?

Ternyata ada metode yang sebenarnya sudah cukup dikenal untuk menyelesaikan sistem persamaan, ya ketaksamaan! Nah, karena metode ketaksamaan ada banyak variasinya, maka kali ini kita akan membahas kasus khusus memakai ketaksamaan pada persamaan siklis dan simetris.


\section{Definisi Sistem Persamaan Siklis dan Simetris}
Misalkan terdapat suatu persamaan yang terdiri dari $n \ge 1$ variabel $(a_1, a_2, \dots, a_n)$.

\subsection{Sistem Persamaan Siklis}
Suatu sistem persamaan bersifat \textbf{siklis} apabila setiap persamaan dari sistem tersebut dapat diperoleh dengan "menggeser" secara siklis atau "memutar" $n$ variabel dari persamaan lainnya pada sistem tersebut. Dengan kata lain, suatu sistem yang terdiri dari $n$ persamaan dengan $(a_1, a_2, \dots,a_{n-1}, a_n), (a_2, a_3, \dots, a_n, a_1),$
$ (a_3, a_4, \dots, a_1, a_2), \dots,(a_{n-1}, a_{n}, \dots, a_{n-2}),$
$(a_n, a_1, \dots, a_{n-2}, a_{n-1})$ merupakan solusi yang memenuhi adalah sistem persamaan yang bersifat siklis.
\vspace{2mm}
\subsection{Sistem Persamaan Simetris}
Sedikit mirip dengan sistem persamaan siklis, suatu sistem persamaan bersifat \textbf{simetris} apabila setiap persamaan dari sistem tersebut dapat diperoleh dari permutasi $n$ variabel dari persamaan lainnya pada sistem tersebut. Dengan kata lain, suatu sistem yang terdiri dari seluruh permutasi $n$ variabel $(a_1, a_2, \dots, a_n)$.


\section{"Teknik WLOG" Ketaksamaan}
Sistem persamaan siklis dan simetris tersebut memiliki satu kemiripan, pergeseran variabel-variabel baik secara siklis ataupun dengan permutasi tidak merusak persamaan yang ada, oleh karena itu, kita dapat mengambil kasus spesifik dari seluruh pergeseran yang mungkin tersebut.

\subsection{Sistem Persamaan Siklis}
Untuk persamaan siklis, karena setiap persamaan didapat dengan menggeser variabel persamaan lainnya, maka tanpa mengurangi keumuman (WLOG --\textit{ Without Loss of Generality}), dari $n$ variabel $a_1,a_2,\dots,a_n$ dapat kita misalkan satu variabel $a_1$ sebagai variabel dengan nilai terkecil atau nilai terbesar dari $n$ variabel tersebut. Secara notasi, dapat kita misalkan $a_1 = \min\{a_1,a_2,\dots,a_n\}$ atau $a_1 = \max\{a_1,a_2,\dots,a_n\}$. 

\subsection{Sistem Persamaan Simetris}
Lalu, untuk persamaan simetris, karena setiap persamaan dapat diperoleh dari permutasi variabel persamaan lainnya, maka WLOG dapat dimisalkan $n$ variabel $a_1,a_2,\dots,a_n$ tersebut memenuhi suatu pengurutan, misalkan $a_1 \le a_2 \le \dots \le a_n$.


\section{Contoh Soal 1}
\begin{example*}
Cari semua solusi real $(x,y,z)$ dari sistem persamaan\\[-10pt] 
		$$\begin{cases}
			\vspace{2mm}
			\dfrac{4x^2}{4x^2+1}=y \\
			\vspace{2mm}
			\dfrac{4y^2}{4y^2+1}=z \\
			\dfrac{4z^2}{4z^2+1}=x
		\end{cases}$$
\end{example*}
\begin{proof}[Solusi.]
Perhatikan bahwa sistem persamaan tersebut siklis karena $(x,y,z)$,
$(y,z,x)$,$(z,x,y)$ adalah solusi sistem persamaan tersebut. Lalu, perhatikan bahwa $x,y,z$ haruslah non-negatif karena ruas kiri dari masing-masing persamaan tersebut bernilai non-negatif (karena berbentuk kuadrat). Maka WLOG dapat dimisalkan $z = \min\{x,y,z\}$ sehingga $z \le x$ dan $z \le y$. Dari sini diperoleh
$$z=1-\dfrac{1}{4y^2+1} \ge 1-\dfrac{1}{4z^2+1}=x \iff z \ge x.$$
Dari asumsi karena $x \ge z$, maka didapat $x \ge z \ge x$ yang menyebabkan $x=z$. Dengan cara serupa diperoleh pula $y=z$. Oleh karena itu kita punya
$$\dfrac{4x^2}{4x^2+1}=x \implies x(2x-1)^2=0 \implies x=0 \text{ atau } x=\frac{1}{2}.$$
Cek ke persamaan, solusi $(0,0,0)$ dan $(\frac{1}{2}, \frac{1}{2}, \frac{1}{2})$ memenuhi.

\end{proof}

\section{Contoh Soal 2}
\begin{example*}
Cari semua solusi real $(x,y,z)$ dari sistem persamaan\\[-10pt] 
		$$\begin{cases}
			x^2+y^2 = z^3\\
			y^2+z^2 = x^3\\
			z^2+x^2 = y^3
		\end{cases}$$
\end{example*}
\begin{proof}[Solusi.] Perhatikan bahwa sistem persamaan tersebut simetris karena permutasi dari $(x,y,z)$, yaitu $(x,z,y)$,  $(y,x,z)$, $(y,z,x)$, $(z,x,y)$, $(z,y,x)$ juga merupakan solusi dari sistem persamaan tersebut. Oleh karena itu WLOG $x \ge y \ge z$. Dapat diperoleh bahwa
$$z^3=x^2+y^2 \ge z^2 + y^2 = x^3 \iff z^3 \ge x^3 \iff z \ge x.$$ 
Berarti dari ketaksamaan tersebut dan dari permisalan didapatkan $z \ge x \ge y \ge z$ yang menyebabkan kesamaan harus terjadi atau $x=y=z$, sehingga kita punya $x^2+x^2=x^3 \implies x^2(x-2)=0 \implies x=0 \text{ atau } x=2$. Cek ke soal, dapat disimpulkan bahwa $(0,0,0)$ dan $(2,2,2)$ adalah solusi yang memenuhi.

\end{proof}


\section{Contoh Soal 3 (OSP 2019 Esai no. 2)}
\begin{example*}
\item  Cari semua solusi bilangan real $k$ sehingga sistem persamaan\\[-10pt] 
		$$\begin{cases}
			a^2+ab=kb^2 \\
			b^2+bc=kc^2 \\
			c^2+ca=ka^2
		\end{cases}$$
memiliki solusi bilangan real positif $a,b,c$
\end{example*}
\begin{proof}[Solusi.] Perhatikan bahwa sistem persamaan tersebut siklis. Oleh karena itu WLOG dapat dimisalkan $a = \max \{a,b,c\} \implies a \ge b \text{ dan } a \ge c$. Tinjau bahwa $k$ haruslah positif karena ruas kiri ketiga persamaan tersebut juga positif. Oleh karena itu, kita punya
$$2b^2 = b^2+b^2 \le a^2+ab = kb^2 \implies 2b^2 \le kb^2 \implies 2 \le k$$
dan 
$$ka^2 = c^2 + ca \le a^2 + a^2 = 2a^2 \implies ka^2 \le 2a^2 \implies k \le 2$$
Dari kedua ketaksamaan tersebut diperoleh $2 \le k \le 2$ yang menyebabkan $k=2$.
Cek ke soal, $k=2$ memenuhi karena meyebabkan $(1,1,1)$ adalah solusi yang memenuhi. Berarti $k$ yang memenuhi hanyalah $k=2$.


\end{proof}

\section{Latihan soal}
Sebagai latihan, coba kerjakan soal-soal berikut dengan teknik yang telah didemonstrasikan :)

\begin{enumerate}
    \item (OSP 2012 Esai no. 1) Tentukan semua pasangan bilangan bulat tak negatif $(a,b,x,y)$ yang memenuhi sistem persamaan\\[-10pt] 
		$$\begin{cases}
			a+b=xy \\
			x+y=ab 
		\end{cases}$$
		
    \item (OSP 2012 Esai no. 2) Cari semua pasangan bilangan real $(x,y,z)$ yang memenuhi sistem persamaan\\[-10pt] 
		$$\begin{cases}
			x=1+\sqrt{y-z^2} \\
			y=1+\sqrt{z-x^2} \\
			z=1+\sqrt{x-y^2}
		\end{cases}$$
		
	\item Carilah seluruh bilangan real berbeda $x,y,z$ yang memenuhi \\[-10pt] 
	$$\begin{cases}
			2^x=2^y+2^z-yz \\
			2^y=2^z+2^x-zx \\
			2^z=2^x+2^y-xy
		\end{cases}$$
		
	\item Cari semua tupel bilangan real $(x,y,z)$ yang memenuhi sistem persamaan\\[-10pt] 
		$$\begin{cases}
			\vspace{2mm}
			2x=y+\dfrac{2}{y} \\
			\vspace{2mm}
			2y=z+\dfrac{2}{z} \\
			2z=x+\dfrac{2}{x}
		\end{cases}$$
		
	\item (OSP 2015 Esai no. 2) Tentukan semua tripel bilangan real $(x,y,z)$ yang memenuhi sistem persamaan\\[-10pt]
	    $$\begin{cases}
			(x+1)^2 = x+y+2 \\
			(y+1)^2 = y+z+2 \\
			(z+1)^2 = z+x+2
		\end{cases}$$
		
	\item Carilah semua pasangan solusi real $(x,y)$ yang memenuhi\\[-10pt] 
				 		$$\begin{cases}\\[-30pt]
				 		x^2+3x+\log(2x+1)=y \\[-5pt]
				 		y^2+3y+\log(2y+1)=x
				 		\end{cases}$$
\end{enumerate}
\end{document}