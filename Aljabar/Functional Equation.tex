\documentclass[11pt]{scrartcl}
\usepackage{graphicx}
\graphicspath{{./}}
\usepackage[sexy]{evan}
\usepackage[normalem]{ulem}
\usepackage{hyperref}
\usepackage{mathtools}
\hypersetup{
    colorlinks=true,
    linkcolor=blue,
    filecolor=magenta,      
    urlcolor=cyan,
    pdfpagemode=FullScreen,
    }

\renewcommand{\dangle}{\measuredangle}

\renewcommand{\baselinestretch}{1.5}

\addtolength{\oddsidemargin}{-0.4in}
\addtolength{\evensidemargin}{-0.4in}
\addtolength{\textwidth}{0.8in}
% \addtolength{\topmargin}{-0.2in}
% \addtolength{\textheight}{1in} 


\setlength{\parindent}{0pt}

\usepackage{pgfplots}
\pgfplotsset{compat=1.15}
\usepackage{mathrsfs}
\usetikzlibrary{arrows}

\title{Functional Equations}
\author{Azzam (IG: haxuv.world)}
\date{\today}

\begin{document}

\maketitle

\section{Heuristik}
\subsection{Dasar-dasar}
\begin{enumerate}
    \item Tebak solusi. Misal tebak bahwa $f(x)=kx+c$. Coba juga bentuk polinomial.
    \item Optimisasi misal scaling $x \mapsto kx$, atau shifting $x \mapsto x + c$.
    \item Colok $x=0, y=0$, $x=y$, $x=-y$, atau lihat soal, apa yang bisa membuat dalemnya $f$ menjadi $f(0)$.
\end{enumerate}

\subsection{Lanjutan}
\begin{enumerate}
    \item Pakai "asersi" misal $P(x,y)$.
    \item Substitusikan suatu nilai sehingga ada yang saling "coret" atau "cancelling".
    \item Lihat sifat injektif, surjektif, atau bijektif.
    \item Lihat kesimetrian (bakal tahu kalo sering ngerjain soal)
    \item Eksploitasi involutions (misalkan seperti $f(f(x))=x$).
    \item Untuk fungsi di $\ZZ$ atau $\NN$ bisa pakai induksi atau pakai keterbagian misal $x-y \mid f(x)-f(y)$.
    \item Tulis fungsi dalam bentuk lain, misalkan $g(x)=\dfrac{f(x)}{x}$, $g(x)=xf(x)$, $g(x)=e^{f(x)}$ dan $g(x)=f(x)+x$. Atau shift to zero juga guna.
\end{enumerate}

\subsection{Terakhir}
\begin{enumerate}
    \item Hati-hati dengan pointwise trap misal seperti pada $f(x)^2=x^2$.
    \item Cek solusinya bener ngga Cuyyyy.
\end{enumerate}

\section{Substitusi, Transformasi, Shifting}
\subsection{Latihan Soal}
Carilah seluruh fungsi $f:\RR \to \RR$ sehingga $\forall x,y \in \RR$ berlaku
\begin{enumerate} 
    \item $f(x)+xf(1-x)=x$
    \item $(f(x)+y)(x+f(y))=f(x^2)+f(y^2)+2f(xy)$
    \item $f(x)f(y)+f(x+y)=xy$
    \item $f(x-y)=f(x)+f(y)-2xy$
    \item $f(x+y)+f(x)f(y)=x^2y^2+2xy$
    \item $f(x+y)=f(x)+f(y)+3(x+y)\sqrt[3]{f(x)f(y)}$
\end{enumerate}

\section{Injektif, Surjektif, Bijektif}
\subsection{Latihan Soal}
Carilah seluruh fungsi $f:\RR \to \RR$ sehingga $\forall x,y \in \RR$ berlaku
\begin{enumerate} 
    \item $f(x^2y)=f(xy)+yf(f(x)+y)$
    \item $f(xf(x)+f(y))=f(x^2)+y$
    \item $f(x)+f(x+f(y))=2x+y$
    \item $f(f(x)+y)=2x+f(f(y)-x)$
    \item $f(x^2+f(y))=xf(x)+y$
    \item $f(x^2f(x)^2+f(y))=x^3f(x)+y$
\end{enumerate}

\section{Cauchy's Functional Equation}
Untuk sebuah fungsi aditif (yang memenuhi operasi penjumlahan $+$) $f:\RR \to \RR$, yang memenuhi
\begin{align*}
    f(x+y) = f(x)+f(y)
\end{align*}
jika minimal salah satu syarat ini terpenuhi:
\begin{itemize}
    \item $f$ kontinu pada interval tertentu.
    \item $f$ terbatas (bounded) pada interval tertentu.
    \item $f$ monoton pada interval tertenu.
\end{itemize}
maka dapat dipastikan $f$ linear, atau $f(x)=kx$

\subsection{Latihan Soal}
\begin{enumerate}
    \item Carilah seluruh fungsi kontinu $f:\RR \to \RR$ sehingga untuk sembarang $x,y \in \RR$ berlaku
    \begin{align*}
        f(x+y)-f(x)-f(y)=2022xy
    \end{align*}
    \item (Iran 2014) Carilah seluruh fungsi $f:\RR^+ \to \RR^+$ sehingga untuk semua $x,y \in \RR^+$ berlaku
    \begin{align*}
        f\left(\dfrac{y}{f(x+1)}\right)+f\left(\dfrac{x+1}{xf(y)}\right)=f(y)
    \end{align*}
    \item Carilah seluruh fungsi $f:\RR \to \RR$ sehingga untuk semua $x \in \RR$ dan suatu $y \in \RR^+$ berlaku
    \begin{align*}
        f(x+y)=f(x)+f(y)
    \end{align*}
    \item (Korea 2008) Carilah seluruh fungsi aditif $f:\RR \to \RR$ sehingga
    \begin{align*}
        f\left(\dfrac{1}{x}\right)=\dfrac{f(x)}{x^2}, \forall x \in \RR
    \end{align*}
\end{enumerate}

\section{Fungsi di $\ZZ$}
\subsection{Latihan Soal}
\begin{enumerate}
    \item (Balkan 2017) Carilah seluruh fngsi $f:\NN \to \NN$ sehingga untuk semua $m,n \in \NN$ berlaku
    \begin{align*}
        n+f(m) \mid f(n)+nf(m)
    \end{align*}
\end{enumerate}


\section{Study Case dari OSN}
\begin{enumerate}
    \item (OSN 2015) Misalkan pasangan fungsi $f,g:\RR^+ \to \RR^+$ memenuhi persamaan fungsi
    \begin{align*}
        f(g(x)y+f(x))=(y+2015)f(x)
    \end{align*}
    untuk setiap $x,y \in \RR^+$.
    \begin{enumerate}
        \item Buktikan bahwa $g(x)=f(x)/2015$ untuk setiap $x \in \RR^+$.
        \item Berikan contoh pasangan fungsi yang memenuhi persamaan di atas dan $f(x),g(x) \ge 1$ untuk setiap $x \in \RR^+$.
    \end{enumerate}

    \item (OSN 2012) Misalkan $\RR^+$ menyatakan himpunan semua bilangan real positif. Tunjukkan bahwa tidak ada fungsi $f:\RR^+ \to \RR^+$ yang memenuhi
    \begin{align*}
        f(x+y)=f(x)+f(y)+\dfrac{1}{2012}
    \end{align*}
    untuk setiap bilangan real positif $x,y$.
\end{enumerate}

\section{Referensi}
\begin{enumerate}
    \item https://web.evanchen.cc/handouts/FuncEq-Intro/FuncEq-Intro.pdf
\item https://brilliant.org/wiki/functional-equations/
\item https://cdn.bc-pf.org/resources/math/algebra/fe/Pang\_Cheng\_Wu-FE.pdf
\end{enumerate}


\end{document}
