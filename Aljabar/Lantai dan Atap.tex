\documentclass[11pt]{scrartcl}
\usepackage{graphicx}
\graphicspath{{./}}
\usepackage[sexy]{evan}
\usepackage[normalem]{ulem}
\usepackage{hyperref}
\usepackage{mathtools}
\hypersetup{
    colorlinks=true,
    linkcolor=blue,
    filecolor=magenta,      
    urlcolor=cyan,
    pdfpagemode=FullScreen,
    }

\renewcommand{\dangle}{\measuredangle}

\renewcommand{\baselinestretch}{1.5}

\addtolength{\oddsidemargin}{-0.4in}
\addtolength{\evensidemargin}{-0.4in}
\addtolength{\textwidth}{0.8in}
% \addtolength{\topmargin}{-0.2in}
% \addtolength{\textheight}{1in} 


\setlength{\parindent}{0pt}

\usepackage{pgfplots}
\pgfplotsset{compat=1.15}
\usepackage{mathrsfs}
\usetikzlibrary{arrows}

\title{Lantai dan Atap}
\author{Azzam (IG: haxuv.world)}
\date{\today}
\begin{document}
\maketitle
\section{Sedikit materi}
\begin{itemize}
    \item $\ceiling{x}$ (\textit{ceiling} $x$) adalah bilangan bulat terbesar yang kurang dari sama dengan $x$.
    \item $\floor{x}$ atau $[x]$ (\textit{floor} $x$) adalah bilangan bulat terkecil yang lebih dari sama dengan $x$.
    \item (Fractional Part) $\{x\} = x - \floor{x}$.
    \item Untuk semua $x,y \in \RR$ berlaku $\floor{x+y} \ge \floor{x}+\floor{y}$.
    \item $\floor{x} \le x < \floor{x}+1$.
    \item $\ceiling{x} \ge x > \ceiling{x}-1$.
    \item Jika $n$ bulat, $\floor{x+n}=\floor{x}+n.$
\end{itemize}
\section{Sampel Soal}
\begin{enumerate}
    \item Let $[x]$ denote the largest integer not exceeding $x$. For example, $[2.1]=2$, $[4]=4$ and $[5.7]=5$. How many positive integers $n$ satisfy the equation $\left[\frac{n}{5}\right]=\frac{n}{6}$.

    \item (A Famous Lemma) For real number $x$, show that $\floor{x+\frac{1}{2}} = \floor{2x}-\floor{x}$.

    \item (Hermite's Identity) Let $x$ be a real number, and let $n$ be a positive integer. Then prove that
    \begin{align*}
         \lfloor nx \rfloor = \lfloor x \rfloor + \left\lfloor x+\frac{1}{n} \right\rfloor + \left\lfloor x+\frac{2}{n} \right\rfloor + \dots + \left\lfloor x+\frac{n-1}{n} \right\rfloor.
    \end{align*}
    

    \item (Hong Kong IMO Prelim 1999-2000) Find the integer $n$ satisfying $\left[\frac{n}{1!}\right]+\left[\frac{n}{2!}\right]+...+\left[\frac{n}{10!}\right]=1999$. Here $[x]$ denotes the greatest integer less than or equal to $x$.

    \item (Putnam 1986) What is the units (i.e., rightmost) digit of
\[\left\lfloor \frac{10^{20000}}{10^{100}+3}\right\rfloor\]

    \item  (USAMO 1981) If $x$ is a positive real number, and $n$ is a positive integer, prove that
\[[nx] \geq \frac{[x]}{1} + \frac{[2x]}{2} + \frac{[3x]}{3} + ... + \frac{[nx]}{n},\]where $[t]$ denotes the greatest integer less than or equal to $t$.

    \item (IMO 1968) Let $[x]$ denote the integer part of $x$, i.e., the greatest integer not exceeding $x$. If $n$ is a positive integer, express as a simple function of $n$ the sum\[\left[\frac{n+1}{2}\right]+\left[\frac{n+2}{4}\right]+...+\left[\frac{n+2^k}{2^{k+1}}\right]+\ldots\]

    \item Let $n$ be a natural number. Prove that
    \begin{align*}
        \left\lfloor \frac{n}{1} \right\rfloor + \left\lfloor \frac{n}{2} \right\rfloor + \dots + \left\lfloor \frac{n}{n} \right\rfloor + \lfloor \sqrt{n} \rfloor
    \end{align*}
    is even.

    \item Define $q(n)=\left\lfloor\frac{n}{\lfloor\sqrt{n}\rfloor}\right\rfloor$ for $n \in \mathbb{N}$. Determine all $n$ such that $q(n)>q(n+1)$.

    \item Let $r \ge 1$ be real number such that for any positive integers $m$ and $n$ whenever $m \mid n$ it is also true that $\lfloor mr \rfloor$ divides $\lfloor nr \rfloor$. Show that $r$ is an integer.
\end{enumerate}
\end{document}