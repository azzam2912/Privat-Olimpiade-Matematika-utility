\documentclass{article}
\usepackage{amsmath}
\usepackage{amssymb}

\renewcommand{\baselinestretch}{1.5}
\addtolength{\oddsidemargin}{-1in}
\addtolength{\evensidemargin}{-1.5in}
\addtolength{\textwidth}{1.9in}

\addtolength{\topmargin}{-1in}
\addtolength{\textheight}{1.9in} 

\title{Sequences and Series}
\author{Compiled by Azzam L. H.}

\date{\today}

\begin{document}
\maketitle
 \begin{enumerate}
    \item An infinite geometric series has sum 2005. A new series, obtained by squaring each term of the original series, has 10 times the sum of the original series. The common ratio of the original series is $\frac mn$ where $m$ and $n$ are relatively prime integers. Find $m+n.$

    \item For each positive integer $n \geq 1$ , we define the recursive relation given by $a_{n+1} = \frac {1}{1+a_{n}}$. Suppose that $a_{1} = a_{2012}$.Find the sum of the squares of all possible values of $a_{1}$.

    \item Except for the first two terms, each term of the sequence $1000, x, 1000 - x,\ldots$ is obtained by subtracting the preceding term from the one before that. The last term of the sequence is the first negative term encounted. What positive integer $x$ produces a sequence of maximum length?

    \item A sequence of numbers $x_{1},x_{2},x_{3},\ldots,x_{100}$ has the property that, for every integer $k$ between $1$ and $100,$ inclusive, the number $x_{k}$ is $k$ less than the sum of the other $99$ numbers. Given that $x_{50} = \frac{m}{n}$, where $m$ and $n$ are relatively prime positive integers, find $m + n$.

    \item Find the smallest prime that is the fifth term of an increasing arithmetic sequence, all four preceding terms also being prime.

    \item Given that\begin{align*}x_{1}&=211,\\ x_{2}&=375,\\ x_{3}&=420,\\ x_{4}&=523,\ \text{and}\\ x_{n}&=x_{n-1}-x_{n-2}+x_{n-3}-x_{n-4}\ \text{when}\ n\geq5, \end{align*}find the value of $x_{531}+x_{753}+x_{975}$.

    \item Find the smallest integer $k$ for which the conditions
        \begin{itemize}
            \item $a_1,a_2,a_3\cdots$ is a nondecreasing sequence of positive integers
            \item $a_n=a_{n-1}+a_{n-2}$ for all $n>2$
            \item $a_9=k$
        \end{itemize}
    are satisfied by more than one sequence.

    \item Consider the sequence defined by $a_k =\dfrac{1}{k^2+k}$ for $k\geq 1$. Given that $a_m+a_{m+1}+\cdots+a_{n-1}=\dfrac{1}{29}$, for positive integers $m$ and $n$ with $m<n$, find $m+n$.

    \item Given that\begin{align*}x_{1}&=211,\\ x_{2}&=375,\\ x_{3}&=420,\\ x_{4}&=523,\ \text{and}\\ x_{n}&=x_{n-1}-x_{n-2}+x_{n-3}-x_{n-4}\ \text{when}\ n\geq5, \end{align*}find the value of $x_{531}+x_{753}+x_{975}$.

    \item Find the integer that is closest to $1000\sum_{n=3}^{10000}\frac1{n^2-4}$.

    \item It is known that, for all positive integers $k$, $$1^2+2^2+3^2+\ldots+k^{2}=\frac{k(k+1)(2k+1)}6.$$
    Find the smallest positive integer $k$ such that $1^2+2^2+3^2+\ldots+k^2$ is a multiple of $200$.

    \item In an increasing sequence of four positive integers, the first three terms form an arithmetic progression, the last three terms form a geometric progression, and the first and fourth terms differ by $30$. Find the sum of the four terms.
\end{enumerate}
\end{document}