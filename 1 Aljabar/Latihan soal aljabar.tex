\documentclass{article}
\usepackage{amsmath}
\usepackage{amssymb}
\usepackage{graphicx}
\graphicspath{{./Mini KSK Simulation Problems/}}

\usepackage{enumitem}
\renewcommand{\baselinestretch}{1.5}
\addtolength{\oddsidemargin}{-1in}
\addtolength{\evensidemargin}{-2in}
\addtolength{\textwidth}{1.8in}

\addtolength{\topmargin}{-1in}
\addtolength{\textheight}{1in} 

\title{Pendalaman Aljabar}
\author{}

\date{Senin, 24 Mei 2021}

\begin{document}
	\maketitle
	
	\section{Isian}
	\begin{enumerate}
		\item Jika $a+\dfrac{5}{\sqrt{a}}=26$, maka  $a^2-27a+10=...$
		
		\item Carilah bilangan real $x$ yang memenuhi $x^x = \dfrac{1}{\sqrt{2}}.$
		
		\item Carilah nilai $x$ yang memenuhi $f(f(f(f(f(x)))))=0$ dimana $f(x)=x^2+6x+6$.
		
		\item Carilah bilangan bulat $-500\le k \le 500$ sehingga $\log(kx)=2\log(x+2)$ mempunyai tepat satu solusi real.
		
		\item Sebuah barisan naik bilangan bulat positif $a_1, a_2, \dots$ memenuhi untuk setiap bilangan bulat positif $k$, subbarisan $a_{2k-1},a_{2k},a_{2k+1}$ adalah barisan geometri dan $a_{2k},a_{2k+1},a_{2k+2}$ adalah barisan aritmatika. Jika $a_{13}=2016$, carilah $a_1$.
		
		\item Misalkan $P(x)$ adalah polinomial tak nol sehingga $(x-1)P(x+1)=(x+2)P(x)$ untuk setiap bilangan real $x$, dan $(P(2))^2 = P(3)$. Carilah nilai $P(\frac{7}{2})$.
		
		\item Misalkan $a > 1$ dan $x > 1$ memenuhi $\log_a(\log_a(\log_a 2) + \log_a 24 - 128) = 128$ and $\log_a(\log_a x) = 256$. Carilah tiga digit terakhir $x$.
		
		\item Misalkan $x,y,$ dan $z$ adalah bilangan real yang memenuhi
		\begin{align*} \log_2(xyz-3+\log_5 x)&=5,\\ \log_3(xyz-3+\log_5 y)&=4,\\ \log_4(xyz-3+\log_5 z)&=4. 
		\end{align*}Carilah nilai dari $|\log_5 x|+|\log_5 y|+|\log_5 z|$.
	
		\item barisan bilangan bulat positif $1,a_2, a_3,...$ dan $1,b_2, b_3,...$ adalah barisan naik aritmatika dan barisan naik geometri secara berturut-turut. Misalkan $c_n=a_n+b_n$. Terdapat bilangan bulat $k$ sehingga $c_{k-1}=100$ dan $c_{k+1}=1000$. Carilah $c_k$.
		
		\item Definisikan polinomial $P(x)=1-\dfrac{1}{3}x+\dfrac{1}{6}x^{2}$, dan definisikan $Q(x)=P(x)P(x^{3})P(x^{5})P(x^{7})P(x^{9})=\sum_{i=0}^{50} a_ix^{i}$. Jika $\sum_{i=0}^{50} |a_i|=\dfrac{m}{n}$, dimana $m$ dan $n$ bilangan bulat yang saling prima, carilah $m+n$.
	\end{enumerate}

	\section{Esai}
	\begin{enumerate}
		\item Carilah seluruh fungsi $f: \mathbb{R}^+ \to \mathbb{R}^+$ sedemikian sehingga
		$$(z + 1)f(x + y) = f(xf(z) + y) + f(yf(z) + x),$$untuk seluruh bilangan real positif $x, y, z$.
		
		\item Carilah seluruh tripel bilangan real $(a,b,c)$ yang memenuhi:
		$$\begin{cases} a+b+c=\frac{1}{a}+\frac{1}{b}+\frac{1}{c} \\a^2+b^2+c^2=\frac{1}{a^2}+\frac{1}{b^2}+\frac{1}{c^2}\end{cases}$$
	\end{enumerate}


\end{document}