\documentclass[11pt]{scrartcl}
\usepackage{graphicx}
\graphicspath{{./}}
\usepackage[sexy]{evan}
\usepackage[normalem]{ulem}
\usepackage{hyperref}
\usepackage{mathtools}
\hypersetup{
    colorlinks=true,
    linkcolor=blue,
    filecolor=magenta,      
    urlcolor=cyan,
    pdfpagemode=FullScreen,
    }

\renewcommand{\dangle}{\measuredangle}

\renewcommand{\baselinestretch}{1.5}

\addtolength{\oddsidemargin}{-0.4in}
\addtolength{\evensidemargin}{-0.4in}
\addtolength{\textwidth}{0.8in}
% \addtolength{\topmargin}{-0.2in}
% \addtolength{\textheight}{1in} 


\setlength{\parindent}{0pt}

\usepackage{pgfplots}
\pgfplotsset{compat=1.15}
\usepackage{mathrsfs}
\usetikzlibrary{arrows}

\title{Kompilasi 30 Soal Kombinatorika}
\author{Pelatihan OSP}
\date{April 2023}

\begin{document}

\maketitle

\section{Combinatorial Games}
\begin{soaljawab}
    Akira dan Benjiro mempunyai tak hingga banyaknya koin bundar yang identik. Akira dan Benjiro bergantian menaruh koin tersebut di meja persegi yang ukurannya terbatas (finit) sehingga tidak ada dua koin yang saling bertumpuk dan setiap koin tepat di atas meja (jadi tidak ada yang koin yang menggantung di pinggiran meja sehingga bisa jatuh). Orang yang tidak bisa menempatkan koin di meja saat gilirannya dinyatakan kalah. Asumsikan setidaknya satu koin dapat ditaruh di meja. Jika Akira main duluan, buktikan bahwa Akira punya strategi menang.
\end{soaljawab}

\begin{soaljawab}
Xeratha dan Haxuv sedang memainkan Cram, dimana giliran pertama dimainkan Xeratha dan mereka bergantian menempatkan domino (ubin berukuran $1 \times 2$ atau $2 \times 1$) pada grid persegi panjang $m \times n$ dan $mn$ bernilai genap. Xeratha maupun Haxuv harus menempatkan ubin domino berukuran secara vertikal atau horizontal, dengan ubin domino tersebut tidak boleh tumpang tindih atau keluar dari papan. Pemain yang tidak bisa melakukan langkah untuk pertama kalinya dinyatakan kalah dan pemain yang dapat melakukan langkah terakhir dinyatakan sebagai pemenang. Jika diberikan $m$ dan $n$, tentukan siapa yang memiliki strategi kemenangan, dan jelaskan strateginya.
\end{soaljawab}

\begin{soaljawab}
Permainan catur ganda adalah permainan seperti catur biasa, tetapi bedanya setiap pemain melakukan dua langkah di setiap gilirannya (putih bermain dua kali, kemudian hitam bermain dua kali, dan seterusnya). Tunjukkan bahwa putih selalu bisa menang atau seri. (Putih bermain duluan)
\end{soaljawab}

\begin{soaljawab}
Sakura dan Hinata bermain sebuah permainan dimana mereka pada awalnya nilai $x=0$ dan mereka bergantian menambahkan salah satu angka dari $S=\{1,2,\dots,10\}$ ke $x$. Pemain yang pertama kali membuat $x$ bernilai $1320$ menang. Jika Sakura memainkan giliran pertama, siapakah yang memiliki strategi menang?
\end{soaljawab}

\begin{soaljawab}
% all russian 2023
Ada $2000$ komponen dalam suatu rangkaian listrik, di mana setiap dua komponen dihubungkan dengan kawat pada awalnya. Pipit dan Pepet iseng melakukan permainan (yang berbahaya) dengan memutuskan kawat pada rangkaian tersebut satu per satu. Pipit, yang mulai duluan, memutuskan satu kawat pada gilirannya, sementara Pepet memutuskan satu atau tiga kawat. Orang yang memutuskan kawat terakhir dari suatu komponen dinyatakan kalah. Siapa yang memiliki strategi menang?
\end{soaljawab}

\begin{soaljawab}
Ada tiga ember kosong di atas meja. Anya, Loid, dan Yor meletakkan kenari satu per satu ke dalam ember secara bergantian, dengan urutan yang ditentukan oleh Loid di awal permainan. Dengan demikian, Anya meletakkan kenari di ember pertama atau kedua, Loid meletakkan kenari di ember kedua atau ketiga, dan Yor meletakkan kenari di ember pertama atau ketiga. Pemain yang setelah gilirannya membuat ada tepat 2023 kenari di salah satu ember dinyatakan sebagai pemain yang kalah. Tunjukkan bahwa Anya dan Yor dapat bekerja sama sehingga membuat Loid kalah.
\end{soaljawab}

\begin{soaljawab}
    % korea 2009
    Diberikan sebuah papan catur berukuran $(m+1) \times m$ untuk $m \ge 10$ (mempunyai $m+1$ baris dan $m$ kolom). Sebuah batu diletakkan di ujung kiri atas papan catur tersebut. Mahiro dan Mihari bermain dengan memindahkan batu sesuai aturan berikut:
    \begin{itemize}
        \item setiap pemain memindahkan batu ke satu kotak di sebelahnya secara bergiliran,
        \item kotak yang telah dilewati sekali oleh batu tersebut tidak boleh dilewati lagi, dan
        \item pemain yang tidak bisa memindahkan batu dinyatakan kalah.
    \end{itemize}
    Jika Mahiro memainkan giliran pertama, tentukan siapa yang mempunyai strategi menang.
\end{soaljawab}

% MO for girl 2022  (UKMT)
\begin{soaljawab}
    Freya dan Greesel sedang memainkan sebuah permainan. Pertama, Freya memilih sebuah bilangan bulat $a$. Lalu, Greesel memilih sebuah bilangan bulat $b$ dimana $a,b \in \{1,2,\dots,2023\}$. Setelah $a$ dan $b$ terpilih, mereka berdua membuar sebuah barisan $(c_n)$ dimana $c_n = an + b$ untuk $n = 1,2,\dots$. Jika setidaknya salah satu suku dari barisan $(c_n)$ habis dibagi 10 maka Freya menang dan jika tidak ada yang habis dibagi 10, Greesel yang menang. Berapa banyak nilai $a$ sehingga dijamin Freya dapat memenangkan permainan tersebut terlepas dari apapun bilangan $b$ yang dipilih Greesel?
\end{soaljawab}

% INMO 2023
\begin{soaljawab}
    Misalkan $k \ge 1$ dan $N > 1$ adalah dua bilangan asli. Chisato dan Takina memainkan sebuah permainan pada sebuah lingkaran yang pada kelilingnya diletakkan $2N+1$ koin dengan keadaan awal semuanya menunjukkan gambar (dimana setiap koin tersebut mempunyai sisi gambar dan angka). Chisato mulai duluan dengan di setiap gilirannya ia bisa membalik koin yang menunjukkan gambar sehingga menjadi koin yang menunjukkan angka. Takina pada gilirannya dapat membalik paling banyak satu koin yang menunjukkan angka di sebelah koin yang barusan dibalik Chisato sehingga menjadi koin yang menunjukkan gambar. Chisato menang pada saat ada $k$ koin yang menunjukkan angka setelah Takina menyelesaikan gilirannya. Tentukan semua nilai $k$ sehingga Chisato dapat memenangkan permainan tersebut.
\end{soaljawab}

% JMO 2023
\begin{soaljawab}
Dua pemain, Budi dan Rayyan, bermain permainan berikut pada sebuah grid tak berhingga yang tersusun atas kotak satuan, yang pada keadaan awal semuanya berwarna putih. Para pemain bergantian dengan Budi bergiliran pertama. Pada giliran Budi, ia memilih satu kotak satuan putih dan memberinya warna biru. Pada giliran Rayyan, ia memilih dua kotak satuan putih dan memberinya warna merah. Para pemain bergantian sampai Budi memutuskan untuk mengakhiri permainan. Pada saat ini, Budi mendapatkan skor yang menyatakan banyaknya kotak satuan di dalam poligon sederhana terbesar (dalam hal luas) yang hanya berisi kotak satuan biru. Berapakah skor terbesar yang bisa didapat oleh Budi?

(Suatu poligon sederhana adalah poligon (tidak harus konveks) atau daerah yang tidak berpotongan dengan dirinya sendiri dan tidak memiliki lubang)
\end{soaljawab}

\section{Coloring}
\begin{soaljawab}
    2023 koin akan diletakkan pada papan berukuran $n \times n$ sedemikian sehingga selisih banyaknya koin pada setiap 2 persegi yang bertetangga adalah 1. Jika 2 persegi disebut bertetangga apabila mereka mempunyai atau berbagi satu sisi yang sama, carilah nilai $n$ terbesar yang mungkin.
\end{soaljawab}

\begin{soaljawab}
    Apakah mungkin untuk berjalan pada taman yang menyerupai papan catur $8 \times 8$ sehingga anda hanya bisa berjalan melewati setiap kotak $1 \times 1$ tepat sekali dengan kotak awal dan kotak akhir perjalanan tersebut berada pada ujung-ujung (corner) yang saling berlawanan? (Anda hanya diperbolehkan berjalan ke kotak yang bertetangga atau tepat bersebelahan dari kotak anda sekarang).
\end{soaljawab}

\begin{soaljawab}
    Diketahui bahwa delapan persegi panjang berukuran $1 \times 3$ dan satu kotak berukuran $1 \times 1$ menutup papan berukuran $5 \times 5$. Tunjukkan bahwa kotak berukuran $1 \times 1$ tersebut harus berada di tengah papan. (persegi panjang berukuran $1 \times 3$ dan $3 \times 1$ dianggap sama)
\end{soaljawab}

\begin{soaljawab}
    Dapatkah papan $8\times8$ ditutupi dengan lima belas persegi panjang $1\times4$ dan satu buah persegi $2\times2$ tanpa tumpang tindih?
\end{soaljawab}

\begin{soaljawab}
    Misalkan $m,n > 2$ adalah bilangan bulat. Warnai setiap persegi $1\times1$ dari papan $m\times n$ dengan warna hitam atau putih (tetapi tidak keduanya). Jika dua persegi $1\times1$ yang tepat saling bersebelahan memiliki warna yang berbeda, maka sebut pasangan persegi ini sebagai pasangan "roman". Definisikan $S$ menjadi jumlah pasangan roman di papan $m\times n$. Buktikan bahwa genap atau ganjilnya $S$ hanya tergantung pada persegi $1\times1$ pada pinggiran papan selain 4 persegi $1\times1$ di ujung-ujung sudut papan.
\end{soaljawab}

\begin{soaljawab}
    Terdapat 1004 titik yang berbeda pada sebuah bidang. Hubungkan setiap pasang titik tersebut dan tandai titik tengah dari segmen garis ini dengan warna hitam. Buktikan bahwa terdapat setidaknya 2005 titik hitam pada bidang tersebut dan buktikan ada satu himpunan berisi 1004 titik berbeda yang menghasilkan tepat 2005 titik hitam pada titik tengah dari segmen garis yang menghubungkan pasangan titik-titik tersebut.
\end{soaljawab}

\begin{soaljawab}
    Cari seluruh cara mewarnai semua bilangan positif sedemikian sehingga 
    \begin{enumerate}[(a)]
        \item setiap bilangan positif diberi warna hitam atau putih (tapi tidak keduanya) dan \item jumlah dari dua bilangan dengan warna yang berbeda selalu berwarna hitam dan hasil kali keduanya selalu berwarna putih.
    \end{enumerate}
     
    Tentukan juga warna dari hasil kali dua bilangan yang diwarnai putih.
\end{soaljawab}

\begin{soaljawab}
    Di bidang kartesius, sebuah titik $(x,y)$ disebut sebagai titik letis jika dan
hanya jika $x$ dan $y$ adalah bilangan bulat. Misalkan ada sebuah segi lima konveks $ABCDE$
yang titik-titiknya adalah titik letis dan panjang kelima sisinya adalah bilangan bulat.
Buktikan bahwa keliling segi lima $ABCDE$ adalah bilangan bulat genap.
\end{soaljawab}

\begin{soaljawab}
    Sebuah kisi berukuran $5 \times 5$ yang setiap kotaknya berisi lampu-lampu, mengalami kerusakan. Kerusakan ini mengakibatkan setiap memencet sakelar sebuah lampu menyebabkan setiap lampu yang bersebelahan di baris yang sama dan kolom yang sama (beserta lampu itu sendiri) berubah keadaannya, dari nyala menjadi mati, atau dari mati menjadi nyala. Awalnya semua lampu dimatikan. Setelah beberapa kali pemutaran sakelar, tepat satu lampu menyala. Temukan semua posisi yang mungkin dari lampu ini.
\end{soaljawab}

\begin{soaljawab}
    % nordic 2020
    Ultrawati memiliki $2n + 1$ kartu dengan sebuah angka ditulis pada setiap kartu. Pada salah satu kartu terdapat angka 0, dan di antara sisa kartu yang ada, bilangan bulat $k = 1, \dots, n$ muncul masing-masing dua kali. Ultrawati ingin meletakkan kartu-kartu tersebut dalam satu baris sedemikian sehingga kartu 0 berada di tengah, dan untuk setiap $k = 1, ..., n$, kedua kartu dengan angka $k$ memiliki jarak $k$ (yang berarti ada tepat $k - 1$ kartu di antara mereka). Untuk nilai $n$ berapa saja, dengan $1 \le n \le 10$, hal ini dimungkinkan?
\end{soaljawab}

\section{Bijection}
\begin{soaljawab}
    Tanpa menggunakan manipulasi aljabar, buktikan secara kombinatorial bahwa
    $$\sum_{k=1}^{n} k^4 = {n+1 \choose 2} + 14{n+1 \choose 3}+36{n+1 \choose 4}+24{n+1 \choose 5}$$ dengan $n\in \NN$ dan $n \ge 4$.
\end{soaljawab}

\begin{soaljawab}
%ISL 2009 C1
Misalkan terdapat $M \ge 1$ kartu, masing-masing memiliki satu sisi berwarna emas dan satu sisi berwarna hitam, diletakkan secara sejajar di atas meja panjang. Awalnya semua kartu menunjukkan sisi emas mereka. Oniel memainkan kartu tersebut sehingga di setiap giliran ia memilih satu blok $k$ kartu berurutan, dengan kartu paling kiri menunjukkan sisi emas, dan membalikkan semua kartu tersebut, sehingga kartu yang sebelumnya menunjukkan sisi emas sekarang menunjukkan sisi hitam dan sebaliknya ($M \ge k \ge 1$). Apakah permainan pasti akan berakhir?
\end{soaljawab}

\begin{soaljawab}
%OTC 2017 paket 1
Barisan bilangan asli $(a_n) = 1,3,4,9,10,12,13,\dots$ adalah barisan dari bilangan yang merupakan bilangan 3 berpangkat (seperti $1,3,9,27,...$ ) dan bilangan yang merupakan penjumlahan dari bilangan berbeda yang berbentuk 3 berpangkat. Tentukanlah nilai suku ke 100.
\end{soaljawab}

\begin{soaljawab}
% USAMO 1983
Terdapat 20 anggota di suatu klub tenis yang menjadwalkan tepat 14 permainan antar dua orang diantara mereka dengan setiap anggota klub bermain minimal satu kali. Buktikan bahwa dalam pembagian ini, terdapat himpunan 6 permainan dengan 12 pemain yang berbeda.
\end{soaljawab}

\begin{soaljawab}
    % pelatnas 1 IMO 2019
    Hitunglah banyaknya polinomial $P(x)$ dengan koefisien-koefisien yang dipilih dari $\{0, 1, 2, 3\}$ sedemikian sehingga $P(2) = 2023$.
\end{soaljawab}

\begin{soaljawab}
    % USAMO 1996
    Sebuah barisan $n$ suku $(x_1,x_2,\dots,x_n)$ dimana setiap sukunya bernilai 0 atau 1 disebut sebagai barisan biner dengan panjang $n$. Definisikan $a_n$ sebagai banyaknya barisan biner dengan panjang $n$ yang tidak mengandung 3 suku berurutan $0,1,0$ dalam urutan tersebut. Definisikan $b_n$ sebagai banyaknya barisan biner dengan panjang $n$ yang tidak mengandung 4 suku berurutan $0,0,1,1$ atau $1,1,0,0$  dalam urutan tersebut. Buktikan bahwa $b_{n+1} = 2a_{n}$ untuk semua bilangan asli $n$.
\end{soaljawab}

\begin{soaljawab}
    Misalkan sebuah layar LED raksasa berukuran $2023 \times 2025$ dinyalakan. Layar tersebut tersusun atas $2023 \times 2025$ layar satuan berukuran $1 \times 1$. Pada awalnya ada lebih dari $2022 \times 2024$ layar satuan yang menyala. Namun ternyata layar tersebut rusak sehingga jika di setiap daerah layar berukuran $2 \times 2$ ada 3 layar satuan yang mati, maka layar satuan ke-4 juga akan ikut mati. Buktikan bahwa layar tersebut tidak akan pernah benar-benar mati (masih ada layar satuan yang menyala).
\end{soaljawab}

\begin{soaljawab}
%ISL 2002 C1
    Diberikan $n$ adalah bilangan bulat positif. Misalkan setiap titik $(x,y)$ pada koordinat kartesius dengan $x$ dan $y$ adalah bilangan bulat nonnegatif yang memenuhi $x+y < n$, diwarnai biru atau merah dengan aturan berikut: jika titik $(x,y)$ berwarna merah, maka semua titik $(x',y')$ dengan $x' \le x$ dan $y' \le y$ juga berwarna merah. Definisikan $A$ sebagai banyaknya cara memilih $n$ titik biru dengan koordinat-$x$ yang berbeda, dan definisikan $B$ sebagai banyaknya cara memilih $n$ titik biru dengan koordinat-$y$ yang berbeda. Buktikan bahwa $A=B$.
\end{soaljawab}

\begin{soaljawab}
    %ISL 2002 C3
    Sebuah barisan $n$ bilangan bulat positif (tidak harus berbeda) disebut \textit{kawaii} jika memenuhi kondisi berikut: untuk setiap bilangan bulat positif $k \ge 2$, jika bilangan $k$ muncul dalam barisan tersebut, maka bilangan $k-1$ juga muncul, dan kemunculan pertama bilangan $k-1$ muncul sebelum kemunculan terakhir dari $k$. Untuk setiap bilangan bulat positif $n$, ada berapa barisan kawaii yang mungkin?
\end{soaljawab}

\begin{soaljawab}
    % putnam 2005
    Misalkan sebuah ruangan berukuran $n \times 3$ (mempunyai $n$ baris dan $3$ kolom) lantainya ditutupi oleh ubin satuan berukuran $1 \times 1$. Misalkan ubin yang berada pada baris ke-$i$ dan kolom ke-$j$ dinotasikan dengan $(i,j)$. Definisikan \textit{langkah satuan} sebagai langkah yang dilakukan antar satu ubin satuan ke ubin satuan yang berada tepat di sebelahnya. Tentukan banyaknya perjalanan atau jalur di ruangan tersebut sehingga kita bisa berjalan dari $(1,1)$ ke $(n,1)$ dengan langkah satuan dan melewati setiap titik di ruangan tersebut tepat sekali.
\end{soaljawab}

\end{document}
