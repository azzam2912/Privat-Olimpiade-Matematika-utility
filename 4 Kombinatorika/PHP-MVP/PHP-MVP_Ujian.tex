\documentclass[11pt]{scrartcl}
\usepackage{graphicx}
\graphicspath{{./}}
\usepackage[sexy]{evan}
\usepackage[normalem]{ulem}
\usepackage{hyperref}
\usepackage{mathtools}
\hypersetup{
    colorlinks=true,
    linkcolor=blue,
    filecolor=magenta,      
    urlcolor=cyan,
    pdfpagemode=FullScreen,
    }

\renewcommand{\dangle}{\measuredangle}

\renewcommand{\baselinestretch}{1.5}

\addtolength{\oddsidemargin}{-0.4in}
\addtolength{\evensidemargin}{-0.4in}
\addtolength{\textwidth}{0.8in}
% \addtolength{\topmargin}{-0.2in}
% \addtolength{\textheight}{1in} 


\setlength{\parindent}{0pt}

\usepackage{pgfplots}
\pgfplotsset{compat=1.15}
\usepackage{mathrsfs}
\usetikzlibrary{arrows}

\title{PigeonHole Principle and Mean Value Principle - Soal Post Test}
\author{Azzam (IG: haxuv.world)}
\date{Sabtu, 27 Januari 2024}

\begin{document}
\maketitle
\textbf{Aturan umum:}
\begin{itemize}
    \item Tulis \textbf{nama lengkap} dan \textbf{asal sekolah} di pojok kiri atas halaman pertama.
    \item \textbf{Soal bertipe esai}. Sertakan argumentasi atau cara mendapatkan jawaban yang ditanyakan di soal.
    \item Waktu standar untuk mengerjakan semua soal berikut adalah 90-120 menit.
    \item Setiap soal bernilai bilangan bulat antara 0 sampai 10 (inklusif).
    \item Soal yang wajib dikerjakan untuk mendapatkan \textit{full points} adalah \textbf{TIGA SOAL}.
\end{itemize}


\section{Soal}
\begin{enumerate}
    \item Given 7 points in $\triangle ABC$, no three of which are collinear. Prove that there exists 3 points of these points which determine a triangle with the area less than or equal to $\frac{1}{4} S_{\triangle ABC}$.

       \item Suppose 10 persons buy 30 books at a book-shop satisfying the following conditions:
    \begin{enumerate}
        \item Each of them buys 3 books,
        \item Among the books which were bought by any two of them, there are at least one is the same. 
        Suppose the number of purchasers who bought some book is largest. Determine what that smallest value of the largest number is? 
    \end{enumerate}
    
    \item Jika $p$ bilangan prima dan bilangan asli $n$ tidak habis dibagi oleh $p$, maka ada bilangan bulat $m$ dengan $1 \leq m \leq p - 1$ sehingga $p$ membagi $mn - 1$.\\

    \item Perlihatkan bahwa untuk sebarang bilangan bulat sebanyak 52 selalu ada dua bilangan yang jumlahnya atau selisihnya habis dibagi 100.

    \item Lima titik terletak pada segitiga sama sisi dengan panjang sisi satu satuan panjang. Buktikan bahwa ada dua titik yang berjarak kurang atau sama dengan $\frac{1}{2}$ satuan panjang.
\end{enumerate}
\end{document}