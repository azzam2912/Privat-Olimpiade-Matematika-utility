\documentclass[11pt]{scrartcl}
\usepackage{graphicx}
\graphicspath{{./}}
\usepackage[sexy]{evan}
\usepackage[normalem]{ulem}
\usepackage{hyperref}
\usepackage{mathtools}
\hypersetup{
    colorlinks=true,
    linkcolor=blue,
    filecolor=magenta,      
    urlcolor=cyan,
    pdfpagemode=FullScreen,
    }

\renewcommand{\dangle}{\measuredangle}

\renewcommand{\baselinestretch}{1.5}

\addtolength{\oddsidemargin}{-0.4in}
\addtolength{\evensidemargin}{-0.4in}
\addtolength{\textwidth}{0.8in}
% \addtolength{\topmargin}{-0.2in}
% \addtolength{\textheight}{1in} 


\setlength{\parindent}{0pt}

\usepackage{pgfplots}
\pgfplotsset{compat=1.15}
\usepackage{mathrsfs}
\usetikzlibrary{arrows}

\title{Combinatorial Games dan Invarian - Soal Post Test}
\author{Azzam (IG: haxuv.world)}
\date{Kamis, 25 Januari 2024}

\begin{document}
\maketitle
\textbf{Aturan umum:}
\begin{itemize}
    \item Tulis \textbf{nama lengkap} dan \textbf{asal sekolah} di pojok kiri atas halaman pertama.
    \item \textbf{Soal bertipe esai}. Sertakan argumentasi atau cara mendapatkan jawaban yang ditanyakan di soal.
    \item Waktu standar untuk mengerjakan semua soal berikut adalah 90-120 menit.
    \item Setiap soal bernilai bilangan bulat antara 0 sampai 10 (inklusif).
    \item Soal yang wajib dikerjakan untuk mendapatkan \textit{full points} adalah \textbf{TIGA SOAL}.
\end{itemize}


\section{Soal}
\begin{enumerate}
    \item There are three piles with $n$ tokens each. In every step we are allowed to choose two piles, take one token from each of those two piles and add a token to the third pile. Using these moves, is it possible to end up having only one token?
    
    \item Sakura dan Hinata bermain sebuah permainan dimana mereka pada awalnya nilai $x=0$ dan mereka bergantian menambahkan salah satu angka dari $S=\{1,2,\dots,10\}$ ke $x$. Pemain yang pertama kali membuat $x$ bernilai $1320$ menang. Jika Sakura memainkan giliran pertama, siapakah yang memiliki strategi menang?
    
    \item Ada tiga ember kosong di atas meja. Anya, Loid, dan Yor meletakkan kenari satu per satu ke dalam ember secara bergantian, dengan urutan yang ditentukan oleh Loid di awal permainan. Dengan demikian, Anya meletakkan kenari di ember pertama atau kedua, Loid meletakkan di ember kedua atau ketiga, dan Yor meletakkan di ember pertama atau ketiga. Pemain yang setelah gilirannya membuat ada tepat 2023 kenari di salah satu ember dinyatakan sebagai pemain yang kalah. Tunjukkan bahwa Anya dan Yor dapat bekerja sama sehingga membuat Loid kalah.
    
    \item A pirate ship has 2009 treasure chests (all chests are closed). Each chest contains some amount of gold. To distribute the gold the pirates are going todo the following: The captain is going to decide first how many chests he wants to keep and tell that number to the rest of the pirates. Then he is going to open all the chests and decide which ones he wants to keep (he can only choose as many as he said before opening them). The captain wants to make sure he can keep at least half of the total gold. However, he wants to say the smallest possible number to keep the rest of the pirates as happy as he can. What number should the captain say?
    Note: The chests may be empty.

    \item On an $8 \times 8$ board there is a lamp in every square. Initially every lamp is turned off. In a move we choose a lamp and a direction (it can be the vertical direction or the horizontal one) and change the state of that lamp and all its neighbors in that direction. After a certain number of moves, there is exactly one lamp turned on. Find all the possible positions of that lamp. %soberon, 3.4, oim shortlist 2009
    % example 3.3.6 soberon
\end{enumerate}
\end{document}