\documentclass[12pt]{scrartcl}
\usepackage[hagavi]{azzam}

\title{\textbf{Latihan Soal Transformasi Geometri}}

\begin{document}

\textbf{\Large Latihan Soal Transformasi Geometri}\\
\textbf{Dikompilasi dari Defantri dan Mathcyber1997}

\renewcommand*\contentsname{Daftar Isi}
\tableofcontents

\section{Mudah}
\begin{enumerate}
    % Soal 1
    \item Diketahui titik $P'(3,-13)$ adalah bayangan titik $P$ oleh translasi $T = \begin{pmatrix}-10 \\ 7 \end{pmatrix}$. Koordinat titik $P$ adalah $\cdots \cdot$
    \begin{enumerate}[label=\Alph*.]
        \item $(13,-20)$
        \item $(13,-4)$
        \item $(4,20)$
        \item $(-5,-4)$
        \item $(-5,-20)$
    \end{enumerate}

    % Soal 2
    \item Bayangan titik $P(a,b)$ oleh rotasi terhadap titik pusat $(0,0)$ sebesar $-90^{\circ}$ adalah $P'(-10,-2)$. Nilai dari $a+2b = \cdots \cdot$
    \begin{enumerate}[label=\Alph*.]
        \item $-18$
        \item $-8$
        \item $8$
        \item $18$
        \item $22$
    \end{enumerate}

    % Soal 3
    \item Bayangan titik $A$ dengan $A(-1,4)$ jika direfleksikan terhadap garis $y=-x$ adalah $\cdots \cdot$
    \begin{enumerate}[label=\Alph*.]
        \item $A'(4,1)$
        \item $A'(-4,1)$
        \item $A'(4,-1)$
        \item $A'(4,3)$
        \item $A'(-4,-1)$
    \end{enumerate}

    % Soal 4
    \item Bayangan titik $P(5,4)$ jika didilatasikan terhadap pusat $(-2,-3)$ dengan faktor skala $-4$ adalah $\cdots \cdot$
    \begin{enumerate}[label=\Alph*.]
        \item $(-30,-31)$
        \item $(-30,7)$
        \item $(-26,-1)$
        \item $(-14,-1)$
        \item $(-14,-7)$
    \end{enumerate}

    % Soal 5
    \item Titik $B(3,-2)$ dirotasikan sebesar $90^{\circ}$ terhadap titik pusat $P(-1,1)$. Bayangan titik $B$ adalah $\cdots \cdot$
    \begin{enumerate}[label=\Alph*.]
        \item $B'(-4,3)$
        \item $B'(-2,1)$
        \item $B'(-1,2)$
        \item $B'(1,4)$
        \item $B'(2,5)$
    \end{enumerate}

    % Soal 6
    \item Bayangan titik $P(2,-3)$ oleh rotasi $R[O,90^{\circ}]$ adalah $\cdots \cdot$
    \begin{enumerate}[label=\Alph*.]
        \item $P'(3,2)$
        \item $P'(2,3)$
        \item $P'(-2,3)$
        \item $P'(-3,2)$
        \item $P'(-3,-2)$
    \end{enumerate}

    % Soal 7
    \item Diketahui koordinat titik $P(-8,12)$. Dilatasi $[P,1]$ memetakan titik $(-4,8)$ ke titik $\cdots \cdot$
    \begin{enumerate}[label=\Alph*.]
        \item $(-4,8)$
        \item $(-4,16)$
        \item $(-4,-8)$
        \item $(4,-16)$
        \item $(4,-8)$
    \end{enumerate}

    % Soal 8
    \item Bayangan titik $B(4,8)$ direfleksikan terhadap sumbu $X$ kemudian dilanjutkan dengan dilatasi $\left[O, \dfrac{1}{2}\right]$ adalah $\cdots \cdot$
    \begin{enumerate}[label=\Alph*.]
        \item $(-2, 4)$
        \item $(2,-4)$
        \item $(8,-2)$
        \item $(-8, 4)$
        \item $(-8,-4)$
    \end{enumerate}

    % Soal 9
    \item Diketahui koordinat titik $T(-1,5)$. Bayangan titik $T$ oleh transformasi yang diwakili oleh matriks $\begin{pmatrix}-4 & 3 \\ 2 & -1 \end{pmatrix},$ dilanjutkan refleksi terhadap garis $x = 8$ adalah $\cdots \cdot$
    \begin{enumerate}[label=\Alph*.]
        \item $T'(30,-7)$
        \item $T'(19, 23)$
        \item $T'(19,-22)$
        \item $T'(3,-7)$
        \item $T'(-3,-7)$
    \end{enumerate}

    % Soal 10
    \item Segitiga $KLM$ dengan $K(6,4), L(-3, 1), M(2,-2)$ didilatasi dengan pusat $(-2, 3)$ dan faktor skala $4.$ Koordinat bayangan $\triangle KLM$ adalah $\cdots \cdot$
    \begin{enumerate}[label=\Alph*.]
        \item $K'(30, 7), L'(-6,-5),$ $M'(14,-17)$
        \item $K'(30, 7), L'(-6,-5),$ $M'(10,-12)$
        \item $K'(30, 7), L'(-3,-7),$ $M'(14,-17)$
        \item $K'(7, 24), L'(-5,-6),$ $M'(14, 8)$
        \item $K'(7, 24), L'(-6,-5),$ $M'(7, 30)$
    \end{enumerate}

    % Soal 11
    \item Segitiga $ABC$ dengan titik $A(-2,3), B(2,3)$, dan $C(0,-4)$ didilatasi dengan pusat $O(0,0)$ dan faktor skala $4$. Luas segitiga setelah didilatasi adalah $\cdots \cdot$
    \begin{enumerate}[label=\Alph*.]
        \item $120$
        \item $224$
        \item $240$
        \item $280$
        \item $480$
    \end{enumerate}

    % Soal 12
    \item Suatu vektor $\overline{a} = (-3,4)$ berturut-turut merupakan hasil pencerminan terhadap garis $y=x$ dan rotasi dengan pusat di titik asal sebesar $90^{\circ}$ searah jarum jam. Vektor awalnya sebelum ditransformasi adalah $\cdots \cdot$
    \begin{enumerate}[label=\Alph*.]
        \item $(3,4)$
        \item $(-3,-4)$
        \item $(-4,3)$
        \item $(4,-3)$
        \item $(-3,4)$
    \end{enumerate}

    % Soal 13
    \item Jika persamaan garis lurus $y = 2x+3,$ maka persamaan garis lurus yang dihasilkan oleh translasi $T = (3, 2)$ adalah $\cdots \cdot$
    \begin{enumerate}[label=\Alph*.]
        \item $y = 3x$
        \item $y = 2x + 6$
        \item $y = 2x-6$
        \item $y = 2x-4$
        \item $y = 2x-1$
    \end{enumerate}

    % Soal 14
    \item Persamaan bayangan garis $2x+y-1=0$ ditransformasikan oleh matriks $\begin{pmatrix} 1 & 1 \\ 1 & 2 \end{pmatrix},$ kemudian dilanjutkan dengan pencerminan terhadap sumbu $X$ adalah $\cdots \cdot$
    \begin{enumerate}[label=\Alph*.]
        \item $3x+y-1=0$
        \item $5x-y+1=0$
        \item $3x+y+1=0$
        \item $5x+y-1=0$
        \item $5x+y+1=0$
    \end{enumerate}

    % Soal 15
    \item Bayangan garis $3x-y+2=0$ apabila dicerminkan terhadap garis $y=x$ dan dilanjutkan dengan rotasi sebesar $90^{\circ}$ dengan pusat $(0,0)$ adalah $\cdots \cdot$
    \begin{enumerate}[label=\Alph*.]
        \item $3x+y+2=0$
        \item $3x+y-2=0$
        \item $-3x+y+2=0$
        \item $-x+3y+2=0$
        \item $x-3y+2=0$
    \end{enumerate}

    % Soal 16
    \item Garis $3x+2y=6$ ditranslasikan oleh $T (3,-4)$, lalu dilanjutkan dilatasi dengan pusat $O$ dan faktor skala $2$. Hasil bayangan transformasinya adalah $\cdots \cdot$
    \begin{enumerate}[label=\Alph*.]
        \item $3x+2y=14$
        \item $3x+2y=7$
        \item $3x+y=14$
        \item $3x+y=7$
        \item $x+3y=14$
    \end{enumerate}

    % Soal 17
    \item Garis $y=2x-3$ ditranslasikan oleh $T = \begin{pmatrix}-2 \\ 3 \end{pmatrix}$. Persamaan bayangan garis tersebut adalah $\cdots \cdot$
    \begin{enumerate}[label=\Alph*.]
        \item $y=2x+4$
        \item $y=2x-4$
        \item $y=2x-3$
        \item $y=-2x+4$
        \item $y=-2x-3$
    \end{enumerate}

    % Soal 18
    \item Bayangan kurva $y=x^2+3x+3$ jika dicerminkan terhadap sumbu $X,$ lalu dilanjutkan dengan dilatasi dengan pusat $O$ dan faktor skala $3$ adalah $\cdots \cdot$
    \begin{enumerate}[label=\Alph*.]
        \item $x^2+9x-3y+27=0$
        \item $x^2+9x+3y+27=0$
        \item $3x^2+9x-y+27=0$
        \item $3x^2+9x+y+27=0$
        \item $3x^2+9x+27=0$
    \end{enumerate}

    % Soal 19
    \item Kurva $y = x^2+3$ didilatasikan dengan pusat $P(-1,2)$ dan faktor skala $3$, lalu dirotasikan sejauh $-\dfrac12 \pi$ dengan pusat $O(0,0)$. Persamaan bayangan kurva tersebut adalah $\cdots \cdot$
    \begin{enumerate}[label=\Alph*.]
        \item $3y=x^2+4x+19$
        \item $3x=y^2+4y+19$
        \item $y=x^2+4x+19$
        \item $x=y^2 + 4y + 19$
        \item $x=y^2+19$
    \end{enumerate}

    % Soal 20
    \item Sebuah mesin fotokopi dapat membuat salinan gambar/tulisan dengan ukuran berbeda. Suatu gambar persegi panjang difotokopi dengan setelan tertentu. Jika setelan tersebut dapat disamakan dengan proses transformasi terhadap matriks $\begin{pmatrix} 2 & 1 \\ 4 & 3 \end{pmatrix}$, kemudian didilatasi dengan titik pusat $(0,0)$ dan faktor skala $3$, maka luas gambar persegi panjang itu akan menjadi $\cdots$ kali dari luas semula.
    \begin{enumerate}[label=\Alph*.]
        \item $12$
        \item $18$
        \item $24$
        \item $30$
        \item $36$
    \end{enumerate}

    % Soal 21
    \item Sebuah kamera memproses gambar dengan mentransformasikan gambar tersebut terhadap matriks $\begin{pmatrix} \dfrac14 & \dfrac58 \\ \dfrac12 & 2 \end{pmatrix}$. Selanjutnya, gambar tersebut ditransformasi lagi terhadap matriks $\begin{pmatrix} 4 & 1 \\ 8 & 1 \end{pmatrix}$. Jika kamera tersebut mengambil gambar suatu benda dengan luas $32~\text{cm}^2$, maka luas benda hasil potretan adalah $\cdots \cdot$
    \begin{enumerate}[label=\Alph*.]
        \item $24~\text{cm}^2$
        \item $28~\text{cm}^2$
        \item $34~\text{cm}^2$
        \item $36~\text{cm}^2$
        \item $40~\text{cm}^2$
    \end{enumerate}

    % Soal 22
    \item Jika segi empat $ABCD$ didilatasi menjadi $A'B'C'D'$ seperti gambar, maka faktor skala yang sesuai adalah $\cdots \cdot$ \\
    \begin{center}
        \includegraphics[width=0.3\textwidth]{0Figure/transformasi nomor 22 mathcyber 1997.png} % Ganti dengan file gambar yang sesuai jika ada
    \end{center}
    \begin{enumerate}[label=\Alph*.]
        \item $2$
        \item $3$
        \item $4$
        \item $6$
        \item $9$
    \end{enumerate}

    % Soal 23
    \item Perhatikan grafik berikut. \\
    \begin{center}
        \includegraphics[width=0.3\textwidth]{0Figure/transformasi nomor 23 mathcyber 1997.png} % Ganti dengan file gambar yang sesuai jika ada
    \end{center}
    Salah satu translasi yang dapat memindahkan garis $g$ ke garis $l$ adalah $\cdots \cdot$
    \begin{enumerate}[label=\Alph*.]
        \item $\begin{bmatrix} 0 \\ 5 \end{bmatrix}$
        \item $\begin{bmatrix} 0 \\ -5 \end{bmatrix}$
        \item $\begin{bmatrix} -5 \\ 0 \end{bmatrix}$
        \item $\begin{bmatrix} 3 \\ 0 \end{bmatrix}$
        \item $\begin{bmatrix} 3 \\ -4 \end{bmatrix}$
    \end{enumerate}

    % Soal 24
    \item Perhatikan gambar garis alfabet berikut. \\
    \begin{center}
        \includegraphics[width=0.3\textwidth]{0Figure/transformasi nomor 24 mathcyber 1997.png} % Ganti dengan file gambar yang sesuai jika ada
    \end{center}
    Bayangan huruf E setelah didilatasi dengan pusat I dan faktor skala $-\dfrac12$ adalah $\cdots \cdot$
    \begin{enumerate}[label=\Alph*.]
        \item huruf A
        \item huruf C
        \item huruf G
        \item Huruf J
        \item huruf K
    \end{enumerate}

    % Soal 25
    \item Garis $y = 2ax-b$ digeser $2$ satuan ke kanan dan $1$ satuan ke bawah, lalu dicerminkan terhadap sumbu $Y$ sehingga menghasilkan garis $y=-4x.$ Nilai $a-b=\cdots \cdot$
    \begin{enumerate}[label=\Alph*.]
        \item $-7$
        \item $1$
        \item $2$
        \item $6$
        \item $11$
    \end{enumerate}

    % Soal 26
    \item Koordinat bayangan titik $(1, 0)$ oleh refleksi terhadap garis $y = x+1$ adalah titik $\cdots \cdot$
    \begin{enumerate}[label=\Alph*.]
        \item $(0, 1)$
        \item $(-2, 2)$
        \item $(-2, 1)$
        \item $(-1, 1)$
        \item $(-1, 2)$
    \end{enumerate}

\end{enumerate}

\section{Medium - Sulit}

\begin{enumerate}[resume]
    % Soal 1
    \item \textbf{Soal UNBK SMA IPA 2018} \\
    Segitiga $ABC$ dengan koordinat titik sudut $A(2,-1)$, $B(6,-2)$, dan $C(5,2)$ dirotasi sejauh $180^{\circ}$ dengan pusat $(3,1)$. Bayangan koordinat titik sudut segitiga $ABC$ adalah...
    \begin{enumerate}[label=(\Alph*)]
        \item $A(4,3),\ B(0,4),\ C(1,0)$
        \item $A(3,4),\ B(4,0),\ C(0,1)$
        \item $A(-4,3),\ B(0,-4),\ C(-1,0)$
        \item $A(-4,-3),\ B(0,-4),\ C(-1,0)$
        \item $A(-4,-3),\ B(0,4),\ C(1,1)$
    \end{enumerate}

    % Soal 2
    \item \textbf{Soal Simulasi UNBK SMA IPA 2019} \\
    Persamaan bayangan garis $y=x+1$ ditransformasikan oleh matriks $ \begin{pmatrix} 1 & 2\\ 0 & 1 \end{pmatrix}$, dilanjutkan dengan pencerminan terhadap sumbu $x$ adalah...
    \begin{enumerate}[label=(\Alph*)]
        \item $x+y-3=0$
        \item $x-y-3=0$
        \item $3x+y+3=0$
        \item $x+3y+1=0$
        \item $3x+y+1=0$
    \end{enumerate}

    % Soal 3
    \item \textbf{Soal OSK Matematika SMP 2018} \\
    Perhatikan gambar berikut ini:
    \begin{center}
        \includegraphics[width=0.3\textwidth]{0Figure/OSK Matematika SMP 2018 R2 no.23.jpg} % Ganti dengan file gambar yang sesuai jika ada
    \end{center}
    Persamaan garis hasil transformasi $R[0,180^{\circ}]$ dilanjutkan dengan pencerminan $y =-x$ terhadap garis $AB$ adalah...
    \begin{enumerate}[label=(\Alph*)]
        \item $y=2x+4$
        \item $y=2x-4$
        \item $y=-2x+4$
        \item $y=-2x-4$
    \end{enumerate}

    % Soal 4
    \item \textbf{Soal UN SMA IPA 2017} \\
    Persamaan bayangan garis $y=3x+2$ oleh transformasi yang bersesuaian dengan matriks $ \begin{pmatrix} 1 & 2\\ 0 & 1 \end{pmatrix}$, dilanjutkan dengan rotasi pusat $O(0,0)$ sebesar $90^{\circ}$ adalah...
    \begin{enumerate}[label=(\Alph*)]
        \item $y=-\frac{7}{3}x-\frac{2}{3}$
        \item $y=-\frac{7}{3}x+\frac{2}{3}$
        \item $y= \frac{7}{3}x+\frac{2}{3}$
        \item $y=-\frac{3}{7}x+\frac{2}{3}$
        \item $y=\frac{3}{7}x+\frac{2}{3}$
    \end{enumerate}

    % Soal 5
    \item \textbf{Soal SNMPTN 2011 Kode 559} \\
    Jika titik $(3,4)$ dirotasikan berlawanan arah jarum jam sejauh $45^{\circ}$ dengan pusat titik asal, kemudian hasilnya dicerminkan terhadap garis $y=x$, maka koordinat bayangannya adalah...
    \begin{enumerate}[label=(\Alph*)]
        \item $\left ( \frac{7\sqrt{2}}{2},\frac{\sqrt{2}}{2} \right )$
        \item $\left ( -\frac{5\sqrt{2}}{2},\frac{\sqrt{2}}{2} \right )$
        \item $\left ( \frac{5\sqrt{2}}{2},-\frac{\sqrt{2}}{2} \right )$
        \item $\left ( \frac{7\sqrt{2}}{2},-\frac{\sqrt{2}}{2} \right )$
        \item $\left ( -\frac{7\sqrt{2}}{2},\frac{\sqrt{2}}{2} \right )$
    \end{enumerate}

    % Soal 6
    \item \textbf{Soal SNMPTN 2011 Kode 559} \\
    Jika titik $(3,4)$ dirotasikan berlawanan arah jarum jam sejauh $45^{\circ}$ dengan pusat titik asal, kemudian hasilnya dicerminkan terhadap garis $y=-x$, maka koordinat bayangannya adalah...
    \begin{enumerate}[label=(\Alph*)]
        \item $\left ( \frac{7\sqrt{2}}{2},\frac{\sqrt{2}}{2} \right )$
        \item $\left ( -\frac{7\sqrt{2}}{2},\frac{\sqrt{2}}{2} \right )$
        \item $\left ( \frac{7\sqrt{2}}{2},-\frac{\sqrt{2}}{2} \right )$
        \item $\left ( \frac{5\sqrt{2}}{2},-\frac{\sqrt{2}}{2} \right )$
        \item $\left ( -\frac{5\sqrt{2}}{2},\frac{\sqrt{2}}{2} \right )$
    \end{enumerate}

    % Soal 7
    \item \textbf{Soal UMB 2011 Kode 152} \\
    Jika setiap titik pada grafik $y=\sqrt{x}$ dicerminkan terhadap $y=x$, maka grafik yang dihasilkan adalah...
    \begin{enumerate}[label=(\Alph*)]
        \item $y=x^{2},\ x \geq 0$
        \item $y=-\sqrt{x},\ x \geq 0$
        \item $y=-x^{2},\ x \leq 0$
        \item $y=\sqrt{-x},\ x \leq 0$
        \item $y=-\sqrt{-x},\ x \leq 0$
    \end{enumerate}

    % Soal 8
    \item \textbf{Soal UMB 2011 Kode 350} \\
    Jika setiap titik pada parabola $y=x^{2}$ di translasikan menurut vektor $(2,1)$ maka parabola yang dihasilkan adalah...
    \begin{enumerate}[label=(\Alph*)]
        \item $y=x^{2}+2x+1$
        \item $y=x^{2}-2x+1$
        \item $y=x^{2}-2x+3$
        \item $y=x^{2}-4x+5$
        \item $y=x^{2}+4x+5$
    \end{enumerate}

    % Soal 9
    \item \textbf{Soal SBMPTN 2015 Kode 507} \\
    Pencerminan garis $y=-x+2$ terhadap $y=3$, menghasilkan garis...
    \begin{enumerate}[label=(\Alph*)]
        \item $y=x+4$
        \item $y=-x+4$
        \item $y=x+2$
        \item $y=x-2$
        \item $y=-x-4$
    \end{enumerate}

    % Soal 10
    \item \textbf{Soal SPMB 2004} \\
    Diketahui lingkaran $L$ berpusat di titik $(-2,3)$ dan melalui titik $(1,5)$. Jika lingkaran $L$ diputar $90^{\circ}$ terhadap titik $(0,0)$ searah jarum jam, kemudian digeser kebawah sejauh $5$ satuan, maka persamaan lingkaran $L'$ yang dihasilkan adalah...
    \begin{enumerate}[label=(\Alph*)]
        \item $x^{2}+y^{2}-6x+6y+5=0$
        \item $x^{2}+y^{2}-6x+6y-5=0$
        \item $x^{2}+y^{2}+6x-6y+5=0$
        \item $x^{2}+y^{2}+6x-6y-5=0$
        \item $x^{2}+y^{2}-6x+6y=0$
    \end{enumerate}

    % Soal 11
    \item \textbf{Soal UN SMA IPA 2007} \\
    Bayangan kurva $y=x^{2}-3$ jika dicerminkan terhadap sumbu $x$ dilanjutkan dengan dilatasi pusat $O$ dan faktor skala $2$ adalah...
    \begin{enumerate}[label=(\Alph*)]
        \item $y=\frac{1}{2}x^{2}+6$
        \item $y=\frac{1}{2}x^{2}-6$
        \item $y=\frac{1}{2}x^{2}-3$
        \item $y=6-\frac{1}{2}x^{2}$
        \item $y=3-\frac{1}{2}x^{2}$
    \end{enumerate}

    % Soal 12
    \item \textbf{Soal UM UGM 2005} \\
    Jika matriks $\begin{pmatrix} a & -3 \\ 4 & b \end{pmatrix}$ mentransformasikan titik $(5,1)$ ke $(7,12)$ dan inversnya mentransformasikan titik $P$ ke titik $(1,0)$, koordinat titik $P$ adalah...
    \begin{enumerate}[label=(\Alph*)]
        \item $(2,-4)$
        \item $(2, 4)$
        \item $(-2,4)$
        \item $(-2,-4)$
        \item $(1,-3)$
    \end{enumerate}

    % Soal 13
    \item \textbf{Soal UN SMA IPA 2006} \\
    Persamaan bayangan parabola $y=x^{2}-3$, karena refleksi terhadap sumbu $x$ dilanjutkan oleh transformasi yang bersesuaian dengan matriks $ \begin{pmatrix} 2 & 1\\ 1 & 1 \end{pmatrix}$, adalah...
    \begin{enumerate}[label=(\Alph*)]
        \item $y^{2}+x^{2}-2xy-x+2y-3=0$
        \item $y^{2}+x^{2}+2xy+x-2y-3=0$
        \item $y^{2}+x^{2}-2xy+x-2y-3=0$
        \item $y^{2}+x^{2}+2xy+x+2y-3=0$
        \item $y^{2}-x^{2}+2xy+x+2y-3=0$
    \end{enumerate}

    % Soal 14
    \item \textbf{Soal SBMPTN 2013 Kode 332} \\
    Titik $(2a,-a)$ diputar $90^{\circ}$ berlawanan arah jarum jam dengan pusat perputaran titik $(1,1)$. Jika hasil rotasi adalah $(2+a,-2)$, maka $a=\cdots$
    \begin{enumerate}[label=(\Alph*)]
        \item $2$
        \item $1$
        \item $0$
        \item $-1$
        \item $-2$
    \end{enumerate}

    % Soal 15
    \item \textbf{Soal UM STIS 2011} \\
    Persamaan bayang kurva $y=x^{2}-2x-3$ oleh rotasi $[0,180^{\circ}]$, kemudian dilanjutkan oleh pencerminan terhadap garis $y=-x$ adalah...
    \begin{enumerate}[label=(\Alph*)]
        \item $y=x^{2}-2x-3$
        \item $x=y^{2}-2y-3$
        \item $y=x^{2}-2x+3$
        \item $x=y^{2}-2y+3$
        \item $y=x^{2}+2x+3$
    \end{enumerate}

    % Soal 16
    \item \textbf{Soal UM STIS 2011} \\
    Vektor $\bar{x}=\begin{pmatrix} x_{1}\\ x_{2} \end{pmatrix}$ diputar mengelilingi pusat koordinat $O$ sejauh $90^{\circ}$ dalam arah berlawanan dengan perputaran jarum jam. Hasilnya dicerminkan terhadap sumbu $x$, menghasilkan vektor $\bar{y}=\begin{pmatrix} y_{1}\\ y_{2} \end{pmatrix}$. Jika $\bar{x}=A\bar{y}$ maka $A=\cdots$
    \begin{enumerate}[label=(\Alph*)]
        \item $\begin{pmatrix} 0 & 1\\ 1 & 0 \end{pmatrix}$
        \item $\begin{pmatrix} 0 & -1\\ -1 & 0 \end{pmatrix}$
        \item $\begin{pmatrix} 0 & -1\\ 1 & 0 \end{pmatrix}$
        \item $\begin{pmatrix} 1 & 0\\ 0 & 1 \end{pmatrix}$
        \item $\begin{pmatrix} -1 & 0\\ 0 & -1 \end{pmatrix}$
    \end{enumerate}

    % Soal 17
    \item \textbf{Soal UNBK SMA IPA 2019} \\
    Misalkan $A'(-1,-2)$ dan $B'(3,7)$ adalah hasil bayangan titik $A(-1,0)$ dan $B(2,1)$ oleh transformasi matriks $x$ berordo $2 \times 2$. Jika $C'(0,1)$ adalah bayangan titik $C$ oleh transformasi tersebut, titik $C$ adalah...
    \begin{enumerate}[label=(\Alph*)]
        \item $(-1,1)$
        \item $(1,1)$
        \item $(1,3)$
        \item $(2,-3)$
        \item $(2,3)$
    \end{enumerate}

    % Soal 18
    \item \textbf{Soal UNBK SMA IPA 2019} \\
    Persamaan peta garis $x-2y-4=0$ yang dirotasikan dengan pusat $O(0,0)$ sebesar $90^{\circ}$ berlawanan arah dengan jarum jam dan dilanjutkan dengan pencerminan terhadap garis $y=x$ adalah...
    \begin{enumerate}[label=(\Alph*)]
        \item $x+2y+4=0$
        \item $x+2y-4=0$
        \item $2x+ y+4=0$
        \item $2x-y-4=0$
        \item $2x+y-4=0$
    \end{enumerate}

    % Soal 19
    \item \textbf{Soal UTBK-SBMPTN 2019} \\
    Jika garis $y=ax+b$ digeser ke atas sejauh $2$ satuan kemudian dicerminkan terhadap sumbu $x$, maka bayangannya adalah garis $y=-2x+1$. Nilai $3a-2b$ adalah...
    \begin{enumerate}[label=(\Alph*)]
        \item $-8$
        \item $-4$
        \item $-1$
        \item $8$
        \item $12$
    \end{enumerate}

    % Soal 20
    \item \textbf{Soal UTBK-SBMPTN 2019} \\
    Jika $y=2x+1$ digeser sejauh $a$ satuan ke kanan dan sejauh $b$ satuan ke bawah, kemudian dicerminkan terhadap sumbu-$x$, bayangannya menjadi $y=ax-b$. Nilai $a+b=\cdots$
    \begin{enumerate}[label=(\Alph*)]
        \item $-\frac{1}{2}$
        \item $-3$
        \item $4$
        \item $3$
        \item $\frac{1}{2}$
    \end{enumerate}

    % Soal 21
    \item \textbf{Soal UTBK-SBMPTN 2019} \\
    Garis $y=2x+1$ dirotasi searah jarum jam sebesar $90^{\circ}$ terhadap titik asal, kemudian digeser ke atas sejauh $b$ satuan dan ke kiri sejauh $a$ satuan, bayangannya menjadi $x-ay=b$. Nilai $a+b=\cdots$
    \begin{enumerate}[label=(\Alph*)]
        \item $5$
        \item $2$
        \item $0$
        \item $-2$
        \item $-5$
    \end{enumerate}

    % Soal 22
    \item \textbf{Soal UTBK-SBMPTN 2019} \\
    Parabola $y=x^{2}-6x+8$ digeser ke kanan sejauh $2$ satuan searah dengan sumbu-$x$ dan digeser ke bawah sejauh $3$ satuan searah sumbu-$y$. Jika parabola hasil pergeseran ini memotong sumbu-$x$ di $x_{1}$ dan $x_{2}$, maka nilai $x_{1}+x_{2}=\cdots$
    \begin{enumerate}[label=(\Alph*)]
        \item $7$
        \item $8$
        \item $9$
        \item $10$
        \item $11$
    \end{enumerate}

    % Soal 23
    \item \textbf{Soal UTBK-SBMPTN 2022} \\
    Diketahui $0 < \alpha < \frac{\pi}{2}$. Jika rotasi $\alpha$ diikuti pencerminan terhadap sumbu-$y$ disajikan oleh matriks
    $\begin{pmatrix} -\frac{2}{\sqrt{5}} & \frac{1}{\sqrt{5}} \\ \frac{1}{\sqrt{5}} & \frac{2}{\sqrt{5}} \end{pmatrix}$, maka $\sin \alpha - 2 \cos \alpha=\cdots$
    \begin{enumerate}[label=(\Alph*)]
        \item $\frac{3}{\sqrt{5}}$
        \item $\frac{1}{\sqrt{5}}$
        \item $0$
        \item $\frac{-1}{\sqrt{5}}$
        \item $\frac{-3}{\sqrt{5}}$
    \end{enumerate}

    % Soal 24
    \item \textbf{Soal UTBK-SBMPTN 2022} \\
    Jika pencerminan terhadap sumbu-$x$ diikuti dengan rotasi sebesar $\alpha^{\circ}$ disajikan oleh matriks
    $\begin{pmatrix} \frac{\sqrt{3}}{2} & \frac{1}{2} \\ \frac{1}{2} & -\frac{\sqrt{3}}{2} \end{pmatrix}$, maka nilai $\alpha$ yang mungkin adalah...
    \begin{enumerate}[label=(\Alph*)]
        \item $-60$
        \item $-30$
        \item $30$
        \item $60$
        \item $120$
    \end{enumerate}

    % Soal 25
    \item \textbf{Soal UTBK-SBMPTN 2022} \\
    Jika parabola $y= x^{2}-8x+4$ dicerminkan terhadap garis $x=-1$ kemudian dicerminkan terhadap sumbu-$x$, maka persamaan parabola yang baru adalah...
    \begin{enumerate}[label=(\Alph*)]
        \item $y=-x^{2}-6x-36$
        \item $y=-x^{2}+6x-48$
        \item $y=-x^{2}+12x-24$
        \item $y=-x^{2}-12x-24$
        \item $y=-x^{2}-12x-48$
    \end{enumerate}

    % Soal 26
    \item \textbf{Soal UTBK-SBMPTN 2022} \\
    Jika parabola $y=-x^{2}+6x$ dicerminkan terhadap garis $y=-x$, kemudian digeser ke atas sejauh $2$ satuan, maka persamaan parabola yang baru adalah...
    \begin{enumerate}[label=(\Alph*)]
        \item $x=y^{2}-2y+4$
        \item $x=y^{2}-2y$
        \item $x=y^{2}- 4$
        \item $x=y^{2}+2y$
        \item $x=y^{2}+2y-8$
    \end{enumerate}

    % Soal 27
    \item \textbf{Soal UTBK-SBMPTN 2022} \\
    Jika parabola $y= x^{2}+6x$ dicerminkan terhadap garis $x=1$, kemudian digeser ke kiri sejauh $4$ satuan, maka persamaan parabola yang baru adalah...
    \begin{enumerate}[label=(\Alph*)]
        \item $y=x^{2}-10x-16$
        \item $y=x^{2}-2x-8$
        \item $y=x^{2}-2x-9$
        \item $y=x^{2}+8x+7$
        \item $y=x^{2}+10x-16$
    \end{enumerate}

    % Soal 28
    \item \textbf{Soal UTBK-SBMPTN 2022} \\
    Jika parabola $y= x^{2}+2x-8$ dicerminkan terhadap sumbu-$x$, kemudian digeser ke kiri sejauh $3$ satuan, maka persamaan parabola yang baru adalah...
    \begin{enumerate}[label=(\Alph*)]
        \item $y=x^{2}+8x+7$
        \item $y=-x^{2}-6x+9$
        \item $y=-x^{2}-6x-18$
        \item $y=-x^{2}-8x-7$
        \item $y=-x^{2}-8x-25$
    \end{enumerate}

    % Soal 29
    \item \textbf{Soal UTBK-SBMPTN 2022} \\
    Jika parabola $x= y^{2}+6y-7$ digeser ke kiri sejauh $3$ satuan, kemudian dicerminkan terhadap garis $y=3$, maka persamaan parabola yang baru adalah...
    \begin{enumerate}[label=(\Alph*)]
        \item $x=y^{2}-18y-72$
        \item $x=y^{2}-18y+81$
        \item $x=y^{2}-18y+67$
        \item $x=y^{2}-18y+62$
        \item $x=y^{2}-18y-81$
    \end{enumerate}

    % Soal 30
    \item \textbf{Soal UTBK-SBMPTN 2022} \\
    Diketahui $0 < \alpha < \frac{\pi}{2}$ dan $\tan \alpha=3$.
    Jika rotasi $p \pi + q \alpha$ diikuti pencerminan terhadap garis $y=x$ disajikan oleh matriks $\begin{pmatrix}
    \frac{1}{\sqrt{10}} & -\frac{3}{\sqrt{10}}  \\
     - \frac{3}{\sqrt{10}} & -\frac{1}{\sqrt{10}}  \\
    \end{pmatrix}$, maka nilai $3pq$ yang mungkin adalah...
    \begin{enumerate}[label=(\Alph*)]
        \item $\frac{5}{2}$
        \item $2$
        \item $\frac{3}{2}$
        \item $1$
        \item $\frac{1}{2}$
    \end{enumerate}

    % Soal 31
    \item \textbf{Soal UTBK-SBMPTN 2022} \\
    Diketahui $0 < \alpha < \frac{\pi}{2}$ dan $\tan \alpha=2$.
    Jika rotasi $p \pi + q \alpha$ diikuti pencerminan terhadap garis $y=x$ disajikan oleh matriks $\begin{pmatrix}
    \frac{2}{\sqrt{5}} & \frac{1}{\sqrt{5}}  \\
     \frac{1}{\sqrt{5}} & -\frac{2}{\sqrt{5}}  \\
    \end{pmatrix}$, maka nilai $\dfrac{p}{q}$ yang mungkin adalah...
    \begin{enumerate}[label=(\Alph*)]
        \item $-\frac{3}{2}$
        \item $-1$
        \item $0$
        \item $\frac{1}{2}$
        \item $2$
    \end{enumerate}

    % Soal 32
    \item \textbf{Soal UTBK-SBMPTN 2022} \\
    Diketahui $0 < \alpha < \frac{\pi}{2}$ dan $\tan \alpha=2$.
    Jika rotasi $p \pi + q \alpha$ diikuti pencerminan terhadap garis $y=x$ disajikan oleh matriks $\begin{pmatrix}
    \frac{1}{\sqrt{5}} & -\frac{2}{\sqrt{5}}  \\
     -\frac{2}{\sqrt{5}} & -\frac{1}{\sqrt{5}}  \\
    \end{pmatrix}$, maka nilai $2pq$ yang mungkin adalah...
    \begin{enumerate}[label=(\Alph*)]
        \item $1$
        \item $\frac{1}{2}$
        \item $\frac{1}{3}$
        \item $-\frac{1}{2}$
        \item $-1$
    \end{enumerate}

    % Soal 33
    \item \textbf{Soal UTBK-SBMPTN 2022} \\
    Jika parabola $y= x^{2}+8$ dirotasi terhadap titik asal dengan sudut $\alpha=90^{\circ}$ searah putaran jarum jam kemudian digeser ke bawah sejauh $5$ satuan, maka persamaan parabola yang baru adalah...
    \begin{enumerate}[label=(\Alph*)]
        \item $x=y^{2}+10y+8$
        \item $x=y^{2}+10y+17$
        \item $x=y^{2}+10y+33$
        \item $x=y^{2}+10y+35$
        \item $x=y^{2}+10y+36$
    \end{enumerate}

    % Soal 34
    \item \textbf{Contoh Soal TKA Matematika SMA} \\
    Bayangan dari kurva $y = 2x^{2} - 5$ yang ditranslasikan oleh matriks $\begin{pmatrix} -3 \\ 2 \end{pmatrix}$ kemudian didilatasikan oleh $(O, 2)$ dengan $O$ merupakan titik koordinat $(0,0)$ adalah....
    \begin{enumerate}[label=(\Alph*)]
        \item $y=\frac{1}{4}x^{2}+3x+\frac{15}{2}$
        \item $y=\frac{1}{4}x^{2}+3x+\frac{11}{2}$
        \item $y= x^{2}+12x+30$
        \item $y= x^{2}-12x+22$
        \item $y=4x^{2}+12x+15$
    \end{enumerate}

    % Soal 35
    \item \textbf{Contoh Soal TKA Matematika SMA} \\
    Segitiga $PQR$ dengan titik $P (-1,3),$ $Q (3,3),$ dan $R(1,-2)$ didilatasi dengan pusat titik $(0,0)$ dan faktor skala $3$. Luas segitiga tersebut setelah dilakukan dilatasi adalah.... satuan luas.
    \begin{enumerate}[label=(\Alph*)]
        \item $10$
        \item $30$
        \item $90$
        \item $120$
        \item $180$
    \end{enumerate}

\end{enumerate}

\end{document}