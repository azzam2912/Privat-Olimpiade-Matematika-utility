\documentclass[12pt]{scrartcl}
\usepackage{graphicx}
\graphicspath{{./}}
\usepackage[sexy]{evan}
\usepackage[normalem]{ulem}
\usepackage{hyperref}
\usepackage{mathtools}
\hypersetup{
    colorlinks=true,
    linkcolor=blue,
    filecolor=magenta,      
    urlcolor=cyan,
    pdfpagemode=FullScreen,
    }
\usepackage[most]{tcolorbox}
\renewcommand{\dangle}{\measuredangle}

\renewcommand{\baselinestretch}{1.5}

\addtolength{\oddsidemargin}{-0.4in}
\addtolength{\evensidemargin}{-0.4in}
\addtolength{\textwidth}{0.8in}
% \addtolength{\topmargin}{-0.2in}
% \addtolength{\textheight}{1in} 


\setlength{\parindent}{0pt}

\usepackage{pgfplots}
\pgfplotsset{compat=1.15}
\usepackage{mathrsfs}
\usetikzlibrary{arrows}

\title{Geometri Exercises}
\author{Azzam Labib (IG: azzam\_29\_12)}
\date{\today}
\begin{document}
\maketitle

\begin{enumerate}
    %(OSK 2015)
    \item Diberikan segitiga $ABC$ dengan sudut $\angle ABC = 90^\circ$. Lingkaran $L_1$ dengan $AB$ sebagai diameter sedangkan lingkaran $L_2$ dengan $BC$ sebagai diameternya. Kedua lingkaran $L_1$ dan $L_2$ berpotongan di $B$ dan $P$. Jika $AB = 5$, $BC = 12$ dan $BP = x$ maka nilai dari $\frac{240}{x}$ adalah \ldots
    
    %(OSK 2015)
    \item Pada segitiga $ABC$, garis tinggi $AD$, garis bagi $BE$ dan garis berat $CF$ berpotongan di satu titik. Jika panjang $AB = 4$ dan $BC = 5$, dan $CD = \frac{m^2}{n^2}$ dengan $m$ dan $n$ relatif prima, maka nilai dari $m - n$ adalah \ldots

    \item Diberikan segitiga $ABC$ dengan panjang $BC = 36$. Misalkan $D$ adalah titik tengah $BC$ dan $E$ adalah titik tengah $AD$. Misalkan pula bahwa $F$ adalah perpotongan $BE$ dengan $AC$. Jika diketahui bahwa $AB$ menyinggung lingkaran luar segitiga $BFC$, hitunglah panjang $BF$.

    \item Diberikan sebuah segiempat siklis $ABCD$ dengan $ABC$ adalah segitiga sama sisi. Jika $AD=2$ dan $CD=3$, panjang $BD=\dots$

    %(OSK 2016)
    \item Pada segitiga $ABC$, titik $M$ terletak pada $BC$ sehingga $AB=7, AM=3, BM=5$, dan $MC=6$. Panjang $AC$ adalah \dots

    %(AIME 2016 I) 
    \item Misalkan $\triangle ABC$ adalah segitiga lancip dengan lingkaran $\omega,$ dan misalkan $H$ adalah titik potong dari garis tinggi $\triangle ABC.$ Garis singgung lingkaran luar $\triangle HBC$ di $H$ memotong $\omega$ pada titik $X$ dan $Y$ dengan $HA=3,HX=2,$ dan $HY=6.$ Carilah luas dari $\triangle ABC$.

    %(OSK 2012)
    \item Diberikan segitiga $ABC$ dengan keliling 3, dan jumlah kuadrat sisi-sisinya sama dengan 5. Jika jari-jari lingkaran luarnya sama dengan 1, maka jumlah ketiga garis tinggi dari segitiga $ABC$ tersebut adalah \dots

    \item Nilai dari $\cos \dfrac{\pi}{7}\cdot \cos \dfrac{2\pi}{7} \cdot \cos \dfrac{4\pi}{7}$ adalah \dots

    \item Tentukan nilai eksak dari $\tan 1^\circ \cdot \tan 2^\circ \cdot \tan 3^\circ \cdot \ldots \cdot \tan 89^\circ$.

    % (OSK 2005) 
    \item Nilai dari $\sin^8 75^\circ - \cos^8 75^\circ$ adalah \dots

    %(OSK 2013)
    \item \textbf{(Bonus: kalo bisa kerjain, nilainya plus plus)}  Diberikan segitiga lancip $ABC$ dengan $O$ sebagai pusat lingkaran luarnya. Misalkan $M$ dan $N$ berturut - turut pertengahan $OA$ dan $BC$. Jika $\angle ABC = 4\angle OMN$ dan $\angle ACB = 6\angle OMN$, maka besarnya $\angle OMN$ sama dengan \dots
\end{enumerate}

\end{document}