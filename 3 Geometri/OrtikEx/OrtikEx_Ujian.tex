\documentclass[11pt]{scrartcl}
\usepackage{graphicx}
\graphicspath{{./}}
\usepackage[sexy]{evan}
\usepackage[normalem]{ulem}
\usepackage{hyperref}
\usepackage{mathtools}
\hypersetup{
    colorlinks=true,
    linkcolor=blue,
    filecolor=magenta,      
    urlcolor=cyan,
    
    pdfpagemode=FullScreen,
    }

\renewcommand{\dangle}{\measuredangle}

\renewcommand{\baselinestretch}{1.5}

\addtolength{\oddsidemargin}{-0.4in}
\addtolength{\evensidemargin}{-0.4in}
\addtolength{\textwidth}{0.8in}
% \addtolength{\topmargin}{-0.2in}
% \addtolength{\textheight}{1in} 


\setlength{\parindent}{0pt}

\usepackage{pgfplots}
\pgfplotsset{compat=1.15}
\usepackage{mathrsfs}
\usetikzlibrary{arrows}

\title{Ortik dan Ex - Soal Post Test}
\author{Azzam (IG: haxuv.world)}
\date{Jumat, 19 Januari 2024}

\begin{document}
\maketitle
\textbf{Aturan umum:}
\begin{itemize}
    \item Tulis \textbf{nama lengkap} dan \textbf{asal sekolah} di pojok kiri atas halaman pertama.
    \item \textbf{Soal bertipe esai}. Sertakan argumentasi atau cara mendapatkan jawaban yang ditanyakan di soal.
    \item Waktu standar untuk mengerjakan semua soal berikut adalah 90-120 menit.
    \item Setiap soal bernilai bilangan bulat antara 0 sampai 10 (inklusif).
    \item Soal yang wajib dikerjakan untuk mendapatkan \textit{full points} adalah \textbf{dua soal}.
\end{itemize}


\section{Soal}
\begin{enumerate}
\item Let $ABXC$ be a cyclic quadrilateral such that $\angle XAB = \angle XAC$. Let $I$ be the incenter of triangle $ABC$ and by $D$ the foot of $I$ on $BC$. Given $AI = 25$, $ID = 7$, and $BC = 14$, find $XI$.

\item Given triangle $ABC$ with incenter $I$ and $A$-excenter $J$. Circle $\omega_b$ centered at point $O_b$ passes through point $B$ and is tangent to line $CI$ at point $I$. Circle $\omega_c$ with center $O_c$ passes through point $C$ and touches line $BI$ at point $I$. Let $O_bO_c$ and $IJ$ intersect at point $K$. Find the ratio $IK/KJ$.

\item Let $ABC$ be an acute triangle with circumcenter $O$, and let $D$, $E$, and $F$ be the feet of altitudes from $A$, $B$, and $C$ to sides $BC$, $CA$, and $AB$, respectively. Denote by $P$ the intersection of the tangents to the circumcircle of $ABC$ at $B$ and $C$. The line through $P$ perpendicular to $EF$ meets $AD$ at $Q$, and let $R$ be the foot of the perpendicular from $A$ to $EF$. Prove that $DR$ and $OQ$ are parallel.
\end{enumerate}

\end{document}
