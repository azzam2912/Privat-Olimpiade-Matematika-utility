\documentclass[11pt]{scrartcl}
\usepackage{graphicx}
\graphicspath{{./}}
\usepackage[sexy]{evan}
\usepackage[normalem]{ulem}
\usepackage{hyperref}
\usepackage{mathtools}
\hypersetup{
    colorlinks=true,
    linkcolor=blue,
    filecolor=magenta,      
    urlcolor=cyan,
    pdfpagemode=FullScreen,
    }

\renewcommand{\dangle}{\measuredangle}

\renewcommand{\baselinestretch}{1.5}

\addtolength{\oddsidemargin}{-0.4in}
\addtolength{\evensidemargin}{-0.4in}
\addtolength{\textwidth}{0.8in}
% \addtolength{\topmargin}{-0.2in}
% \addtolength{\textheight}{1in} 


\setlength{\parindent}{0pt}

\usepackage{pgfplots}
\pgfplotsset{compat=1.15}
\usepackage{mathrsfs}
\usetikzlibrary{arrows}

\title{Geometry from (Old) OSN}
\author{Selected and compiled by Azzam}
\date{\today}

\begin{document}
	\maketitle
	\section{Tips dan trik untuk untuk mengalahkan soal geometri olimpiade :}
	\\Lakukan langkah-langkah ini untuk mempermudah pengerjaan soal geo terutama di OSN
	\begin{enumerate}
		\item
		Gambarlah diagramnya dengan ukuran besar dan rapi menggunakan penggaris dan jangka (kecuali anda sudah imba).
		\item
		Selalu cari solusi Euclidnya dulu minimal 15 menit
		\item
		Pakai directed angle
		\item
		Hampir seluruh soal geo OSN bisa pakai angle chasing
		\item
		Kalo pure angle chasing gagal, cari kesebangunan / kekongruenan. 
		\begin{itemize}
			\item
			Kadangkala kesebangunannya bisa jadi hasil dari suatu rotasi / translasi. Bisa jadi kesebangunannya berhubungan dengan perbandingan luas atau dalil sinus.
			\item
			Bisa juga pakai dalil Menelaus atau dalil Ceva untuk dapat perbandingannya.
		\end{itemize}
		\item
		Gunakan lemma-lemma yang anda ketahui. Tanyakan ke pengawas apakah boleh menggunakan lemma tersebut tanpa bukti / langsung pakai.
		\item
		Kalau belum ketemu, definisikan titik baru atau garis baru yang sejajar, tegak lurus, atau hasil translasi / rotasi suatu titik / garis.
		
		\item
		Kalau ada lingkaran coba pakai power of a point atau mungkin ptolemy kadang berguna.
		
		
		\item
		Ngga ketemu juga? Pakai trigon. Usahakan dalil sinus dulu. Atau pakai dalil Ceva versi Trigon.
		\begin{itemize}
			\item
			Biasanya kalo di soal diketahui sudut-sudut istimewa, bisa gampang banget kalo make trigon.
		\end{itemize}
		\item
		blank? pakai dalil cosinus.
		\item
		Pakai phantom point (Misal ingin buktikan A, B, C segaris. Maka kita sengaja buat A', B, C segaris, lalu buktikan fakta di soal)
		\item
		Gagal juga? pakai analit. Tapi kalo mau make analit, usahakan ada fakta-fakta atau solusi yang dicari make euclid, biar ada jaminan nilai parsial. Lalu, biasanya kalo udah analit, emang paling enak make trigon.
	
	\end{enumerate}
	
	\section{Soal}
	\begin{enumerate}
		\item
		Diberikan segitiga $ABC$ dengan $AC>BC$. Pada lingkaran luar segitiga $ABC$ terletak titik $D$ yang merupakan titik tengah busur $AB$ yang memuat titik $C$. Misalkan $E$ adalah titik pada $AC$ sehingga $DE$ tegak lurus $AC$. Buktikan bahwa $AE=EC+CB$.
		
		\item
		Misalkan $M$ suatu titik di dalam segitiga $ABC$ sedemikian sehingga $\angle AMC = 90 ^{\circ}, \angle AMB = 150 ^{\circ}, \text{dan} \angle BMC = 120 ^\circ$. Titik pusat lingkaran luar dari segitiga-segitiga $AMC, AMB,$ dan $ BMC$ berturut-turut adalah $P,Q$, dan $R$. Buktikan bahwa luas segitiga $PQR$ lebih besar dari luas segitiga $ABC$
		
		\item
		Misalkan $ABCD$ sebuah segiempat konveks. Persegi $AB_1A_2B$ dibuat sehingga kedua titik $A_2,B_1$ terletak di luar segiempat $ABCD$. Dengan cara serupa diperoleh persegi-persegi $BC_1B_2C, CD_1C_2D$, dan $DA_1D_2A$. Misalkan $K$ adalah titik potong $AA_2$ dengan $BB_1$, $L$ adalah titik potong $BB_2$ dengan $CC_1$, $M$ adalah titik potong $CC_2$ dengan $DD_1$, dan $N$ adalah titik potong $DD_2$ dengan $AA_1$. Buktikan bahwa $KM = LN$ dan $KM$ tegak lurus $LN$.
		
		\item
		Titik-titik $A,B,C,D$ terletak pada lingkaran $S$ sdemikian sehingga $AB$ merupakan diameter $S$, tetapi $CD$ bukan merupakan diameter $S$. Diketahui pula bahwa $C$ dan $D$ berada pada sisi yang berbeda terhadap $AB$. Garis singgung terhadap $S$ di $C$ dan $D$ berpotongan di tiitk $P$. Titik-titik $Q$ dan $R$ berturut-turut adalah perpotongan garis $AC$ dengan garis $BD$ dan garis $AD$ dengan garis $BC$.
		\begin{enumerate}
			\item[a.] 
			Buktikan bahwa $P,Q$, dan $R$ segaris.
			\item[b.]
			Buktikan bahwa garis $QR$ tegak lurus terhadap garis $AB$.
		\end{enumerate}
	
		\item
		Pada segitiga $ABC$, titik-titik $D,E,$ dan $F$ berturut-turut terletak pada segmen $BC,CA$, dan $AB$. Nyatakan $P$ sebagai titik perpotongan $AD$ dan $EF$. Tunjukkan bahwa $$\frac{AB}{AF}\times CD + \frac{AC}{AE}\times BD = \frac{AD}{AP} \times BC$$
		
		\item
		Diberikan segitiga lancip $ABC$ dengan $AC > BC$ dan titik pusat lingkaran luar $O$. Garis tinggi segitiga $ABC$ dari $C$ memotong $AB$ dan lingkaran luar segitiga $ABC$ lagi berturut-turut di titik $D$ dan $E$. Garis melalui $O$ sejajar $AB$ memotong garis $AC$ di titik $F$. Buktikan bahwa garis $CO$, garis melalui $F$ yang tegak lurus dengan $AC$, dan garis melalui $E$ yang sejajar $DO$ bertemu di satu titik.
		
		\item
		Diberikan segitiga lancip $ABC$ dengan $AB > AC$ dan memiliki titik pusat lingkaran luar $O$. Garis $BO$ dan $CO$ memotong garis bagi $\angle BAC$ berturut-turut di titik $P$ dan $Q$. Selanjutnya garis $BQ$ dan $CP$ berpotongan di titik $R$. Buktikan bahwa garis $AR$ tegak lurus terhadap garis $BC$.
		
		\item
		Diberikan sembarang segitiga $ABC$ dan garis bagi $\angle BAC$ memotong sisi $BC$ dan lingkaran luar segitiga $ABC$ berturut-turut di $D$ dan $E$. Misalkan $M$ dan $N$ berturut-turut adalah titik tengah $BD$ dan $CE$. Lingkaran luar segitiga $ABD$ memotong $AN$ di titik $Q$. Lingkaran yang melalui $A$ dan menyinggung $BC$ di $D$, memotong garis $AM$ dan sisi $AC$ berturut-turut di titik $P$ dan $R$. Tunjukkan bahwa empat titik $B,P,Q,R$ terletak pada satu garis.
		
		\item
		Diberikan segitiga lancip $ABC$ dengan lingkaran luar $\omega$. Garis bagi $\angle BAC$ memotong $\omega$ di titik $M$. Misalkan $P$ adalah suatu titik pada garis $AM$ dengan $P$ di dalam segitiga $ABC$. Garis melalui $P$ yang sejajar $AB$ dan garis melalui $P$ yang sejajar $AC$ memotong sisi $BC$ berturut-turut di titik $E$ dan $F$. Garis $ME$ dan $MF$ memotong $\omega$ lagi berturut-turut di titik $K$ dan $L$. Buktikan bahwa garis-garis $AM,BL$, dan $CK$ konkuren (bertemu di satu titik).
		
		\item
		Diberikan trapesium $ABCD$ dengan $AB$ sejajar $CD$ dan $AB<CD$. Misalkan diagonal $AC$ dan $BD$ bertemu di titik $E$, lalu misalkan garis $AD$ dan $BC$ bertemu di titik $F$. Jika $AEDK$ dan $BECL$ merupakan jajar genjang, buktikan bahwa garis $EF$ melalui titik tengah segmen $KL$.
		
		
	\end{enumerate}
	
	
\end{document}