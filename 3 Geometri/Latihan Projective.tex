\documentclass[12pt]{scrartcl}

\usepackage[hagavi]{azzam}
\title{Proyektif}
\date{Iklan: Menuju JKT48 x AKB48 Meet and Greet 25 Oktober 2025}

\begin{document}


\begin{enumerate}
    \item Points $A, C, B, D$ are on a line in this order, so that $(A,B;C,D)$ is harmonic. Let $M$ be the midpoint of $AB$. Prove that $AM^2 = MC \cdot MD$. (There is a purely algebraic way to do this, as well as a way using poles and polars. Try to find the latter).

    \item The tangents to the circumcircle of $\triangle ABC$ at $B$ and $C$ intersect at $D$. Prove that $AD$ is the symmedian of $\triangle ABC$.

    \item $AD$ is the altitude of an acute $\triangle ABC$. Let $P$ be an arbitrary point on $AD$. $BP, CP$ meet $AC, AB$ at $M, N$, respectively. $MN$ intersects $AD$ at $Q$. $F$ is an arbitrary point on side $AC$. $FQ$ intersects line $CN$ at $E$. Prove that $\angle FDA = \angle EDA$.

    \item Point $M$ lies on diagonal $BD$ of parallelogram $ABCD$. Line $AM$ intersects side $CD$ and line $BC$ at points $K$ and $N$, respectively. Let $C_1$ be the circle with center $M$ and radius $MA$ and $C_2$ be the circumcircle of triangle $KCN$. $C_1, C_2$ intersect at $P$ and $Q$. Prove that $MP, MQ$ are tangent to $C_2$.
    
    \item The incircle $\omega$ of $\triangle ABC$ has centre $I$ and touches $BC$ at $D$. $DE$ is the diameter of $\omega$. If $AE$ intersects $BC$ at $F$, prove that $BD=FC$.

    \item The incircle of $\triangle ABC$ touches $BC$ at $E$. $AD$ is the altitude in $\triangle ABC$; $M$ is the midpoint of $AD$. Let $I_a$ be the centre of the excircle opposite to $A$ of $\triangle ABC$. Prove that $M, E, I_a$ are collinear.

    \item A circle $\omega$ is internally tangent to a circle $\Gamma$ at $P$. $A$ and $B$ are points on $\Gamma$ such that $AB$ is tangent to $\omega$ at $K$. Show that $PK$ bisects the arc $AB$ not containing point $P$.

    \item Let $\Gamma$ be the circumcircle of $\triangle ABC$ and $D$ an arbitrary point on side $BC$. The circle $\omega$ is tangent to $AD, DC, \Gamma$ at $F, E, K$ respectively. Prove that the incenter $I$ of $\triangle ABC$ lies on $EF$.

    \item $\Gamma$ is the circumcircle of $\triangle ABC$. The incircle $\omega$ is tangent to $BC, CA, AB$ at $D, E, F$ respectively. A circle $\omega_A$ is tangent to $BC$ at $D$ and to $\Gamma$ at $A'$, so that $A'$ and $A$ are on different sides of $BC$. Define $B', C'$ similarly. Prove that $DA', EB', FC'$ are concurrent.

    \item (Sawayama-Thébault's Theorem) Given a triangle $\triangle ABC$, construct its circumcircle $T$. Now take any point $D$ on side $BC$ and draw line $AD$. Now construct the two circles that are internally tangent to $T$, and also tangent to the segments $BC$ and $AD$, with centers $O_1$ and $O_2$ respectively.

    Let $I$ be the incenter of $\triangle ABC$. The points $O_1$, $O_2$, and $I$ are collinear.
\end{enumerate}

\end{document}