\documentclass[12pt]{article}
\usepackage[hagavi]{azzam}

\begin{document}

\title{Dimensi tiga dan Analit (Koordinat Kartesius)}
\maketitle
\section{Dimensi Tiga}
\begin{enumerate}
        \item (OSP SMP 2016) Diberikan Kubus $ABCD.EFGH$ dengan panjang rusuk $2~cm$. Titik $P$ terletak pada perpanjangan $HE$ sehingga $PE = 1~cm$. Tentukan jarak titik $P$ ke bidang yang memuat segitiga $AHF$.
    \begin{center}
        \begin{tikzpicture}
        \coordinate (D_cube) at (1,1.5); % Front-bottom-left
        \coordinate (C_cube) at (4,1.5); % Front-bottom-right
        \coordinate (B_cube) at (3,0);    % Back-bottom-right
        \coordinate (A_cube) at (0,0);    % Back-bottom-left
    
        \coordinate (H_cube) at (1,4.5); % Front-top-left
        \coordinate (G_cube) at (4,4.5); % Front-top-right
        \coordinate (F_cube) at (3,3);    % Back-top-right
        \coordinate (E_cube) at (0,3);    % Back-top-left
    
        % Draw dashed lines
        \draw[dashed] (D_cube) -- (A_cube); % Back to front bottom-left
        \draw[dashed] (D_cube) -- (C_cube); % Back bottom-left to back bottom-right
        \draw[dashed] (D_cube) -- (H_cube); % Back bottom-left to back top-left
    
        % Draw solid lines
        \draw (A_cube) -- (B_cube) -- (C_cube); % Front bottom and bottom-right
        \draw (A_cube) -- (E_cube) -- (F_cube) -- (B_cube); % Front face
        \draw (G_cube) -- (C_cube); % Right face
        \draw (E_cube) -- (H_cube) -- (G_cube) -- (F_cube); % Top face
        \draw (E_cube) -- (H_cube); % Left top edge (already part of top face)
    
        % Draw the additional line segments (diagonals/internal)
        \draw (A_cube) -- (H_cube);
        \draw (H_cube) -- (F_cube);
        \draw (A_cube) -- (F_cube); 
    
        % Draw line segment P-E
        % Extend E-A to P
        \coordinate (P_point) at ($ (E_cube)!-0.5!(H_cube) $);
        \draw (P_point) -- (E_cube);
        
        % Label the points
        \node[above right] at (D_cube) {D};
        \node[below left] at (A_cube) {A};
        \node[below right] at (B_cube) {B};
        \node[below right] at (C_cube) {C};
        
        \node[above left] at (E_cube) {E};
        \node[above right] at (F_cube) {F};
        \node[above right] at (G_cube) {G};
        \node[above left] at (H_cube) {H};
        
        \node[left] at (P_point) {P};
    
    \end{tikzpicture}
    \end{center}

    \item (OSN SMP 2022) Kubus $ABCD.EFGH$ memiliki volume 1 liter. Titik $I$ pada $BF$ dan titik $J$ pada $CG$ sehingga $IJ \parallel BC$. Titik $K$ dan $L$ adalah titik Tengah $EH$ dan $FG$ berturut-turut. Titik $O$ pada bidang $IJK$ yang paling dekat ke $L$. jarak $F$ ke garis $IK$ adalah $\frac{5}{2}\sqrt{70}$. Tentukan volume limas $L.OIJ$.

    \newpage
    \item (OSN SMP 2024) Panjang salah satu rusuk dari suatu balok adalah $7 + \sqrt{3}~cm$ dan luas permukaannya $176 cm^2$. Volume balok adalah $V~cm^3$ dan total panjang rusuk adalah $k~cm$, dengan $V$ dan $k$ merupakan bilangan bulat. Tentukan panjang diagonal ruang balok tersebut.

    \item (OSN SMP 2023) Terdapat sebuah kubus $ABCD.EFGH$ yang diletakkan dengan sisi $ABCD$ berada di lantai. Kubus itu diangkat sehingga posisi titik A tetap dan diagonal bidang ABGH tegak lurus lantai. Jarak titik $B$ dan titik $H$ ke lantai adalah 4 dan 12 berturut-turut. Kubus itu dipotong menjadi tiga bagian oleh dua bidang sejajar lantai dengan bidang pertama melalui $B$ dan bidang kedua berjarak 14 ke lantai. Tentukan volume terbesar dari ketiga bagian tersebut.

    \item (OSN SMP 2016) Diberikan kubus $ABCD.EFGH$ dengan panjang rusuk 1 dm. Terdapat persegi $PQRS$ pada bidang diagonal $ABGH$ dengan titik $P$ pada $HG$ dan $Q$ pada $AH$ seperti ditunjukkan pada gambar di bawah. Titik $T$ adalah titik pusat persegi $PQRS$. Garis $HT$ diperpanjang sehingga memotong garis diagonal $BG$ di $N$. Titik $M$ adalah proyeksi $N$ terhadapat $BC$. Tentukan volume prisma terpancung $DCM.HGN$.
    \begin{center}
    \begin{tikzpicture}[scale=3.5]
        % Define the coordinates for the cube vertices
        \coordinate (D) at (0.5,0.5);
        \coordinate (A) at (0, 0);
        \coordinate (C) at (1.5, 0.5);
        \coordinate (B) at ($(A)+(C)-(D)$);
        \coordinate (H) at (0.5, 1.5);
        \coordinate (E) at ($(A)+(H)-(D)$);
        \coordinate (G) at ($(C)+(H)-(D)$);
        \coordinate (F) at ($(B)+(H)-(D)$);

        % Find intersection point N
        \coordinate (N) at ($(G)!0.5!(B)$);
        % Define center T
        \coordinate (T) at ($(H)!0.3!(N)$);

        % Define points P, Q, R, S based on visual placement
        \coordinate (P) at ($(H)!0.3!(G)$);
        \coordinate (Q) at ($(H)!0.15!(A)$); % Q is on AH
        \coordinate (R) at ($(T)!-1.0!(P)$);
        \coordinate (S) at ($(T)!-1.0!(Q)$); % S is on GB
        
        % Draw the square PQRS (some edges dotted)
        \draw[thick] (Q) -- (P) -- (S);
        \draw[thick] (Q) -- (R) -- (S);

        % Define line paths for intersection
        \path[name path=HT_line] (H) -- ($(H)!1.5!(T)$);
        \path[name path=BG_line] (B) -- (G);

        % Find projection M on BC
        % For a standard cube, projection of N(nx,ny,nz) onto line BC is (nx, s, 0)
        % In our perspective, we find the point on BC that has the same 'depth' as N.
        \coordinate (M) at ($(B)!0.5!(C)$); % Project N onto the line BC

        % Draw additional lines
        \draw[thick] (P) -- (R);
        \draw[thick] (S) -- (Q);
        \draw[thick, dotted] (H) -- (N); % The full line HTN
        
        % Draw the prism shape DCM.HGN
        \draw[thick] (N) -- (M);
        \draw[thick] (C) -- (M);
        \draw[thick] (D) -- (C); % Already drawn
        \draw[thick] (H) -- (G); % Already drawn
        \draw[thick] (G) -- (N);
        \draw[thick, dotted] (H) -- (D); % Dotted
        \draw[thick] (G) -- (C); % Already drawn

        
        \draw[thick, dotted] (D) -- (M);
        \draw[thick, dotted] (H) -- (A);
        
        \draw[thick] (D) -- (A);
        \draw[thick] (D) -- (C);
        \draw[thick] (D) -- (H);
        \draw[thick] (E) -- (A);

    
        \draw[thick] (E) -- (F) -- (G) -- (H) -- cycle;
        \draw[thick] (A) -- (B) -- (C);
        \draw[thick] (B) -- (F);
        \draw[thick] (C) -- (G);
        \draw[thick] (E) -- (H);
        \draw[thick] (G) -- (B);

        % Label all the points
        \foreach \point/\pos in {A/left, B/below, C/right, D/below, E/left, F/above, G/right, H/left, P/above, Q/left, R/below, S/right, T/above, M/right, N/right}
        {
            \node[\pos] at (\point) {$\point$};
        }
    \end{tikzpicture}
    \end{center}
\end{enumerate}

\newpage
\section{Geometri Euclid dengan Analit: Coordinate Bashing}
\subsection{Rumus-rumus}
Tidak semua, tapi cukup banyak:
\begin{itemize}
    \item Persamaan garis yang menghubungkan titik $(x_1, y_1)$ dan titik $(x_2, y_2)$:
    \[ y - y_1 = m(x - x_1) \quad \text{dimana} \quad m = \frac{y_2 - y_1}{x_2 - x_1} \text{ adalah kemiringan garis.} \]
    
    \item Jarak antara titik $(x_1, y_1)$ dan $(x_2, y_2)$:
    \[ d = \sqrt{(x_2 - x_1)^2 + (y_2 - y_1)^2} \]
    
    \item Titik tengah antara dua titik $(x_1, y_1)$ dan $(x_2, y_2)$:
    \[ M = \left(\frac{x_1 + x_2}{2}, \frac{y_1 + y_2}{2}\right) \]
    
    \item Persamaan lingkaran dengan radius $r$ dan pusat $(a, b)$:
    \[ (x - a)^2 + (y - b)^2 = r^2 \]
    
    \item Formula untuk garis tegak lurus
    \[ m_1m_2 = -1, \quad \text{dimana } m_1, m_2 \text{ adalah kemiringan dari dua garis yang saling tegak lurus.} \]
    
    \item Kemiringan sebuah garis yang melalui titik asal $O$:
    \[ m = \tan(\theta) \quad \text{dimana } \theta \text{ adalah sudut yang dibentuk garis dengan sumbu x.} \]
    
    \item Luas segitiga dari koordinat $(x_1, y_1)$, $(x_2, y_2)$, $(x_3, y_3)$:
    \[ A = \frac{1}{2} 
    \left| 
    x_1(y_2 - y_3) + x_2(y_3 - y_1) + x_3(y_1 - y_2) 
    \right| \]
\end{itemize}

\subsection{Soal Coordinate Bashing}
\begin{enumerate}[resume]
    \item (PUMAC 2016) Let $ABCD$ be a square with side length 8. Let $M$ be the midpoint of $BC$ and let $\omega$ be the circle passing through $M$, $A$, and $D$. Let $O$ be the center of $\omega$, $X$ be the intersection point (besides $A$) of $\omega$ with $AB$, and $Y$ be the intersection point of $OX$ and $AM$. Find the length $OY$.

    \item (AMC 2004 10B) In the right triangle $\triangle ACE$, we have $AC = 12$, $CE = 16$, and $EA = 20$. Points $B$, $D$, and $F$ are located on $AC$, $CE$, and $EA$, respectively, so that $AB = 3$, $CD = 4$, and $EF = 5$. What is the ratio of the area of $\triangle DBF$ to that of $\triangle ACE$?

    \item (AMC 2005 10B) Equilateral $\triangle ABC$ has side length 2, $M$ is the midpoint of $AC$, and $C$ is the midpoint of $BD$. What is the area of $\triangle CDM$?

    \item (AMC 2004 10B) A triangle with sides of 5, 12, and 13 has both an inscribed and a circumscribed circle. What is the distance between the centers of those circles?

    \item Two circles centered at $O$ and $P$ have radii of length $5$ and $6$ respectively. Circle $O$ passes through point $P$. Let the intersection points of circles $O$ and $P$ be $M$ and $N$. Find the area of $\triangle MNP$.

    \item Let $ABC$ be a triangle with $AB = 13$, $BC = 14$, $CA = 15$. Let $O$ be its circumcenter and let $AD \perp BC$ with $D$ on $BC$. Suppose that $X$ is on $DC$ and $Y$ is on $AD$ such that $XY \parallel AO$ and $AX \perp YO$. Find the length of $BX$.

    \item Let $\triangle ABC$ be a triangle with $D$ on $BC$. Suppose $AB = \sqrt{2}$, $AC = \sqrt{3}$, $\angle BAD = 30^\circ$, $\angle CAD = 45^\circ$. Find $AD$.

    \item (AMC 2009 10B) Rectangle $ABCD$ has $AB = 8$ and $BC = 6$. Point $M$ is the midpoint of diagonal $AC$, and $E$ is on $AB$ with $ME \perp AC$. What is the area of $\triangle AME$?


    \item (AMC 2018 10B) Let $ABCDEF$ be a regular hexagon with side length $1$. Denote by $X$, $Y$ , and $Z$ the midpoints of sides $AB$, $CD$, and $EF$, respectively. What is the area of the convex hexagon whose interior is the intersection of the interiors of $\triangle ACE$ and $\triangle XYZ$?


    \item (PUMAC 2017) Triangle $ABC$ has $AB = BC = 10$ and $CA = 16$. The circle $\Omega$ is drawn with diameter $BC$. $\Omega$ meets $AC$ at points $C$ and $D$. Find the area of $\triangle ABD$.


    \item (PUMAC 2012) Two circles centered at $O$ and $P$ have radii of length $5$ and $6$ respectively. Circle $O$ passes through point $P$. Let the intersection points of circles $O$ and $P$ be $M$ and $N$. Find the area of $\triangle MNP$.


    \item (PUMAC 2018) Triangle $ABC$ has $\angle A = 90^\circ$, $\angle C = 30^\circ$, and $AC = 12$. Let the circumcircle of this triangle be $\omega$. Define $D$ to be the point on arc $BC$ not containing $A$ so that $\angle CAD = 60^\circ$. Define points $E$ and $F$ to be the foots of the perpendiculars from $D$ to lines $AB$ and $AC$, respectively. Let $J$ be the intersection of line $EF$ with $\omega$, where $J$ is on the minor arc $AC$. The line $DF$ intersects $W$ at $H$ other than at $D$. Find the area of the triangle $FHJ$.


    \item (PUMAC 2018) Consider rectangle $ABCD$ with $AB = 30$ and $BC = 60$. Construct circle $T$ whose diameter is $AD$. Construct circle $S$ whose diameter is $AB$. Let circles $T$ and $S$ intersect at $P$, so that $P \neq A$. Let $AP$ intersect $BC$ at $E$. Let $F$ be the point on $AB$ so that $EF$ is tangent to the circle with diameter $AD$. Find the area of $\triangle AEF$.
\end{enumerate}

\section{Persamaan Koordinat Kartesiuss}
\begin{enumerate}[resume]
    \item (\textbf{OSN SMP 2013}) Misalkan A, B dan P adalah paku - paku yang ditanam pada papan ABP. Panjang $AP=a$ satuan dan $BP=b$ satuan. Papan ABP diletakkan pada lintasan $X_{1}X_{2}$ dan $Y_{1}Y_{2}$ sehingga A hanya dapat bergerak bebas sepanjang lintasan $X_{1}X_{2}$ dan B hanya bergerak bebas sepanjang lintasan $Y_{1}Y_{2}$ seperti pada gambar berikut. Misalkan $x$ adalah jarak titik P terhadap lintasan $Y_{1}Y_{2}$ dan $y$ jarak titik P terhadap lintasan $X_{1}X_{2}$. Tunjukkan bahwa persamaan lintasan titik P adalah $\frac{x^{2}}{b^{2}}+\frac{y^{2}}{a^{2}}=1$.
    \begin{figure}[H]
    \centering
    \includegraphics[width=0.5\textwidth]{0Figure/osn-smp-2013-4.png}
    \end{figure}

    \item (\textbf{OSN SMP 2013}) Diketahui parabola $y=ax^{2}+bx+c$ melalui titik (-3,4) dan (3, 16), serta tidak memotong sumbu-X. Carilah semua nilai absis yang mungkin untuk titik puncak parabola tersebut.

    \item (\textbf{OSN SMP 2015}) Diketahui persamaan $ax^{2}+bx+c=0$ dengan $a>0$ mempunyai dua akar real yang berbeda dan persamaan $ac^{2}x^{4}+2acdx^{3}+(bc+ad^{2})x^{2}+bdx+c=0$ tidak mempunyai akar real. Apakah $ad^{4}+2ad^{2}<4bc+16c^{3}$?

    \item (\textbf{OSN SMP 2017}) Parabola $y=ax^{2}+bx$ dengan $a<0$ memiliki puncak di titik C dan memotong sumbu x di titik A dan B yang berbeda. Garis $y=ax$ memotong parabola tersebut di titik berbeda A dan D. Jika luas segitiga ABC sama dengan $|ab|$ kali luas segitiga ABD, tentukan nilai $b$ sebagai fungsi dari $a$ tanpa menggunakan nilai mutlak.\\
    Catatan: $|x|$ disebut nilai mutlak $x$ dengan $|x|=\begin{cases}x, & \text{jika } x \ge 0 \\ -x, & \text{jika } x<0\end{cases}$

    \item (\textbf{OSN SMP 2018}) Misalkan fungsi $f, g: \mathbb{R} \to \mathbb{R}$ diberikan dalam bentuk garfik berikut.
    \begin{figure}[H]
    \includegraphics[width=0.45\textwidth, left]{0Figure/osn-smp-2018-2-f.png}
    \includegraphics[width=0.45\textwidth, right]{0Figure/osn-smp-2018-2-g.png}
    \end{figure}
    Didefinisikan fungsi $gof$ dengan $(gof)(x) = g(f(x))$ untuk semua $x \in D_{f}$ dengan $D_{f}$ adalah daerah asal $f$.
    \begin{enumerate}
        \item[a).] gambarlah grafik fungsi $gof$.
        \item[b).] tentukan semua bilangan $x$ agar $-\frac{1}{2} \le (gof)(x) \le 6$.
    \end{enumerate}

    \item (\textbf{OSN SMP 2020}) Diberikan grafik fungsi $f:\mathbb{R} \to \mathbb{R}$ dan $g:\mathbb{R} \to \mathbb{R}$. Tentukan banyak nilai $x$ agar $(f(x))^{2}-2g(x)-x$ merupakan anggota himpunan $\{-10,-9,...,0,1,2,...,10\}$.
    \begin{figure}[H]
    \centering
    \includegraphics[width=0.4\textwidth]{0Figure/osn-smp-2020-3.png}
    \end{figure}

    \item (\textbf{OSN SMP 2022}) Misalkan $a$ dan $b$ adalah bilangan real yang memenuhi $a^{2}+b^{2} \le 1$. Tentukan peluang grafik $f(x)=ax^{2}-2bx-a$ memotong grafik $g(x)=2x^{2}$ setidaknya satu titik.

    \item (\textbf{OSN SMP 2022}) Suatu jaring-jaring laba-laba terbentuk dari komponen berikut.
    \begin{itemize}
        \item Beberapa lingkaran dengan persamaan $x^{2}+y^{2}=r^{2}$ dengan jari-jari $r=0,1,3,5,7,9$ satuan.
        \item Garis $x=0$, $y=0$, $x+y=0$, $x-y=0$.
    \end{itemize}
    Pergerakan laba-laba memiliki aturan sebagai berikut:
    \begin{itemize}
        \item Pergerakan dari lingkaran yang satu ke lingkaran terdekat dapat dilakukan melalui jaring-jaring yang ada di lingkaran awal atau garis $x=0, y=0, x+y=0,$ dan $x-y=0$, atau
        \item Membuat lintasan baru dengan menghubungkan suatu titik di jaring-jaring yang sudah ada ke titik $(a, b)$ yang terletak pada lingkaran terdekat, dimana $a, b$ bilangan bulat, atau sebaliknya.
        \item Laba-laba hanya bisa menggunakan lingkaran dan garis di jaring-jaring yang sudah ada maupun lintasan baru satu kali saja. Sebagai contoh, jika misalkan garis $x=0$ sudah dipergunakan, maka untuk pindah dari satu lingkaran ke lingkaran lain tidak boleh menggunakan garis ini kembali.
        \item Kecepatan perjalanan melalui jaring-jaring yang sudah ada adalah 1 satuan/menit.
        \item Kecepatan perjalanan melalui lintasan baru adalah setengah kecepatan perjalanan melalui jaring-jaring yang sudah ada.
    \end{itemize}
    Apabila seekor laba-laba berada pada titik $(0,0)$ dan akan menuju ke lingkaran terluar, tentukan rute dengan durasi tercepat.

    \item (\textbf{OSN SMP 2023}) Sebuah segitiga pada koordinat kartesius dikatakan PAS jika titik beratnya berada pada titik letis. Tentukan nilai $k$ terkecil sehingga jika dipilih $k$ titik letis dengan syarat tidak ada tiga titik yang segaris agar pasti terdapat tiga titik yang membentuk segitiga PAS.

    \item (\textbf{OSN SMP 2024}) Seorang perancang bangunan merancang jembatan gantung sepanjang 120 meter dan lebar 5 meter menggunakan lima pasang tiang gantung dengan rancangan tampak depan (satu sisi) seperti gambar berikut.
    \begin{figure}[H]
    \centering
    \includegraphics[width=\textwidth]{0Figure/osn-smp-2024-10.png}
    \end{figure}
    Pada satu sisi depan, jembatan berada 2 meter di atas permukaan air. Tinggi tiang gantung yang diukur dari atas permukaan air masing-masing $BL=4$ m, $CM=20$ m, $DN=4$ m, $EO=8$ m, dan $FP=4$ m. Kawat penyangga KLM dan MNO berbentuk parabola, sedangkan OPQ berbentuk garis. Jarak titik K ke tiang BL adalah 20 m dan jarak titik Q ke tiang FP adalah 10 m. Untuk menahan jembatan, disusun tali-tali yang menggantung dan diikat pada kawat penyangga KLMNOPQ dan pinggir jembatan dengan jarak antar tali adalah 1 meter. Diasumsikan bahwa panjang tali yang digunakan adalah panjang tali dari kawat penyangga ke pinggir jembatan dan tegak lurus terhadap jembatan. Apabila harga tali per meter adalah Rp100.000, hitunglah biaya total tali yang diperlukan untuk kedua sisi jembatan tersebut.

\end{enumerate}
\end{document}