\documentclass[11pt]{scrartcl}
\usepackage[sexy]{evan}
\usepackage[normalem]{ulem}

\renewcommand{\baselinestretch}{1.5}



\begin{document}
	\title{Medium - Advanced Algebra} % Beginner
	\date{\today}
	\author{Compiled by Azzam}
	\maketitle
	
	\begin{soalbaru}
	Diketahui $a$,$b$,$c$ adalah bilangan real yang memenuhi persamaan
	$$\sqrt{a-4}+\sqrt{b}+\sqrt{c+1}=\dfrac{a+b+c}{2}$$
	tentukan nilai $a+b+c$.
	\end{soalbaru}
	
	\begin{soalbaru}
	Nilai dari $\left(0,2 ^{\left(0,5 ^{\left(0,8 ^{...} \right)} \right)} \right)^{-1}$ adalah $\dots$ (perhatikan bahwa pangkatnya membentuk barisan aritmatika dengan beda 0,3)
	\end{soalbaru}
	
	\begin{soalbaru}
	Tentukan semua tupel bilangan real $(x,y,z)$ yang memenuhi
	\begin{align*}
	  x+y &= 2 \\
	xy-z^2 &= 1  
	\end{align*}
	\end{soalbaru}
	
	\begin{soalbaru}
	Tentukan semua solusi bulat $(x,y)$ yang memenuhi 
	$x+y=x^2-xy+y^2.$
	\end{soalbaru}
	
    \begin{soalbaru}
    Jika diketahui bilangan real positif $x,y,z$ memenuhi $xy+2(x+y)z=12$, tentukan nilai maksimal $xyz.$
    \end{soalbaru}
    
    \begin{soalbaru}
    Misalkan $a$ dan $b$ adalah bilangan real sedemikian sehingga $$a\sqrt{1-b^2}+b\sqrt{1-a^2}=1.$$
    Hitunglah nilai dari $a^2+b^2$.
    \end{soalbaru}
    
    \begin{soalbaru}
    Tentukan semua solusi real $(x,y,z)$ dari
    \begin{align*}
        x^2 &= yz+1\\
        y^2 &= zx+2\\
        z^2 &= xy+4
    \end{align*}
    \end{soalbaru}
    
    \begin{soalbaru}
    Tentukan semua penyelesaian real $(x,y)$ yang memenuhi
    $$20\sin x-21 \cos x = 81y^2-18y+30.$$
    \end{soalbaru}
    
    \begin{soalbaru}
    Carilah semua solusi real dari sistem persamaan
    \begin{align*}
        (x-1)(y^2+6) &= y(x^2+1)\\
        (y-1)(x^2+6) &= x(y^2+1)
    \end{align*}
    \end{soalbaru}
    
    \begin{soalbaru}
    Misalkan $a,b,c$ adalah bilangan real bukan nol yang berbeda sehingga 
    $$a+\dfrac{1}{b}=b+\dfrac{1}{c}=c+\dfrac{1}{a}.$$
    Tentukan nilai dari $|abc|$
    \end{soalbaru}
    
    \begin{soalbaru}
		Diberikan persamaan $x^3-x+1=0$ yang mempunyai akar-akar $a,b,$ dan $c$. Carilah nilai $a^8+b^8+c^8$.
	\end{soalbaru}
	
	\begin{soalbaru}
			Diberikan $\dfrac{\pi}{4} = 1 - \dfrac{1}{3}+\dfrac{1}{5}-\dfrac{1}{7}+\dots$. Jika  $\dfrac{1}{1 \times 3 \times 5}+\dfrac{3}{5 \times 7 \times 9}+\dfrac{5}{9 \times 11 \times 13}+\dots = \dfrac{a+b\pi}{c}$, dimana $a,b,c$ adalah bilangan asli dengan $b$ relatif prima dengan $c$, hitunglah nilai $a+b+c$.
		\end{soalbaru}
	
	\begin{soalbaru}
		Diberikan polinomial $p(x)=x^3-ax^2+bx-c$ mempunyai tiga akar bulat positif berbeda dan $p(2002)=2001$. Misalkan $q(x)=x^2-2x+2002$. Diketahui pula bahwa $p(q(x))$ tidak mempunyai akar real. Tentukan nilai $a$.
	\end{soalbaru}
	
	\begin{soalbaru}
		Jumlah seluruh bilangan real $x$ yang memenuhi $10^x+11^x+12^x=13^x+14^x$ adalah... %probs 15 101
	\end{soalbaru}

    \begin{soalbaru}(JBMO 2020)
    Find all triples $(a,b,c)$ of real numbers such that the following system holds:
$$\begin{cases} a+b+c=\frac{1}{a}+\frac{1}{b}+\frac{1}{c} \\a^2+b^2+c^2=\frac{1}{a^2}+\frac{1}{b^2}+\frac{1}{c^2}\end{cases}$$
    \end{soalbaru}
    
        \begin{soalbaru}(EGMO 2019)
    Find all triples $(a, b, c)$ of real numbers such that $ab + bc + ca = 1$ and $$a^2b + c = b^2c + a = c^2a + b.$$
    \end{soalbaru}
    
    \begin{soalbaru}(BMO 2017)
    Find all ordered pairs of positive integers$ (x, y)$ such that:$$x^3+y^3=x^2+42xy+y^2.$$
    \end{soalbaru}
    
    
    \begin{soalbaru}(BMO 2019)
    Let $\mathbb{P}$ be the set of all prime numbers. Find all functions $f:\mathbb{P}\rightarrow\mathbb{P}$ such that:
$$f(p)^{f(q)}+q^p=f(q)^{f(p)}+p^q$$holds for all $p,q\in\mathbb{P}$.
    \end{soalbaru}

    
    \begin{soalbaru}(JBMO 2004)
    Prove that the inequality\[ \frac{ x+y}{x^2-xy+y^2 } \leq \frac{ 2\sqrt 2 }{\sqrt{ x^2 +y^2 } } \]holds for all real numbers $x$ and $y$, not both equal to 0.
    \end{soalbaru}
    
    \begin{soalbaru}(BMO 2014)
    Let $x,y$ and $z$ be positive real numbers such that $xy+yz+xz=3xyz$. Prove that\[ x^2y+y^2z+z^2x \ge 2(x+y+z)-3 \]and determine when equality holds.
    \end{soalbaru}
\end{document}


