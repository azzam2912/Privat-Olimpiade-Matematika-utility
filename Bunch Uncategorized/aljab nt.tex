\documentclass[12pt]{article}
\usepackage{systeme}
\usepackage{amsmath}
\usepackage{amssymb}

\newcommand{\R}{\mathbb{R}}

\addtolength{\oddsidemargin}{-.875in}
\addtolength{\evensidemargin}{-.875in}
\addtolength{\textwidth}{1.75in}

\addtolength{\topmargin}{-1.0in}
\addtolength{\textheight}{1.75in} 
%opening
\title{Paket Soal Aljabar dan Teori Bilangan Olimpiade SMA}
\author{compiled by: Azzam L. H.}
\date{7 Oktober}

\begin{document}
	
	\maketitle
	
	\section{Aljabar}
	\begin{enumerate}
		\item
		Carilah seluruh fungsi $f:\R \rightarrow \R$ sedemikian sehingga
		\[
		f(x^2y)=f(xy)+yf(f(x)+y)
		\]
		untuk seluruh $x,y \in \R$
		
		\item
		Untuk setiap bilangan real positif $x,y$, tunjukkan bahwa $$\frac{(x+y)^6}{x^2y^2(y+2x)^2} \geq \frac{27}{4}$$
		
		\item
		Cari semua solusi dari sistem persamaan:
		$$
		\begin{cases}
			\vspace{2mm}
			\dfrac{4x^2}{4x^2+1}=y \\
			\vspace{2mm}
			\dfrac{4y^2}{4y^2+1}=z \\
			\dfrac{4z^2}{4z^2+1}=x
		\end{cases}
		$$
		
		\item
		Carilah bilangan real $x$ yang memenuhi:
		\[
		\sqrt{1+\sqrt{1-x^2}}\left(\sqrt{(1+x)^3}-\sqrt{(1-x)^3}\right)=2+\sqrt{1-x^2}
		\]
		
		\item
		Selesaikan sistem persamaan berikut:
		$$
		\begin{cases}
			x^2+y^2+z^2+t^2=50 \\
			x^2-y^2+z^2-t^2=-24 \\
			xz = yt \\
			x-y+z+t=0
		\end{cases}
		$$
		
		\item
		Untuk setiap $x,y \in \R$, carilah fungsi $f: \R \rightarrow  \R$ yang memenuhi:
		$$f(x+y)-f(x-y)=f(x)f(y)$$
		
		\item
		Diberikan akar-akar dari persamaan $x^3-x+1=0$ adalah $\alpha, \beta, \gamma$. Carilah nilai dari $\alpha^8+\beta^8+\gamma^8$.
		
		\item
		Untuk setiap $x,y \in \R$, carilah fungsi $f: \R \rightarrow  \R$ yang memenuhi:
		$$(x-y)f(x+y)-(x+y)f(x-y) = 4xy(x^2-y^2)$$
		
		\item
		Misalkan $a_1,a_2,a_3,\dots$ adalah barisan bilangan real positif yang memenuhi
		\[
		a_{k+1} \geq \dfrac{k(a_k)}{a^2_k+k-1} , \quad k \geq 1
		\]
		Buktikan bahwa $a_1 + a_2 + \dots + a_n \geq n$ untuk semua $n\geq 2$.
		
		\item
		Tentukan semua fungsi $f:\mathbb{Q} \rightarrow \mathbb{Q}$ yang memenuhi
		\[
		f(x+y)-f(x-y)=2f(x)+2f(y)
		\]
		untuk semua $x,y \in \mathbb{Q}$.
		
		\item
		Misalkan $P(x)$ dan $Q(x)$ adalah dua polinomial dengan koefisien bilangan real dan memenuhi $P(x)=Q(x)$ untuk setiap $x \in \R$. Buktikan bahwa $P(x)=Q(x)$ untuk setiap $x \in \mathbb{C}$.
		
		\item
		Tentukan semua polinomial $P(x)$ dengan koefisien real yang memenuhi $P(x^2)=(P(x))^2$.
		
	\end{enumerate}
	
	
	\section{Teori Bilangan}
	\begin{enumerate}
		\item
		Sebuah bilangan bulat positif 
		$k$ disebut \textbf{keren} apabila persamaan 
		$$a+b+c=k^a+k^b+k^c$$
		memiliki solusi bulat nonnegatif $a,b,c$.
		Carilah jumlah seluruh bilangan keren.
		
		\item
		Carilah seluruh pasangan bilangan asli $(x,y)$ yang memenuhi \[ 
		x^2+(x+1)^2 = y^2+(y+1)^2
		\]
		
		\item
		Carilah banyaknya pasangan bulat nonnegatif $(a,b)$ yang memenuhi
		\[
		a^3+b^2+1=7ab
		\]
		
		\item
		Tentukan semua bilangan asli $a,b,c$ dimana  $ 1 \leq a \le b \le c$ sedemikian sehingga
		\[
		(a-1)(b-1)(c-1) | abc -1
		\]
		
		\item
		Carilah seluruh tripel prima $(p,q,r)$ yang memenuhi $3p^4-5q^4-4r^2=26$
		
		\item
		Carilah seluruh pasangan bilangan asli terurut $(a,b)$ dimana
		\[
		\dfrac{a^3b-1}{a+1} \quad \textrm{dan} \quad \dfrac{b^3a+1}{b-1}
		\]
		keduanya adalah bilangan bulat positif.
		
		\item
		Carilah seluruh bilangan asli $n$ sehingga $n \times 2^{n+1}+1$ adalah bilangan kuadrat sempurna.
		
		\item
		Carilah seluruh bilangan bulat $m$ dan $n$ sehingga 
		\[
		m^5-n^5=16mn
		\]
		
		\item
		Jika $m,n \in \mathbb{Z}$, tentukan semua fungsi $f:\mathbb{Z} \rightarrow \mathbb{Z}$ yang memenuhi 
		\[
		f(m+n)+f(mn-1)=f(m)f(n)+2
		\]
		
		\item
		Misalkan $x$ dan $y$ adalah bilangan asli. Jika $k$ adalah bilangan bulat yang memenuhi
		\[
		\dfrac{x+1}{y}+\dfrac{y+1}{x}=k,
		\]
		carilah seluruh $k$ yang memenuhi.
		
		\item
		Tentukan seluruh pasangan bilangan asli $(a,b)$ sehingga $ab^2+b+7|a^2b+a+b$.
		
		\item
		Tentukan banyaknya bilangan asli berbeda $n$ sehingga $n+9$ dan $n^2+27$ keduanya adalah pangkat tiga dari suatu bilangan bulat.
	\end{enumerate}
	
	
\end{document}
