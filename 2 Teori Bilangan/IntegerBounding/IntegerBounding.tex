\documentclass[11pt]{scrartcl}
\usepackage{graphicx}
\graphicspath{{./}}
\usepackage[sexy]{evan}
\usepackage[normalem]{ulem}
\usepackage{hyperref}
\usepackage{mathtools}
\hypersetup{
    colorlinks=true,
    linkcolor=blue,
    filecolor=magenta,      
    urlcolor=cyan,
    pdfpagemode=FullScreen,
    }

\renewcommand{\dangle}{\measuredangle}

\renewcommand{\baselinestretch}{1.5}

\addtolength{\oddsidemargin}{-0.4in}
\addtolength{\evensidemargin}{-0.4in}
\addtolength{\textwidth}{0.8in}
% \addtolength{\topmargin}{-0.2in}
% \addtolength{\textheight}{1in} 


\setlength{\parindent}{0pt}

\usepackage{pgfplots}
\pgfplotsset{compat=1.15}
\usepackage{mathrsfs}
\usetikzlibrary{arrows}

\title{Integer Bounding}
\author{Azzam (IG: haxuv.world)}
\date{\today}

\begin{document}
\maketitle

\section{General}
\begin{enumerate}
    \item Carilah seluruh bilangan bulat positif $a,b,c$ dengan $FPB(a,b,c)=1$ yang memenuhi $a<b<c$ dan
    \begin{align*}
        abc \mid a+b+c+ab+bc+ca.
    \end{align*}

    \item (St. Petersburg 2001) Tunjukkan bahwa terdapat tak hingga banyaknya bilangan asli $n$ sedemikian sehingga faktor prima terbesar dari $n^4+1$ lebih dari $2n$.

    \item (IMO 1999) Tentukan seluruh pasangan bilangan bulat positif $(n,p)$ dengan $p$ adalah bilangan prima, $n \le 2p$, dan $n^{p-1} \mid (p-1)^n+1$.
\end{enumerate}

\section{No Square Between Two Consecutive Squares}
\begin{enumerate}[resume]
    \item (AIME II 2013) Carilah bilangan bulat positif terkecil $N$ sehingga himpunan $1000$ bilangan bulat berurutan yang dimulai dari $1000\cdot N$ tidak mengandung kuadrat sempurna dari suatu bilangan bulat.
\end{enumerate}

\section{Fermat's Method of Infinite Descent}
\begin{enumerate}[resume]
    \item (Hungaria 2000) Carilah seluruh bilangan prima $p$ sehingga terdapat bilangan bulat $x,y,n$ yang memenuhi $p^n=x^3+y^3$.
\end{enumerate}

\section{Arithmetics Function}
\begin{itemize}
    \item $\phi(n)$ adalah fungsi Euler Totient Phi yang menyatakan banyaknya bilangan asli kurang dari sama dengan $n$ yang relatif prima dengan $n$.
    \item $\sigma(n)$ menyatakan jumlah seluruh faktor positif berbeda dari $n$.
    \item $\tau(n)$ menyatakan banyaknya faktor positif berbeda dari $n$.
\end{itemize}

\begin{enumerate}[resume]
    \item Jika $n$ adalah bilangan asli komposit, buktikan bahwa
    \begin{align*}
        \phi(n) \le n - \sqrt{n}.
    \end{align*}

    \item Untuk sembarang bilangan asli $n$ dengan $n \neq 2$ dan $n \neq 6$, buktikan bahwa
    \begin{align*}
        \phi(n) \ge \sqrt{n}.
    \end{align*}
\end{enumerate}

\section{Latihan Soal Tambahan}
\begin{enumerate}
    \item (India TST 2001) Misalkan $a_1,a_2,\dots$ adalah sebuah barisan bilangan asli monoton naik dimana $FPB(a_m,a_n)=a_{FPB(m,n)}$ untuk setiap bilangan asli $m$ dan $n$. Terdapat bilangan asli terkecil $k$ sedemikian sehingga terdapat bilangan asli $r<k$ dan $s>k$ dengan yang memenuhi $a_k^2=a_ra_s$. Buktikan bahwa $r \mid k$ dan $k \mid s$.

    \item (Bulgaria TST 2001) Carilah seluruh tripel bilangan bulat positif $(a,b,c)$ sedemikian sehingga $a^3+b^3+c^3$ habis dibagi $a^2b$, $b^2c$, dan $c^2a$.

    \item Carilah seluruh bilangan prima $a,b,c$ yang memenuhi
    \begin{align*}
        ab+bc+ca > abc.
    \end{align*}
    
    \item Jika $f: \ZZ \to \ZZ$ adalah fungsi dengan $f(x)=x^2+x$ untuk sembarang $x \in \ZZ$. Apakah terdapat $a,b \in \ZZ$ sehingga berlaku $4f(a)=f(b)$ ?
    
    \item Carilah seluruh tripel bilangan bulat nonnegatif $(x,y,z)$ yang memenuhi
    \begin{align*}
        x^3+2y^3=4z^3.
    \end{align*}

    \item Buktikan bahwa $\sqrt{3}$ adalah bilangan tak rasional.
    
    \item Jika $n$ adalah bilangan asli komposit, buktikan bahwa
    \begin{align*}
        \sigma(n) \ge n + \sqrt{n} + 1.
    \end{align*}

    \item Untuk setiap bilangan asli $n \ge 7$, tunjukkan bahwa
    \begin{align*}
        \sigma(n) < n \ln n.
    \end{align*}

        \item Misalkan $a,b,c,d$ adalah bilangan bulat positif sehingga
    \begin{align*}
        (a+b)(b+c)(c+a) = abcd.
    \end{align*}
    Carilah banyaknya pasangan terurut $(a,b,c,d)$ dimana setiap pasang dari $a,b,c,d$ relatif prima.
    
    \item Hitunglah banyaknya pasangan bilangan bulat positif $(x,y)$ yang memenuhi
    \begin{align*}
        x^2+(x+1)^2 = y^4 + (y+1)^4.
    \end{align*}

    \item Untuk setiap bilangan asli $n \ge 2$, tunjukkan bahwa
    \begin{align*}
        \dfrac{\sigma(n)}{\tau(n)} \ge \sqrt{n}.
    \end{align*}
\end{enumerate}

\end{document}
