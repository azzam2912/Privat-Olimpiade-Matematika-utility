\documentclass[11pt]{scrartcl}
\usepackage{graphicx}
\graphicspath{{./}}
\usepackage[sexy]{evan}
\usepackage[normalem]{ulem}
\usepackage{hyperref}
\usepackage{mathtools}
\hypersetup{
    colorlinks=true,
    linkcolor=blue,
    filecolor=magenta,      
    urlcolor=cyan,
    
    pdfpagemode=FullScreen,
    }

\renewcommand{\dangle}{\measuredangle}

\renewcommand{\baselinestretch}{1.5}

\addtolength{\oddsidemargin}{-0.4in}
\addtolength{\evensidemargin}{-0.4in}
\addtolength{\textwidth}{0.8in}
% \addtolength{\topmargin}{-0.2in}
% \addtolength{\textheight}{1in} 


\setlength{\parindent}{0pt}

\usepackage{pgfplots}
\pgfplotsset{compat=1.15}
\usepackage{mathrsfs}
\usetikzlibrary{arrows}

\title{Integer Bounding - Soal Post Test}
\author{Azzam (IG: haxuv.world)}
\date{Senin, 15 Januari 2024}

\begin{document}
\maketitle
\textbf{Aturan umum:}
\begin{itemize}
    \item \textbf{Soal bertipe esai}. Sertakan argumentasi atau cara mendapatkan jawaban yang ditanyakan di soal.
    \item Waktu standar untuk mengerjakan semua soal berikut adalah 120 menit.
    \item Setiap soal bernilai bilangan bulat antara 0 sampai 10 (inklusif).
\end{itemize}

\section{Soal}
\begin{enumerate}
    \item Misalkan $a,b,c,d$ adalah bilangan bulat positif sehingga
    \begin{align*}
        (a+b)(b+c)(c+a) = abcd.
    \end{align*}
    Carilah banyaknya pasangan terurut $(a,b,c,d)$ dimana setiap pasang dari $a,b,c,d$ relatif prima.
    
    \item Hitunglah banyaknya pasangan bilangan bulat positif $(x,y)$ yang memenuhi
    \begin{align*}
        x^2+(x+1)^2 = y^4 + (y+1)^4.
    \end{align*}
\end{enumerate}

\begin{enumerate}[resume]
    \item Untuk setiap bilangan asli $n \ge 2$, tunjukkan bahwa
    \begin{align*}
        \dfrac{\sigma(n)}{\tau(n)} \ge \sqrt{n}.
    \end{align*}
\end{enumerate}

\textbf{Catatan:}
\begin{itemize}
    \item $\phi(n)$ adalah fungsi Euler Totient Phi yang menyatakan banyaknya bilangan asli kurang dari sama dengan $n$ yang relatif prima dengan $n$.
    \item $\sigma(n)$ menyatakan jumlah seluruh faktor positif berbeda dari $n$.
    \item $\tau(n)$ menyatakan banyaknya faktor positif berbeda dari $n$.
\end{itemize}

\end{document}
