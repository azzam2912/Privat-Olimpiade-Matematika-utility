\documentclass[12pt]{article}
\usepackage{systeme}
\usepackage{amsmath}
\usepackage{amssymb}

\newcommand{\R}{\mathbb{R}}

\addtolength{\oddsidemargin}{-.875in}
\addtolength{\evensidemargin}{-.875in}
\addtolength{\textwidth}{1.75in}

\addtolength{\topmargin}{-1.0in}
\addtolength{\textheight}{1.75in} 
%opening
\title{Paket Soal Kombinatorika dan Geometri Olimpiade SMA}
\author{compiled by: Azzam L. H.}
\date{7 Oktober 2020}
\begin{document}
	\maketitle
	
\section{Kombinatorika}
\begin{enumerate}
	\item
	Dalam sebuah pesta, setiap peserta mengenal tepat 201 peserta lainnya. Untuk setiap dua peserta yang saling mengenal satu sama lain, terdapat tepat 8 peserta lainnya yang dikenal kedua orang tersebut dan untuk setiap dua peserta, yang tidak saling mengenal, terdapat tepat 24 peserta lainnya yang dikenal kedua peserta tersebut. Perhatikan
	bahwa $a$ mengenal $b$ jika dan hanya jika $b$ mengenal $a$. Tentukan banyaknya peserta dalam pesta tersebut.
	
	\item
	Hitunglah banyaknya polinomial $P(x)$ dengan koefisien-koefisien dipilih dari ${0,1,2,3}$ sedemikian sehingga $P(2)=2019$.
	
	\item
	Terdapat $n$ tumpukan batu dengan masing-masing tumpukan terdiri dari 2018 batu. Berat masing-masing batu adalah salah satu diantara bilangan-bilangan $1,2,\dots,25$ dan setiap dua tumpukan yang berbeda memiliki berat total yang berbeda. Diketahui bahwa untuk setiap dua tumpukan, jika batu terberat dan teringan  pada kedua tumpukan tersebut dibuang, maka tumpukan yang mulanya lebih berat menjadi lebih ringan. Tentukan nilai $n$ terbesar yang mungkin. 
	
	\item
	Ali dan Baba mempunyai tak hingga banyaknya koin yang identik. Ali dan Baba bergiliran (Ali duluan) menaruh koin di meja persegi yang ukuran sisinya berhingga, sehingga tidak ada koin yang bertumpuk dan semua koin tersebut ada di atas meja (jadi tidak ada yang terlalu kepinggir sampe hampir jatuh). Pemain yang tidak bisa melakukan hal tersebut dinyatakan kalah. Asumsikan setidaknya satu koin bisa ditaruh di meja, siapakah yang mempunyai strategi menang?
	
	\item
	Pada sebuah kompetisi, ada $a$ peserta dan $b$ juri, dimana $b \geq 3$ adalah bilangan ganjil. Setiap juri menilai setiap peserta dengan $\textit{lulus}$ atau $\textit{gagal}$. Misalkan bilangan $k$ menyatakan banyaknya cara, sedemikian sehingga untuk setiap dua juri, penilaian mereka sama untuk paling banyak $k$ peserta. Buktikan bahwa \[ \dfrac{k}{a} \geq \dfrac{b-1}{2b} \]
	
	\item 
	Ada  $n$ titik di bidang datar sedemikian sehingga tidak ada dari tiga titik yang segaris. 
	Jika $k$ adalah banyaknya segitiga yang mempunyai luas 1, dan titik-titik sudutnya diambil dari $n$ titik tersebut, buktikan bahwa 
	\[
	k \leq \frac{2}{3}(n^2-n)
	\]
	
	\item
	Misalkan $n$ dan $k$ adalah bilangan bulat positif dan $S$ adalah himpunan $n$ titik yang memenuhi:
	\begin{itemize}
		\item Tidak ada tiga titik di $S$ yang segaris.
		
		\item Untuk setiap titik $P$ di $S$, ada setidaknya $k$ titik di $S$ yang berjarak sama dengan $P$.
	\end{itemize}
	Buktikan bahwa $k < \frac{1}{2} + \sqrt{2n}$
	
	\item
	Diberikan $n$ lampu yang berjejer di satu baris. Pada awalnya lampu paling kiri menyala dan sisanya padam. Pada suatu menit, sebuah lampu padam jika dan hanya jika pada menit sebelumnya, lampu tersebut beserta tetangga-tetangganya berstatus sama (semua menyala atau semua mati). Buktikan terdapat tak hingga banyaknya $n$ sehingga pada akhirnya semua lampu padam.
	
	\item
	Ada $2n$ orang yang menghadiri suatu pesta. Setiap orang mempunyai sebanyak genap teman di pesta (disini jika $a$ berteman dengan $b$, dipastikan $b$ juga berteman dengan $a$). Buktikan bahwa ada dua orang di pesta yang mempunyai sebanyak genap teman yang sama (common friends).
	
	\item
	Ada sebanyak 119 penghuni yang tinggal di asrama dengan 120 kamar. Kita sebut sebuah kamar $\textit{PENGAP}$ jika sedikitnya ada 15 orang yang tinggal di kamar tersebut. Setiap hari, setiap penghuni kamar yang PENGAP bertengkar satu sama dengan penghuni lain kamar itu (karena sumpek) dan akhirnya setiap penghuni kamar tersebut pindah ke kamar lain di asrama itu dimana setiap penghuni kamar PENGAP itu pindah ke kamar yang berbeda-beda. Buktikan bahwa suatu hari, di asrama tersebut tidak akan ada lagi kamar yang PENGAP.
\end{enumerate}

\section{Geometri}
\begin{enumerate}
	\item
	Diberikan persegi panjang $ABCD$. Misalkan titik $E$ di sisi $AB$ dan misalkan titik $F$ pada $CD$ sehingga $EF$ tegak lurus $CD$. Sebuah lingkaran yang melewati $A,B$ dan $F$ memotong $AD$ dan $BC$ di $X$ dan $Y$, secara berturut-turut. Buktikan bahwa $XY$ melewati titik tinggi segitiga $CDE$.
	
	\item
	Diberikan segitiga lancip $ABC$. Misalkan $X$ di $AB$ dan $Y$ di $AC$ sedemikian sehingga $BXYC$ siklis. Misalkan juga $R_1,R_2,$ dan $R_3$ berturut-turut adalah jari-jari lingkaran luar segitiga $AXY, BXY,$ dan $ABC$. Buktikan bahwa jika $R_1^2+R_2^2=R_3^2$, maka $BC$ adalah diameter dari lingkaran yang melewati $B,X,Y,C$.
	
	\item
	Misalkan $P$ dan $Q$ adalah titik-titik di sisi $AB$ dari segitiga $ABC$ (dengan $P$ diantara $A$ dan $Q$) sehingga $\angle ACP = \angle PCQ = \angle QCB$ serta misalkan $AD$ adalah garis bagi $\angle BAC$. Garis $AD$ memotong garis $CP$ ddan $CQ$ berturut-turut di $M$ dan $N$. Jika $PN=CD$ dan $3\angle BAC=2\angle BCA$, buktikan bahwa segitiga $CQD$ dan $QNB$ mempunyai luas yang sama.
	
	\item
	Misalkan $ABC$ adalah segitiga dengan titik bagi (perpotongan ketiga garis bagi segitiga) $I$. Jika $P$ adalah sebuah titik di dalam segitiga $ABC$ yang memenuhi
	\[
	\angle PBA + \angle PCA = \angle PBC + \angle PCB.
	\]
	Tunjukkan bahwa $AP \geq AI$ dan tunjukkan bahwa kesamaan dicapai jika dan hanya jika $P=I$. 
	
	\item
	Misalkan $ABC$ adalah segitiga dengan $D$ pada $BC$ sehingga $AD$ tegak lurus $BC$. Garis yang melalui $A$ dan sejajar $BC$ memotong lingkaran luar segitiga $ABC$ untuk kedua kalinya di $E$. Buktikan bahwa garis $DE$ melewati titik berat segitiga $ABC$.
	
	\item
	Diberikan segiempat $ABCD$ dimana diagonal-diagonalnya saling tegak lurus dan berpotongan di $O$. Buktikan titik hasil pencerminan $O$ terhadap $AB,BC,CD,DA$ siklis.
	
	\item
	Pada segitiga $ABC$ dengan sudut $\angle BAC = 120^\circ$, notasikan $D,E,F$ sebagai perpotongan garis bagi sudut $A,B,C$ dengan $BC,CA,AB$ secara berturut-turut. Carilah nilai $\angle EDF$.
	
	\item
	Misalkan $H$ adalah titik tinggi dari segitiga lancip $ABC$. Jika $x$ adalah segitiga yang titik-titik sudutnya dibentuk oleh titik pusat lingkaran luar segitiga $ABH,BCH$, dan $CAH$, buktikan bahwa $x$ kongruen dengan segitiga $ABC$.
	
	\item
	Pada segiempat siklis $ABCD$, titik $X$ dan $Y$ adalah titik-titik tinggi dari segitiga $ABC$ dan $BCD$ secara berturut-turut. Tunjukkan bahwa $AXYD$ adalah jajargenjang.
	
	\item
	Diberikan $M$ sebagai titik tengah sisi $BC$ dari segitiga $ABC$. Titik $K$ pada segmen $AM$ sedemikian sehingga $CK=AB$. Notasikan $L$ sebagai perpotongan $CK$ dan $AB$. Buktikan bahwa segitiga $AKL$ sama kaki.
\end{enumerate}
\end{document}