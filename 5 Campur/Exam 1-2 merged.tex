\documentclass[a4paper, 12pt]{article}
\usepackage[hagavi]{azzam}

\usepackage{CJKutf8}
\begin{CJK*}{UTF8}{min}
\title{さまざまな知識}
\begin{document}

\maketitle
\begin{enumerate}
    %% kebanyakan dari lmnas smp 30
    \item Diketahui $a_1, a_2, \dots, a_k$ merupakan barisan aritmetika dengan:
    \begin{align*}
        a_4 + a_7 + a_{10} = 17\\
    a_4 + a_5 + a_6 + a_7 + a_8 + a_9 + a_{10} + a_{11} + a_{12} + a_{13} + a_{14} = 77\\
    a_k = 13
    \end{align*}
    Berapa nilai $k$?

    \item Untuk setiap bilangan real $x$, didefinisikan $\lfloor x \rfloor$ sebagai bilangan bulat terbesar yang tidak lebih dari $x$. Sebagai contoh, $\lfloor 1,2 \rfloor = 1$ dan $\lfloor 3,567 \rfloor = 3$. Nilai dari
    \[ \left\lfloor \dfrac{1}{\lfloor \sqrt[3]{1} \rfloor} + \dfrac{1}{\lfloor \sqrt[3]{2} \rfloor} + \dfrac{1}{\lfloor \sqrt[3]{3} \rfloor} + \dots + \dfrac{1}{\lfloor \sqrt[3]{1000} \rfloor} \right\rfloor \]
    adalah...

    % Babak Penyisihan SMP nomor 13
    \item Diketahui $A = \dfrac{1+\dfrac{1}{3}+\dfrac{1}{5}+\dots+\dfrac{1}{99}}{\dfrac{1}{1\times99}+\dfrac{1}{3\times97}+\dfrac{1}{5\times95}+\dots+\dfrac{1}{99\times1}}$. Nilai $\lfloor A \rfloor =\dots$
    
    % Babak Penyisihan SMP nomor 15
    \item Diketahui bahwa $a$ dan $b$ adalah bilangan real serta terdapat bilangan real positif $x$ dan $y$ sedemikian hingga $a=1+\dfrac{2x}{y}$ dan $b=1+\dfrac{2y}{x}$. Jika $a^2+b^2=29$ maka digit terakhir dari $a^{2019}+b^{2019}$ adalah...

    \item Tentukan jumlahan semua bilangan prima $x$ yang membuat $\left\lfloor \dfrac{x^2}{6} \right\rfloor + \left\lfloor \dfrac{x}{2} \right\rfloor$ merupakan bilangan prima.

    % buku olimpiade.org
    \item Apabila $a,b,c$ adalah penyelesaian dari sistem persamaan
    \begin{align*}
        a+b+(c^2-8c+14)\sqrt{a+b-2} &= 1\\
        2a+5b+\sqrt{ab+c} &= 3,
    \end{align*}
    tentukan jumlah semua kemungkinan nilai $abc$.

    % babak 30 besar
    \item Diketahui $a,b,c,d$ merupakan bilangan real yang memenuhi
    \begin{align*}
        a^2+b^2+c^2+138 = d+20\sqrt{4a+6b+10c-d}
    \end{align*}
    Tentukan nilai $\lfloor a+b+c+d \rfloor$

    % paket soal osp
    \item Jumlah seluruh bilangan asli $n$ yang memenuhi $3^{n-1}+5^{n-1} \mid 3^n + 5^{n}$ adalah...

    % paket soal osp
    \item Misalkan tripel bilangan prima $(p,q,r)$ memenuhi $3p^4-5q^4-4r^2=26$. Jika $S = p+q+r$, hitunglah jumlah semua $S$ yang mungkin.

    % modifikasi jbmo 2013?
    \item Misalkan $(a,b)$ adalah pasangan bilangan bulat positif sehingga $\dfrac{a^3b-1}{a+1}$ dan $\dfrac{b^3a+1}{b-1}$ keduanya merupakan bilangan bulat positif. Jika $S$ adalah jumlah seluruh nilai $a \times b$ yang mungkin dari semua pasangan terurut $(a,b)$, tentukan nilai $|S|$.

    % paket soal osp : 10
    \item Diberikan persamaan $x^3-x+1=0$ yang mempunyai akar-akar $a,b,$ dan $c$. Carilah nilai $a^8+b^8+c^8$.

    % paket soal osp : 5951
    \item Diberikan polinomial $p(x)=x^3-ax^2+bx-c$ mempunyai tiga akar bulat positif berbeda dan $p(2002)=2001$. Misalkan $q(x)=x^2-2x+2002$. Diketahui pula bahwa $p(q(x))$ tidak mempunyai akar real. Tentukan nilai $a$.
\end{enumerate}
\begin{enumerate}[resume]
    %% paket soal osp
    \item Diberikan segitiga $ABC$ lancip. Garis tinggi terpanjang adalah dari titik sudut $A$ tegak lurus pada $BC$, dan panjangnya sama dengan panjang median (garis berat) dari titik sudut $B$. Nilai terbesar $\angle ABC$ adalah $\dots^\circ$
    \\
    %\textbf{Answer: 60}
	
	\item Pada segitiga $ABC$ terdapat titik $P$ di dalamnya sehingga $\angle PAB = 10^\circ, \angle PBA = 20^\circ , \angle PAC = 40^\circ, \angle PCA = 30^\circ$. Besar sudut $\angle ABC$ adalah $\dots^\circ$ 
    \\
    %\textbf{Answer: 80}
	
	\item Pada sembarang segitiga $ABC$, titik $D,E,F$ berturut-turut pada $BC,CA,AB$ dimana $AD,BE,CF$ bertemu di $M$. Nilai eksak dari $\dfrac{AM}{AD}+\dfrac{BM}{BE}+\dfrac{CM}{CF}$ adalah $\dots$
    \\
    %\textbf{Answer: 2}
	
	\item Misalkan $ABCD$ adalah segiempat konveks dengan $\angle DAC=\angle BDC = 36^\circ$, $\angle CBD = 18^\circ$, dan $\angle BAC = 72^\circ$. Diagonal $AC$ dan $BD$ berpotongan di titik $P$. Tentukan besar sudut $\angle APD$ dalam derajat. %ko ss spring camp 30 maret 2019 
    \\
    %\textbf{Answer: 100}
	
\end{enumerate}

\end{document}
\end{CJK*}