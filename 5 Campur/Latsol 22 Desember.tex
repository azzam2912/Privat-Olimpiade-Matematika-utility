\documentclass[11pt]{scrartcl}
\usepackage[sexy]{evan}
\usepackage[normalem]{ulem}

\renewcommand{\baselinestretch}{1.5}

% kunci jawaban: Simulasi KSN-K 8 -9

\begin{document}
	\title{Latihan Soal Medium - High} % Beginner
	\date{\today}
	\author{Compiled by Azzam}
	\maketitle
	\section{Kemampuan Dasar}
	Pada bagian ini setiap jawaban yang benar bernilai 2 poin dan setiap jawaban yang salah
	atau kosong bernilai nol.
	\begin{enumerate}
		\item
		Tentukan digit ke 2019 dari kiri bilangan 1223334444555556666667777777....
		
		\item
		Berapa banyak cara memilih bilangan bulat berbeda $x, y$ dengan $0 \le x, y \le 19$ dan 5 habis membagi $x + y$?
		
		\item
		Wadah pertama berisi 5 kelereng merah dan 4 kelereng biru. Wadah kedua berisi 7 kelereng merah dan
		9 kelereng biru. Sebuah kelereng dipindahkan dari wadah pertama ke wadah kedua. Selanjutnya, sebuah
		kelereng diambil dari wadah kedua, jika peluang yang terambil bola merah adalah
		$\frac{a}{b}$. Tentukan nilai dari $a+b$ dimana $a, b$ bilangan bulat positif yang saling prima.
		
		\item
		Diberikan sebuah trapesium siku-siku $ABCD$ dimana $BC$ tegak lurus $CD$ dengan $AB$ sejajar $CD$ dan $AB < CD$. jika panjang $AB = 1$, $BD = \sqrt{7}$, dan $AD = CD$, Jika luas trapesium tersebut adalah $S$, maka nilai $4S^2$ adalah \dots
		
		\item Nilai dari $\left(0,2 ^{\left(0,5 ^{\left(0,8 ^{...} \right)} \right)} \right)^{-1}$ adalah $\dots$ (perhatikan bahwa pangkatnya membentuk barisan aritmatika dengan beda 0,3)
		
		\item Banyak bilangan asli $n$ yang tidak lebih besar dari 2019 sehingga $3^n+4^n$ habis dibagi 49 adalah $\dots$
		
		\item Segitiga $ABC$ memiliki luas 120 satuan. Titik $E$ dan $F$ dipilih di sisi $AC$ sehigga $AE=EF=FC$. Jika $D$ dan $G$ berturut-turut merupakan titik tengah $AB$ dan $EF$, luas segitiga $DFG$ adalah $\dots$
		
		\item
		Berapa banyak persegi panjang bukan persegi yang dibentuk dari petak satuan papan catur $16 \times 16$ dimana sisi-sisinya paralel dengan sisi papan catur?
		
	\end{enumerate}
\section{Kemampuan Lanjut}
Pada bagian ini setiap jawaban yang benar bernilai 4 poin, jawaban kosong bernilai nol
dan jawaban \textbf{salah} bernilai -1 (\textbf{minus satu})

\begin{enumerate}[resume]
	\item (Tobi Moektijono)
	Carilah digit terakhir dari $$\sum_{k=1}^{2018} \left \lfloor \sqrt{2k} \right \rfloor $$
	
	\item Diberikan segitiga $ABC$. Garis bagi sudut $A$ memotong $BC$ di titik $D$. Garis bagi
	dalam $\angle ADB$ memotong $AB$ di titik $E$. Jika $BE = 7, AE = 14$, dan $DE$ sejajar $AC$,
	tentukan nilai $AD^2$.
	
	\item (Tobi Moektijono)
	Carilah banyak solusi bulat nonnegatif $(a,b)$ yang memenuhi $$a^3+b^2+1=7ab.$$
	
	\item
	Berapa banyak bilangan bulat positif 8 digit yang memenuhi perkalian dari digit-digitnya adalah $2^{21}$
	?
	
	\item Bilangan $\dfrac{2019}{2^{2019}}$ merepresentasikan bilangan desimal. Tentukan digit ke-2017 di belakang koma dari bilangan tersebut.
	
	\item Banyaknya bilangan real $x$ sehingga $|x| \le 19$ dan $\left \lfloor x \right \rfloor\left \lceil x \right \rceil = x^2$ adalah $\dots$
	
	\item Diketahui segitiga $ABC$ dengan $AB=c$, $BC=a$, dan $AC=b$. Jika $\dfrac{b}{c-a} - \dfrac{a}{b+c} = 1$. Carilah sudut terbesar pada segitiga $ABC$ dalam derajat.
	
	\item
	Misalkan banyak subset tak kosong dari $\{1,2,3,\dots,2018\}$ yang tidak mengandung dua bilangan yang memiliki jumlah 2019 adalah $a^c-b$ dimana $a,b,c$ adalah bilangan asli dan nilai $a$ sekecil mungkin. Tentukan nilai $a+b+c$.

\end{enumerate}
	
	
	
\end{document}