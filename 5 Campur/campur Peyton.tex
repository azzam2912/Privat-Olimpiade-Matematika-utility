\documentclass{scrartcl}
\usepackage[hagavi]{azzam}

\usepackage{CJKutf8}
\begin{CJK*}{UTF8}{min}
\title{愛しいナターシャ}
\begin{document}
    \maketitle
    Abbreviations:
    \begin{itemize}
        \item JOM = Junior Olympiad of Malaysia
        \item JMO = Junior Math Olympiad
        \item SMO = Singapore Math Olympiad
        \item SMOJ = SMO Junior
        \item SMOO = SMO Open
    \end{itemize}
    \begin{enumerate}
        \item (JOM 2025) Let $ABC$ be a triangle with $AB<AC$ and with its incircle touching the sides $AB$ and $BC$ at $M$ and $J$ respectively. A point $D$ lies on the extension of $AB$ beyond $B$ such that $AD=AC$. Let $O$ be the midpoint of $CD$. Prove that the points $J$, $O$, $M$ are collinear.

        \item (SMOJ 2018) In $\vartriangle ABC, AB=AC=14 \sqrt2 , D$ is the midpoint of $CA$ and $E$ is the midpoint of $BD$. Suppose $\vartriangle CDE$ is similar to $\vartriangle ABC$. Find the length of $BD$.

        \item (Spain MO 2025) Let $ABC$ be an acute triangle with circumcenter $O$ and orthocenter $H$, satisfying $AB<AC$. The tangent line at $A$ to the circumcicle of $ABC$ intersects $BC$ in $T$. Let $X$ be the midpoint of $AH$. Prove that $\angle ATX=\angle OTB$.

        \item (Polish JMO 2025) In a rhombus $ABCD$, angle $\angle ABC=100^{\circ}$. Point $P$ lies on $CD$ such that $\angle PBC=20^{\circ}$. Line parallel to $AD$ passing trough $P$ intersects $AC$ at $Q$. Prove that $BP=AQ$.

        \item (SMOO 2024) In triangle $ABC$, $\angle B=90^\circ$, $AB>BC$, and $P$ is the point such that $BP=BC$ and $\angle APB=90^\circ$, where $P$ and $C$ lie on the same side of $AB$. Let $Q$ be the point on $AB$ such that $AP=AQ$, and let $M$ be the midpoint of $QC$. Prove that the line through $M$ parallel to $AP$ passes through the midpoint of $AB$.

        \item (Polish JMO 2018) Let $ABCD$ be a trapezium with bases $AB$ and $CD$ in which $AB + CD = AD$. Diagonals $AC$ and $BD$ intersect in point $E$. Line passing through point $E$ and parallel to bases of trapezium cuts $AD$ in point $F$. Prove that $\angle BFC = 90 ^{\circ}$.

        \item (Japan MO 2025 Prelim) 円に内接する四角形 $ABCD$ が半径 $6$ の円に外接している. また, 半直線 $AB$ と半直線 $DC$ が点 $P$ で交わり, 半直線 $AD$ と半直線 $BC$ が点 $Q$ で交わっている. 三角形 $PBC, QCD$ の内接円の半径がそれぞれ $5, 3$ であるとき, $\frac{BC}{CD}$ の値を求めよ.\\

        Quadrilateral $ABCD$ is inscribed in a circle and circumscribed about a circle with radius $6$. Furthermore, rays $AB$ and $DC$ intersect at point $P$, and rays $AD$ and $BC$ intersect at point $Q$. If the inradii of triangles $PBC$ and $QCD$ are $5$ and $3$ respectively, find the value of $\frac{BC}{CD}$.

        \begin{figure}[H]
            \centering
            \includegraphics[width=0.5\linewidth]{0Figure/jmo-2025-prelim-5.png}
        \end{figure}
    \end{enumerate}
\end{document}
\end{CJK*}