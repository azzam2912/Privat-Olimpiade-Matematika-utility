\documentclass[12pt]{article}
\usepackage{azzam}

\begin{document}
\title{Semifinal Lolos Semua Inimah}
\maketitle
\begin{enumerate}
    \item Diberikan lingkaran $\Gamma(O)$ yang berpusat di $O$ dan $P$ titik di luar $\Gamma(O)$. $A$ dan $B$ titik di $\Gamma(O)$ sedemikian sehingga $PA$ dan $PB$ adalah garis singgung $\Gamma(O)$. Garis $l$ melalui $P$ memotong $\Gamma(O)$ berturut-turut di titik $C$ dan $D$ ($C$ terletak di antara $P$ dan $D$). Garis $BF$ sejajar garis $PA$ dan memotong garis $AC$ dan garis $AD$ berturut-turut di $E$ dan $F$. Buktikan bahwa $BE = BF$.

    \item Terdapat $k$ siswa dalam suatu kelas. Masing-masing siswa menghitung banyaknya siswa lain yang memiliki makanan favorit yang sama dengannya dan banyaknya siswa lain yang memiliki minuman favorit yang sama dengannya (Diasumsikan masing-masing anak hanya memiliki 1 makanan favorit dan 1 minuman favorit). Hasilnya kemudian dituliskan di papan tulis (Masing-masing anak menuliskan 2 buah bilangan). Diketahui ternyata bilangan $0, 1, 2, \dots , 7$ ada pada papan tulis tersebut. Tentukan nilai $k$ terkecil sehingga selalu dapat dipastikan bahwa terdapat dua siswa dengan makanan dan minuman favorit yang sama.

    \item Tentukan semua bilangan asli $n$ yang memenuhi dua bilangan 
    \[ n^2+4 \quad \text{dan} \quad n^4+2n^2+3 \]
    merupakan bilangan kubik!
\end{enumerate}

\end{document}