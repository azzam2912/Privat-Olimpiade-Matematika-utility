\documentclass[11pt]{scrartcl}
\usepackage{graphicx}
\graphicspath{{./}}
\usepackage[sexy]{evan}
\usepackage[normalem]{ulem}
\usepackage{hyperref}
\usepackage{mathtools}
\hypersetup{
    colorlinks=true,
    linkcolor=blue,
    filecolor=magenta,      
    urlcolor=cyan,
    pdfpagemode=FullScreen,
    }

\renewcommand{\dangle}{\measuredangle}

\renewcommand{\baselinestretch}{1.5}

\addtolength{\oddsidemargin}{-0.4in}
\addtolength{\evensidemargin}{-0.4in}
\addtolength{\textwidth}{0.8in}
% \addtolength{\topmargin}{-0.2in}
% \addtolength{\textheight}{1in} 


\setlength{\parindent}{0pt}

\usepackage{pgfplots}
\pgfplotsset{compat=1.15}
\usepackage{mathrsfs}
\usetikzlibrary{arrows}


\title{POP: Power Of a Point - Soal Latihan}
\author{Azzam (IG: haxuv.world)}
\date{Rabu, 24 Januari 2024}

\begin{document}
\maketitle
\begin{enumerate}
    \item (OSK 2011,2012,2013,2018) Diberikan segitiga $ABC$ dan lingkaran $\Gamma$ yang berdiameter $AB$. Lingkaran $\Gamma$ memotong sisi $AC$ dan $BC$ berturut-turut di titik $D$ dan $E$. Jika $AD = \frac13 AC, BE =\frac14 BC$ dan $AB = 30$, maka luas segitiga $ABC$ adalah \dots

    \item (OSP 2018) Misalkan $\Gamma_1$ dan $\Gamma_2$ lingkaran berbeda dengan panjang jari-jari sama dan berturut-turut berpusat di titik $O_1$ dan $O_2$. Lingkaran $\Gamma_1$ dan $\Gamma_2$ bersinggungan di titik $P$. Garis $\ell$ melalui $O_1$ menyinggung $\Gamma_2$ di titik $A$. Garis $\ell$ memotong $\Gamma_1$ di titik $X$ dengan $X$ di antara $A$ dan $O_1$. Misalkan $M$ titik tengah $AX$ dan $Y$ titik potong $PM$ dengan $\Gamma_2$ dengan $Y \neq P$. Buktikan bahwa $XY$ sejajar $O_1O_2$.
    
    \item (OSN 2019) Diberikan lingkaran dengan pusat $O$. Titik $A$ di dalam lingkaran namun tidak pada keliling lingkaran. Titik $B$ merupakan refleksi $A$ terhadap $O$. Sebarang titik $P$ terletak pada keliling lingkaran. Garis yang tegak lurus $AP$ dan melewati $P$ memotong lingkaran di $Q$. Buktikan $AP \times BQ$ konstan selama $P$ bergerak di lingkaran.
    
    \item (OSN 2023) Diberikan segitiga $ABC$ dengan $\angle ACB = 90^\circ$. Misalkan $\omega$ lingkaran luar $ABC$. Garis singgung terhadap $\omega$ di titik $B$ dan $C$ bertemu di $P$. Misalkan $M$ titik tengah $PB$. Garis $CM$ memotong $\omega$ di $N$ dan garis $PN$ memotong $AB$ di $E$. Titik $D$ pada $CM$ sehingga $ED \parallel BM$. Buktikan bahwa lingkaran luar $CDE$ menyinggung $\omega$.

    \item Let $ABC$ be a triangle with $I_A$, $I_B$, and $I_C$ as excenters. Prove that triangle $I_AI_BI_C$ has orthocenter $I$ and that triangle $ABC$ is its orthic triangle.
    
    \item (The Pitot Theorem). Let $ABCD$ be a quadrilateral. If a circle can be inscribed† in it, prove that $AB + CD = BC + DA$.
    
    \item (USAMO 1990/5). An acute-angled triangle $ABC$ is given in the plane. The circle with diameter $AB$ intersects altitude $CC'$ and its extension at points $M$ and $N$, and the circle with diameter $AC$ intersects altitude $BB'$ and its extensions at $P$ and $Q$. Prove that the points $M$, $N$, $P$, $Q$ lie on a common circle.
    
    \item (BAMO 2012/4). Given a segment $AB$ in the plane, choose on it a point $M$ different from $A$ and $B$. Two equilateral triangles $AMC$ and $BMD$ in the plane are constructed on the same side of segment $AB$. The circumcircles of the two triangles intersect in point $M$ and another point $N$.
    
    \item (USAJMO 2012/1). Given a triangle $ABC$, let $P$ and $Q$ be points on segments $AB$ and $AC$, respectively, such that $AP = AQ$. Let $S$ and $R$ be distinct points on segment $BC$ such that $S$ lies between $B$ and $R$, $\angle BPS = \angle PRS$, and $\angle CQR = \angle QSR$. Prove that $P, Q, R, S$ are concyclic.
    \begin{itemize}[(a)]
            \item  Prove that $AD$ and $BC$ pass through point $N$.
            \item  Prove that no matter where one chooses the point $M$ along segment $AB$, all lines $MN$ will pass through some fixed point $K$ in the plane.
    \end{itemize}
    
    \item (IMO 2008/1). Let $H$ be the orthocenter of an acute-angled triangle $ABC$. The circle $\Gamma_A$ centered at the midpoint of $BC$ and passing through $H$ intersects the sideline $BC$ at points $A_1$ and $A_2$. Similarly, define the points $B_1$, $B_2$, $C_1$, and $C_2$. Prove that six points $A_1$, $A_2$, $B_1$, $B_2$, $C_1$, and $C_2$ are concyclic.
    
    \item (USAMO 1997/2). Let $ABC$ be a triangle. Take points $D$, $E$, $F$ on the perpendicular bisectors of $BC$, $CA$, $AB$ respectively. Show that the lines through $A$, $B$, $C$ perpendicular to $EF$, $FD$, $DE$ respectively are concurrent.
    
    \item (IMO 1995/1). Let $A, B, C, D$ be four distinct points on a line, in that order. The circles with diameters $AC$ and $BD$ intersect at $X$ and $Y$. The line $XY$ meets $BC$ at $Z$. Let $P$ be a point on the line $XY$ other than $Z$. The line $CP$ intersects the circle with diameter $AC$ at $C$ and $M$, and the line $BP$ intersects the circle with diameter $BD$ at $B$ and $N$.Prove that the lines $AM$, $DN$, $XY$ are concurrent.
    
    \item Let ${\cal C}_1$ and ${\cal C}_2$ be concentric circles, with ${\cal C}_2$ in the interior of ${\cal C}_1$. From a point $A$ on ${\cal C}_1$ one draws the tangent $AB$ to ${\cal C}_2$ ($B\in {\cal C}_2$). Let $C$ be the second point of intersection of $AB$ and ${\cal C}_1$, and let $D$ be the midpoint of $AB$. A line passing through $A$ intersects ${\cal C}_2$ at $E$ and $F$ in such a way that the perpendicular bisectors of $DE$ and $CF$ intersect at a point $M$ on $AB$. Find, with proof, the ratio $AM/MC$.
    
    \item (IMO 2000/1). Two circles $G1$ and $G2$ intersect at two points $M$ and $N$. Let $AB$ be the line tangent to these circles at $A$ and $B$, respectively, so that $M$ lies closer to $AB$ than $N$. Let $CD$ be the line parallel to $AB$ and passing through the point $M$, with $C$ on $G1$ and $D$ on $G2$. Lines $AC$ and $BD$ meet at $E$; lines $AN$ and $CD$ meet at $P$; lines $BN$ and $CD$ meet at $Q$. Show that $EP = EQ$.
    
    \item (Canada 1990/3). Let $ABCD$ be a cyclic quadrilateral whose diagonals meet at $P$. Let $W, X, Y , Z$ be the feet of $P$ onto $AB$, $BC$, $CD$, $DA$, respectively. Show that $WX + YZ = XY + WZ$.
    
    \item (IMO 2009/2). Let $ABC$ be a triangle with circumcenter $O$. The points $P$ and $Q$ are interior points of the sides $CA$ and $AB$, respectively. Let $K, L$, and $M$ be the midpoints of the segments $BP$, $CQ$, and $PQ$, respectively, and let $\Gamma$ be the circle passing through $K, L$, and $M$. Suppose that the line $PQ$ is tangent to the circle $\Gamma$. Prove that $OP = OQ$.
    
    \item Let $AD, BE, CF$ be the altitudes of a scalene triangle $ABC$ with circumcenter $O$. Prove that $(AOD), (BOE)$, and $(COF)$ intersect at point $X$ other than $O$.
    
    \item Let the incircle of triangle $ABC$ touch sides $BC$, $CA$, and $AB$ at $D$, $E$, and $F$, respectively. Let $\omega$, $\omega_1$, $\omega_2$, and $\omega_3$ denote the circumcircles of triangles $ABC$, $AEF$, $BDF$, and $CDE$ respectively. Let $\omega$ and $\omega_1$ intersect at $A$ and $P$, $\omega$ and $\omega_2$ intersect at $B$ and $Q$, $\omega$ and $\omega_3$ intersect at $C$ and $R$.
    \begin{itemize}[(a)]
        \item Prove that $\omega_1$, $\omega_2$, and $\omega_3$ intersect in a common point.
        \item Show that lines $PD$, $QE$, and $RF$ are concurrent.
    \end{itemize}
    
    \item (Iran TST 2011/1). In acute triangle $ABC$, $\angle B$ is greater than $\angle C$. Let $M$ be the midpoint of $BC$ and let $E$ and $F$ be the feet of the altitudes from $B$ and $C$, respectively. Let $K$ and $L$ be the midpoints of $ME$ and $MF$, respectively, and let $T$ be on line $KL$ such that $TA\parallel BC$. Prove that $TA = TM$.
\end{enumerate}

\end{document}
