\documentclass[11pt]{scrartcl}
\usepackage{graphicx}
\graphicspath{{./}}
\usepackage[sexy]{evan}
\usepackage[normalem]{ulem}
\usepackage{hyperref}
\usepackage{mathtools}
\hypersetup{
    colorlinks=true,
    linkcolor=blue,
    filecolor=magenta,      
    urlcolor=cyan,
    pdfpagemode=FullScreen,
    }

\renewcommand{\dangle}{\measuredangle}

\renewcommand{\baselinestretch}{1.5}

\addtolength{\oddsidemargin}{-0.4in}
\addtolength{\evensidemargin}{-0.4in}
\addtolength{\textwidth}{0.8in}
% \addtolength{\topmargin}{-0.2in}
% \addtolength{\textheight}{1in} 


\setlength{\parindent}{0pt}

\usepackage{pgfplots}
\pgfplotsset{compat=1.15}
\usepackage{mathrsfs}
\usetikzlibrary{arrows}

\title{POP: Power Of a Point - Soal Post Test}
\author{Azzam (IG: haxuv.world)}
\date{Rabu, 24 Januari 2024}

\begin{document}
\maketitle
\textbf{Aturan umum:}
\begin{itemize}
    \item Tulis \textbf{nama lengkap} dan \textbf{asal sekolah} di pojok kiri atas halaman pertama.
    \item \textbf{Soal bertipe esai}. Sertakan argumentasi atau cara mendapatkan jawaban yang ditanyakan di soal.
    \item Waktu standar untuk mengerjakan semua soal berikut adalah 90-120 menit.
    \item Setiap soal bernilai bilangan bulat antara 0 sampai 10 (inklusif).
    \item Soal yang wajib dikerjakan untuk mendapatkan \textit{full points} adalah \textbf{dua soal}.
    \item Silakan mengerjakan dengan cara \textbf{apa saja}, tidak harus mengikuti materi yang diajarkan hari ini.
\end{itemize}


\section{Soal}
\begin{enumerate}
    \item Pada segitiga $ABC$, misalkan $D$ adalah titik tengah $BC$, dan $BE$, $CF$ adalah garis tinggi. Buktikan bahwa $DE$ dan $DF$ keduanya adalah garis singgung lingkaran luar $\triangle AEF$. %(OSN SL 2010) 

    %AIME 2016 I and AIME 2016 I
    \item Misalkan $\triangle ABC$ adalah segitiga lancip dengan lingkaran $\omega,$ dan misalkan $H$ adalah titik potong dari garis tinggi $\triangle ABC.$ Garis singgung lingkaran luar $\triangle HBC$ di $H$ memotong $\omega$ pada titik $X$ dan $Y$ dengan $HA=3,HX=2,$ dan $HY=6.$ Carilah luas dari $\triangle ABC$.

    \item Lingkaran $\omega_1$ dan $\omega_2$ bertemu di titik $X$ dan $Y$. Garis $\ell$ menyinggung lingkaran $\omega_1$ dan $\omega_2$ di $A$ dan $B$, berturut-turut, dengan garis $AB$ lebih dekat ke titik $X$ daripada $Y$. Lingkaran $\omega$ yang melewati $A$ dan $B$, memotong $\omega_1$ lagi di $D \neq A$ dan memotong $\omega_2$ lagi di $C \neq B$. Ketiga titik $C$, $Y$, $D$ segaris dengan $XC = 67$, $XY = 47$, dan $XD = 37$. Carilah panjang $AB$.
    
\end{enumerate}

\end{document}
