\documentclass[11pt]{scrartcl}
\usepackage[sexy]{evan}
\usepackage[normalem]{ulem}

\renewcommand{\baselinestretch}{1.5}



\begin{document}
	\title{Geometry from OSK-OSP} % Beginner
	\date{\today}
	\author{Compiled by Azzam}
	\maketitle
	Semua soal tidak terurut berdasarkan kesulitan.
	Tetapi diurutkan berdasarkan tahun.
	Gudlak :)
	\section{Isian Singkat}
	
	\begin{soalbaru}(\textbf{Soal Legend: OSK 2011,2012,2013,2018}) Diberikan segitiga $ABC$ dan lingkaran $\Gamma$ yang berdiameter $AB$ . Lingkaran $\Gamma$ memotong sisi $AC$ dan $BC$
berturut-turut di titik $D$ dan $E$. Jika $AD = \frac13 AC, BE =\frac14 BC$ dan $AB = 30$, maka luas segitiga $ABC$ adalah \dots
	
	\end{soalbaru}
	
	\begin{soalbaru}
	(OSK 2013) Misalkan $P$ adalah titik interior dalam daerah segitiga $ABC$ sehingga besar $\angle PAB = 10^\circ, \angle PBA = 20^\circ, \angle PCA = 30^\circ, \angle PAC = 40^\circ$. Besar $\angle ABC = \dots$
	\end{soalbaru}
	
	\begin{soalbaru}
	(OSK 2013) Diberikan segitiga lancip $ABC$ dengan $O$ sebagai pusat lingkaran luarnya. Misalkan $M$ dan $N$
berturut - turut pertengahan $OA$ dan $BC$. Jika $\angle ABC = 4\angle OMN$ dan $\angle ACB = 6\angle OMN$,
maka besarnya $\angle OMN$ sama dengan \dots
	\end{soalbaru}
	\begin{soalbaru}(OSK 2015)
	Tiga titik berbeda $B, C,$ dan $D$ terletak segaris dengan $C$ diantara $B$ dan $D$. Titik $A$ adalah suatu
titik yang tidak terletak di garis $BD$ dan memenuhi $|AB| = |AC| = |CD|$. Jika diketahui
$$\dfrac{1}{|CD|}-\dfrac{1}{|BD|}=\dfrac{1}{|CD|+|BD|}$$
maka besar sudut $\angle BAC$ adalah \dots
	\end{soalbaru}
	
	\begin{soalbaru}
	(OSK 2016) Segitiga $ABC$ mempunyai panjang sisi $AB = 20, AC = 21$ dan $ BC = 29$. Titik $D$ dan $ E$
terletak pada segmen garis $BC$, dengan $BD = 8$ dan $EC = 9$. Besar $\angle DAE$ adalah \dots
derajat.
	\end{soalbaru}
	
	\begin{soalbaru}(OSK 2016)
	Diberikan empat titik pada satu lingkaran $\Gamma$ dalam urutan $A, B, C, D$. Sinar garis $AB$ dan
$DC$ berpotongan di $E$, dan sinar garis $AD$ dan $BC$ berpotongan di $F$. Misalkan $EP$ dan
$FQ$ menyinggung lingkaran $\Gamma$ berturut-turut di $P$ dan $Q$. Misalkan pula bahwa $EP = 60$
dan $FQ = 63$, maka panjang $EF$ adalah \dots
	\end{soalbaru}
	
	\begin{soalbaru}
	(OSK 2017) Pada segitiga $ABC$ titik $K$ dan $L$ berturut-turut adalah titik tengah $AB$ dan $AC$. Jika $CK$ dan $BL$
saling tegak lurus, maka nilai minimum dari $\cot B + \cot C$ adalah \dots
	\end{soalbaru}
	
	\begin{soalbaru}
	(OSK 2017) Diberikan segitiga $ABC$ dengan $AB = 12, BC = 5$ dan $AC = 13$. Misalkan $P$ suatu titik pada garis
bagi $\angle A$ yang terletak di dalam $ABC$ dan misalkan $M$ suatu titik pada sisi $AB$ (dengan $A\neq  M \neq
B$). Garis $AP$ dan $MP$ memotong $BC$ dan $AC$ berturut-turut di $D$ dan $N$. Jika $\angle MPB = \angle PCN$
dan $\angle NPC = \angle MBP$, maka nilai $\dfrac{AP}{PD}$
adalah \dots
	\end{soalbaru}
	
	\begin{soalbaru}
	(OSK 2018)
	Panjang sisi-sisi dari segitiga merupakan bilangan asli yang berurutan. Diketahui bahwa garis berat dari
segitiga tegak lurus dengan salah satu garis baginya. Keliling segitiga itu adalah \dots
	\end{soalbaru}
	
	\begin{soalbaru}
	(OSK 2018) Pada segitiga $ABC$, panjang sisi $BC$ adalah 1 satuan. Ada tepat satu titik $D$ pada sisi $BC$ yang memenuhi
$|DA|^2 = |DB| \cdot |DC|$. Jika $k$ menyatakan keliling $ABC$, jumlah semua nilai $k$ yang mungkin adalah \dots
	\end{soalbaru}
	
	\section{Esai}
	Warning: Diambil yang paling sulit :)
	\begin{soalbaru}
	(OSP 2012) Diberikan segitiga lancip $ABC$. Titik $H$ menyatakan titik kaki dari garis tinggi yang ditarik dari $A$. Buktikan bahwa
	$$AB+AC \ge BC \cos \angle BAC + 2AH \sin \angle BAC$$
	\end{soalbaru}
	\begin{soalbaru}
	(OSP 2016) Misalkan $PA$ dan $PB$ adalah garis singgung lingkaran $\omega$ dari suatu titik $P$ di luar lingkaran.
Misalkan $M$ adalah sebarang titik pada $AP$ dan $N$ adalah titik tengah $AB$. Perpanjangan
$MN$ memotong $\omega$ di $C$ dengan $N$ di antara $M$ dan $C$. Misalkan $PC$ memotong $\omega$ di $D$ dan
perpanjangan $ND$ memotong $PB$ di $Q$.
Tunjukkan bahwa $MQ$ sejajar dengan $AB$.
	\end{soalbaru}
	\begin{soalbaru} (OSP 2019)
	Diberikan segitiga $ABC$, dengan $AC > BC$ , dan lingkaran luarnya yang berpusat di $O$. Misalkan $M$ adalah titik pada lingkaran luar segitiga $ABC$ sehingga $CM$ adalah garis bagi $\angle ACB$. Misalkan $\Gamma$ adalah lingkaran berdiameter $CM$. Garis bagi $\angle BOC$ dan garis bagi $\angle AOC$ memotong $\Gamma$ berturut-turut di $P$ dan $Q$. Jika $K$ adalah titik tengah $CM$, buktikan bahwa $P,Q,O,K$ terletak pada satu lingkaran.
	\end{soalbaru}

\end{document}


