\documentclass[11pt]{scrartcl}
\usepackage[sexy]{evan}
\usepackage[normalem]{ulem}

\renewcommand{\baselinestretch}{1.5}



\begin{document}
	\title{Geometry Bunch 1} % Beginner
	\date{\today}
	\author{Compiled by Azzam}
	\maketitle
	
	Recommended allocation time: 90 minutes total.
	
	\begin{soalbaru}
	(JMO Preliminary 2012) Given a square $ABCD$. Let $P\in{AB},\ Q\in{BC},\ R\in{CD}\ S\in{DA}$ and $PR\Vert BC,\ SQ\Vert AB$ and let $Z=PR\cap SQ$. If $BP=7,\ BQ=6,\ DZ=5$, then find the side length of the square.
	%https://artofproblemsolving.com/community/c4h458141p2571968
	\end{soalbaru}
	
    \begin{soalbaru}
    (JMO Preliminary 2012)  Given $A,\ B,\ C,\ D$ on a circle in this order and suppose the angle of the tangent at $B$ and the line $AB$ is $30^\circ$, and that of the tangent at $C$ and the line $CD$ is $10^\circ$. If $AB\Vert DC$ and they are located across the center of the circle, then find $\angle{BDC}$.
    %https://artofproblemsolving.com/community/c4h458149p2572017
    \end{soalbaru}
    
    \begin{soalbaru}
    (JMO Preliminary 2012) Given a triangle $ABC$ with circumcenter $O$. Let $D\in{AB},\ E\in{AC}$ such that $AD=8,\ BD=3$ and $AO=7$. If $O$ is the midpoint of $DE$, then find $CE$.
    %https://artofproblemsolving.com/community/c1090h1026164
    \end{soalbaru}
    
    \begin{soalbaru}
    (JMO Preliminary 2014) The circle $C_1$ is tangent to the circle $C_2$ internally at the point $A$. Let $O$ be the center of $C_2$. The point $P$ is on $C_1$ and the tangent line at $P$ passes through $O$. Let the ray $OP$ intersects with $C_2$ at $Q$, and let the tangent line of $C_1$ passing through $A$ intersects with the line $OP$ at $R$. If the radius of $C_2$ is 9 and $PQ=QR$, then find the length of the segment $OP$. %https://artofproblemsolving.com/community/c6h571137p3355142
    \end{soalbaru}
    
    \begin{soalbaru}
    (JMO Preliminary 2014) 6 points $A,\ B,\ C,\ D,\ E,\ F$ are on a circle in this order and the segments $AD,\ BE,\ CF$ intersect at one point. If $AB=1,\ BC=2,\ CD=3,\ DE=4,\ EF=5$, then find the length of the segment $FA$.
    %https://artofproblemsolving.com/community/c6h571046p3354460
    \end{soalbaru}
    
    \begin{soalbaru}
    (JMO Preliminary 2014) Let $ABCD$ be a square with the point of intersection $O$ of the diagonals and let $P,\ Q,\ R,\ S$ be the points which are on the segments $OA,\ OB,\ OC,\ OD$, respectively such that $OP=3,\ OQ=5,\ OR=4$. If the points of intersection of the lines $AB$ and $PQ$, the lines $BC$ and $QR$, the lines $CD$ and $RS$ are collinear, then find the length of the segment $OS$.
    \end{soalbaru}
    
    \begin{soalbaru}
    (JMO Preliminary 2016) A hexagon $ABCDEF$ is inscribed in a circle. Let $P, Q, R, S$ be intersections of $AB$ and $DC$, $BC$ and $ED$, $CD$ and $FE$, $DE$ and $AF$, then $\angle BPC=50^{\circ}$, $\angle CQD=45^{\circ}$, $\angle DRE=40^{\circ}$, $\angle ESF=35^{\circ}$. Let $T$ be an intersection of $BE$ and $CF$. Find $\angle BTC$.
    %https://artofproblemsolving.com/community/c6h1195500p5852493
    \end{soalbaru}
    
    \begin{soalbaru}
    (JMO Preliminary 2016) Let $ABCD$ be a quadrilateral with $AC=20$, $AD=16$. We take point $P$ on segment $CD$ so that triangle $ABP$ and $ACD$ are congruent. If the area of triangle $APD$ is $28$, find the area of triangle $BCP$. Note that $XY$ expresses the length of segment $XY$.
    %https://artofproblemsolving.com/community/c6h1195502p5852496
    \end{soalbaru}
    
    \begin{soalbaru}
    (JMO Preliminary 2016) Let $\omega$ be an incircle of triangle $ABC$. Let $D$ be a point on segment $BC$, which is tangent to $\omega$. Let $X$ be an intersection of $AD$ and $\omega$ against $D$. If $AX : XD : BC = 1 : 3 : 10$, a radius of $\omega$ is $1$, find the length of segment $XD$. Note that $YZ$ expresses the length of segment $YZ$.
    %https://artofproblemsolving.com/community/c6h1195507p5852514
    \end{soalbaru}
    
    %below problem
    %https://www.hkage.org.hk/file/competitions/4942/Prelim2019.pdf
    %https://www.hkage.org.hk/file/competitions/5615/Prelim2019_sol.pdf
    \begin{soalbaru}
    (HKIMO Preliminary 2019) In $\triangle ABC$, $\angle ABC = 90^\circ$. $D$ is the midpoint of $BC$, while $E$ is a point on $AC$ such that $DE$ bisects $\angle ADC$. If $DE=DC=1$, find the length of $AD$. 
    \end{soalbaru}
    
    \begin{soalbaru}
    (HKIMO Preliminary 2019) Let $ABC$ be a triangle. $D$ and $E$ are points on $AB$ such that $AD:DE:EB=2:3:2$. $F$ and $G$ are points on $BC$ such that $BF:FG:GC=1:3:1$. $H$ and $K$ are points on $AC$ such that $AH:HK:KC=2:3:1$. The line $DG$ meets $FH$ and $EK$ at $P$ and $Q$ respectively. Find $DP:QG$.
    \end{soalbaru}
    
    \begin{soalbaru}
    (HKIMO Preliminary 2019) $A, B, C$ are three points on a circle while $P$ and $Q$ are two points on $AB$. The extensions of $CP$ and $CQ$ meet the circle at $S$ and $T$ respectively. If $AP = 2, AQ = 7, AB = 11, AS = 5$ and $BT = 2$, find the length of $ST$.
    \end{soalbaru}
	
\end{document}


