\documentclass[11pt]{scrartcl}
\usepackage{graphicx}
\graphicspath{{./}}
\usepackage[sexy]{evan}
\usepackage[normalem]{ulem}
\usepackage{hyperref}
\usepackage{mathtools}
\hypersetup{
    colorlinks=true,
    linkcolor=blue,
    filecolor=magenta,      
    urlcolor=cyan,
    
    pdfpagemode=FullScreen,
    }

\renewcommand{\dangle}{\measuredangle}

\renewcommand{\baselinestretch}{1.5}

\addtolength{\oddsidemargin}{-0.4in}
\addtolength{\evensidemargin}{-0.4in}
\addtolength{\textwidth}{0.8in}
% \addtolength{\topmargin}{-0.2in}
% \addtolength{\textheight}{1in} 


\setlength{\parindent}{0pt}

\usepackage{pgfplots}
\pgfplotsset{compat=1.15}
\usepackage{mathrsfs}
\usetikzlibrary{arrows}

\title{Ortik dan Ex - Soal Latihan}
\author{Azzam (IG: haxuv.world)}
\date{Jumat, 19 Januari 2024}

\begin{document}
\maketitle
\section{Ortik}
\begin{enumerate}
\item Diberikan segitiga lancip $ABC$ dengan $AB>AC$. $BE$ dan $CF$ adalah garis tinggi $\triangle ABC$. Perpanjangan $EF$ memotong $BC$ di $P$. Jika $H$ adalah tiitk tinggi $\triangle ABC$ dan $D$ titik tengah $BC$, buktikan bahwa $PH \perp AD$.

\item $\triangle ABC$ memilliki titik tinggi $H$ dan titik pusat lingkaran dalam $I$. Buktikan bahwa $A,B,H,I$ konsiklis $\iff$ $\angle ACB = 60^\circ$.

\item For any triangle $ABC$, if $H$ denotes its orthocenter, $r$ as its inradius, and $O$ as its circumcenter, prove that 
 \begin{align*}
     2AO = AH + 2r \iff AC+AB=2BC
 \end{align*}

\item Diberikan segitiga lancip $ABC$ dengan $\angle A = 60^\circ$. Misalkan $H$ dan $I$ adalah berturut-turut adalah titik tinggi dan titik pusat lingkaran dalam segitiga $ABC$. Buktikan bahwa $AH \ge AI$ dengan kesamaan terjadi jika dan hanya jika $ABC$ sama sisi.

\item (OSP 2017) Diberikan segitiga $ABC$ yang ketiga garis tingginya berpotongan di titik $H$. Tentukan semua titik $X$ pada sisi $BC$ sehingga pencerminan $H$ terhadap titik $X$ terletak pada lingkaran luar segitiga $ABC$.

\item (OSN 2010) Diberikan segitiga lancip $ABC$ dengan titik pusat lingkaran luar $O$ dan titik tinggi $H$. Misalkan $K$ sebarang titik di dalam segitiga $ABC$ yang tidak sama dengan $O$ maupun $H$. Titik $L$ dan $M$ terletak di luar segitiga $ABC$ sedemikian sehingga $AKCL$ dan $AKBM$ jajaran genjang. Terakhir, misalkan $BL$ dan $CM$ berpotongan di titik $N$ dan misalkan juga $J$ adalah titik tengah $HK$. Buktikan bahwa $KONJ$ jajaran genjang.

\item (IMO SL 2010) Let $\triangle ABC$ be an acute triangle with $D$, $E$, $F$ the feet of the altitudes lying on $\overline{BC}$, $\overline{CA}$, and $\overline{AB}$ respectively. One of the intersection points of the line $\overline{EF}$ and the circumcircle is $P$. The lines $\overline{BP}$ and $\overline{DF}$ meet at point $Q$. Prove that $|AP| = |AQ|$.

%\item (Korea 2022) Let $ABC$ be an acute triangle with circumcenter $O$, and let $D$, $E$, and $F$ be the feet of altitudes from $A$, $B$, and $C$ to sides $BC$, $CA$, and $AB$, respectively. Denote by $P$ the intersection of the tangents to the circumcircle of $ABC$ at $B$ and $C$. The line through $P$ perpendicular to $EF$ meets $AD$ at $Q$, and let $R$ be the foot of the perpendicular from $A$ to $EF$. Prove that $DR$ and $OQ$ are parallel.

\item (USAMO 2015) Quadrilateral $APBQ$ is inscribed in circle $\omega$ with $\angle P = \angle Q = 90^{\circ}$ and $AP = AQ < BP$. Let $X$ be a variable point on segment $\overline{PQ}$. Line $AX$ meets $\omega$ again at $S$ (other than $A$). Point $T$ lies on arc $AQB$ of $\omega$ such that $\overline{XT}$ is perpendicular to $\overline{AX}$. Let $M$ denote the midpoint of chord $\overline{ST}$. As $X$ varies on segment $\overline{PQ}$, show that $M$ moves along a circle.

\item (USA TST for IMO 2016) \textbf{Extremely Hard} Let $ABC$ be an acute scalene triangle and let $P$ be a point in its interior. Let $A_1$, $B_1$, $C_1$ be projections of $P$ onto triangle sides $BC$, $CA$, $AB$, respectively. Find the locus of points $P$ such that $AA_1$, $BB_1$, $CC_1$ are concurrent and $\angle PAB + \angle PBC + \angle PCA = 90^{\circ}$.
 \end{enumerate}
 
\section{Ex}
\begin{enumerate}[resume]
\item (IMO 2006) Let $ABC$ be triangle with incenter $I$. A point $P$ in the interior of the triangle satisfies
\begin{align*}
    \angle PBA + \angle PCA = \angle PBC + \angle PCB.
\end{align*}
Show that $AP \ge AI$ and that equality holds if and only if $P=I$.

\item (OSN 2018) Misalkan $I$ dan $O$ masing-masing menyatakan titik pusat lingkaran dalam dan lingkaran luar segitiga $ABC$. Lingkaran singgung luar $\omega_A$ dari segitiga $ABC$ menyinggung sisi $BC$ di $N$ serta menyinggung perpanjangan sisi $AB$ dan $AC$ masing-masing di $K$ dan $M$. Jika titik tengah dari ruas garis $KM$ berada pada lingkaran luar segitiga $ABC$, buktikan bahwa $O,I$, dan $N$ segaris.

% \item (All Russian 2022 Grade 9) Given triangle $ABC$ with incenter $I$ and $A$-excenter $J$. Circle $\omega_b$ centered at point $O_b$ passes through point $B$ and is tangent to line $CI$ at point $I$. Circle $\omega_c$ with center $O_c$ passes through point $C$ and touches line $BI$ at point $I$. Let $O_bO_c$ and $IJ$ intersect at point $K$. Find the ratio $IK/KJ$.

\item (CGMO 2022) In triangle $ABC,AB>AC,I$ is the incenter, $AM$ is the midline. The line crosses $I$ and is perpendicular to $BC $ intersect $AM$ at point $L$, and the symmetry of $I$ with respect to point $A$ is $J$
Prove that: $\angle ABJ= \angle LBI$.

\item (Philippine 2022) In $\triangle ABC$, let $D$ be the point on side $BC$ such that $AB+BD=DC+CA.$ The line $AD$ intersects the circumcircle of $\triangle ABC$ again at point $X \neq A$. Prove that one of the common tangents of the circumcircles of $\triangle BDX$ and $\triangle CDX$ is parallel to $BC$.

\item 	(Russia 2014) Let $ABC$ be a triangle with $AB>BC$ and circumcircle $\Omega$. Points $M$, $N$ lie on the sides $AB$, $BC$ respectively, such that $AM=CN$. Lines $MN$ and $AC$ meet at $K$. Let $P$ be the incenter of the triangle $AMK$, and let $Q$ be the $K$-excenter of the triangle $CNK$. If $R$ is midpoint of arc $ABC$ of $\Omega$ then prove that $RP=RQ$.

\item Let $ABC$ be a triangle with circumcircle $\Omega$, and let $D$ be any point on $\ol{BC}$. We draw a \emph{curvilinear incircle} tangent to $\overline{AD}$ at $L$, to $\overline{BC}$ at $K$ and internally tangent to $\Omega$. Show that the incenter of triangle $ABC$ lies on $\overline{KL}$.

\item (USAMO 1988). Triangle $ABC$ has incenter $I$. Consider the triangle whose vertices are the circumcenters of $\triangle IAB$, $\triangle IBC$, $\triangle ICA$. Show that its circumcenter coincides with the circumcenter of $\triangle ABC$.

\item (CGMO 2012). The incircle of a triangle $ABC$ is tangent to sides $AB$ and $AC$ at $D$ and $E$ respectively, and $O$ is the circumcenter of triangle $BCI$. Prove that $\angle ODB = \angle OEC$.

\item (CHMMC Spring 2012). In triangle $ABC$, the angle bisector of $\angle A$ meets the perpendicular bisector of $BC$ at point $D$. The angle bisector of $\angle B$ meets the perpendicular bisector of $AC$ at point $E$. Let $F$ be the intersection of the perpendicular bisectors of $BC$ and $AC$. Find $DF$, given that $\angle ADF = 5^\circ$, $\angle BEF = 10^\circ$ and $AC = 3$.

\item (Nine-Point Circle). Let $ABC$ be an acute triangle with orthocenter $H$. Let $D$, $E$, $F$ be the feet of the altitudes from $A$, $B$, $C$ to the opposite sides. Show that the midpoint of $AH$ lies on the circumcircle of $\triangle DEF$.

\item (HMMT 2011). Let $ABCD$ be a cyclic quadrilateral, and suppose that $BC = CD = 2$. Let $I$ be the incenter of triangle $ABD$. If $AI = 2$ as well, find the minimum value of the length of diagonal $BD$.

\item (HMMT 2013). Let triangle $ABC$ satisfy $2BC = AB + AC$ and have incenter $I$ and circumcircle $\omega$. Let $D$ be the intersection of $AI$ and $\omega$ (with $A$, $D$ distinct). Prove that $I$ is the midpoint of $AD$.

\item (Online Math Open 2014/F19). In triangle $ABC$, $AB = 3$, $AC = 5$, and $BC = 7$. Let $E$ be the reflection of $A$ over $BC$, and let line $BE$ meet the circumcircle of $ABC$ again at $D$. Let $I$ be the incenter of $\triangle ABD$. Compute $\cos \angle AEI$.

%\item (NIMO 2012). Let $ABXC$ be a cyclic quadrilateral such that $\angle XAB = \angle XAC$. Let $I$ be the incenter of triangle $ABC$ and by $D$ the foot of $I$ on $BC$. Given $AI = 25$, $ID = 7$, and $BC = 14$, find $XI$.

\item Let $ABC$ be an acute triangle such that $\angle A = 60^\circ$. Prove that $IH = IO$, where $I$, $H$, $O$ are the incenter, orthocenter, and circumcenter.

\item (APMO 2007). In triangle $ABC$, we have $AB > AC$ and $\angle A = 60^\circ$. Let $I$ and $H$ denote the incenter and orthocenter of the triangle. Show that $2\angle AHI = 3\angle B$.

\item (ELMO 2013, Evan Chen). Triangle $ABC$ is inscribed in circle $\omega$. A circle with chord $BC$ intersects segments $AB$ and $AC$ again at $S$ and $R$, respectively. Segments $BR$ and $CS$ meet at $L$, and rays $LR$ and $LS$ intersect $\omega$ at $D$ and $E$, respectively. The internal angle bisector of $\angle BDE$ meets line $ER$ at $K$. Prove that if $BE = BR$, then $\angle ELK = \frac{1}{2}\angle BCD$.

\item (Online Math Open 2012/F27). Let $ABC$ be a triangle with circumcircle $\omega$. Let the bisector of $\angle ABC$ meet segment $AC$ at $D$ and circle $\omega$ at $M \neq B$. The circumcircle of $\triangle BDC$ meets line $AB$ at $E \neq B$, and $CE$ meets $\omega$ at $P \neq C$. The bisector of $\angle PMC$ meets segment $AC$ at $Q \neq C$. Given that $PQ = MC$, determine the degree measure of $\angle ABC$.
\end{enumerate}

\end{document}
