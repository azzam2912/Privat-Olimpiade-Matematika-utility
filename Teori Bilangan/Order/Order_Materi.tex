\documentclass[11pt]{scrartcl}
\usepackage{graphicx}
\graphicspath{{./}}
\usepackage[sexy]{evan}
\usepackage[normalem]{ulem}
\usepackage{hyperref}
\usepackage{mathtools}
\hypersetup{
    colorlinks=true,
    linkcolor=blue,
    filecolor=magenta,      
    urlcolor=cyan,
    
    pdfpagemode=FullScreen,
    }

\renewcommand{\dangle}{\measuredangle}

\renewcommand{\baselinestretch}{1.5}

\addtolength{\oddsidemargin}{-0.4in}
\addtolength{\evensidemargin}{-0.4in}
\addtolength{\textwidth}{0.8in}
% \addtolength{\topmargin}{-0.2in}
% \addtolength{\textheight}{1in} 


\setlength{\parindent}{0pt}

\usepackage{pgfplots}
\pgfplotsset{compat=1.15}
\usepackage{mathrsfs}
\usetikzlibrary{arrows}

\title{Order Modulo - Materi}
\author{Azzam (IG: haxuv.world)}
\date{Rabu, 17 Januari 2024}

\begin{document}

\maketitle

\section{Materi Singkat}
\subsection{Order}
\begin{enumerate}
    \item (Eksistensi Order) Berdasarkan Euler's Theorem, untuk setiap $a \in \ZZ$ dengan $gcd(a,m)=1$, terdapat bilangan asli $t$ sehingga $a^t \equiv 1 \mod m$.
    \item Jika $s$ adalah bilangan asli terkecil yang memenuhi $a^s \equiv 1 \mod m$ maka $s = ord_m a$ disebut sebagai order dari $a \mod m$.
    \item Jika $s = ord_m a$ dan $t \in \NN$ memenuhi $a^t \equiv 1 \mod m$ maka $s \mid t$.
\end{enumerate}

\subsection{Primitve Roots}
Biarkan $p$ menjadi bilangan prima. Maka ada bilangan bulat $g$, yang disebut \textbf{akar primitif (primitive roots),} sehingga urutan dari $g \mod p$ sama dengan $p - 1$.

Akibatnya:
$p \equiv 1 \pmod{4}$ is a prime if and only if there exists an $n$ such that $n^2 \equiv -1 \pmod{p}$.
a
\subsection{p-adic valuation (pengantar LTE)}
\begin{enumerate}
    \item $p$ bilangan prima. $x \in \ZZ$. Didefinisikan \textbf{p-adic valuation} dari $x$ sebagai pangkat tertinggi dari $p$ yang membagi $x$. Notasinya adalah $v_p(x)=a$ berarti $p^a \mid x$ tetapi $p^{a+1} \nmid x$.
    \item $v_p(xy) = v_p(x) + v_p(y)$.
    \item $v_p(x+y) \ge \min(v_p(x),v_p(y))$.
\end{enumerate}

\section{Contoh Soal}
\begin{enumerate}
    \item Prove that the order of $a \mod m$ (with $a$ and $m$ relatively prime) is less than or equal to $\phi(m)$.

    \item For relatively prime positive integers $a$ and $m$ prove that $a^n \equiv 1 \pmod{m}$ if and only if $\text{ord}_m a | n$.

    \item For relatively prime positive integers $a$ and $m$, $\text{ord}_m a | \phi(m)$.
    
    \item For positive integers $a > 1$ and $n$ find $\text{ord}_{a^n-1} (a)$.

    \item If $a$ and $b$ are positive integers relatively prime to $m$ with $a^x \equiv b^x \pmod{m}$ and $a^y \equiv b^y \pmod{m}$ prove that $a^{\text{gcd}(x,y)} \equiv b^{\text{gcd}(x,y)} \pmod{m}$.

    \item Prove that for $n > 1$ we have $n$ doesn't divide $2^{n-1} + 1$.

    \item Let $n$ be an integer with $n \geq 2$. Prove that $n$ doesn’t divide $2^n - 1$.

    \item Prove that if $p$ is prime, then every prime divisor of $2^p - 1$ is greater than $p$.
\end{enumerate}

\end{document}
