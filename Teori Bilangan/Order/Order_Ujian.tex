\documentclass[11pt]{scrartcl}
\usepackage{graphicx}
\graphicspath{{./}}
\usepackage[sexy]{evan}
\usepackage[normalem]{ulem}
\usepackage{hyperref}
\usepackage{mathtools}
\hypersetup{
    colorlinks=true,
    linkcolor=blue,
    filecolor=magenta,      
    urlcolor=cyan,
    
    pdfpagemode=FullScreen,
    }

\renewcommand{\dangle}{\measuredangle}

\renewcommand{\baselinestretch}{1.5}

\addtolength{\oddsidemargin}{-0.4in}
\addtolength{\evensidemargin}{-0.4in}
\addtolength{\textwidth}{0.8in}
% \addtolength{\topmargin}{-0.2in}
% \addtolength{\textheight}{1in} 


\setlength{\parindent}{0pt}

\usepackage{pgfplots}
\pgfplotsset{compat=1.15}
\usepackage{mathrsfs}
\usetikzlibrary{arrows}

\title{Order Modulo - Soal Post Test}
\author{Azzam (IG: haxuv.world)}
\date{Rabu, 17 Januari 2024}

\begin{document}
\maketitle
\textbf{Aturan umum:}
\begin{itemize}
    \item Tulis \textbf{nama lengkap} dan \textbf{asal sekolah} di pojok kiri atas halaman pertama.
    \item \textbf{Soal bertipe esai}. Sertakan argumentasi atau cara mendapatkan jawaban yang ditanyakan di soal.
    \item Waktu standar untuk mengerjakan semua soal berikut adalah 90-120 menit.
    \item Setiap soal bernilai bilangan bulat antara 0 sampai 10 (inklusif).
\end{itemize}


\section{Soal}
\begin{enumerate}
    \item Let $n$ be an integer with $n \geq 2$. Prove that $n$ doesn’t divide $2^n-1$.
    \item Misalkan $p$ adalah bilangan prima ganjil. Misalkan pula $q$ dan $r$ adalah bilangan prima sehingga $p \mid q^r+1$. Buktikan bahwa salah satu diantara pernyataan berikut terpenuhi: 
    \begin{align*}
        2r \mid p-1 \text{ atau } p \mid q^2-1.
    \end{align*}
    \item Find all pairs of primes $p$, $q$ such that $pq | 5^p + 5^q$.
\end{enumerate}

\end{document}
