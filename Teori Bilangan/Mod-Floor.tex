\documentclass[11pt]{scrartcl}
\usepackage{graphicx}
\graphicspath{{./}}
\usepackage[sexy]{evan}
\usepackage[normalem]{ulem}
\usepackage{hyperref}
\usepackage{mathtools}
\hypersetup{
    colorlinks=true,
    linkcolor=blue,
    filecolor=magenta,      
    urlcolor=cyan,
    pdfpagemode=FullScreen,
    }

\renewcommand{\dangle}{\measuredangle}

\renewcommand{\baselinestretch}{1.5}

\addtolength{\oddsidemargin}{-0.4in}
\addtolength{\evensidemargin}{-0.4in}
\addtolength{\textwidth}{0.8in}
% \addtolength{\topmargin}{-0.2in}
% \addtolength{\textheight}{1in} 


\setlength{\parindent}{0pt}

\usepackage{pgfplots}
\pgfplotsset{compat=1.15}
\usepackage{mathrsfs}
\usetikzlibrary{arrows}

\title{Modulo-Prima Dan Floor-Ceiling}
\author{Azzam Labib (IG: haxuv.world)}
\date{\today}
\begin{document}
\maketitle
% \renewcommand*\contentsname{Daftar Isi}
% \tableofcontents

% \newpage
% \section{Modulo}
% \subsection{Sedikit Materi}
%     \subsubsection{Definisi dan Properti Dasar}
%     Untuk suatu bilangan asli $m$ dan bilangan bulat $a,b,c$ dan $d$, notasikan $m\mid a-b \iff a \equiv b \mod m$ (dibaca $a$ kongruen $b$ modulo $m$). Simpelnya $a \equiv b \mod m$ adalah $a$ dibagi $m$ bersisa $b$. Contohnya $5 \equiv 2 \mod 3$. $13 \equiv 3 \mod 5$. $10 \equiv -2 \mod 12$.
%     \begin{enumerate}
%         \item $a \equiv a \mod m$.
%         \item $a \equiv 0 \mod m \iff m\mid a$.
%         \item $a \equiv b \mod m \iff b \equiv a \mod m$.
%         \item $a \equiv b \mod m \text{ dan } b \equiv c \mod m \implies a \equiv c \mod m$.
%         \item Jika $a \equiv b \mod m$ dan $d\mid m$ maka $a \equiv b \mod d$.
%         \item Untuk semua bilangan asli $k$, $a \equiv b \mod m \iff a^k \equiv b^k \mod m$.
%         \item $a \equiv b \mod m \text{ dan } c \equiv d \mod m \implies a+c \equiv b+d \mod m$.
%         \item $a \equiv b \mod m \text{ dan } c \equiv d \mod m \implies a-c \equiv b-d \mod m$.
%         \item $a \equiv b \mod m \text{ dan } c \equiv d \mod m \implies ac \equiv bd \mod m$.
%         \item $\forall k\in \ZZ^+, (am+b)^k \equiv b^k \mod m$.
%         \item Jika $ca \equiv cb \mod m$ dengan $FPB(c,m)=1$, maka $a \equiv b \mod m$.
%     \end{enumerate}
    
%     Catatan: Penggunaan sifat nomor 8 dapat dimodifikasi sehingga menjadi konsep \textbf{Chinese Remainder Theorem}.

%     \subsubsection{Banyaknya Bilangan Relatif Prima}
%     \begin{remark*}
%              Bilangan $b$ dikatakan relatif prima dengan $a$ jika dan hanya jika $FPB(a,b)=1$.
%         \end{remark*}
%      Definisikan fungsi \textbf{Euler Totient Phi $\phi(n)$ sebagai banyaknya bilangan bulat positif $b$ yang kurang dari sama dengan $n$ dimana $n$ relatif prima dengan $b$}. Rumus eksplisit untuk menghitung fungsi ini adalah
%     $$\phi(n) = n\left(1-\dfrac{1}{p_1}\right)\left(1-\dfrac{1}{p_2}\right)\dots\left(1-\dfrac{1}{p_n}\right).$$
    
%     Contoh: $\phi(4)=4(1-\frac{1}{2})=2$ karena ada 2 bilangan yang realtif prima dengan 4, yaitu 1 dan 3.
    
%     Catatan: Untuk semua bilangan prima $p$, nilai $\phi(p) = p-1$. (Silakan dibuktikan sendiri :D)

%     \subsubsection{Euler's Theorem on Modulo}
%     Untuk bilangan asli $a$ dan $n$ yang saling relatif prima kita punya
%     $$a^{\phi(n)} \equiv 1 \mod n.$$

%     \subsubsection{Fermat's Little Theorem}
%     Teorema ini merupakan kasus khusus dari Euler's Theorem saat $n$ prima sehingga $\phi(n)=n-1$. Untuk bilangan prima $p$, kita punya
%     $$a^{p-1} \equiv 1 \mod p.$$

%     \subsubsection{Wilson's Theorem}
%     Untuk suatu bilangan asli $p$, kita punya $p$ adalah bilangan prima jika dan hanya jika
%     $$(p-1)! \equiv -1 \mod p.$$
% \subsection{Latihan Soal}
% \begin{enumerate}
%     \item (AIME 1986) Tentukan bilangan asli $n$ terbesar sehingga $n+10 \mid n^3+100$.
    
%     \item Tentukan digit satuan dari $7^{7^7}$.
    
%     \item Jika $S=1!+2!+3!+\dots+2021!$, tentukan sisa $S$ saat dibagi 6.
    
%     \item (OSK 2009) Sisa saat $10^{999999999}$ saat dibagi oleh 7 adalah \dots
    
%     \item (OSK 2011) Bilangan asli terkecil $n>2011$ yang bersisa 1 jika dibagi $2,3,4,5,6,7,8,9,10$ adalah \dots.

%     \item (OSK 2015) Bilangan $x$ adalah bilangan bulat positif terkecil yang membuat
%     \[31^n + x \cdot 96^n\]
%     merupakan kelipatan 2015 untuk setiap bilangan asli $n$. Nilai $x$ adalah \ldots

%     \item (OSK 2023) Sisa pembagian bilangan $5^{2022}+11^{2022}$ oleh $64$ adalah \ldots

%     \item (OSK 2022) Untuk setiap bilangan asli $n$, misalkan $S(n)$ menyatakan hasil penjumlahan semua digit dari $n$. Diberikan barisan $\{a_n\}$ dengan $a_1 = 5$ dan $a_n = (S(a_{n-1}))^2 - 1$ untuk $n \geq 2$. Sisa pembagian $a_1 + a_2 + \cdots + a_{2022}$ oleh $21$ adalah \ldots
    
%     \item (OSK 2021) Diketahui dua digit terakhir dari $a^{777}$ adalah $77$, maka dua digit terakhir dari $a$ adalah \ldots

%     \item (OSK 2020) Misalkan $n \geq 2$ adalah bilangan asli sedemikian sehingga untuk setiap bilangan asli $a$, $b$ dengan $a + b = n$ berlaku $a^2 + b^2$ merupakan bilangan prima. Hasil penjumlahan semua bilangan asli $n$ semacam itu adalah \ldots

%     \item (OSK 2019) Sisa pembagian $1111^{2019}$ oleh $11111$ adalah \ldots

%     \item (OSK 2017) Bilangan prima terbesar yang dapat dinyatakan dalam bentuk $a^4+b^4+13$ untuk suatu bilangan-bilangan prima $a$ dan $b$ adalah \ldots

%     \item (OSK 2016) Palindrom adalah bilangan yang sama dibaca dari depan atau dari belakang. Sebagai contoh 12321 dan 32223 merupakan palindrom. Palindrom 5 digit terbesar yang habis dibagi 303 adalah \ldots
% \end{enumerate}

% \newpage
\section{Lantai dan Atap}
\subsection{Sedikit Materi}
Definisikan $\floor{x}$ (\textit{floor} $x$) sebagai bilangan bulat terbesar yang kurang dari sama dengan $x$. Simpelnya, $\floor{x}$ dapat dikatakan sebagai "pembulatan ke bawah". Contoh: $\floor{\pi}=3$, $\floor{2}=2$, $\floor{10,51}=10$, $\floor{-1,5}=-2$.

Definisikan $\ceiling{x}$ (\textit{ceiling} $x$) sebagai bilangan bulat terkecil yang lebih dari sama dengan $x$. Simpelnya, $\ceiling{x}$ dapat dikatakan sebagai "pembulatan ke atas". Contoh: $\ceiling{\pi}=4$, $\ceiling{2}=2$, $\ceiling{10,51}=11$, $\ceiling{-1,5}=-1$.

Definisikan $\{x\}$ sebagai \textit{fractional part} dari $x \in \RR$ dimana $\{x\} = x - \floor{x}$.

Beberapa properti:
\begin{enumerate}
    \item $\floor{x}=\ceiling{x}$ untuk $x \in \ZZ$.
    \item $\floor{x}=\ceiling{x}-1$ untuk $x \not \in \ZZ$.
    \item $\floor{x} \le  x < \floor{x}+1$ untuk $x \in \RR$.
    \item $\ceiling{x}-1 < x \le \ceiling{x}$ untuk $x \in \RR$.
    \item $\floor{n+x}=n+\floor{x}$ dan $\ceiling{n+x}=n+\ceiling{x}$ untuk $n\in \ZZ$ dan $x \in \RR$.
    \item Untuk semua $x,y \in \RR$ berlaku $\floor{x+y} \ge \floor{x}+\floor{y}$.
    
    \item $0 \le \{x\} < 1$ untuk $x \in \RR$.
\end{enumerate}

\subsection{Latihan Soal}
\begin{enumerate}
    \item (OSK 2013) Misalkan $\floor{x}$ menyatakan bilangan bulat terbesar yang lebih kecil atau sama dengan $x$ dan $\ceiling{x}$ menyatakan bilangan bulat terkecil yang lebih besar atau sama dengan $x$. Tentukan semua $x$ yang memenuhi $\floor{x}$ + $\ceiling{x}$ = 5.
    
    \item (OSK 2016) Banyaknya bilangan asli $n \in \{1,2,3,\dots,1000\}$ sehingga terdapat bilangan real positif $x$ yang memenuhi $x^2+\floor{x}^2=n$ adalah \dots
    
    \item (OSK 2018) Untuk setiap bilangan real $z$, $\lfloor z \rfloor$ menyatakan bilangan bulat terbesar yang lebih kecil dari atau sama dengan $z$. Jika diketahui $\lfloor x \rfloor + \lfloor y \rfloor + y = 43.8$ dan $x + y - \lfloor x \rfloor = 18.4$. Nilai $10(x + y)$ adalah...
    
    \item Let $[x]$ denote the largest integer not exceeding $x$. For example, $[2.1]=2$, $[4]=4$ and $[5.7]=5$. How many positive integers $n$ satisfy the equation $\left[\frac{n}{5}\right]=\frac{n}{6}$.

    \item (OSK 2019) Untuk sebarang bilangan real $x$, simbol $\lfloor x \rfloor$ menyatakan bilangan bulat terbesar yang tidak lebih besar daripada $x$, sedangkan $\lceil x \rceil$ menyatakan bilangan bulat terkecil yang tidak lebih kecil dibanding $x$. Interval $[a, b)$ adalah himpunan semua bilangan real $x$ yang memenuhi
    $$\lfloor 2x \rfloor^2 = \lceil x \rceil + 7.$$
    Nilai $a \cdot b$ adalah ...

    \item (OSK 2021) Jika $a > 1$ suatu bilangan asli sehingga hasil penjumlahan semua bilangan riil $x$ yang memenuhi persamaan
    $$\lfloor x \rfloor^2 - 2ax + a = 0$$
    adalah $p$, maka $a$ adalah ...

    \item Carilah banyaknya bilangan real $x$ yang memenuhi
    $$\floor{x^2}=4x+3.$$

    \item Jika $x$ adalah suatu bilangan real yang memenuhi persamaan berikut
    $$\floor{x}+\floor{2x}+\floor{3x}+\floor{4x}=2024.$$
    Tentukan nilai dari $\floor{6x}$.

    \item Jumlah dari semua solusi bilangan real $x$ dari persamaan
    $$2\floor{x}^2+3\{x\}^2=\frac{7}{4}x\floor{x}$$
    dapat dinyatakan dalam bentuk $\frac{p}{q}$ dimana $p$ dan $q$ merupakan bilangan bulat yang saling relatif prima. Tentukan nilai dari $10p+q$.

    
    \item (Modifikasi JBMO 2021) Carilah seluruh penyelesaian dari persamaan $2\cdot \lfloor{\frac{1}{2x}}\rfloor - 7 = 9(1 - 8x)$.
\end{enumerate}

\subsection{Latihan Soal Jika Anda Sudah Sepuh}
\begin{enumerate}
    \item (A Famous Lemma) For real number $x$, show that $\floor{x+\frac{1}{2}} = \floor{2x}-\floor{x}$.

    \item (Hermite's Identity) Let $x$ be a real number, and let $n$ be a positive integer. Then prove that
    \begin{align*}
         \lfloor nx \rfloor = \lfloor x \rfloor + \left\lfloor x+\frac{1}{n} \right\rfloor + \left\lfloor x+\frac{2}{n} \right\rfloor + \dots + \left\lfloor x+\frac{n-1}{n} \right\rfloor.
    \end{align*}
    

    \item (Hong Kong IMO Prelim 1999-2000) Find the integer $n$ satisfying $\left[\frac{n}{1!}\right]+\left[\frac{n}{2!}\right]+...+\left[\frac{n}{10!}\right]=1999$. Here $[x]$ denotes the greatest integer less than or equal to $x$.

    \item (Putnam 1986) What is the units (i.e., rightmost) digit of
\[\left\lfloor \frac{10^{20000}}{10^{100}+3}\right\rfloor\]

    \item  (USAMO 1981) If $x$ is a positive real number, and $n$ is a positive integer, prove that
\[[nx] \geq \frac{[x]}{1} + \frac{[2x]}{2} + \frac{[3x]}{3} + ... + \frac{[nx]}{n},\]where $[t]$ denotes the greatest integer less than or equal to $t$.

    \item (IMO 1968) Let $[x]$ denote the integer part of $x$, i.e., the greatest integer not exceeding $x$. If $n$ is a positive integer, express as a simple function of $n$ the sum\[\left[\frac{n+1}{2}\right]+\left[\frac{n+2}{4}\right]+...+\left[\frac{n+2^k}{2^{k+1}}\right]+\ldots\]

    \item Let $n$ be a natural number. Prove that
    \begin{align*}
        \left\lfloor \frac{n}{1} \right\rfloor + \left\lfloor \frac{n}{2} \right\rfloor + \dots + \left\lfloor \frac{n}{n} \right\rfloor + \lfloor \sqrt{n} \rfloor
    \end{align*}
    is even.

    \item Define $q(n)=\left\lfloor\frac{n}{\lfloor\sqrt{n}\rfloor}\right\rfloor$ for $n \in \mathbb{N}$. Determine all $n$ such that $q(n)>q(n+1)$.

    \item Let $r \ge 1$ be real number such that for any positive integers $m$ and $n$ whenever $m \mid n$ it is also true that $\lfloor mr \rfloor$ divides $\lfloor nr \rfloor$. Show that $r$ is an integer.
\end{enumerate}

\end{document}