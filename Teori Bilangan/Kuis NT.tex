\documentclass[a4paper,12pt,bahasa]{article}
\usepackage[utf8]{inputenc}

\title{\textbf{Kuis Teori Bilangan}}
\author{Azzam Labib Hakim}
\date{Mei 2023}


\usepackage{graphicx,amsmath,amssymb,amsthm,amsfonts,babel}
\usepackage{tikz, tkz-euclide}
\usetikzlibrary{calc,through,intersections}
\usepackage[a4paper,margin=1in]{geometry}
\usepackage{enumerate}

\newtheorem{lemma}{Lemma} %kasih * kalau gamau angka
\newtheorem*{claim}{Claim}
\newtheorem*{remark}{Remark}
\usepackage{enumitem}



\begin{document}

\maketitle

\section{Soal Isian}
\begin{enumerate}
    \item Carilah jumlah seluruh bilangan bulat positif terurut $a$ sehingga $b$, $\dfrac{a^3b+1}{a+1}$ dan $\dfrac{2021b-2025}{b-1}$ ketiganya merupakan bilangan bulat.


\item  Tentukan jumlah semua bilangan prima $p$ sehingga $p^3-3p^2+2$ juga merupakan bilangan prima.

\item Nilai dari $FPB(2002+2,2002^2+2,2002^3+2,\dots)$ adalah $\dots$
\end{enumerate}

\section{Soal Uraian}
Apakah terdapat bilangan asli $x$ dan $y$ sedemikian sehingga $x^3+xy^3+y^2+3$ habis membagi $x^2+y^3+3y-1$?
	
	
\section{Solusi Soal Isian}
\begin{enumerate} 
\item Carilah jumlah seluruh bilangan bulat positif terurut $a$ sehingga $b$, $\dfrac{a^3b+1}{a+1}$ dan $\dfrac{2021b-2025}{b-1}$ ketiganya merupakan bilangan bulat.

			
\begin{proof}[\textbf{Solusi.}]
\textbf{Solusi. }Perhatikan bahwa $$\dfrac{a^3b+1}{a+1}=\dfrac{b(a^3+1)-(b-1)}{a+1} =\dfrac{b(a^3+1)}{a+1}-\dfrac{b-1}{a+1}$$ adalah bilangan bulat. Karena $a+1 \mid a^3+1$, maka $\dfrac{b(a^3+1)}{a+1}$ adalah bilangan bulat yang menyebabkan $\dfrac{b-1}{a+1}$ bulat atau $a+1 \mid b-1$.
			
			Selanjutnya, perhatikan bahwa $$\dfrac{2021b-2025}{b-1}=\dfrac{2021(b-1)-4}{b-1}=2021-\dfrac{4}{b-1}$$ adalah bilangan bulat. Karena 2021 bilangan bulat maka menyebabkan $\dfrac{4}{b-1}$ bulat atau $b-1 \mid 4$. 
			
			Dari $a+1 \mid b-1$ dan $b-1 \mid 4$, didapatkan bahwa $a+1 \mid 4$ sehingga diperoleh $a=1$ dan $a=3$. Oleh karena itu, jumlah seluruh bilangan bulat positif $a$ yang memenuhi adalah $1+3=\boxed{4}$. $\square$
\end{proof}

\item  Tentukan jumlah semua bilangan prima $p$ sehingga $p^3-3p^2+2$ juga merupakan bilangan prima.
\begin{proof}[\textbf{Solusi.}] Perhatikan bahwa
\begin{align*}
p^3-3p^2+2 = (p-1)(p^2-2p-2)
\end{align*}
Akan dibagi kasus berdasarkan nilai $p$.
\begin{itemize}[]
\item Jika $p=2$, maka $p^3-3p^2+2=-2$, tidak memenuhi.
\item Jika $p=3$, maka $p^3-3p^2+2=2$, memenuhi.
\item Jika $p>3$, maka $p^2-2p-2 > p-1$ sehingga, agar $(p-1)(p^2-2p-2)$ prima, haruslah $p-1=1 \implies p=2$, kontradiksi. 
\end{itemize}
Berarti bilangan prima $p$ yang memenuhi hanyalah $p=3$. Oleh karena itu, jumlah semua bilangan prima $p$ yang memenuhi adalah $\boxed{3}$.
\end{proof}

\item Nilai dari $FPB(2002+2,2002^2+2,2002^3+2,\dots)$ adalah $\dots$
\begin{proof}[\textbf{Solusi.}]
		Misalkan $d=FPB(2002+2,2002^2+2,2002^3+2,\dots)$. Perhatikan bahwa $2002^2+2=2002(2000+2)+2=2002\cdot 2000+2002\cdot 2+2=2002\cdot 2000+2000\cdot 2 2\cdot 2+2=2000(2002+2)+6$. 
			\begin{lemma}
			Algoritma Euclid : $FPB(a,b)=FPB(a,b-a)=FPB(a,b-2a)=\dots =FPB(a,b-ka)$
			\end{lemma}
		Maka menurut Algoritma Euclid, $FPB(2002+2,2002^2+2)=FPB(2002+2,(2000(2002+2)+6)-2000(2002+2))=FPB(2002+2,6)=6$. Maka kita punya $d \mid FPB(2002+2,2002^2+2) \implies d \mid 6$. 
		
		\begin{lemma}
					Ekspansi Binomial: $(a+b)^n=\sum_{i=0}^{n} {n \choose i}a^{n-i}b^i$
		\end{lemma}
		
		Kemudian untuk sembarang bilangan asli $n$, dengan Ekspansi Binomial didapat
		\begin{align*}
		2002^n+2 &= 2+(2001+1)^n\\
			S	&= 2 + \sum_{i=0}^{n} {n \choose i}1^{n-i}2001^i\\
			S	&= 2 + {n \choose 0}2001^0 + \sum_{i=1}^{n} {n \choose i}1^{n-i}2001^i\\
			S	&= 3 + {n \choose 0}2001^0 + \sum_{i=1}^{n} {n \choose i}1^{n-i}2001^i
		\end{align*}
			
		Perhatikan, karena $3 \mid 2001$, maka $3 \mid S$ sehingga menyebabkan $3 \mid 2002^n+2$. Di lain sisi, perhatikan bahwa $2002^n+2$ bernilai genap, maka $2 \mid 2002^n +2$. Dari sini didapat $3\cdot2 \mid 2002^n+2$ atau $\mid 2002^n+2$ untuk sembarang $n \in \mathbb{N}$. Ini berakibat $6 \mid d$. Namun, karena kita juga punya $d \mid 6$, berarti $FPB(2002+2,2002^2+2,2002^3+2,\dots)=d=6$. \qed
  \end{proof}

\end{enumerate}

\section{Solusi Soal Esai}
Apakah terdapat bilangan asli $x$ dan $y$ sedemikian sehingga $x^3+xy^3+y^2+3$ habis membagi $x^2+y^3+3y-1$?
	\textbf{Jawaban. }
	Tidak.\\
 
	\textbf{Solusi. }
	Andaikan $\exists x,y \in \mathbb{N}$ sehingga $x^3+xy^3+y^2+3 \mid x^2+y^3+3y-1$. Maka haruslah 
	 $$x^3+xy^3+y^2+3 \le x^2+y^3+3y-1 \dots (1)$$
	Padahal untuk sembarang $x,y \in \mathbb{N}$, kita punya $x^3 \ge x^2$,   $xy^3 \ge y^3$, dan\\ $y^2-3y+4 = (y-2)^2+y \ge 0 + y > 0 \implies y^2+3 > 3y -1$. Berarti kita punya $$x^3+xy^3+y^2+3 > x^2+y^3+3y-1,$$ yang kontradiksi dengan (1).
	
	Jadi, tidak ada $x,y \in \mathbb{N}$ yang memenuhi. \qed

\end{document}
