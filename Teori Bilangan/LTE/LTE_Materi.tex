\documentclass[11pt]{scrartcl}
\usepackage{graphicx}
\graphicspath{{./}}
\usepackage[sexy]{evan}
\usepackage[normalem]{ulem}
\usepackage{hyperref}
\usepackage{mathtools}
\hypersetup{
    colorlinks=true,
    linkcolor=blue,
    filecolor=magenta,      
    urlcolor=cyan,
    pdfpagemode=FullScreen,
    }

\renewcommand{\dangle}{\measuredangle}

\renewcommand{\baselinestretch}{1.5}

\addtolength{\oddsidemargin}{-0.4in}
\addtolength{\evensidemargin}{-0.4in}
\addtolength{\textwidth}{0.8in}
% \addtolength{\topmargin}{-0.2in}
% \addtolength{\textheight}{1in} 


\setlength{\parindent}{0pt}

\usepackage{pgfplots}
\pgfplotsset{compat=1.15}
\usepackage{mathrsfs}
\usetikzlibrary{arrows}

\title{LTE - Materi}
\author{Azzam (IG: haxuv.world)}
\date{Senin, 22 Januari 2024}

\begin{document}

\maketitle

\section{Materi Singkat}
\subsection{p-adic valuation (pengantar LTE)}
\begin{enumerate}
    \item $p$ bilangan prima. $x \in \ZZ$. Didefinisikan \textbf{p-adic valuation} dari $x$ sebagai pangkat tertinggi dari $p$ yang membagi $x$. Notasinya adalah $\upsilon_p(x)=a$ berarti $p^a \mid x$ tetapi $p^{a+1} \nmid x$.
    \item $\upsilon_p(xy) = \upsilon_p(x) + \upsilon_p(y)$.
    \item $\upsilon_p(x^k)=k\upsilon_p(x)$.
    \item $x \mid y$ maka $\upsilon_p(x) \le \upsilon_p(y)$.
    \item $\upsilon_p(x+y) \ge \min(\upsilon_p(x),\upsilon_p(y))$.
    \item Jika $\upsilon_p(x) \neq \upsilon_p(y)$ maka $\upsilon_p(x+y) = \min(\upsilon_p(x),\upsilon_p(y))$.
\end{enumerate}

\subsection{Legendre}
\begin{enumerate}
    \item Untuk semua bilangan bulat positif $n$ dan bilangan prima $p$, berlaku
\begin{align*}
    \upsilon_p(n!) = \sum_{i=1}^{\infty} \floor{\frac{n}{p^i}}.
\end{align*}

    \item Untuk semua bilangan bulat positif $n$ dan bilangan prima $p$, berlaku
\begin{align*}
    \upsilon_p(n!) = \dfrac{n-s_p(n)}{p-1}
\end{align*}
    dimana $s_p(n)$ menyatakan jumlah digit-digit $n$ dalam basis $p$.
\end{enumerate}


\section{Contoh Soal}
\begin{enumerate}
    \item Carilah nilai $\upsilon_3(17!)$.
    \item Tentukan banyak angka 0 berturut-turut di belakang angka $103!$ .
    \item Buktikan bahwa $\upsilon_2(\binom{2n}{n})$ bernilai sama dengan banyaknya angka $1$ pada representasi biner dari $n$.
    \item Diberikan bilangan komposit $n$ dan $p$ pembagi prima dari $n$. Buktikan $\upsilon_p((2n)!) \geq 2+\upsilon_p(n!)$.
    \item Buktikan apabila $3^n-2^n$ merupakan perpangkatan dari suatu bilangan prima ($p^a$), maka $n$ merupakan bilangan prima.
\end{enumerate}

Coba buktikan \textbf{Lemma Kecil Berikut} yang dapat membantu banyak hal di kehidupan LTE anda :).

\begin{enumerate}[resume]
    \item Diberikan sembarang $x,y$ bilangan bulat dan $n$ bilangan asli. Misal $p$ merupakan sembarang bilangan prima sehingga $gcd(n,p)=1$ dan $p|x-y$ namun $p\nmid x$ dan $p\nmid y$. Maka
    \[\upsilon_p(x^n-y^n) = \upsilon_p(x-y)\]
    \item Diberikan sembarang $x,y$ bilangan bulat dan $n$ bilangan asli ganjil. Misal $p$ merupakan sembarang bilangan prima sehingga $gcd(n,p)=1$ dan $p|x+y$ namun $p\nmid x$ dan $p\nmid y$. Maka
    \[\upsilon_p(x^n+y^n) = \upsilon_p(x+y)\]
\end{enumerate}

\subsection{LTE (Lifting The Exponent)}
\begin{enumerate}
    \item [Lemma 1] $x,y \in \ZZ$, $n \in \NN$, $p$ bilangan prima ganjil dengan $p \mid x-y$, $p \nmid x$, dan $p \nmid y$ maka $\upsilon_p(x^n-y^n)=\upsilon_p(x-y)+\upsilon_p(n)$.
    \item [Lemma 2] $x,y \in \ZZ$, $n \in \NN$ bilangan ganjil, $p$ bilangan prima ganjil dengan $p \mid x+y$, $p \nmid x$, dan $p \nmid y$ maka $\upsilon_p(x^n+y^n)=\upsilon_p(x+y)+\upsilon_p(n)$.
    \item [Lemma 3] $x,y \in \ZZ$, $n \in \NN$, $2 \nmid x$, $2 \nmid y$,  $4 \mid x-y$  maka $v_2(x^n-y^n)=v_2(x+y)+v_2(n)$.
    \item [Lemma 4] $x,y \in \ZZ$, $n \in \NN$ bilangan genap, $2 \nmid x$, $2 \nmid y$,  $2 \mid x-y$ maka $v_2(x^n-y^n)=v_2(x+y)+v_2(x-y)+v_2(n)-1$.
\end{enumerate}


\section{Contoh Soal}
\begin{enumerate}
\item (IMO 1990) Tentukan semua bilangan asli $n > 1$ sehingga 
    \begin{align*}
        \dfrac{2^n+1}{n^2} \in \ZZ.
    \end{align*}
\item (PEN Book) Diberikan $n$ merupakan bilangan asli, buktikan bahwa
\[\frac{1}{2} + \frac{1}{3} +...+ \frac{1}{n}\]
tidak mungkin bernilai bulat.

\item Untuk setiap bilangan asli $k>1$, didefinisikan $S_k$ sebagai himpunan yang memuat triple bilangan asli $(n,a,b)$, dengan $n$ ganjil dan $\gcd (a,b)=1$, sehingga $a+b=k$ dan $n$ habis membagi $a^n+b^n$. Tentukan semua bilangan asli $k$ sehingga $S_k$ memiliki berhingga anggota.

\item (IMO SL 2014) Tentukan semua triple bilangan asli $(p, x, y)$ dengan $p$ merupakan bilangan prima sehingga memenuhi $x^{p -1} + y$ and $x + y^ {p -1}$ berbentuk perpangkatan dari $p$ ($p^k$). 
\end{enumerate}

\end{document}
