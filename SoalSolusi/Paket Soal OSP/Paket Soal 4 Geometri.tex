\documentclass[11pt]{scrartcl}
\usepackage[sexy]{evan}

%\addtolength{\textheight}{1.5in} 
\renewcommand{\baselinestretch}{1.5}



\begin{document}
	\title{Paket Soal 4 : Geometri} % Beginner
	\date{\today}
	\author{Compiled by Azzam}
	\maketitle
	\newpage
	
	\section{Soal}
\subsection{Isian Singkat}
Jawablah soal-soal berikut dengan jawaban akhir tanpa menyertakan cara.

	\begin{soalbaru}
		Diberikan segitiga $ABC$ lancip. Garis tinggi terpanjang adalah dari titik sudut $A$ tegak
			lurus pada $BC$, dan panjangnya sama dengan panjang median (garis berat) dari titik sudut $B$.
			Nilai terbesar $\angle ABC$ adalah $\dots^\circ$
	\end{soalbaru}
	
	\begin{soalbaru}
		Pada segitiga $ABC$ terdapat titik $P$ di dalamnya sehingga $\angle PAB = 10^\circ, \angle PBA = 20^\circ , \angle PAC = 40^\circ, \angle PCA = 30^\circ$. Besar sudut $\angle ABC$ adalah $\dots^\circ$ 
		
	\end{soalbaru}
	
	\begin{soalbaru}
			Pada sembarang segitiga $ABC$, titik $D,E,F$ berturut-turut pada $BC,CA,AB$ dimana $AD,BE,CF$ bertemu di $M$. Nilai eksak dari $\dfrac{AM}{AD}+\dfrac{BM}{BE}+\dfrac{CM}{CF}$ adalah $\dots$
		\end{soalbaru}
	
	\begin{soalbaru}
		Misalkan $ABCD$ adalah segiempat konveks dengan $\angle DAC=\angle BDC = 36^\circ$, $\angle CBD = 18^\circ$, dan $\angle BAC = 72^\circ$. Diagonal $AC$ dan $BD$ berpotongan di titik $P$. Tentukan besar sudut $\angle APD$ dalam derajat. %ko ss spring camp 30 maret 2019 
	\end{soalbaru}
	
	\begin{soalbaru}
		Diberikan segitiga sama kaki $ABC$ dengan $AB = AC$ dan $\angle BAC < 60^\circ$. Titik $D$ dan $E$ dipilih pada titik $AC$ sehingga $EB = ED$ dan $\angle ABD = \angle CBE$. Notasikan $O$ sebagai perpotongan antara garis bagi dalam $\angle BDC$ dan garis bagi dalam $\angle ACB$. Hitunglah besar sudut $\angle COD$ dalam derajat. %ko ss 30 maret
	\end{soalbaru}
	
	\begin{soalbaru}
		Titik $A$ dan $B$ dari segitiga sama sisi $ABC$ berada pada lingkaran $k$ yang berjari-jari 1 dengan titik $C$ berada di dalam lingkaran $k$ tersebut. Sebuah titik $D$ yang berbeda dari titik $B$, berada pada lingkaran $k$ sehingga $AD=AB$. Garis $DC$ memotong lingkaran $k$ untuk yang kedua kalinya di titik $E$. Panjang garis $CE$ adalah $\dots$ %ko ss 30 maret
	\end{soalbaru}

	\begin{soalbaru}
		Diberikan sebuah segilima $ABCDE$ dengan masing-masing titik sudutnya berada pada satu lingkaran. Jika $AB=DC=3$, $BC=DE=10$, dan $AE=14$. Jika jumlah panjang seluruh diagonal segilima $ABCDE$ tersebut adalah $a$, hitunglah nilai $\floor{a}$.
	\end{soalbaru}
	
	\begin{soalbaru}
		Pada $\triangle BAC$, $\angle BAC = 40^\circ$, $AB=10$, and $AC=6$. Titik $D$ dan $E$ berada pada segmen $AB$ dan $AC$ secara berturut-turut. Berapakah nilai minimum $\floor{BE+DE+CD}$?
	\end{soalbaru}

\subsection{Esai}
Jawablah soal-soal berikut dengan menyertakan cara pengerjaan atau argumentasinya.
	
	\begin{soalbaru} 
		Suatu segitiga $ABC$ memiliki titik tinggi $H$ dan titik pusat lingkaran dalam $I$. Buktikan bahwa $A,B,H,I$ berada pada satu lingkaran (siklis) jika dan hanya jika $\angle ACB = 60^\circ$.
	\end{soalbaru}
	
	\begin{soalbaru} 
		 Misalkan $C$ adalah titik pada setengah lingkaran dengan diameter $AB$ (terletak di keliling lingkarannya, bukan di diameternya). Misalkan pula $D$ adalah titik tengah busur $AC$. Notasikan $E$ sebagai proyeksi $D$ pada garis $BC$ dan $F$ adalah perpotongan $AE$ dengan setengah lingkaran. Buktikan bahwa $BF$ membagi garis $DE$ sama panjang.
	\end{soalbaru}
	
	\begin{soalbaru} 
		Diberikan $\triangle ABC$ dimana $A',B',C'$ berturut-turut adalah pencerminan $A,B,C$ terhadap $BC,CA,AB$. Perpotongan lingkaran luar $\triangle ABB'$ dan $\triangle ACC'$ adalah $A_1$. Definisikan $B_1$ dan $C_1$ secara serupa. Buktikan bahwa $AA_1,BB_1,$ dan $CC_1$ konkuren (bertemu di satu titik).
	\end{soalbaru}
	
	\begin{soalbaru}
		Misalkan $X,Y,Z$ berturut-turut pada segmen $BC,CA,AB$ dari $\triangle ABC$ sehingga $AX, BY, CZ$ konkuren di $P$. Selanjutnya, misalkan $D,E,F$ berturut-turut pada segmen $YZ,ZX,XY$ sehingga $XD,YE,ZF$ konkuren di $Q$. Buktikan bahwa $AD,BE,CF$ konkuren.
	\end{soalbaru}
	
	\begin{soalbaru}
		Diberikan segitiga $ABC$, dengan garis berat $AD, BE, CF$. Misalkan $m=AD+BE+CF$ dan misalkan $s=AB+BC+CA$. Buktikan bahwa $$\dfrac{3s}{2} > m > \dfrac{3s}{4}.$$
	\end{soalbaru}
	
\end{document}
