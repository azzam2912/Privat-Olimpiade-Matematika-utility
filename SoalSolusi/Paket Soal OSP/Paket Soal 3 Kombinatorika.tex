\documentclass[11pt]{scrartcl}
\usepackage[sexy]{evan}

%\addtolength{\textheight}{1.5in} 
\renewcommand{\baselinestretch}{1.5}



\begin{document}
	\title{Paket Soal 3 : Kombinatorika} % Beginner
	\date{\today}
	\author{Compiled by Azzam}
	\maketitle
	\newpage
	
	\section{Soal}
\subsection{Isian Singkat}
Jawablah soal-soal berikut dengan jawaban akhir tanpa menyertakan cara.

	\begin{soalbaru}
		Misalkan $A$ adalah himpunan semua pembagi positif dari $6^8$ Jika dipilih dua bilangan sebarang $x$ dan $y$ di $A$ (boleh sama), tentukan peluang dari kejadian $x \mid y$. 
	\end{soalbaru}
	
	\begin{soalbaru}
		Misalkan Hitomi ingin mengejar Hyewon menaiki tangga dalam perlombaan kebugaran. Karena kelelahan, Hitomi hanya bisa menaiki satu atau dua anak tangga setiap satu langkah ia berlari. Jika banyak anak tangga yang tersisa untuk mengejar Hyewon adalah 10 anak tangga, berapa banyak kemungkinan Hitomi dalam melangkah?
	\end{soalbaru}
	
	\begin{soalbaru}
		Kata sandi tanpa perulangan karakter dibentuk dengan menggunakan huruf kapital. Sebuah kata sandi dikatakan \textit{sempurna} bila tidak memuat untaian karakter $XYZ$ maupun $ZYX$. Hitunglah peluang untuk membentuk kata sandi \textit{sempurna} yang terdiri dari atas 8 huruf.
	\end{soalbaru}
	
	\begin{soalbaru}
			Sejumlah $n$ siswa duduk mengelilingi suatu meja bundar. Diketahui siswa laki-laki sama banyak dengan siswa perempuan. Jika banyaknya pasangan 2 orang yang duduk berdampingan
			dihitung, ternyata perbandingan antara pasangan bersebelahan yang berjenis kelamin sama dan
			pasangan bersebelahan yang berjenis kelamin berbeda adalah $7 : 3$. Tentukan minimal $n$ yang
			mungkin.
		\end{soalbaru}
	
	\begin{soalbaru}
		Misalkan $A$ adalah himpunan yang berisi $10$ elemen dan $A_1,A_2,\dots,A_k$ adalah himpunan bagian dari $A$ sedemikian sehingga untuk sembarang dua himpunan bagian berbeda $A_i$ dan $A_j$ hanya memenuhi salah satu dari tiga kemungkinan: $A_i \cap A_j = \emptyset$, $A_i \subset A_j$, atau $A_j \subset A_i$. Tentukan nilai terbesar yang mungkin untuk $k$. %ko SS kombin 15 juni 19
	\end{soalbaru}
	
	\begin{soalbaru}
		Misalkan $|X|$ menyatakan kardinalitas atau banyaknya anggota himpunan $X$, dan $S=\{1,2,\dots,10\}$. Carilah banyaknya pasangan himpunan $(A,B)$ sehingga\\[-20pt]
		$$A \cup B = S,\text{ } A \cap B = \emptyset, \text{ } (|A|+|B|) \in A,\text{ } (|A|-|B|) \in B. $$ %ktom februari 2016
	\end{soalbaru}
	\begin{soalbaru}
		Untuk tiap bilangan asli $n$, definisikan $a_n$ sebagai banyaknya bilangan asli $n$
			digit yang digitnya hanya terdiri atas angka 1 dan 2, serta tidak ada
			dua buah angka 2 yang terletak persis bersebelahan. Nilai dari $a_{10}$ adalah...
	%akeyla
		\end{soalbaru}

	\begin{soalbaru}
		Seorang laki - laki memiliki 5 teman. Suatu malam di restoran McHarto, dia bertemu
		dengan masing - masing dari mereka 11 kali, setiap 2 dari mereka 5 kali, setiap 3 dari mereka 4 kali, setiap 4 dari mereka 3 kali, dan semua mereka 10 kali. Dia makan di McHarto 10 kali tanpa bertemu mereka. Berapa kali total dia makan di McHarto?
	\end{soalbaru}

\subsection{Esai}
Jawablah soal-soal berikut dengan menyertakan cara pengerjaan atau argumentasinya.
	
	\begin{soalbaru} 
		Mungkinkah menempatkan masing-masing angka-angka $1,2,\dots,9$ tepat sekali ke dalam kotak-kotak satuan pada papan catur berukuran $3\times 3$, sehingga untuk setiap dua persegi yang bertetangga, baik secara vertikal ataupun horizontal, jumlah dari dua bilangan yang ada di dalamnya selalu prima?
	\end{soalbaru}
	
	\begin{soalbaru} 
		 Tunjukkan bahwa ada dua bilangan di $A$ yang jumlahnya 83, dengan $A$ adalah himpunan 15 bilangan bulat berbeda yang diambil dari barisan aritmatika $1,4,\dots,79$.
	\end{soalbaru}
	
	\begin{soalbaru} 
		Misalkan $A = \{1,11,111,1111,\dots\}$ adalah himpunan semua bilangan asli yang hanya tersusun oleh angka 1. Buktikan bahwa ada bilangan $x \in A$ sehingga $2021 \mid x$.
	\end{soalbaru}
	
	\begin{soalbaru}
		Akira dan Benjiro mempunyai tak hingga banyaknya koin bundar yang identik. Akira dan Benjiro bergantian menaruh koin tersebut di meja persegi yang ukurannya terbatas (finit) sehingga tidak ada dua koin yang saling bertumpuk dan setiap koin tepat di atas meja (jadi tidak ada yang koin yang menggantung di pinggiran meja sehingga bisa jatuh). Orang yang tidak bisa menempatkan koin di meja saat gilirannya dinyatakan kalah. Asumsikan setidaknya satu koin dapat ditaruh di meja. Jika Akira main duluan, buktikan bahwa Akira punya strategi menang.
	\end{soalbaru}
	
	\begin{soalbaru}
		Misalkan di papan tulis pada awalnya kita mempunyai 3 bilangan yaitu $(5,2,3)$. Setiap langkah kita memilih dua bilangan berbeda dari 3 bilangan yang kita punya tersebut, misalkan $a$ dan $b$, lalu menghapusnya  masing-masing dengan $0.6a-0.8b$ dan $0.8a+0.6b$. Bisakah pada akhirnya kita mendapatkan tiga angka $(1,5,4)$?
	\end{soalbaru}
	
\end{document}
