\documentclass[11pt]{scrartcl}
\usepackage[sexy]{evan}

%\addtolength{\textheight}{1.5in} 
\renewcommand{\baselinestretch}{1.5}



\begin{document}
	\title{Paket Soal 5 : Aljabar} % Beginner
	\date{\today}
	\author{Compiled by Azzam}
	\maketitle
	\newpage
	
	\section{Soal}
\subsection{Isian Singkat}
Jawablah soal-soal berikut dengan jawaban akhir tanpa menyertakan cara.

	\begin{soalbaru}
		Untuk sembarang barisan bilangan real $A=(a_1,a_2,a_3,\ldots)$, definisikan $\Delta A^{}_{}$ sebagai barisan $(a_2-a_1,a_3-a_2,a_4-a_3,\ldots)$, dimana suku ke $n$ nya adalah  $a_{n+1}-a_n^{}$. Misalkan setiap suku dari barisan  $\Delta(\Delta A^{}_{})$ adalah $1^{}_{}$, dan $a_{19}=a_{92}^{}=0$. Carilah nilai $a_1^{}$.
		%101 25 / AIME 1992
	\end{soalbaru}
	
	\begin{soalbaru}
		Diberikan persamaan $x^3-x+1=0$ yang mempunyai akar-akar $a,b,$ dan $c$. Carilah nilai $a^8+b^8+c^8$.
	\end{soalbaru}
	
	\begin{soalbaru}
			Diberikan $\dfrac{\pi}{4} = 1 - \dfrac{1}{3}+\dfrac{1}{5}-\dfrac{1}{7}+\dots$. Jika  $\dfrac{1}{1 \times 3 \times 5}+\dfrac{3}{5 \times 7 \times 9}+\dfrac{5}{9 \times 11 \times 13}+\dots = \dfrac{a-b\pi}{c}$, dimana $a,b,c$ adalah bilangan asli dengan $b$ relatif prima dengan $c$, hitunglah nilai $a+b+c$.
		\end{soalbaru}
	
	\begin{soalbaru}
		Diberikan polinomial $p(x)=x^3-ax^2+bx-c$ mempunyai tiga akar bulat positif berbeda dan $p(2002)=2001$. Misalkan $q(x)=x^2-2x+2002$. Diketahui pula bahwa $p(q(x))$ tidak mempunyai akar real. Tentukan nilai $a$.
	\end{soalbaru}
	
	\begin{soalbaru}
		Diberikan bilangan real $a,b,c,d$ yang memenuhi\\[-20pt] $$a+b+c+d+e=8 \text{ dan }a^2+b^2+c^2+d^2+e^2=16.$$\\[-25pt] Misalkan $M$ dan $m$ berturut-turut adalah nilai maksimum dan minimum dari $a$. Tentukan nilai dari $\left \lfloor M-m \right \rfloor$.  %101 no 41
	\end{soalbaru}
	
	\begin{soalbaru}
		Jumlah seluruh bilangan real $x$ yang memenuhi $10^x+11^x+12^x=13^x+14^x$ adalah... %probs 15 101
	\end{soalbaru}

	\begin{soalbaru}
		Carilah bilangan bulat terkecil $m$ sehingga\\[-15pt] $${2n \choose n}^{\frac{1}{n}}<m$$\\[-20pt] untuk semua bilangan asli $n$.% 101 11
	\end{soalbaru}
	
	\begin{soalbaru}
		Misalkan $f(x)=a\sin((x+1)\pi) + b\sqrt[3]{x-1}+2$, dimana $a$ dan $b$ adalah bilangan real. Jika $f(\log 5)=5$, carilah nilai dari $f(\log 20)$.
	\end{soalbaru}

\subsection{Esai}
Jawablah soal-soal berikut dengan menyertakan cara pengerjaan atau argumentasinya.
	
	
	\begin{soalbaru} Untuk sembarang bilangan real positif $a,b,c$ yang memenuhi $a^6+b^6+c^6=9$, buktikan bahwa $$\dfrac{a+b}{(a^3\sqrt{b}+b^3\sqrt{a})^2}+\dfrac{b+c}{(b^3\sqrt{c}+c^3\sqrt{b})^2}+\dfrac{c+a}{(c^3\sqrt{a}+a^3\sqrt{c})^2} \ge \dfrac{1}{2}.$$
		\end{soalbaru}
	
	
	\begin{soalbaru} 
		 Carilah semua solusi real  $x,y,z$ yang memenuhi: 
		 		$$\begin{cases}
		 		\dfrac{4x^2}{4x^2+1}=y \\[5pt]
		 		\dfrac{4y^2}{4y^2+1}=z \\[5pt]
		 		\dfrac{4z^2}{4z^2+1}=x.
		 		\end{cases}$$
	\end{soalbaru}
	
	\begin{soalbaru} 
		 Untuk $\forall a,b,c \in\mathbb{R^+}$ dengan $a+b+c=1$, buktikan bahwa $$\left( a+\frac{1}{a} \right)^2+\left( b+\frac{1}{b} \right)^2+\left( c+\frac{1}{c} \right)^2 \ge \frac{100}{3}$$
	\end{soalbaru}
	
	\begin{soalbaru}
		Sebuah polinomial $P(x)=x^3+ax^2+bx+c$ mempunyai tiga akar real yang berbeda. Polinomial $P(Q(x))$ tidak mempunyai akar real dimana $Q(x)=x^2+x+2021$. Buktikan bahwa $P(2021)>\frac{1}{64}$.
	\end{soalbaru}
	
	\begin{soalbaru}
		Carilah semua pasangan solusi real $(x,y)$ yang memenuhi:\\[-10pt] 
				 		$$\begin{cases}\\[-30pt]
				 		x^2+3x+\log(2x+1)=y \\[-5pt]
				 		y^2+3y+\log(2y+1)=x
				 		\end{cases}$$
	\end{soalbaru}
	
\end{document}
