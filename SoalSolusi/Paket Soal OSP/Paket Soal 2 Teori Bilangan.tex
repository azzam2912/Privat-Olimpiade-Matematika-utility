\documentclass[11pt]{scrartcl}
\usepackage[sexy]{evan}

%\addtolength{\textheight}{1.5in} 
\renewcommand{\baselinestretch}{1.5}



\begin{document}
	\title{Paket Soal 2 : Teori Bilangan} % Beginner
	\date{\today}
	\author{Compiled by Azzam}
	\maketitle
	\newpage
	
	\section{Soal}
\subsection{Isian Singkat}
Jawablah soal-soal berikut dengan jawaban akhir tanpa menyertakan cara.
	
	
	\begin{soalbaru}
		Jumlah seluruh bilangan asli $n$ yang memenuhi $3^{n-1}+5^{n-1} \mid 3^n + 5^{n}$ adalah...
	\end{soalbaru}
	
	\begin{soalbaru}
		Bilangan asli terkecil $n$ sehingga $N = 2010 \times 2012 \times 2016 \times 2018 + n$ merupakan kuadrat sempurna suatu bilangan asli adalah....
	\end{soalbaru}
	
	\begin{soalbaru}
		Sebuah bilangan asli disebut $TRI$ apabila representasi basis $3$ bilangan tersebut jika dibaca dalam basis $10$ habis dibagi 7. Sebagai contoh, 21 merupakan bilangan $TRI$ karena representasi basis 3 dari 21 adalah $(210)_3$ dan 210 habis dibagi 7. Tentukan banyaknya bilangan $TRI$ yang representasi basis tiganya memiliki tepat 9 digit jika dibaca dalam basis 10.
	\end{soalbaru}
	
	\begin{soalbaru}
		Misalkan tripel bilangan prima $(p,q,r)$ memenuhi $3p^4-5q^4-4r^2=26$. Jika $S = p+q+r$, hitunglah jumlah semua $S$ yang mungkin.
	\end{soalbaru}
	
	\begin{soalbaru}
		Carilah bilangan bulat $n$ terbesar sehingga $3^n \mid 10^{3^{1999}} - 1$.
	\end{soalbaru}
	
	\begin{soalbaru}
		Jumlah seluruh bilangan asli $n$ sehingga $n\cdot 2^{n+1}+1$ merupakan bilangan kuadrat sempurna adalah $\dots$
	\end{soalbaru}

	\begin{soalbaru}
		Berapa banyak bilangan asli $n$ berbeda sehingga bilangan $n+9$ dan $n^2+27$ keduanya adalah bilangan kubik atau pangkat tiga dari suatu bilangan asli?
	\end{soalbaru}

	\begin{soalbaru}
		Hitunglah banyaknya pasangan bilangan bulat positif $(x,y)$ yang memenuhi\\[-20pt] $$x^2+(x+1)^2=y^4+(y+1)^4.$$
	\end{soalbaru}
	\newpage

\subsection{Esai}
Jawablah soal-soal berikut dengan menyertakan cara pengerjaan atau argumentasinya.
	
	\begin{soalbaru} 
		Diberikan polinomial $f(x)=x^n+a_1x^{n-1}+\dots+a_{n-1}x+a_n$ dengan koefisien bilangan bulat. Lalu, diketahui bahwa ada empat bilangan bulat berbeda $a,b,c,$ dan $d$ sehingga $f(a)=f(b)=f(c)=f(d)=5$. Tunjukkan bahwa tak ada bilangan bulat $k$ sehingga $f(k)=8$.
		
	\end{soalbaru}
	
	\begin{soalbaru} Tentukan semua pasangan bilangan bulat positif $(a,b)$ sehingga\\[-18pt] $$ab^2+b+7 \mid a^2b+a+b.$$
	\end{soalbaru}
	\begin{soalbaru} 
		Carilah seluruh pasangan bilangan bulat positif terurut $(a,b)$ sehingga $\dfrac{a^3b-1}{a+1}$ dan $\dfrac{b^3a+1}{b-1}$ keduanya merupakan bilangan bulat positif.
	\end{soalbaru}
	
	\begin{soalbaru}
		Misalkan $m$ dan $s$ adalah bilangan asli dengan $2 \le s \le 3m^2.$ Definisikan barisan $a_1,a_2,\dots$ secara rekursif dengan $a_1=s$ dan \\[-18pt]$$a_{n+1} = 2n + a_n \text{    untuk   }  n=1,2,\dots.$$\\[-23pt]
		Jika $a_1,a_2,\dots,a_m$ adalah bilangan-bilangan prima, buktikan bahwa $a_{s-1}$ juga prima. 
	\end{soalbaru}
	
	\begin{soalbaru}
		Buktikan bahwa tidak ada bilangan ganjil $n > 1$ sehingga $n \mid 3^n+1$.
	\end{soalbaru}
	
\end{document}
