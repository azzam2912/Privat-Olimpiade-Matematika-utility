\documentclass[11pt]{scrartcl}
\usepackage[sexy]{evan}

%\addtolength{\textheight}{1.5in} 
\renewcommand{\baselinestretch}{1.5}



\begin{document}
	\title{Paket Soal 1 : Campur} % Beginner
	\date{\today}
	\author{Compiled by Azzam}
	\maketitle
	\newpage
	
	\section{Isian Singkat}
	Jawablah soal-soal berikut dengan jawaban akhir tanpa cara.
	
	\begin{soalbaru}
		Jihyo memilih secara acak satu bilangan dari himpunan $\{1,-1\}$, lalu menulis bilangan yang dipilihnya di papan tulis. Proses tersebut dilakukan 2021 kali. Jika $p$ adalah peluang di mana hasil kali ke-2021 angka di papan tulis adalah positif, tentukan nilai dari $100p$.
		
		
	\end{soalbaru}
	
	
	\begin{soalbaru}
		Misalkan $a_n$ dan $b_n$ berturut-turut menyatakan banyaknya digit dari $4^n$ dan $25^n$. Tentukan nilai dari $a_1+a_2+\dots +a_{50}+b_{50}+\dots+b_1$.
		
	\end{soalbaru}
	
	\begin{soalbaru}
		Tentukan jumlah semua bilangan real $x$ yang memenuhi persamaan $$\sqrt[3]{\frac{x-3}{2020}}+\sqrt[3]{\frac{x-2}{2021}}+\sqrt[3]{\frac{x-1}{2022}}=\sqrt[3]{\frac{x-2020}{3}}+\sqrt[3]{\frac{x-2021}{2}}+\sqrt[3]{x-2022}.$$
	\end{soalbaru}
	
	\begin{soalbaru}
		Diberikan segitiga $ABC$ dengan panjang sisi $BC,CA,$ dan $AB$ berturut-turut adalah $11,13,$ dan $20$. Misalkan $I$ adalah titik pusat lingkaran dalam segitiga $ABC$. Luas segitiga yang ketiga titik sudutnya adalah titik berat dari segitiga $ABI,BCI$ dan $ACI$ dapat dinyatakan dalam bentuk $\frac{a}{b}$, dengan $a$ dan $b$ adalah bilangan asli yang relatif prima. Tentukan nilai dari $a+b$.
	\end{soalbaru}
	
	\begin{soalbaru}
		Sebuah papan berukuran $5 \times 5$ dibagi menjadi 25 kotak satuan. Bilangan-bilangan $1,2,3,4,$ dan $5$ dituliskan pada kotak satuan tersebut sedemikian sehingga setiap baris, setiap kolom, dan setiap diagonal yang mengandung lima bilangan hanya diisi oleh masing-masing dari bilangan tersebut sekali saja. Misalkan $S$ adalah jumlah bilangan pada empat kotak yang terletak di bawah kedua diagonal. Tentukan nilai terbesar $S$.
	\end{soalbaru}
	\begin{soalbaru}
		Misalkan $X = 3 + 33 + 333 + \dots + \underbrace{333 \dots 333}_{2016 \text{ angka } 3}$ dan $Y$ adalah jumlah digit dari $X$. Tentukanlah nilai $Y$.
	\end{soalbaru}
	
	
	
	\begin{soalbaru}
		Diberikan segitiga $ABC$ dengan panjang $BC = 21$. Misalkan $D $ adalah titik tengah
		$BC $ dan $E $ adalah titik tengah $AD$. Misalkan pula bahwa $F $ adalah perpotongan $BE$
		dengan $AC$. Jika diketahui bahwa $AB $ menyinggung lingkaran luar segitiga $BFC$,
		hitunglah nilai dari $BF^2$.
	\end{soalbaru}
	
	\begin{soalbaru}
		Nilai dari $FPB(2002+2,2002^2+2,2002^3+2,...)$ adalah $\dots$
	\end{soalbaru}
	
	\begin{soalbaru}
		Carilah bilangan asli terbesar $n$ sehingga $n+10 \mid n^3+100$.
	\end{soalbaru}
	
	\begin{soalbaru}
		Carilah jumlah semua akar real dari persamaan $x+\dfrac{x}{\sqrt{x^2-1}}=2021.$
	\end{soalbaru}
	
	\begin{soalbaru}
		Negara Hankoku memiliki 5 kota. Menteri transportasi negara tersebut berencana untuk
		membangun 9 jalan identik dengan ketentuan bahwa setiap jalan menghubungkan
		tepat 2 kota. Jika diketahui bahwa setiap pasang kota terhubung oleh 0, 1, atau 2
		buah jalan, tentukan banyaknya konfigurasi pembangunan jalan yang mungkin.
	\end{soalbaru}
	
	\begin{soalbaru}
		Dipunyai segitiga $ABC $ dengan $AB = \sqrt{52}$, $AC=\sqrt{61}$ dan $BC = 9$. Misalkan $\gamma$
		adalah lingkaran yang melalui $A$ dan menyinggung $BC$ di titik tengahnya. Jika $\gamma$
		memotong lingkaran luar segitiga $ABC$ di $A$ dan $P$, tentukan panjang segmen $AP$.
	\end{soalbaru}
	
	\section{Esai}
	Jawablah soal-soal berikut dengan menyertakan cara pengerjaan atau argumentasinya.
	
	\begin{soalbaru}
	Untuk sembarang bilangan real positif $x$ dan $y$, buktikan bahwa $$\dfrac{1}{(1+\sqrt{x})^2}+\dfrac{1}{(1+\sqrt{y})^2} \ge \dfrac{2}{x+y+2}.$$
	\end{soalbaru}
	
	\begin{soalbaru} Apakah terdapat bilangan asli $x$ dan $y$ sedemikian sehingga $x^3+xy^3+y^2+3$ habis membagi $x^2+y^3+3y-1$?
	\end{soalbaru}
	
	\begin{soalbaru} Terdapat 20 anggota di suatu klub tenis yang menjadwalkan tepat 14 permainan antar dua orang diantara mereka dengan setiap anggota klub bermain minimal satu kali. Buktikan bahwa dalam pembagian ini, terdapat himpunan 6 permainan dengan 12 pemain yang berbeda.
	\end{soalbaru}
	
	\begin{soalbaru} Dua lingkaran berpotongan di $A$ dan $B$. Suatu garis melalui $B$ memotong lingkaran pertama di $C$ dan lingkaran kedua di $D$ ($B \neq C, B \neq D$).  Garis singgung lingkaran pertama yang melewati $C$ dan garis singgung lingkaran kedua melewati $D$, keduanya berpotongan di $M$. Melalui perpotongan $AM$ dan $CD$, suatu garis sejajar $CM$ memotong $AC$ di $K$. Buktikan bahwa $BK$ menyinggung lingkaran kedua.
	\end{soalbaru}
	
\end{document}
