\documentclass[11pt]{scrartcl}
\usepackage{graphicx}
\graphicspath{{./}}
\usepackage[sexy]{evan}
\usepackage[normalem]{ulem}
\usepackage{hyperref}
\usepackage{mathtools}
\hypersetup{
    colorlinks=true,
    linkcolor=blue,
    filecolor=magenta,      
    urlcolor=cyan,
    pdfpagemode=FullScreen,
    }
\usepackage[most]{tcolorbox}
\renewcommand{\dangle}{\measuredangle}

\renewcommand{\baselinestretch}{1.5}

\addtolength{\oddsidemargin}{-0.4in}
\addtolength{\evensidemargin}{-0.4in}
\addtolength{\textwidth}{0.8in}
% \addtolength{\topmargin}{-0.2in}
% \addtolength{\textheight}{1in} 


\setlength{\parindent}{0pt}

\usepackage{pgfplots}
\pgfplotsset{compat=1.15}
\usepackage{mathrsfs}
\usetikzlibrary{arrows}

\title{2024 Asia International Mathematical Olympiad Open Trial Simulation 2}
\author{Question Paper}
\date{17.00, 1 May 2024 - 23.59, 3 May 2024}

\begin{document}
\maketitle
\begin{center}
    \Huge
    \boxed{\textbf{Grade 3 - 4}}
\end{center}
\vspace{3cm}

\begin{flushright}
    \huge
   Time allowed: 90 minutes
\end{flushright}

\vspace{3cm}
\normalsize
Write down the answer according to the instruction given in questions. \textbf{The calculation result are guaranteed in the integers form.} If the answer is in surd form, represent the answer which is in the simplest form. \textbf{No need to write down any unit.}

\vspace{2cm}
Link to submit the answers: \url{https://forms.gle/mucWuTaanZhq814k8}

\pagestyle{plain}
\newpage
Section A – each question carries 4 marks

\hrulefill %linefill
\begin{enumerate}

\end{enumerate}


\newpage
Section B – each question carries 5 marks

\hrulefill %linefill
\begin{enumerate}[resume]
    
\end{enumerate}

\newpage
Section C – each question carries 7 marks

\hrulefill %linefill
\begin{enumerate}[resume]

\end{enumerate}

\end{document}