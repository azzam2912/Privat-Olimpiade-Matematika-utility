\documentclass[11pt]{scrartcl}
\usepackage{graphicx}
\graphicspath{{./}}
\usepackage[sexy]{evan}
\usepackage[normalem]{ulem}
\usepackage{hyperref}
\usepackage{mathtools}
\hypersetup{
    colorlinks=true,
    linkcolor=blue,
    filecolor=magenta,      
    urlcolor=cyan,
    pdfpagemode=FullScreen,
    }
\usepackage[most]{tcolorbox}
\renewcommand{\dangle}{\measuredangle}

\renewcommand{\baselinestretch}{1.5}

\addtolength{\oddsidemargin}{-0.4in}
\addtolength{\evensidemargin}{-0.4in}
\addtolength{\textwidth}{0.8in}
% \addtolength{\topmargin}{-0.2in}
% \addtolength{\textheight}{1in} 


\setlength{\parindent}{0pt}

\usepackage{pgfplots}
\pgfplotsset{compat=1.15}
\usepackage{mathrsfs}
\usetikzlibrary{arrows}

\title{Number Operations}
\author{Azzam Labib (IG: haxuv.world)}
\date{Wednesday, 8 May 2024}
\begin{document}
\maketitle


\begin{tcolorbox}[colback=red!10,colframe=red!75!black]
No calculator 0.0 and do it as efficient as possible i.e. don't count one by one, find the trick!
\end{tcolorbox}


\section{Commutative, Associative, Distributive Operations}
\begin{enumerate}
\item Evaluate $11 \times 34 + 66 \times 11$.
\vspace{8cm}\item Compute $8 \times 16 \times 3 \times 25$.
\vspace{8cm}\item Find the value of the following expression: $35 \times 99 + 11 \times 99 + 53 \times 99$.
\vspace{8cm}\item Find the value of the following expression: $45 \times 11 + 33 \times 11 + 53 \times 33$.
\vspace{8cm}\item Compute $7359 + 6379 - 359$.
\vspace{8cm}\item Evaluate
$(12469+31753-21002) \times (4869-4689+180)$.
\vspace{8cm}\item Evaluate $640 \times 128 + 640 \times 64 + 640 \times 8$.
\vspace{8cm}\item Evaluate $2016 \times 803 - 1016 \times 803 + 999 \times 803 + 803$.
\vspace{8cm}\item Find the value of $2019 \div 7 + 2018 \div 9 - 2005 \div 7 - 2000 \div 9$.
\vspace{8cm}\item Find the value ofthe following expression:
$50-49+48-47+46-45+\dots+4-3+2-1$.
\vspace{8cm}\item Find the value ofthe following expression:\\
$1+2+3-4+5+6+7-8+9+10+11-12+\dots+33+34+35-36$.
\end{enumerate}

\newpage
\section{Smart Operations}
\begin{enumerate}[resume]
    \item Defined that $a \otimes b \oplus c = a \times b + b \times c + c \times a$. Evaluate $8 \otimes 8 \oplus 8$.

    \vspace{8cm}\item Define the operation symbol $\oplus$ following this pattern:
    \begin{align*}
        1 \oplus 5 &= 1 + 2 + 3 + 4 - 5\\
        3 \oplus 8 &= 3 + 4 + 5 + 6 + 7 - 8\\
        5 \oplus 9 &= 5 + 6 + 7 + 8 - 9\\
        \dots& \text{(and so on ...)}
    \end{align*}
    Find the value of $(5 \oplus 10) - (4 \oplus 9).$
    
    \vspace{16cm}\item Defined that the symbol $\triangle$ as an operation satisfies the following conditions:
    \begin{enumerate}[(i)]
        \item When $a$ is odd, $a \triangle b = a \times 2 + b$,
        \item When $a$ is even, $a \triangle b = a \div 2 + b$,
    \end{enumerate}
    evaluate $[(21 \triangle 12) \triangle 9] \triangle 3$.

    \vspace{8cm}\item If $p \oplus q = p + 2 \times p + 3 \times q - 4 \times q + 5 \times p$, find the value of $3 \oplus 7$.

    \vspace{8cm}\item Define $F(a)$ as $3 \times a+1$. Find the value of $n$ if $F(F(n))=40$.
\end{enumerate}

\newpage
\section{Secret Arsenal: Difference of Two Squares}
\begin{remark*}
    Notes: The symbol $^2$ is read as "squared" meaning like this: $a^2 = a \times a$ ($a$ squared equal to $a$ times $a$.\\
    Example: $9^2 = 9 \times 9 = 81$.
\end{remark*}
\begin{lemma*}
    If you have two numbers $a$ and $b$ then $$a^2-b^2 = (a+b) \times (a-b)$$
    Example: 
    \begin{align*}
        13^2-7^2 &= (13+7) \times (13-7)\\
        13^2-7^2 &= 20 \times 6\\
        13^2-7^2 &= \boxed{120}.
    \end{align*}
\end{lemma*}
\begin{enumerate}[resume]
    \item Evaluate $23^2-22^2$.
    \vspace{8cm}\item Evaluate $(23 \times 17 + 9) \div (12 \times 8 + 4).$
    \vspace{8cm}\item Evaluate the value of $3 \times (2024^2 - 2^2) \div (2022 \times 2026)$.
    \vspace{8cm}\item Evaluate $20^2-18^2+16^2-14^2+\dots+4^2-2^2$.
    \vspace{8cm}\item Evaluate $\dfrac{1}{1 \times 2} + \dfrac{1}{2 \times 3} + \dfrac{1}{3 \times 4} + \dots + \dfrac{1}{9 \times 10}$.
\end{enumerate}
\end{document}