\documentclass[11pt]{scrartcl}
\usepackage{graphicx}
\graphicspath{{./}}
\usepackage[sexy]{evan}
\usepackage[normalem]{ulem}
\usepackage{hyperref}
\usepackage{mathtools}
\hypersetup{
    colorlinks=true,
    linkcolor=blue,
    filecolor=magenta,      
    urlcolor=cyan,
    pdfpagemode=FullScreen,
    }
\usepackage[most]{tcolorbox}
\renewcommand{\dangle}{\measuredangle}

\renewcommand{\baselinestretch}{1.5}

\addtolength{\oddsidemargin}{-0.4in}
\addtolength{\evensidemargin}{-0.4in}
\addtolength{\textwidth}{0.8in}
% \addtolength{\topmargin}{-0.2in}
% \addtolength{\textheight}{1in} 


\setlength{\parindent}{0pt}

\usepackage{pgfplots}
\pgfplotsset{compat=1.15}
\usepackage{mathrsfs}
\usetikzlibrary{arrows}

\title{Problems Drill again - Now not by Orland :)}
\author{Azzam Labib (Math IG: haxuv.world)}
\date{\today}
\begin{document}
\maketitle

\begin{enumerate}
    \item Tentukan digit ke 2024 dari kiri pada bilangan $1223334444555556666667777777\cdots$ ? (setiap bilangan $n$ muncul sebanyak $n$ kali pada bilangan tersebut)
    
    \item Berapa banyak cara memilih bilangan bulat berbeda $x$, $y$ dengan $0 \leq x$, $y \leq 19$ dan $5$ habis membagi $x + y$?
    
    \item $x$, $y$, $z$ tiga bilangan yang memenuhi $\left(x + \frac{1}{y}\right)\left(y + \frac{1}{z}\right)\left(z + \frac{1}{x}\right) = 19$. Jumlah dari semua nilai yang mungkin dari $\left(x + \frac{1}{z}\right)\left(z + \frac{1}{y}\right)\left(y + \frac{1}{x}\right)$ adalah \ldots
    
    \item Segitiga $ABC$ memiliki luas $120$ satuan. Titik $E$ dan $F$ dipilih di sisi $AC$ sehingga $AE = EF = FC$. Jika $D$ dan $G$ berturut-turut merupakan titik tengah $AB$ dan $EF$, luas segitiga $DFG$ adalah \ldots
    
    \item Nilai dari $$\left(\left(0,2\right)^{\left(0,5\right)^{(0,8)^{\left(1,1^{\dots}\right)}}}\right)^{-1}$$ adalah \ldots (perhatikan bahwa pangkatnya mengikuti barisan aritmatika)
    
    \item Parabola $y = x^2$ dan $x = y^2$ berpotongan di titik $O = (0, 0)$ dan $X = (1, 1)$. Lingkaran $\omega$, yang berpusat di $(0, 0)$ dan melewati $X$, berpotongan dengan parabola $y = x^2$ di titik $X$ dan $A$. Lingkaran $\omega$ berpotongan dengan parabola $x = y^2$ di titik $X$ dan $B$. Luas $XAB$ adalah \ldots
    
    \item Wadah pertama berisi $5$ kelereng merah dan $4$ kelereng biru. Wadah kedua berisi $7$ kelereng merah dan $9$ kelereng merah. Sebuah kelereng dipindahkan dari wadah pertama ke wadah kedua. Selanjutnya, sebuah kelereng diambil dari wadah kedua, jika peluang yang terambil bola merah adalah $\frac{a}{b}$. Tentukan nilai dari $a + b$ dimana $a$, $b$ bilangan bulat positif yang saling prima.

    \item Berapa banyak persegi panjang (panjang dan lebar tidak sama) yang dibentuk dari petak satuan pada papan catur ukuran $16 \times 16$ dimana sisi-sisinya paralel dengan sisi pada papan catur?
    
    \item $AT$ dan $BT$ menyinggung sebuah lingkaran berturut-turut di $A$ dan $B$, dan kedua garis bertemu tegak lurus di $T$. $Q$ dan $S$ titik berturut-turut di $AT$ dan $BT$, dan $S$ titik di lingkaran sehingga $QRST$ persegi panjang dengan $QT = 8$, $ST = 9$. Jari-jari lingkaran ini adalah \ldots
    
    \item Tentukan nilai dari 
    \begin{align*}
        \left(\frac{\sqrt{10+\sqrt{1}}+\sqrt{10+\sqrt{3}}+\sqrt{10+\sqrt{5}}+\cdots+\sqrt{10+\sqrt{99}}}{\sqrt{10-\sqrt{1}}+\sqrt{10-\sqrt{3}}+\sqrt{10-\sqrt{5}}+\cdots+\sqrt{10-\sqrt{99}}}\right)^{2}
    +\\ \left(\frac{\sqrt{10-\sqrt{1}}+\sqrt{10-\sqrt{3}}+\sqrt{10-\sqrt{5}}+\cdots+\sqrt{10-\sqrt{99}}}{\sqrt{10+\sqrt{1}}+\sqrt{10+\sqrt{3}}+\sqrt{10+\sqrt{5}}+\cdots+\sqrt{10+\sqrt{99}}}\right)^{2}.
    \end{align*}

\end{enumerate}

\end{document}

\end{document}