\documentclass[11pt]{scrartcl}
\usepackage{graphicx}
\graphicspath{{./}}
\usepackage[sexy]{evan}
\usepackage[normalem]{ulem}
\usepackage{hyperref}
\usepackage{mathtools}
\hypersetup{
    colorlinks=true,
    linkcolor=blue,
    filecolor=magenta,      
    urlcolor=cyan,
    pdfpagemode=FullScreen,
    }

\renewcommand{\dangle}{\measuredangle}

\renewcommand{\baselinestretch}{1.5}

\addtolength{\oddsidemargin}{-0.4in}
\addtolength{\evensidemargin}{-0.4in}
\addtolength{\textwidth}{0.8in}
% \addtolength{\topmargin}{-0.2in}
% \addtolength{\textheight}{1in} 


\setlength{\parindent}{0pt}

\usepackage{pgfplots}
\pgfplotsset{compat=1.15}
\usepackage{mathrsfs}
\usetikzlibrary{arrows}

\title{Exam 2 - Aljabar, Teori Bilangan, Kombinatorika} % Beginner
\date{\today}
\author{Compiled by Azzam}

\begin{document}

\maketitle
\begin{remark*}
    \begin{itemize}
    \item Kerjakan tanpa meminta bantuan orang lain.
    \item \textbf{SOAL URAIAN/ESAI: KERJAKAN DENGAN CARA MENDAPATKANNYA}.
    \item Cara boleh ditulis tangan atau diketik. 
    \item \textbf{Tidak boleh melihat resources selain ppt camp / modul basic yang disediakan oleh tutor.}
    \item Ujian terdiri dari 10 soal uraian/esai. Setiap soal bernilai 0-10 poin
    \end{itemize}
\end{remark*}
\section{Aljabar}
\begin{enumerate}
      \item Jumlah $n$ suku pertama suatu barisan aritmetika adalah 195. Jika suku pertama deret tersbut adalah $n$ dan suku ke-$n$ adalah 127, maka selisih barisan tersebut adalah...

      \item Nilai minimum dari $$\frac{a^2+2b^2+\sqrt{2}}{\sqrt{ab}}$$ dengan $a,b$ bilangan real positif adalah...

      \item Carilah seluruh bilangan asli $n$ sehingga $(n^2-5n+5)^{n^2-n-2}=1$.
\end{enumerate}

\section{Teori Bilangan}
\begin{enumerate}[resume]
    \item Tentukan banyaknya pasangan bilangan prima $(p,q)$ yang memenuhi $p^q-q^p = p+q$.

    \item Banyaknya pasangan terurut bilangan bulat $(a, b)$ dimana $1 \le a, b \le 2015$ yang memenuhi $b + 1|a$ dan $b|2016 - a$ adalah ...

    \item Jika $a,b,c,d$ adalah bilangan asli berbeda sehingga $abcd=2020$, maka nilai terkecil yang mungkin dari $\dfrac{a+b}{c+d}$ adalah...

    \item Carilah seluruh bilangan bulat positif terkecil yang memenuhi $x^2+x+1 \equiv 0 \mod 49$
\end{enumerate}

\section{Kombinatorika}
\begin{enumerate}[resume]
    \item Berapa banyak persegi panjang bukan persegi yang dibentuk dari petak satuan papan catur $16 \times 16$ dimana sisi-sisinya paralel dengan sisi papan catur?
    
    \item Ada berapa bilangan 5 digit yang setiap angkanya diulang setidaknya dua kali? Contoh: 20220 dan 33333 memenuhi syarat tersebut, tapi 25222 tidak.

    \item Sebuah pohon ajaib memiliki 25 buah lemon dan 30 buah jeruk. Pada setiap langkah, dipetik dua buah sekaligus. Jika dua buah yang jenisnya sama dipetik, maka tumbuh lagi satu buah jeruk. Jika dua buah yang jenisnya berbeda dipetik, maka tumbuh lagi satu buah lemon. Jika langkah-langkah itu dilakukan terus, tentukan buah apa yang terakhir kali tumbuh.
\end{enumerate}
	
	
\end{document}