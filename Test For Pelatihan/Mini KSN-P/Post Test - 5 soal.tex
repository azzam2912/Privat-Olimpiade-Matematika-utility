\documentclass[11pt]{scrartcl}
\usepackage[sexy]{evan}

%\addtolength{\textheight}{1.5in} 
\renewcommand{\baselinestretch}{1.35}



\begin{document}
	\title{Post Test} % Beginner
	\date{\today}
	\author{Compiled by Azzam}
	\maketitle
	
	\section{Esai}
	
	\begin{soalbaru}
	%ahsme 1989
	Misalkan $x$ adalah bilangan real yang dipilih secara acak diantara 100 dan 200. Jika $\floor{\sqrt{x}}=12$, carilah peluang terjadinya $\floor{\sqrt{100x}}=120$.
	\end{soalbaru}
	
	\begin{soalbaru}
	% geometri hayattir
	Diberikan segitiga $ABC$ siku-siku di $B$.Misalkan $D$ dan $E$ berturut-turut di segmen $AC$ dan $BC$ sedemikian sehingga $\angle EDC= 90^\circ$ dan $\angle CED = \angle AEB$. Jika $AE = 24$ dan $CE=EB$, tentukan panjang $DE$.
	\end{soalbaru}
	
	\begin{soalbaru}
	%ahsme 1988 prob 22
	Berapa banyak bilangan bulat $n$ sehingga terdapat segitiga lancip yang mempunyai sisi $10,24$, dan $n$?
	\end{soalbaru}

	\begin{soalbaru}
	%aime 1986
	Polinomial $1-x+x^2-x^3+\cdots+x^{16}-x^{17}$ dapat ditulis dalam bentuk $a_0+a_1y+a_2y^2+\cdots +a_{16}y^{16}+a_{17}y^{17}$, dimana $y=x+1$ dan semua $a_i$ adalah konstanta. Nilai dari $a_2$ adalah$\dots$
	\end{soalbaru}
 
	\begin{soalbaru}
	%bmc bary
	Segitiga $ABC$ mempunyai lingkaran luar $\Omega$. Garis bagi dalam $\angle A$ memotong sisi $BC$ dan $\Omega$ secara beruturut-turut di $D$ dan $L$ (selain $A$). Misalkan $M$ adalah titik tengah sisi $BC$. Lingkaran luar $\triangle  ADM$ memotong sisi $AB$ dan $AC$ di $Q$ dan $P$ (selain $A$) secara berutrut-turut. Jika $N$ adalah titik tengah segmen $PQ$, tunjukkan bahwa $MN \parallel AD$. 
	\end{soalbaru}
	
	\begin{soalbaru}
	%imo 1964
	Misalkan $a$,$b$, dan $c$ adalah sisi-sisi suatu segitiga. Buktikan bahwa$$\sum_{cyc} a^2(b+c-a) \le 3abc.$$
	\end{soalbaru}

\end{document}


