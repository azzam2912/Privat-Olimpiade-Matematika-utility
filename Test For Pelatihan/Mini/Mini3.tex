\documentclass[11pt]{scrartcl}
\usepackage{graphicx}
\graphicspath{{./}}
\usepackage[sexy]{evan}
\usepackage[normalem]{ulem}
\usepackage{hyperref}
\usepackage{mathtools}
\hypersetup{
    colorlinks=true,
    linkcolor=blue,
    filecolor=magenta,      
    urlcolor=cyan,
    pdfpagemode=FullScreen,
    }

\renewcommand{\baselinestretch}{1.5}

\addtolength{\oddsidemargin}{-0.4in}
\addtolength{\evensidemargin}{-0.4in}
\addtolength{\textwidth}{0.8in}

\setlength{\parindent}{0pt}

\usepackage{pgfplots}
\pgfplotsset{compat=1.15}
\usepackage{mathrsfs}
\usetikzlibrary{arrows}

\begin{document}
	\title{Daily Problems Set} % Beginner
	\date{}
	\author{Mini 3}
	\maketitle

	\section{Soal}
	\begin{enumerate}
            \item (OSK 2012) Banyaknya bilangan bulat $n$ yang memenuhi $$(n-1)(n-3)(n-5)\dots(n-2013)=n(n+2)(n+4)\dots (n+2012)$$ adalah \dots
        
            \item (OSK 2013) Diketahui $x_1,x_2$ adalah dua bilangan bulat berbeda yang merupakan akar-akar dari persamaan kuadrat $x^2+px+q+1=0$. Jika $p$ dan $p^2+q^2$ adalah bilangan-bilangan prima, maka nilai terbesar yang mungkin dari $x_1^{2013}+x_2^{2013}$ adalah \dots
            
            \item (OSK 2017) Misalkan $a,b,c$ bilangan real positif yang memenuhi $a+b+c=1$. Nilai minimum dari $\dfrac{a+b}{abc}$ adalah \dots
            
            \item (OSK 2017) Pada segitiga $ABC$ titik $K$ dan $L$ berturut-turut adalah titik tengah $AB$ dan $AC$. Jika $CK$ dan $BL$ saling tegak lurus, maka nilai minimum dari $\cot B + \cot C$ adalah \dots
	\end{enumerate}
	
\end{document}