\documentclass[11pt]{scrartcl}
\usepackage{graphicx}
\graphicspath{{./}}
\usepackage[sexy]{evan}
\usepackage[normalem]{ulem}
\usepackage{hyperref}
\usepackage{mathtools}
\hypersetup{
    colorlinks=true,
    linkcolor=blue,
    filecolor=magenta,      
    urlcolor=cyan,
    pdftitle={Overleaf Example},
    pdfpagemode=FullScreen,
    }

\renewcommand{\baselinestretch}{1.5}

\addtolength{\oddsidemargin}{-0.4in}
\addtolength{\evensidemargin}{-0.4in}
\addtolength{\textwidth}{0.8in}

\setlength{\parindent}{0pt}

\usepackage{pgfplots}
\pgfplotsset{compat=1.15}
\usepackage{mathrsfs}
\usetikzlibrary{arrows}

\begin{document}
    \title{Daily Problems Set} % Beginner
    \date{}
    \author{Mini 5}
    \maketitle
    
    \section{Soal}
    \begin{enumerate}
        \item (OSK 2017) Pada suatu kotak ada sekumpulan bola berwarna merah dan hitam yang secara keseluruhannya kurang dari 1000 bola. Misalkan diambil dua bola. Peluang terambilnya dua bola merah adalah $p$ dan peluang terambilnya dua bola hitam adalah $q$ dengan $p-q =\frac{23}{37}$. Selisih terbesar yang mungkin dari banyaknya bola merah dan hitam adalah \dots

        \item (OSK 2013) Sepuluh kartu ditulis dengan angka satu sampai sepuluh (setiap kartu hanya terdapat satu angka dan tidak ada dua kartu yang memiliki angka yang sama). Kartu - kartu tersebut dimasukkan kedalam kotak dan diambil satu secara acak. Kemudian sebuah dadu dilempar. Probabilitas dari hasil kali angka pada kartu dan angka pada dadu menghasilkan bilangan kuadrat adalah \dots

        \item (OSK 2017) Terdapat enam anak, $A, B, C, D, E$ dan $F$, akan saling bertukar kado. Tidak ada yang menerima kadonya sendiri, dan kado dari $A$ diberikan kepada $B$. Banyaknya cara membagikan kado dengan cara demikian adalah \dots

        \item (OSK 2013) Koefisien $x^{2013}$ pada ekspansi
        $$(1+x)^{4026}+x(1+x)^{4025}+x^2(1+x)^{4024}+\dots x^{2013}(1+x)^{2013}$$
        adalah \dots

        \item (AIME II 2000) Given that 
        $$\frac 1{2!17!}+\frac 1{3!16!}+\frac 1{4!15!}+\frac 1{5!14!}+\frac 1{6!13!}+\frac 1{7!12!}+\frac 1{8!11!}+\frac 1{9!10!}=\frac N{1!18!}$$
        find the greatest integer that is less than $\frac N{100}$.

        \item (OSK 2010) Polinom $P(x)=x^3-x^2+x-2$ mempunyai tiga pembuat nol yaitu $a,b,$ dan $c$. Nilai dari $a^3+b^3+c^3$ adalah \dots
        
        \item Diberikan polinomial $P(x)=x^4+ax^3+bx^2+cx+d$ dengan $a,b,c,$ dan $d$ konstanta. Jika $P(1)=10$, $P(2)=20$, dan  $P(3)=30$, maka nilai
        $$\dfrac{P(12)+P(-8)}{10}=\dots$$
    \end{enumerate}
\end{document}