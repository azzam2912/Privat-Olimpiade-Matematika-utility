\documentclass[11pt]{scrartcl}
\usepackage{graphicx}
\graphicspath{{./}}
\usepackage[sexy]{evan}
\usepackage[normalem]{ulem}
\usepackage{hyperref}
\usepackage{mathtools}
\hypersetup{
    colorlinks=true,
    linkcolor=blue,
    filecolor=magenta,      
    urlcolor=cyan,
    pdfpagemode=FullScreen,
    }

\renewcommand{\baselinestretch}{1.5}

\addtolength{\oddsidemargin}{-0.4in}
\addtolength{\evensidemargin}{-0.4in}
\addtolength{\textwidth}{0.8in}

\setlength{\parindent}{0pt}

\usepackage{pgfplots}
\pgfplotsset{compat=1.15}
\usepackage{mathrsfs}
\usetikzlibrary{arrows}

\begin{document}
	\title{Daily Problems Set} % Beginner
	\date{}
	\author{Mini 4}
	\maketitle

	\section{Soal}
	\begin{enumerate}
            \item (OSK 2016) Misalkan $x,y,z$ adalah bilangan real positif yang memenuhi $$3 \log_x (3y) = 3 \log_{3x} (27z) = \log_{3x^4} (81yz) \neq 0.$$ Nilai dari $x^5y^4z$ adalah \dots
    
            \item (Nesbitt's Inequality) Untuk bilangan real positif $a,b,c$ tentukan nilai minimum dari $$\dfrac{a}{b+c}+\dfrac{b}{c+a}+\dfrac{c}{a+b}.$$

            \item (OSK 2014) Diberikan segitiga $ABC$ yang sisi-sisinya tidak sama panjang sehingga panjang garis berat $AN$ dan $BP$ berturut-turut 3 dan 6. Jika luas segitiga $ABC$ adalah $3\sqrt{15}$, maka panjang garis berat ketiga $CM$ adalah \dots
        
            \item (OSK 2016) Pada segitiga $ABC$, titik $M$ terletak pada $BC$ sehingga $AB=7, AM=3, BM=5$, dan $MC=6$. Panjang $AC$ adalah \dots (pakai dalil cosinus ya kalo bisa)
            
            \item (IMO 1984) Buktikan bahwa $0 \leq yz + zx + xy – 2xyz \leq \dfrac{7}{27}$, dimana $x,y,z$ adalah bilangan riil positif yang memenuhi $x + y + z = 1$.
	\end{enumerate}

        Hint/petunjuk: Ada di halaman 3

        \newpage

        \textit{halaman ini memang kosong}

        \newpage

        \section{Hint}
        \begin{enumerate}
            \item Ingat soal log yang pernah kita bahaw, yang dibalik-balik itu. Kalo masih kesusahan, coba konversi jadi bentuk eksponensialnya. Terus nanti manipulasi aljabar tralala trilili dan ... selesai :)

            \includegraphics[scale=0.3]{waduh.jpg}

            \item Pakai AM-HM. Coba munculin bentuk $\frac{a+b+c}{a+b}$ dan kawan-kawannya.

            \item Bisa pake stewart + dalil cosinus. Mungkin ada kesebangunannya.

            \item Pakai dalil cosinus. Stewart juga bisa sih... Tapi coba pakai dalil cosinus yah, kan stewart mah kegampangan :)

            \item AM-HM juga, kalo pusing liat aja di internet pembahasannya :)) 
        \end{enumerate}
	
\end{document}