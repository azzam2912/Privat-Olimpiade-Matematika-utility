\documentclass{article}
\usepackage{amsmath}
\usepackage{graphicx}
\graphicspath{{./Mini KSK Simulation Problems/}}

\usepackage{enumitem}
\renewcommand{\baselinestretch}{1.5}
\addtolength{\oddsidemargin}{-1in}
\addtolength{\evensidemargin}{-1.5in}
\addtolength{\textwidth}{1.9in}

\addtolength{\topmargin}{-1in}
\addtolength{\textheight}{2in} 

\title{Mini KSK Simulation 15}
\author{50 menit}

\date{Jumat, 2 April 2021}

\begin{document}
	\maketitle
	
	\section{Kemampuan Dasar}
	Pada bagian ini setiap jawaban yang benar bernilai 2 poin dan setiap jawaban yang salah
	atau kosong bernilai nol.
	\begin{enumerate}
		\item Tentukan nilai eksak dari $\sqrt{\dfrac{11^4+100^4+111^4}{2}}$.
		
		\item Untuk $1\leq n\leq 2016$, berapa banyak bilangan bulat $n$ yang memenuhi: sisa $n$ saat dibagi $20$ lebih kecil daripada saat dibagi $16$.
		
		\item Sebuah segienam $ABCDEF$ titik-titik sudutnya berada di sebuah lingkaran. Misalkan $P, Q, R, S$adalah perpotongan $AB$ dan $DC$, $BC$ dan $ED$, $CD$ dan $FE$, $DE$ dan $AF$ secara berturut-turut. Selanjutnya diketahui $\angle BPC=50^{\circ}$, $\angle CQD=45^{\circ}$, $\angle DRE=40^{\circ}$, $\angle ESF=35^{\circ}$.
		Jika $T$ adalah perpotongan $BE$ dan $CF$, carilah $\angle BTC$.
		
		\item Misalkan pada papan catur $11 \times 11$ kita buat 5 persegi panjang yang tersusun atas kotak-kotak satuan papan catur tersebut. Berapa banyak cara membuat persegi panjang tersebut dimana tidak ada titik sudut persegi panjang tersebut yang berada pada pinggiran atau edge dari papan catur tersebut? (Catatan: Persegi panjang yang didapat dengan cara diputar atau refleksi dari persegi panjang lainnya dihitung berbeda)
	\end{enumerate}

\section{Kemampuan Lanjut}
Pada bagian ini setiap jawaban yang benar bernilai 4 poin, jawaban kosong bernilai nol
dan jawaban \textbf{salah} bernilai -1 (\textbf{minus satu})

\begin{enumerate}[resume]
		
		
		\item Misalkan $ABCD$ adalah sebuah segiempat dengan $AC=20$, $AD=16$. Misalkan $P$ pada segmen $CD$ sehingga segitiga $ABP$ dan $ACD$ kongruen. jika luas segitiga $APD$ adalah $28$, carilah luas segitiga $BCP$.\\
		\\
		\\
		
		\item Misalkan $a, b, c, d$ adalah bilangan-bilangan real yang memenuhi sistem persamaan
		$$(a+b)(c+d)=2$$
		$$(a+c)(b+d)=3$$
		$$(a+d)(b+c)=4$$
		carilah nilai minimum dari $a^2+b^2+c^2+d^2$. 
		
		\item Banyaknya pasangan terurut bilangan bulat $(a, b)$ dimana $1 \le a, b \le 2015$ yang memenuhi $b + 1|a$ dan $b|2016 - a$ adalah ...
		
		\item Sebuah NanoBot berada pada sebuah lingkaran dengan keliling 1 satuan yang tersusun atas 2016 bendera mikro. NanoBot diprogram untuk mengambil semua bendera tersebut dengan cara bergerak di keliling lingkaran tersebut. Carilah nilai terkecil $m$ dimana untuk sembarang posisi NanoBot dan bendera, NanoBot bisa mendapatkan seluruh bendera tersebut dengan bergerak sejauh $\le m$. (Catatan: NanoBot tidak perlu ke titik awal jika ia mendapat seluruh benderanya)
		
\end{enumerate}
\end{document}