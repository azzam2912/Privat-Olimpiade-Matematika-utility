\documentclass{article}
\usepackage{amsmath}
\usepackage{graphicx}
\graphicspath{{./Farhan/}}


\renewcommand{\baselinestretch}{1.5}
\addtolength{\oddsidemargin}{-1in}
\addtolength{\evensidemargin}{-1.5in}
\addtolength{\textwidth}{1.9in}

\addtolength{\topmargin}{-2in}
\addtolength{\textheight}{3in} 

\title{Mini KSK Simulation 6}
\author{60 menit}

\date{Jumat, 19 Februari 2021}

\begin{document}
	\maketitle
	
	Nomor 1-4 disesuaikan dengan kesulitan bagian 1 KSK (mudah-sedang). Nomor 5-8 disesuaikan dengan kesulitan bagian 2 KSK (sedang-sulit).
	\begin{enumerate}
		\item
		Misalkan fungsi $f$ memenuhi $f(x+4) \ge f(x) +4$ dan $f(x+1) \le f(x)+1$ untuk seluruh bilangan bulat non-negatif $x$. Jika $f(2) = 2021$, maka nilai dari $f(2021)$ adalah ...
		
		\item
		Diberikan satu koin yang tidak seimbang. Bila koin tersebut ditos satu kali, peluang muncul angka
		adalah $\frac14$. Jika ditos $n$ kali, peluang muncul tepat dua angka sama dengan peluang muncul tepat tiga
		angka. Nilai $n$ adalah ...
		
		\item
		Misalkan pada sebuah segitiga $ABC$ terdapat titik $D$ yang terletak pada perpanjangan $AB$ sedemikian sehingga $BD = \frac12 AD$. Diketahui $AB = 8, BC=7,$ dan $AC = 6$, maka nilai $CD^2$ adalah...
		
		\item
		Diketahui $x+y=\frac34 xy$ , $y+z=\frac{5}{12}yz$, dan $x+z=\frac23xz$ untuk $x,y,$ dan $z$ real positif. Tentukan nilai $x^3+y^3-z^3$ adalah...
		
		\item
		Untuk bilangan real $0\le x,y,z \le 1$, tentukan banyaknya pasangan terurut $(x,y,z)$ dimana $$\frac{x}{1+y+xz} + \frac{y}{1+z+xy}+\frac{z}{1+x+yz}=\frac{3}{x+y+z}$$
		
		\item
		Diberikan persegi $ABCD$. $P$ adalah titik yang terletak di dalam segitiga $ABC$ sedemikian sehingga $\angle CAP = \angle BCP = 15^\circ$. Titik $Q$ dipilih sedemikian sehingga $PC$ sejajar dengan $AQ$, $AP=CQ$, dan $AP$ tidak sejajar dengan $CQ$. Jika $N$ adalah titik tengah $PQ$, maka besar sudut $\angle CAN$ adalah...
		
		\item
		Misalkan $p$ dan $q$ adalah dua bilangan prima sehingga $p$ merupakan faktor dari $7q-1$ dan $q$ merupakan faktor dari $7p-1$. Jumlah dari seluruh nilai $p$ yang mungkin adalah...
		
		\item
		Barisan sebut-digit adalah barisan bilangan bulat positif, dimana
		setiap suku adalah banyaknya huruf yang diperlukan untuk membaca
		digit-digit suku sebelumnya dalam bahasa Indonesia. Sebagai contoh,
		suku setelah 2018, yang dibaca ‘dua-nol-satu-delapan’, adalah 17.
		Ada pengecualian khusus, dimana barisan tersebut akan berhenti pada
		suku dengan nilai ‘5’.
		Jika suku pertama barisan sebut-digit bernilai tak lebih dari 2018, maka
		berapa banyak suku maksimal barisan tersebut?
	\end{enumerate}
		

\end{document}