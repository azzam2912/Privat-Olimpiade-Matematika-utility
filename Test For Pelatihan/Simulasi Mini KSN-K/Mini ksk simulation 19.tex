\documentclass{article}
\usepackage{amsmath}
\usepackage{amssymb}
\usepackage{graphicx}
\graphicspath{{./Mini KSK Simulation Problems/}}

\usepackage{enumitem}
\renewcommand{\baselinestretch}{1.5}
\addtolength{\oddsidemargin}{-1.4in}
\addtolength{\evensidemargin}{-1.5in}
\addtolength{\textwidth}{2.5in}

\addtolength{\topmargin}{-1in}
\addtolength{\textheight}{2in} 

\title{Mini KSK Simulation 19}
\author{50 menit}

\date{Kamis, 14 April 2021}

\begin{document}
	\maketitle
	
	\section{Kemampuan Dasar}
	Pada bagian ini setiap jawaban yang benar bernilai 2 poin dan setiap jawaban yang salah
	atau kosong bernilai nol.
	\begin{enumerate}
		\item Jika untuk $f(x)=3x-4x^3$ untuk sembarang bilangan real $x$. Hitunglah nilai eksak dari $(f(\sin \theta))^2+(f(\cos \theta))^2$.
		
		\item Untuk suatu bilangan asli $n$, tentukan jumlah seluruh bilangan asli $m$ yang memenuhi $1^3+2^3+3^3+\dots+n^3=10^m$.
		
		\item Diberikan persegi $ABCD$ dimana titik $E$ dan $F$ berturut-turut terletak pada sisi $AB$ dan $DA$ dengan $BE=AF$. Jika $G$ adalah perpotongan $DE$ dan $CF$ dimana $FG=4$ dan $GC=9$, tentukan panjang $GE$.
		
		\item Kenichi dan Victoria mengikuti suatu lomba lari marathon. Saat Kenichi mencapai garis finish, banyaknya pelari yang finish setelah Kenichi sama dengan dua kali banyaknya pelari yang finish sebelum Kenichi. Lalu, banyaknya pelari yang finish sebelum Victoria ada sebanyak dua kali pelari yang finish setelah Victoria. Jika Victoria mencapai garis finish di posisi ke-10 setelah Kenichi (bukan ke-10 di secara keseluruhan), tentukan banyaknya orang yang mengikuti lomba tersebut.
	\end{enumerate}

\section{Kemampuan Lanjut}
Pada bagian ini setiap jawaban yang benar bernilai 4 poin, jawaban kosong bernilai nol
dan jawaban \textbf{salah} bernilai -1 (\textbf{minus satu})

\begin{enumerate}[resume]
		\item Tentukan bilangan bulat terbesar $n$ sehingga $\sqrt{4^{500}+4^{27}+4^{n}}\in \mathbb{Z}$.
		
		\item Dua belas siswa-siswi terpilih menjadi penari pada festival seni tari yang terdiri dari $n$ pertunjukan. Misalkan ada enam orang yang menari di setiap pertunjukan dan untuk setiap dua pertunjukan, paling banyak ada dua orang penari yang sama (two common dancer). Tentukan nilai terbesar yang mungkin untuk $n$.
		
		\item Diberikan sebuah setengah lingkaran dengan diameter $AB$ dan pusat $O$. Misalkan $C$ pada busur $AB$ sehingga $CO$ tegak lurus dengan $AB$. Titik $M$ pada $AC$ sehingga $AM=MC$ dan titik $N$ pada busur $BC$ sehingga $CN=NB$. Jika titik $P$ pada $AB$ sehingga $\angle PMN = 90^\circ$, dan $\angle MPN = \left(\dfrac{n}{2}\right)^\circ$, tentukan nilai $4n$.
		
		\item Tentukan bilangan bulat $n$ yang memenuhi $133^5+110^5+84^5+27^5=n^5$. (Ekspresi tersebut dijamin mempunyai solusi)
\end{enumerate}
\end{document}