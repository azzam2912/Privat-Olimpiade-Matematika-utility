\documentclass{article}
\usepackage{amsmath}
\usepackage{graphicx}


\renewcommand{\baselinestretch}{1.5}
\addtolength{\oddsidemargin}{-1in}
\addtolength{\evensidemargin}{-1.5in}
\addtolength{\textwidth}{1.9in}

\addtolength{\topmargin}{-1in}
\addtolength{\textheight}{1.9in} 

\title{Mini KSK Simulation 4}
\author{60 menit}

\date{Jumat, 12 Februari 2021}

\begin{document}
	\maketitle
	
	Soal sudah diusahakan terurut sesuai tingkat kesulitannya.
	\begin{enumerate}
		\item
		Notasikan $\tau(n)$ sebagai banyaknya faktor positif dari $n$. Nilai dari $$1 \times(-1)^{\tau(1)} + 2 \times (-1)^{\tau(2)} + 3 \times (-1)^{\tau(3)} + \dots + 200 \times (-1)^{\tau(200)}$$ adalah ...
		
		\item
		Banyak bilangan asli kurang dari 1000 yang memiliki jumlah digit habis dibagi 8 dan bilangan tersebut habis dibagi 3 adalah...
		
		\item Diberikan persegi panjang $ABCD$ dengan $AB = 3$ dan $BC = 6$. $M$ merupakan titik pada $BC$ sedemikian sehingga $\angle AMB = \angle AMD$. Besar sudut $\angle BAM$ adalah...
		
		\item
		Misalkan $a,b,c$ adalah tiga bilangan real positif yang memenuhi $a+2b+3c=36$. Tentukan nilai minimum dari $(a^3+35)(b^3+224)(c^3+243).$
		
		\item
		Sebut bilangan asli $k$ dengan \textit{sasageyo} jika untuk setiap bilangan asli $n$ berlaku $gcd(4n+1,kn+1)=1$. Jumlah seluruh bilangan \textit{sasageyo} yang lebih kecil dari 200 adalah...
		
		\item
		Diberikan segitiga lancip $ABC$ dimana $H$ adalah proyeksi $C$ terhadap $AB$. $M$ dan $N$ secara berturut-turut adalah proyeksi $H$ ke $BC$ dan $CA$. Diketahui pula bahwa titik pusat lingkaran luar segitiga $ABC$ terletak di garis $MN$ dan ppanjang jari-jari lingkaran luar tersebut adalah 2. Tentukan nilai $CH^2$.
		
		\item
		Untuk setiap bilangan asli $n$, definisikan $a_n$ sebagai banyaknya bilangan asli $n$ digit yang digit-digitnya hanya tersusun oleh angka 1 atau 2, serta tidak ada dua angka 2 yang bertetangga (terletak persis bersebelahan). Nilai dari $a_{10}$ adalah...
		
		\item
		Diketahui barisan Fibonacci ${F_n}$ dimana $F_0 = 0 , F_1 = 1,$ dan $F_n = F_{n-1} + F_{n-2}$ untuk $n \ge 1$. Jika diketahui bahwa $$\lim\limits_{n \rightarrow \infty} \dfrac{F_n}{F_{n-1}} = \dfrac{1+\sqrt{5}}{2}$$ dan dimisalkan $$N = \sum_{n = 0}^{\infty} \dfrac{1}{F_{2^n}}$$ nilai dari $4N^2-14N +40$ adalah...
	\end{enumerate}
		

\end{document}