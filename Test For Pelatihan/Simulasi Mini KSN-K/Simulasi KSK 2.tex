\documentclass[12pt]{extarticle}
\usepackage[legalpaper, portrait, margin=2in]{geometry}
\usepackage{amsmath}
\usepackage{amssymb}
\usepackage{graphicx}
\graphicspath{{./Mini KSK Simulation Problems/}}

\usepackage{enumitem}
\renewcommand{\baselinestretch}{1.5}
\addtolength{\oddsidemargin}{-1in}
\addtolength{\evensidemargin}{-1in}
\addtolength{\textwidth}{2in}

\addtolength{\topmargin}{-1.2in}
\addtolength{\textheight}{2.2in} 

\title{Simulasi 2 Kompetisi Sains Nasional Bidang Matematika SMA/MA Seleksi Tingkat Kota/Kabupaten Tahun 2021}
\author{120 menit}

\date{Rabu, 12 Mei 2021}

\begin{document}
	\maketitle
	
	\section{Kemampuan Dasar}
	Pada bagian ini setiap jawaban yang benar bernilai 2 poin dan setiap jawaban yang salah
	atau kosong bernilai nol.
	\begin{enumerate}
		\item Tentukan banyaknya bilangan asli di dalam himpunan $\{1,2,\dots,2021\}$ yang hasil jumlah digit-digitnya habis dibagi 3.
		
		\item Diketahui $p,q,r$ bilangan prima yang memenuhi $p+q=r$. Jika $p$ anggota $\{1,2,\dots,100\}$, tentukan nilai terbesar $p$ yang mungkin.
		
		\item Diberikan bilangan real positif $a,b,c$ yang memenuhi \begin{equation*}
			\begin{cases}
				ab+a+b=5 \\
				ac+a+c=9 \\
				bc+b+c=14. \\
			\end{cases}
		\end{equation*}
		Carilah nilai $a+b+c$.
		
		\item Diberikan persegi $ABCD$. Misalkan $E$ dan $F$ berturut-turut titik tengah dari sisi $AD$ dan $AB$. Misalkan pula $G$ merupakan titik potong antara garis $CE$ dan $DF$. Diketahui bahwa luas segitiga $DEG$ adalah 1. Hitunglah luas persegi $ABCD$.
		
		\item Tentukan banyaknya pasangan terurut bilangan asli $(k,l,m)$ se/demikian sehingga \\ $k+2l+m=2k+l-2m=2015$.
		
		\item Diketahui suatu bilangan real $a,b,c,$ dan $d$ memenuhi $$\frac{a-b}{c-d}=2 \text{ dan } \frac{a-c}{b-d}=3.$$ Tentukan hasil jumlah yang mungkin untuk $\dfrac{d-a}{b-c}$.
		
		\item Diberikan segitiga $ABC$ dengan $A_1,B_1,C_1$ berutut-turut merupakan titik tengah sisi $BC,CA,$ dan $AB$. Misalkan $D$ adalah kaki tegak lurus dari titik $A$ ke sisi $BC$ dimana $D$ terletak diantara $B$ dan $A_1$. Diketahui bahwa $B_1D$ tegak lurus dengan $A_1C_1$. Diketahui bahwa $A_1C_1 = 10$ dan $[ABC] = 150$. Tentukan nilai $2[A_1C_1D]$.
		
		\item Terdapat 17 kota yang dapat dituju dari Isekai City dengan Singa Air. Diketahui bahwa jika terdapat $k$ orang yang akan berangkat dari Isekai City ke 17 kota tujuan ini, pasti ada dua kota tujuan yang banyak penumpangnya sama (bisa jadi tidak terdapat penumpang sama sekali atau 0 penumpang). Tentukan nilai terbesar $k$ yang mungkin.
		
		\item Suatu segitiga memiliki panjang sisi berupa bilangan bulat positif. Diketahui bahwa 5 dan 10 merupakan panjang dua dari tiga sisinya dan $s$ merupakan sisi yang satunya lagi. Tentukan nilai terbesar $s$ yang mungkin.
		
		\item Misalkan $r,s,t$ adalah akar-akar dari suku banyak $p(x)=x^3-x-127$. Tentukan nilai dari $$\left (r+\frac{1}{s} \right)\left(s+\frac{1}{t}\right)\left(t+\frac{1}{r}\right).$$  
	\end{enumerate}

\section{Kemampuan Lanjut}
Pada bagian ini setiap jawaban yang benar bernilai 4 poin, jawaban kosong bernilai nol
dan jawaban \textbf{salah} bernilai -1 (\textbf{minus satu})

\begin{enumerate}
		\item Misalkan $ABCD$ merupakan sebuah persegi dengan titik pusat $O$. Titik $P$ diberikan di dalam persegi $ABCD$ sehingga $\angle OPB = 45^\circ$. Diketahui $PO=40\sqrt{2}$ dan $PA=20$. Tentukan panjang $PB$.
		
		\item Jika $a$ merupakan bilangan real terbesar yang memenuhi persamaan $$x^2+\frac{1}{x}=2,$$ tentukanlah nilai $a^2+a+1$.
		
		\item Diberikan himpunan $A = \{1,2,3,4,5,6,7\}$. Tentukan banyaknya fungsi $f:A \rightarrow A$ yang memenuhi $f(f(x))=x$ untuk setiap $x \in A$.
		
		\item Definisikan barisan $a_n$ dengan $a_n = \left \lceil (9+\sqrt{69})^n \right \rceil$ untuk $n$ asli. Tentukan banyaknya bilangan dari $a_1, a_2,\dots, a_{1000}$ yang habis dibagi 9.
		
		\item Diberikan persegi $ABCD$. Garis singgung dari titik $C$ terhadap lingkaran luar segitiga $ACD$ bertemu dengan perpanjangan $AB$ di $E$, dan titik $F$ adalah titik pertemuan kedua lingkaran luar segitiga $BCD$ dengan $DE$. Tentukan nilai $10 \times EF\times DF$, jika diketahui $EC=21$.
		
		\item Misalkan $x$ merupakan bilangan real sehingga $2^x+4^x+8^x=1$. Tentukan nilai dari $2^{x+1}-2^{4x}$.
		
		\item Bilangan asli $k$ disebut \textit{keras} jika memenuhi sifat bahwa tidak terdapat bilangan asli $a,b$ sehingga $a+b+ab=k$. Tentukan banyaknya bilangan \textit{keras} di dalam himpunan $\{1,2,\dots,100\}$.
		
		\item Sebuah turnamen tenis diikuti 6 pemain sehingga setiap pemain berhadapan satu sama lain tepat sekali. Untuk setiap permainan, pasti salah satu pemain menang dan pemain yang satunya lagi kalah. Misalkan $N$ adalah banyaknya kemungkinan hasil semua pertandingan sehingga tidak ada pemain yang tidak terkalahkan (semuanya pernah kalah). Tentukan tiga digit terakhir dari $N$.
		
		\item Diketahui bahwa untuk setiap bilangan real $x$ berlaku $ax^2+bx+c \ge 0$ dimana $a,b,c$ merupakan bilangan real yang tidak semuanya $0$ dan $a<b$. Tentukan nilai terkecil yang mungkin dari $$\frac{a+5b+3c}{b-a}.$$
		
		\item Misalkan $I$ adalah titik pusat lingkaran dalam segitiga $ABC$ dan $O$ adalah titik pusat \textit{excircle} terhadap titik $B$ (yaitu lingkaran yang menyinggung sisi $AC$, serta perpanjangan $AB$ dan $BC$ di luar segitiga $ABC$). Jika $BI=12$, $OI=18$, dan $BC=15$, hitunglah panjang $AB$.  
\end{enumerate}
\end{document}