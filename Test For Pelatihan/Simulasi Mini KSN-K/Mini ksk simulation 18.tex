\documentclass{article}
\usepackage{amsmath}
\usepackage{graphicx}
\graphicspath{{./Mini KSK Simulation Problems/}}

\usepackage{enumitem}
\renewcommand{\baselinestretch}{1.5}
\addtolength{\oddsidemargin}{-1in}
\addtolength{\evensidemargin}{-1.5in}
\addtolength{\textwidth}{1.9in}

\addtolength{\topmargin}{-2in}
\addtolength{\textheight}{5 in} 

\title{Mini KSK Simulation 18}
\author{50 menit}

\date{Senin, 12 April 2021}

\begin{document}
	\maketitle
	
	\section{Kemampuan Dasar}
	Pada bagian ini setiap jawaban yang benar bernilai 2 poin dan setiap jawaban yang salah
	atau kosong bernilai nol.
	\begin{enumerate}
		\item Diberikan persegi $ABCD$. Misalkan $P\in{AB},\ Q\in{BC},\ R\in{CD}\ S\in{DA}$ dan $PR\Vert BC,\ SQ\Vert AB$ dan misalkan $Z=PR\cap SQ$. Jika $BP=7,\ BQ=6,\ DZ=5$, Carilah panjang sisi persegi tersebut.
		
		\item Misalkan $1 \le a,b,c,d,e,f,g,h,i \le 9$ adalah bilangan asli yang berbeda-beda. Notasikan $N$ sebagai nilai maksimum dari $abc, def, ghi$. Tentukan nilai minimum $N$ yang mungkin.
		
		\item
		Tentukan banyaknya akar real dari polinomial $p(x)=x^5+x^4-x^3-x^2-2x-2.$.
		
		\item Terdapat sepuluh murid yang mengikuti ujian matematika. Diketahui bahwa setiap soal dikerjakan oleh tepat tujuh murid. Jika sembilan murid pertama masing-
		masing mengerjakan empat soal, tentukan banyaknya soal yang dikerjakan murid yang ke-10.
	\end{enumerate}

\section{Kemampuan Lanjut}
Pada bagian ini setiap jawaban yang benar bernilai 4 poin, jawaban kosong bernilai nol
dan jawaban \textbf{salah} bernilai -1 (\textbf{minus satu})

\begin{enumerate}[resume]
		\item Diberikan segitiga $ABC$ dengan titik pusat lingkaran luar $O$. Misalkan $D\in{AB},\ E\in{AC}$ sehingga $AD=8,\ BD=3$ dan $AO=7$. Jika $O$ adalah titik tengah $DE$, carilah panjang $CE$. 
		
		\item Diberikan grid $2 \times 100$ dimana setiap kotak satuannya diwarnai oleh biru atau merah dengan syarat: 
		\begin{itemize}
			\item Ada setidaknya satu kotak satuan yang diwarnai biru dan ada setidaknya satu kotak satuan yang diwarnai merah
			\item Semua kotak satuan berwarna merah \textit{connected}, begitu juga dengan kotak satuan biru.
		\end{itemize}
		Berapa banyak kemungkinan pewarnaan yang ada?
		
		Catatan: Dua kotak satuan disebut $\textit{connected}$ jika bertetangga.
		
		\item Carilah semua pasangan bilangan prima $(p,q)$ yang memenuhi $p^5-q^3=(p+q)^2$
		
		\item Tentukan banyaknya solusi real $x$ yang memenuhi $2x^7+x^{28}=3x^{21}.$
\end{enumerate}
\end{document}