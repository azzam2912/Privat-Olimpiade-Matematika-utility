\documentclass{article}
\usepackage{amsmath}
\usepackage{graphicx}
\graphicspath{{./Farhan/}}


\renewcommand{\baselinestretch}{1.5}
\addtolength{\oddsidemargin}{-1in}
\addtolength{\evensidemargin}{-1.5in}
\addtolength{\textwidth}{1.9in}

\addtolength{\topmargin}{-2in}
\addtolength{\textheight}{3in} 

\title{Mini KSK Simulation 7 }
\author{50 menit}

\date{Jumat, 22 Februari 2021}

\begin{document}
	\maketitle
	
	\section{Kemampuan Dasar}
	Pada bagian ini setiap jawaban yang benar bernilai 2 poin dan setiap jawaban yang salah
	atau kosong bernilai nol.
	\begin{enumerate}
		
		\item
		Barisan bilangan real $a_1,a_2,a_3\dots$ memenuhi $$\frac{na_1+(n-1)a_2+\dots+2a_{n-1}+a_n}{n^2} = 1$$ untuk setiap bilangan asli $n$. Nilai dari $a_1a_2a_3\dots a_{2019}$ adalah...
		\item
		Diberikan trapesium $ABCD$ dengan $AD$ sejajar $BC$. Diketahui $BD=1$, $\angle DBA = 23^\circ$, dan $\angle BDC = 46^\circ$. Jika perbandingan $BC:AD=9:5$, maka panjang sisi $CD$ adalah...
		
		
		
		\item
		Banyaknya cara memilih empat bilangan dari $\{1,2,3,\dots,15\}$ dengan syarat selisih sebarang dua bilangan paling sedikit 3 adalah...
		
		\item
		Banyaknya pasangan bilangan asli $(m,n)$ sehingga $FPB(m,n)=2$ dan $KPK(m,n)=1000$ adalah...
	\end{enumerate}
\section{Kemampuan Lanjut}
Pada bagian ini setiap jawaban yang benar bernilai 4 poin, jawaban kosong bernilai nol
dan jawaban \textbf{salah} bernilai -1 (\textbf{minus satu})
\begin{enumerate}
		\item
		Bilangan asli terkecil $n$ sehingga $\dfrac{(2n)!}{(n!)^2}$ habis dibagi 30 adalah...
	


		
		\item
		Nilai minimum dari $$\frac{a^2+2b^2+\sqrt{2}}{\sqrt{ab}}$$ dengan $a,b$ bilangan real positif adalah...
		
		\item
		Himpunan $S$ terdiri $n$ bilangan bulat dengan sifat berikut: Untuk setiap tiga anggota berbeda dari $S$ ada dua diantaranya yang hasil penjumlahannya merupakan anggota $S$. Nilai terbesar dari $n$ adalah...
		
		\item
		Diberikan segitiga $ABC$ dengan $\angle ABC = 135^\circ$ dan $BC>AB$. Titik $D$ terletak pada sisi $BC$ sehingga $AB=CD$. Misalkan $F$ titik pada perpanjangan sisi $AB$ sehingga $DF$ tegak lurus $AB$. Titik $E$ terletak pada sinar $DF$ sehingga $DE>DF$ dan $\angle ACE = 45^\circ$. Besar sudut $\angle AEC $ adalah ...
		
	\end{enumerate}
		

\end{document}