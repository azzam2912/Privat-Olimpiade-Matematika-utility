\documentclass{article}
\usepackage{amsmath}
\usepackage{graphicx}
\graphicspath{{./Farhan/}}


\renewcommand{\baselinestretch}{1.5}
\addtolength{\oddsidemargin}{-1in}
\addtolength{\evensidemargin}{-1.5in}
\addtolength{\textwidth}{1.9in}

\addtolength{\topmargin}{-2in}
\addtolength{\textheight}{3in} 

\title{Mini KSK Simulation 5}
\author{60 menit}

\date{Senin, 15 Februari 2021}

\begin{document}
	\maketitle
	
	Soal sudah diusahakan terurut sesuai tingkat kesulitannya.
	\begin{enumerate}
		\item
		Diketahui $ab+cd=1$ dan $abc=d+3$. Nilai dari $a^2b^2+d^2$ jika diketahui $|c|=1$ adalah...
		
		\item
		Diketahui bahwa $3^{27}+1$ memiliki tepat 2 faktor prima lebih dari 100. Jumlah dari ketiga bilangan tersebut adalah...
		
		\item 
		Delapan buah persegi identik dengan panjang sisi 5 satuan disusun di dalam sebuah persegi besar seperti pada gambar. Luas persegi besar adalah...\\
		\includegraphics[scale=0.3]{no 3 ksk mini simulation 5}
		
		\item
		Arima, Bernard, Chika, Donn, Ezi, Fanny, Grace adalah tujuh orang yang akan mengikuti suatu pertandingan. Setiap orang bertanding melawan orang lain sebanyak satu kali. Kemenangan mendapat 3 poin, seri mendapat 1 poin, dan kekalahan mendapat 0 poin. Diketahui pada akhir pertandingan, jumlah poin semua orang adalah 48. Grace mendapat peringkat 1 dengan selisih poin 1 dari orang yang berperingkat 2. Jika Grace tidak pernah seri, hasil kali poin seluruh tim adalah...
		
		\item
		Titik $M$ dan $P$ dipilih di sisi $AB$ dan $BC$ berturut-turut dari persegi $ABCD$ sehingga $AM=CP$. Lingkaran dengan diameter $DP$ memotong $CM$ di titik $K$. Besar sudut $\angle MKB$ adalah...
		
		\item
		 Jumlah seluruh bilangan prima $p$ sedemikian sehingga $1+p+p^2+p^3+p^4$ merupakan bilangan kuadrat adalah...
		 
		 \item
		 Tiga orang kawan bernama Abu, Babu, Cacu sedang bermain bola dengan aturan buatan mereka sendiri. Dua penyerang berusaha mencetak gol dan satu orang menjadi orang kiper. Orang yang berhasil mencetak gol di pertandingan ke-$n$ akan menjadi kiper di pertandingan ke-$n+1$. Sampai permainan selesai, Abu menjadi penyerang 12 kali, Babu menjadi penyerang 21 kali, dan Cacu menjadi kiper 8 kali. Siapakah yang mencetak gol keenam?
		 
		 \item
		 Sebuah bilangan $n$ disebut sebagai \textit{bagus} jika setiap dua digit berurutan dari $n$ merupakan bilangan prima. Sebagai contoh, 3719 adalah bilangan \textit{bagus} karena 37, 71, dan 19 adalah bilangan prima. Banyak bilangan \textit{bagus} 4 digit yang mungkin adalah...
	\end{enumerate}
		

\end{document}