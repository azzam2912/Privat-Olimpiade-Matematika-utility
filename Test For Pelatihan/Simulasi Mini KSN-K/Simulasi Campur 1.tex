\documentclass{article}
\usepackage{amsmath}
\usepackage{amssymb}
\usepackage{graphicx}
\graphicspath{{./Farhan/}}

\usepackage{enumitem}
\renewcommand{\baselinestretch}{1.5}
\addtolength{\oddsidemargin}{-1in}
\addtolength{\evensidemargin}{-1.5in}
\addtolength{\textwidth}{1.9in}

\addtolength{\topmargin}{-1in}
\addtolength{\textheight}{3in} 

\title{Simulasi Campur 1}
\date{Senin, 10 Mei 2021}
\begin{document}
	\maketitle
	
\section{Isian Singkat}
Alokasi waktu : 5 menit/soal

\begin{enumerate}
	\item Carilah solusi real $x$ yang memenuhi $x^{\left \lfloor x \right \rfloor} = 37.$
	
	\item Carilah pasangan bilangan bulat non-negatif $(a,b)$ yang memenuhi $(1+a!)(1+b!)=(a+b)!$

\end{enumerate}

\section{Esai}
Alokasi waktu: 20-30 menit/soal
\begin{enumerate}[resume]
	\item Jika $a \in \mathbb{R}$ memenuhi $0 \le a < \varepsilon$ untuk setiap $\varepsilon \in \mathbb{R^+}$, buktikan bahwa $a = 0.$
	
	\item Diberikan segitiga $ABC$ dengan titik bagi $I$ dan $\angle BAC = 90^\circ$. Garis $BI$ dan $CI$ memotong $AC$ dan $AB$ berturut-turut di $D$ dan $E$. Titik $P$ dan $Q$ terletak pada sisi $BC$ sehingga $IP \parallel AB$ dan $IQ \parallel AC$. Buktikan bahwa $BE+CD=2PQ$.
	
	\item Misalkan ada 15 anak masing-masing memiliki sebuah bola. Kelima belas anak ini berada pada suatu lapangan dimana jarak setiap dua anak berbeda-beda. Kemudian mereka serentak melemparkan bola mengenai anak yang terdekat dengannya. 
	\begin{enumerate}
		\item Buktikan bahwa ada anak yang tidak terkena bola.
		\item Buktikan bahwa tidak ada anak yang terkena lebih dari 5 bola.
	\end{enumerate}
	
\end{enumerate}
\end{document}

