\documentclass{article}
\usepackage{amsmath}
\usepackage{graphicx}
\graphicspath{{./Mini KSK Simulation Problems/}}

\usepackage{enumitem}
\renewcommand{\baselinestretch}{1.5}
\addtolength{\oddsidemargin}{-1in}
\addtolength{\evensidemargin}{-1.5in}
\addtolength{\textwidth}{1.9in}

\addtolength{\topmargin}{-1in}
\addtolength{\textheight}{2in} 

\title{Mini KSK Simulation 16}
\author{50 menit}

\date{Senin, 5 April 2021}

\begin{document}
	\maketitle
	
	\section{Kemampuan Dasar}
	Pada bagian ini setiap jawaban yang benar bernilai 2 poin dan setiap jawaban yang salah
	atau kosong bernilai nol.
	\begin{enumerate}
		\item Budi mempunyai nama panjang yang terdiri dari tiga kata dimana setiap initial dari ketiga tersebut berurutan sesuai abjad (cotntohnya: Armin Budi Prasetyo - ABP). Jika inisial setiap kata tak ada yang sama, ada berapa kemungkinan inisial nama panjang Budi?
		
		\item Tentukan nilai minimum dari $\dfrac12 (\cos 2x +1) + \dfrac{4}{1-\sin^2 x}$ dimana $0 < x < \frac{\pi}{2}$.
		
		\item Pada suatu segiempat konveks $ABCD$, diketahui bahwa kedua diagonalnya bertemu di $O$. Jika luas segitiga $AOB$ sama dengan luas segitga $COD$, carilah jumlah semua nilai yang mungkin untuk sudut terkecil yang dibentuk oleh $BC$ dan $AD$.
		
		\item Carilah seluruh bilangan asli $n$ sehingga $(n^2-5n+5)^{n^2-n-2}=1$.
	\end{enumerate}

\section{Kemampuan Lanjut}
Pada bagian ini setiap jawaban yang benar bernilai 4 poin, jawaban kosong bernilai nol
dan jawaban \textbf{salah} bernilai -1 (\textbf{minus satu})

\begin{enumerate}[resume]
		\item Akilah hanya memiliki sepatu berwarna hitam dan putih. Setiap hari, Bona selalu menggunakan sepatu dan hanya menggunakan satu jenis warna sepatu. Tentukan banyaknya cara Akilah memilih sepatu yang dia gunakan dalam sepuluh hari dengan syarat bahwa satu jenis warna sepatu tidak digunakan sebanyak tiga kali berturut-
		turut.
		
		\item Misalkan $ABC$ adalah segitiga dengan sisi-sisi $AB = 2, BC = 3\sqrt{3}, $ dan $ AC = \sqrt{13}$. Apabila $O_1$ dan $O_2$ adalah pusat segitiga sama sisi $ABC_1 $ dan $ BCA_1$, berturut-turut, dengan $C_1$ terletak pada sisi yang berbeda dengan $C$ terhadap $AB$ dan $A_1$ terletak pada sisi yang berbeda dengan $A$ terhadap $BC$, tentukan nilai $3(O_1O_2)^2$.
		
		\item Carilah banyaknya bilangan real $x$ yang memenuhi $2^x+2^{\frac{1}{x}}=4$.
		
		\item Diberikan bilangan asli $a,b,c,d$ yang memenuhi $a+b+c+d=63$. Tentukan nilai maksimum $ab+bc+cd$.
\end{enumerate}
\end{document}