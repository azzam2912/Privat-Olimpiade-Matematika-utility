\documentclass{article}
\usepackage{amsmath}
\usepackage{graphicx}
\graphicspath{{./Farhan/}}

\usepackage{enumitem}
\renewcommand{\baselinestretch}{1.5}
\addtolength{\oddsidemargin}{-1in}
\addtolength{\evensidemargin}{-1.5in}
\addtolength{\textwidth}{1.9in}

\addtolength{\topmargin}{-1in}
\addtolength{\textheight}{2in} 

\title{Mini KSK Simulation 11 (Harder Version :o)}
\author{60 menit}

\date{Senin, 8 Maret 2021}

\begin{document}
	\maketitle
\section{Kemampuan Lanjut}
Pada bagian ini setiap jawaban yang benar bernilai 4 poin, jawaban kosong bernilai nol
dan jawaban \textbf{salah} bernilai -1 (\textbf{minus satu})	
	\begin{enumerate}
		\item Jika $p$ dan $q$ adalah bilangan prima yang memenuhi $p^2|q^3+1$ dan $q^2|p^3+1$, tentukan jumlah semua nilai $p$ yang memenuhi.
		
		\item Pada segitiga $ABC$ diketahui $AB=4$, $BC=8$, dan $CA=6$. Misalkan $P$ dan $Q$ berturut-turut adalah titik singgung lingkaran dalam sgitiga $ABC$ yang berpusat di $O$ dengan sisi $AB$ dan $BC$. Jika $I$ adalah titik bagi segitiga $ABC$, dan $CI$ berpotongan dengan $PQ$ di $R$, tentukan nilai $\dfrac{QR}{RP}$.
		
		\item Tentukan jumlah seluruh bilangan real $x$ yang memenuhi $\sqrt{x}+\sqrt{x-\sqrt{1-x}}=1$.
		
		\item Sebuah dek kartu remi tersusun tanpa kartu King, Queen, dan Jack (jadi totalnya hanya 40 kartu). Sepasang kartu $\textit{matching}$ (pasangan dengan kartu berangka sama) dibuang dari dek dan tak dikembalikan lagi ke dek. Berapa peluang mendapatkan pasangan $\textit{matching}$ dari dek kartu yang tersisa? 
	\end{enumerate}



\begin{enumerate}[resume]
	\item Misalkan tiga bilangan prima $p$, $q$, dan $r$ memenuhi persamaan $p^2+pq+q^2=r^2+3.$ Jika $S=p+q+r$, tentukan jumlah semua nilai $S$ yang mungkin. 
	
	\item Pada sebuah papan catur $2021 \times 2021$, setiap kotak $1\times1$ diisi dengan salah satu dari empat huruf $G,E,O,M$. Papan catur tersebut dikatakan sebagai $\textit{sempurna}$ jika untuk setiap kotak $2\times2$ pada papan catur tersbut mengandung empat huruf yang berbeda. Berapa banyak konfigurasi papan catur $\textit{sempurna}$ yang mungkin?
	
	\item Tentukan nilai eksak dari $$\prod_{k=1}^{7}\cos \left ( \frac{k\pi}{15} \right ).$$
	
	\item Diberikan segitiga sama kaki $ABC$ dengan $AB=AC$ dan $\angle BAC = 15^\circ$. Titik $T$ berada di luar segitiga sehingga $\angle TBA = 135^\circ$ dan $TB = BA$. Jika $\angle TCB$ adalah $\textbf{sudut lancip}$, tentukan besar $\angle TCB$.
\end{enumerate}
\end{document}