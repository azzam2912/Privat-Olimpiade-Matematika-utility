\documentclass{article}
\usepackage{amsmath}
\usepackage{graphicx}
\graphicspath{{./Pelatihan 4/}}

\renewcommand{\baselinestretch}{1.5}
\addtolength{\oddsidemargin}{-1in}
\addtolength{\evensidemargin}{-1.5in}
\addtolength{\textwidth}{1.9in}

\addtolength{\topmargin}{-1in}
\addtolength{\textheight}{1.9in} 

\title{Mini KSK Simulation 2}
\author{alokasi waktu : 65 menit}
\date{Senin, 25 Januari 2021}

\begin{document}
	\maketitle
	
	\begin{enumerate}
		\item
		Pada persegi panjang $ABCD$, $AB = 12$ dan $BC=10$. Titik $E$ dan $F$ berada di dalam persegi panjang sehingga $BE=9$, $DF=8$, $BE$ sejajar $DF$, $EF$ sejajar $AB$, dan perpanjangan $BE$ memotong segmen $AD$. Bila panjang $EF$ dapat direpresentasikan dalam $m\sqrt n - p$ dimana $m,n,p$ adalah bilangan bulat positif dalam bentuk paling sederhana, tentukan nilai $m+n+p$.
		
		\item
		Pada suatu konferensi, ada 9 delegasi dari 3 negara berbeda dengan setiap negara mengirim tepat 3 delegasi. mereka duduk pada meja bundar yang memiliki 9 kursi. Misalkan kemungkinan setiap delegasi duduk bersebelahan dengan setidaknya 1 delegasi dari negara lain adalah $\frac{m}{n}$ dimana $m$ dan $n$ adalah bilangan bulat positif yang saling relatif prima, berapakah nilai $m+n$? (delegasi dari negara yang sama dianggap tidak dapat dibedakan)
		
		
		\item
		Sebuah lingkaran dengan pusat $O$ memiliki jari-jari 25. Tali busur $AB$ sepanjang 30 dan $CD$ sepanjang 14 berpotongan pada titik $P$. Jarak antara titik tengah kedua tali busur adalah 12. Jika $OP^2$ dapat dinyatakan dalam $\frac{m}{n}$ dimana $m$ dan $n$ saling relatif prima, tentukan nilai $m+n$.
		
		\item
		Pada sebuah segi-$n$ beraturan, diketahui kemungkinan bila 3 titik yang dipilih secara acak membentuk sebuah segitiga tumpul adalah $\frac{93}{125}$. Tentukan jumlah dari semua nilai $n$ yang mungkin.
		
		\item
		Titik $P$ berada pada diagonal $AC$ dari persegi $ABCD$ dengan $AP>CP$. Misalkan $O_1$ dan $O_2$ adalah titik pusat lingkaran luar segitiga $ABP$ dan $CDP$ secara berturut-turut. Diketahui $AB=12$, $\angle O_1PO_2 = 120^\circ$, dan $AP=\sqrt{a}+\sqrt{b}$ dimana $a$ dan $b$ adalah bilangan bulat positif. Tentukan nilai $a+b$.
		
		\item
		Carilah banyaknya bilangan bulat positif $n$ yang kurang dari 1000 sehingga ada sebuah bilangan real positif $x$ dengan $n=x \left \lfloor{x} \right \rfloor$.
		
		\item
		Azuma dan Berthold ingin meninggalkan isekai secepatnya. Untuk keluar secepat mungkin, mereka bergantian berjalan dan menunggangi satu-satunya kuda mereka, Uma. Pertama, Azuma mulai berjalan kaki sedangkan Berthold naik kuda. Diketahui di setiap km, ada pos istirahat. Saat Berthold mencapai pos pertama, ia mengikat Uma disana dan melanjutkan perjalanan dengan berjalan. Saat Azuma mencapai Uma, ia menungganginya hingga melewati Berthold, meninggalkan Uma pada pos berikutnya, dan mereka melanjutkan perjalanan dengan cara seperti tadi. Diketahui Uma, Azuma, dan Berthold memiliki kecepatan $6 km/jam$, $4 km/jam$, dan $2,5 km/jam$ secara berturut-turut. Misalkan Azuma dan Berthold bertemu untuk pertama kalinya di pos yang berjarak $n$ km dari isekai, dan mereka telah melakukan perjalanan selama $t$ menit. Tentukan nilai $n+t$.
		
		\item
		Misalkan $S$ adalah himpunan seluruh bilangan kuadrat sempurna yang 3 digit terakhirnya adalah 256. Misalkan $T$ adalah himpunan yang berisikan semua bilangan berbentuk $\frac{x-256}{1000}$ dengan $x$ adalah anggota $S$. Carilah nilai dari elemen ke-10 terkecil dari $T$.
		
		\item
		Misalkan $ABC$ adalah segitiga siku-siku di $C$. $D$ dan $E$ adalah titik-titik di $AB$ dengan $D$ terletak diantara $A$ dan $E$, serta $CD$ dan $CE$ membagi sudut $C$ menjadi 3 sudut sama besar. bila $\frac{DE}{BE} = \frac{8}{15}$ dan $\tan B$ dapat dinyatakan dalam bentuk $\frac{m}{n}\sqrt{p}$ dalam bentuk paling sederhana, tentukan nilai dari $m+n+p$.
	\end{enumerate}
\end{document}