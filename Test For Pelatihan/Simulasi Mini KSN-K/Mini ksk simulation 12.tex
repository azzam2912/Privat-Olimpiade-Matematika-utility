\documentclass{article}
\usepackage{amsmath}
\usepackage{graphicx}
\graphicspath{{./Farhan/}}

\usepackage{enumitem}
\renewcommand{\baselinestretch}{1.5}
\addtolength{\oddsidemargin}{-1in}
\addtolength{\evensidemargin}{-1.5in}
\addtolength{\textwidth}{1.9in}

\addtolength{\topmargin}{-1in}
\addtolength{\textheight}{2in} 

\title{Mini KSK Simulation 12}
\author{50 menit}

\date{Sabtu, 13 Maret 2021}

\begin{document}
	\maketitle
	
	\section{Kemampuan Dasar}
	Pada bagian ini setiap jawaban yang benar bernilai 2 poin dan setiap jawaban yang salah
	atau kosong bernilai nol.
	\begin{enumerate}
		\item 
		Carilah FPB dari $21n + 4$ dan $14n + 3$ untuk sembarang bilangan asli $n$.
	
		
		\item Tentukan bilangan bulat positif $m$ sehingga ada bilangan asli $x$ yang merupakan solusi dari\\ $x^2-2(m-480)x+4m+97=0$.
		
		\item Tiga buah bilangan 1447, 1005, dan 1231 mempunyai kesamaan: setiap bilangan itu diawali dengan angka 1, dan mempunyai tepat dua digit yang sama. Berapa banyak bilangan 4 digit yang seperti itu?
		
		\item Jika $$\left(1-\tan^2\frac{x}{2^{2011}}\right)\left(1-\tan^2\frac{x}{2^{2010}}\right)\left(1-\tan^2\frac{x}{2^{2009}}\right)\dots\left(1-\tan^2\frac{x}{2}\right) = 2^{2011}\sqrt{3}\tan \frac{x}{2^{2011}},$$ dan $\sin 2x = \sqrt{\dfrac{a}{b}}$ dimana $a$ dan $b$ adalah dua bilangan asli yang saling prima, tentukan nilai $a+b$.
	\end{enumerate}

\section{Kemampuan Lanjut}
Pada bagian ini setiap jawaban yang benar bernilai 4 poin, jawaban kosong bernilai nol
dan jawaban \textbf{salah} bernilai -1 (\textbf{minus satu})

\begin{enumerate}[resume]
		\item Tentukan nilai terkecil $p = \dfrac{a}{b}$ dimana $a$ dan $b$ adalah bilangan asli dengan $p > \dfrac{31}{17}$ dan $b<17$.
	\item Carilah seluruh bilangan real $x$ yang memenuhi $2^x = \sqrt{3^x}+1$.
	
	\item Misalkan $p_n(k)$ adalah banyaknya permutasi himpunan $\{1,2,3\dots,n\}$ yang mempunyai tepat $k$ $\textit{fixed points}$. Hitunglah nilai $p_6(1)+2p_6(2)+3p_6(3)+4p_6(4)+5p_6(5)+6p_6(6)$.
	
	\item Diketahui terdapat dua lingkaran $\omega_1$ dan $\omega_2$ berpotongan di dua titik salah satunya di $P$. $O_1$ dan $O_2$ berturut-turut adalah titik pusat dari $\omega_1$ dan $\omega_2$. Diketahui garis $O_1O_2$ memotong $\omega_1$ di $A$ dan $B$, serta memotong $\omega_2$ di $C$ dan $D$ dengan $B$ dan $C$ diantara $O_1$ dan $O_2$. Jika $AC=10$ dan $DB=24$ serta $\angle O_1PO_2 = 90^\circ$, tentukan panjang $BC$
	
\end{enumerate}
\end{document}