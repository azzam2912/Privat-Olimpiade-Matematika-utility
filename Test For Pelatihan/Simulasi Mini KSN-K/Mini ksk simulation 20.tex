\documentclass{article}
\usepackage{amsmath}
\usepackage{amssymb}
\usepackage{graphicx}
\graphicspath{{./Mini KSK Simulation Problems/}}

\usepackage{enumitem}
\renewcommand{\baselinestretch}{1.5}
\addtolength{\oddsidemargin}{-1in}
\addtolength{\evensidemargin}{-2in}
\addtolength{\textwidth}{1.8in}

\addtolength{\topmargin}{-1in}
\addtolength{\textheight}{2in} 

\title{Mini KSK Simulation 20}
\author{50 menit}

\date{Minggu, 2 Mei 2021}

\begin{document}
	\maketitle
	
	\section{Kemampuan Dasar}
	Pada bagian ini setiap jawaban yang benar bernilai 2 poin dan setiap jawaban yang salah
	atau kosong bernilai nol.
	\begin{enumerate}
		\item Tentukan nilai $x$ dengan $0 < x < 90$ yang memenuhi $$\tan x^\circ = \frac{\sin 12^\circ + \sin 24^\circ}{\cos 12^\circ + \cos 24^\circ}.$$
		
		\item Misalkan $S = \{0,1,2,3,4,5\}.$ Tentukan banyaknya fungsi $f:S\rightarrow S$ sehingga $n+f(n)$ merupakan bilangan genap.
		
		\item Shinichi dan Nagisa tinggal di kamar berbeda pada sebuah apartemen dengan 10 kamar di setiap lantai. Sistem penomoran kamar pada apartemen tersebut adalah sebagai berikut: kamar nomor 1-10 berada di lantai 1, kamar nomor 11-20 berada di lantai 2, dan seterusnya. Diketahui bahwa nomor kamar Shinichi sama dengan nomor lantai dari kamar Nagisa. Jika $n$ adalah jumlah nomor kamar Shinichi dan Nagisa, tentukan banyaknya nilai yang mungkin untuk $n$ dengan syarat $1 \le n \le 1000.$
		
		\item Lingkaran $O_1$ dan $O_2$ masing-masing mempunyai jari-jari 6 dan 8. Jarak kedua titik pusat lingkaran tersbut adalah 10. Lingkaran $O_3$ menyinggung dalam kedua lingkaran $O_1$ dan $O_2$ serta menyinggung garis yang menghubungkan pusat lingkaran $O_1$ dan $O_2$. Misalkan panjang jari-jari lingkaran $O_3$ dapat dinyatakan sebagai $a\sqrt{b}+c$ dimana $a,b,c$ adalah bilangan bulat, $b>0$ dan $b$ tidak habis dibagi kuadrat sempurna bilangan bulat selain 1. Tentukan nilai dari $|a+b+c|$.
		\newpage
	\end{enumerate}

\section{Kemampuan Lanjut}
Pada bagian ini setiap jawaban yang benar bernilai 4 poin, jawaban kosong bernilai nol
dan jawaban \textbf{salah} bernilai -1 (\textbf{minus satu})

\begin{enumerate}[resume]
		\item Tentukan banyaknya pasangan bilangan bulat nonnegatif $(a,b)$ sehingga $a+b^2$ dan $a^2+b$ keduanya merupakan kuadrat sempurna dan $0 \le a+b^2 \le a^2+b \le 2016.$
		
		\item Suatu bilangan asli $k$ disebut "luar biasa" jika dapat ditulis dalam bentuk $x^2-x+1$ untuk suatu bilangan rasional $x$. Tentukan banyaknya bilangan asli "luar biasa" dalam himpunan $\{1,2,\dots,999,1000\}$.
		
		\item Diberikan 3 buah gelas. Gelas pertama memiliki kapasitas 5 liter dan berisi air 3 liter. Gelas kedua memiliki kapasitas 7 liter dan berisi air 4 liter. Gelas ketiga memiliki kapasitas 9 liter dan berisi air 5 liter. Definisikan satu langkah penuangan sebagai berikut: 
		
		memilih gelas A dan menuang isinya ke gelas B hingga salah satu kondisi berikut terpenuhi:
		\begin{itemize}
			\item air di gelas A habis, atau
			\item air di gelas B penuh.
		\end{itemize}
	Berapa banyak langkah penuangan minimal yang diperlukan untuk mendapatkan dua buah gelas yang masing-masing berisi air 6 liter?
	
	\item Misalkan $ABC$ adalah segitiga lancip. Titik $D$,$E$, dan $F$ terletak pada sisi $BC$,$CA$, dan $AB$, berturut-turut, sedemikian sehingga $AD$,$BE$, dan $CF$ adalah garis tinggi segitiga $ABC$. Titik $H$ adalah titik tinggi segitiga $ABC$. Jika $DE=4$, $DF=3$,dan $EF=5$, tentukan panjang $AH$.
\end{enumerate}
\end{document}