\documentclass{article}
\usepackage{amsmath}
\usepackage{graphicx}
\graphicspath{{./Farhan/}}

\usepackage{enumitem}
\renewcommand{\baselinestretch}{1.5}
\addtolength{\oddsidemargin}{-1in}
\addtolength{\evensidemargin}{-1.5in}
\addtolength{\textwidth}{1.9in}

\addtolength{\topmargin}{-1in}
\addtolength{\textheight}{2in} 

\title{Mini KSK Simulation 10}
\author{50 menit}

\date{Jumat, 5 Maret 2021}

\begin{document}
	\maketitle
	
	\section{Kemampuan Dasar}
	Pada bagian ini setiap jawaban yang benar bernilai 2 poin dan setiap jawaban yang salah
	atau kosong bernilai nol.
	\begin{enumerate}
		\item Untuk dua dadu enam sisi yang tak seimbang, peluang mendapat angka 1,2,3,4,5,6 pada masing-masing dadu saat dilempar adalah 1:2:3:4:5:6. Berapakah peluang jumlah angka yang muncul pada kedua dadu adalah 7?
		
		\item 
		Didefinisikan $S_1 = 1$, $S_2 = 2+3$, $S_3 = 4+5+6$, $S_4 = 7+8+9+10$, dan seterusnya. Berapakah nilai $S_{21}$?
		
		\item Jika pasangan bilangan asli terurut $(x,y)$ memenuhi persamaan $$x^2y-x^2-3y-14=0,$$ tentukan nilai $x+y$.
		
		\item Pada segitiga $ABC$ diketahui $AB = 5$, $BC = 4$, dan $CA = 4$. Titik $D$ terletak di sisi $AB$ sehingga $CD$ adalah garis bagi $\angle ACB$. Jika lingkaran dalam $ADC$ dan $BDC$ berturut-turut memiliki panjang jari-jari $r_a$ dan $r_b$, nilai $\frac{r_a}{r_b}$ adalah...
	\end{enumerate}

\section{Kemampuan Lanjut}
Pada bagian ini setiap jawaban yang benar bernilai 4 poin, jawaban kosong bernilai nol
dan jawaban \textbf{salah} bernilai -1 (\textbf{minus satu})

\begin{enumerate}[resume]
	\item Pada segitiga $ABC$, diketahui $\angle ACB = 90^\circ$, $\angle ABC = 60^\circ$, dan $AB=10$. Jika titik $P$ dipilih secara acak di dalam segitiga $ABC$ dan perpanjangan $BP$ memotong sisi $AC$ di $D$, maka peluang $BD > 5\sqrt{2}$ adalah ...\\
	\\
	
	\item Sebuah operasi biner $*$ mempunyai properti $$a*(b*c) = (a*b)\times c \text{  dan  } a*a = 1$$
	untuk seluruh bilangan real non-negatif $a$,$b$, dan $c$ (dengan $\times$ menotasikan perkalian antar dua bilangan real). Jika pada $2016*(6*x)=100$, $x = \frac{p}{q}$ dimana $p$ dan $q$ adalah bilangan asli yang relatif prima, berapakah nilai $p+q$?
	
	\item Di sebuah meja dengan tempat duduk panjang warung Upssnormal, Angel tidak mau duduk di sebelah Budi atau Cynthia karena mereka sedang bertengkar. Ternyata Didi tahu kalau Emir lah yang mengadu domba mereka, sehingga Didi tak mau duduk di sebelah Emir. Jika syarat-syarat tersebut tak terpenuhi, maka akan ada perkelahian hebat yang bisa membuat mereka viral di jagat raya. Berapakah cara mengatur mereka duduk agar mereka tidak viral?
	
	\item Jika bilangan bulat $n>8$ adalah solusi $x^2-ax+b=0$ dan representasi $a$ dalam basis-$n$ adalah $18$, maka nilai $b$ dalam basis-$n$ adalah...
	
	
\end{enumerate}
\end{document}