\documentclass{article}
\usepackage{amsmath}
\usepackage{amssymb}
\usepackage{graphicx}
\graphicspath{{./Mini KSK Simulation Problems/}}

\usepackage{enumitem}
\renewcommand{\baselinestretch}{1.5}
\addtolength{\oddsidemargin}{-1in}
\addtolength{\evensidemargin}{-2in}
\addtolength{\textwidth}{1.8in}

\addtolength{\topmargin}{-2in}
\addtolength{\textheight}{3in} 

\title{Mini KSK Simulation 21}
\author{50 menit}

\date{Sabtu, 8 Mei 2021}

\begin{document}
	\maketitle
	
	\section{Kemampuan Dasar}
	Pada bagian ini setiap jawaban yang benar bernilai 2 poin dan setiap jawaban yang salah
	atau kosong bernilai nol.
	\begin{enumerate}
		\item Diberikan$ \dfrac{((3!)!)!}{4!} = k \cdot n!, $
		
		dimana $ k $ dan $ n $ adalah bilangan bulat positif dengan $ n $ sebesar mungkin. Carilah $ k + n. $
		
		\item Definisikan sebuah kata "trios" jika hanya tersusun dari huruf $A,B,C$ (setiap huruf tersebut boleh tidak muncul di kata "trios"). Huruf-huruf tersebut disusun dari kiri ke kanan. Jika $A$ tidak diikuti $B$, $B$ tidak diikuti $C$, $C$ tidak diikuti $A$, berapa banyak "trios" 10 digit yang mungkin?
		
		\item Diberikan $\log \sin x + \log \cos x = -1$ dan $\log (\sin x+\cos x) = \dfrac{1}{2}(\log n - 1)$. Carilah $n$.
		
		
		
		\item Diketahui $a,b,c$ adalah panjang sisi-sisi berbeda dari suatu segitiga dengan $R$ adalah radius lingkaran luarnya. Jika $R(b+c)=a\sqrt{bc}$, carilah seluruh kemungkinan sudut yang diapit oleh sisi $a$ dan $c$.
		
	\end{enumerate}

\section{Kemampuan Lanjut}
Pada bagian ini setiap jawaban yang benar bernilai 4 poin, jawaban kosong bernilai nol
dan jawaban \textbf{salah} bernilai -1 (\textbf{minus satu})

\begin{enumerate}[resume]
		\item Misalkan  $S = \{1,2,3,5,8,13,21,34\}$. Untuk setiap subset dua elemen dari $S$, diambil bilangan $a$ yang merupakan bilangan terbesar dari subset dua anggota tersebut. Carilah jumlah seluruh bilangan $a$ tersebut.
	
		\item Untuk sembarang bilangan real $a,b,c$ yang memenuhi $a^2+b^2+c^2=1$, carilah nilai minimum dari $$\sum_{cyc}^{} \frac{a^2}{1+2bc}.$$
		
		\item Diberikan segitiga $ABC$ dengan $\angle C = 60^\circ$ dan $D$ merupakan titik tengah $BC$. Jika $BC=4$, cari nilai maksimum dari $\tan \angle BAD$.
		
		\item Tentukan semua tupel solusi asli $(x,y,z)$ yang memenuhi $x^{2009}+2009!=y^z$, dengan $y$ adalah bilangan prima yang tak lebih dari 2009.

\end{enumerate}
\end{document}