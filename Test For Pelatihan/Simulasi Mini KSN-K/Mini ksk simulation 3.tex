\documentclass{article}
\usepackage{amsmath}
\usepackage{graphicx}


\renewcommand{\baselinestretch}{1.5}
\addtolength{\oddsidemargin}{-1in}
\addtolength{\evensidemargin}{-1.5in}
\addtolength{\textwidth}{1.9in}

\addtolength{\topmargin}{-1in}
\addtolength{\textheight}{1.9in} 

\title{Mini KSK Simulation 3}
\author{60 menit}

\date{Jumat, 5 Februari 2021}

\begin{document}
	\maketitle
	
	
	\begin{enumerate}
		\item Carilah semua pasangan bilangan prima $(p,q)$ yang memenuhi $p^3-q^3=(p+q)^2$
		
		\item
		Banyaknya pasangan bilangan bulat positif 2 digit $(x,y,z)$ yang memenuhi $2x^x=y^y+z^z$ adalah...
		
		\item
		Untuk setiap bilangan bulat positif $n$, didefinisikan $S_n$ adalah: $$ S_n = \sum_{k=1}^{2018} \left( \cos \frac{k!\pi}{2018} \right)^n$$
		Saat $n$ mendekati tak hingga, maka nilai $S_n$ akan mendekati ...
		
		
		\item
		Tentukan banyaknya akar real dari polinomial $p(x)=x^5+x^4-x^3-x^2-2x-2.$
		
		
			
		
		
		
		\item 
		Terdapat sepuluh murid yang mengikuti ujian matematika. Diketahui bahwa setiap soal dikerjakan oleh tepat tujuh murid. Jika sembilan murid pertama masing-
		masing mengerjakan empat soal, tentukan banyaknya soal yang dikerjakan murid yang ke-10.
		
		\item
		Diberikan segitiga $ABC$. Garis bagi sudut $A$ memotong $BC$ di titik $D$. Garis bagi
		$\angle ADB$ memotong $AB$ di titik $E$. Jika $BE = 7, AE = 14$, dan $DE$ sejajar $AC$,
		tentukan nilai $AD^2$.
		
		\item
		Misalkan $x$ dan $y$ adalah dua bilangan real berbeda yang memenuhi persamaan $y+6 = (x-6)^2$ dan $x+6 = (y-6)^2$ secara bersamaan. Tentukan nilai $x^3+y^3$.
		
		\item
		Tentukan hasil penjumlahan semua bilangan asli $n$ yang kurang dari 100 dan memenuhi $100|n^3+n^2+n+1$.
		
		\item
		Diberikan segitiga $ABC$ dimana terdapat titik $P$ di dalam segitiga sedemikian sehingga $\angle PBC = \angle PBA = 22^\circ$ dan $\angle PCA = 30^\circ, \angle PCB = 8^\circ$ Tentukan besar $\angle APB$.
		
		\item
		Misalkan $A_1, A_2, ... A_n$ adalah himpunan bagian dari $S = \{1,2,3,...,14 \}$ yang memiliki 4 elemen dan memenuhi syarat $|A_i \cap A_j| \le 1$ untuk semua $i,j = 1,2,...,n$ dan $i \neq j$. Carilah nilai maksimum $n$.
		
		
		
	\end{enumerate}
		

\end{document}