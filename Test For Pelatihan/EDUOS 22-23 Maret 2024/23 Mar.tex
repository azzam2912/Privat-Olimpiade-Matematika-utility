\documentclass[11pt]{scrartcl}
\usepackage{graphicx}
\graphicspath{{./}}
\usepackage[sexy]{evan}
\usepackage[normalem]{ulem}
\usepackage{hyperref}
\usepackage{mathtools}
\hypersetup{
    colorlinks=true,
    linkcolor=blue,
    filecolor=magenta,      
    urlcolor=cyan,
    pdfpagemode=FullScreen,
    }

\renewcommand{\dangle}{\measuredangle}

\renewcommand{\baselinestretch}{1.5}

\addtolength{\oddsidemargin}{-0.4in}
\addtolength{\evensidemargin}{-0.4in}
\addtolength{\textwidth}{0.8in}
% \addtolength{\topmargin}{-0.2in}
% \addtolength{\textheight}{1in} 


\setlength{\parindent}{0pt}

\usepackage{pgfplots}
\pgfplotsset{compat=1.15}
\usepackage{mathrsfs}
\usetikzlibrary{arrows}

\usepackage[most]{tcolorbox}

\title{Soal Campur}
\author{Compiled by Azzam}

\date{\today}
\begin{document}
\maketitle
\textbf{Waktu: 180 menit}

\section{Kemampuan Dasar}
\textbf{JAWABAN AKHIR SAJA.} Caranya kalau ada, tidak akan dinilai. Setiap soal bernilai 3 jika benar, 0 jika salah atau kosong.
\begin{enumerate}
    \item Misalkan $ABC$ adalah segitiga dengan panjang sisi $AB = 7$, $BC = 8$, dan $CA = 9$. Titik $D$ dan $E$ berturut-turut terletak pada sisi $BC$ dan $CA$, sehingga $AD$ dan $BE$ berturut-turut adalah garis bagi $\angle A$ dan $\angle B$. Proyeksi titik $C$ ke garis $AD$ dan $BE$ berturut-turut adalah $X$ dan $Y$. Hitung panjang $XY$.
    \item Jika $a$, $b$, dan $c$ adalah akar-akar dari polinomial $f(x) = x^3 - 17x + 1$ hitunglah nilai dari $(a + b + ab)(b + c + bc)(c + a + ca)$.
    \item Misalkan $N = 1020304050607$, tentukan dua digit terakhir dari $N^N$.
    \item Tentukan banyaknya bilangan asli yang tidak lebih dari 2018 yang dapat dinyatakan sebagai $x^2 + x + 1$ untuk suatu bilangan rasional positif $x$.
    \item Misalkan $ABC$ adalah segitiga lancip. Titik $D$, $E$, dan $F$ terletak pada sisi $BC$, $CA$, dan $AB$, berturut-turut, sedemikian sehingga $AD$, $BE$, dan $CF$ adalah garis tinggi segitiga $ABC$. Titik $H$ adalah titik tinggi segitiga $ABC$. Jika $DE = 8$, $DF = 15$, dan $EF = 17$ tentukan panjang $AH$.
    \item Tentukan nilai minimum dan maksimum dari $p$ sehingga dapat dibuat suatu susunan melingkar yang terdiri atas 8 koin emas, 13 koin perak, dan $p$ koin perunggu sehingga tidak ada koin sejenis yang bersebelahan.
    \item Misalkan fungsi $f$ mempunyai aturan bahwa untuk setiap bilangan bulat $x$, berlaku: $f(x) + f(x + 1) = x^2$. Jika diketahui bahwa $f(100) = 2018$, tentukan nilai dari $f(1)$.
    \item Diberikan $n$ adalah bilangan komposit dan $m$ adalah bilangan terbesar yang kurang dari $n$ dan habis membagi $n$. Tentukan banyaknya bilangan asli $n$ yang banyak faktornya tidak lebih dari 12 sehingga $n + m = p^a$ untuk suatu prima $p$ dan bilangan asli $a$.
    \item Sebuah papan catur berukuran $14 \times 14$ dibentuk dari 15 garis vertikal dan 15 garis horizontal. Tentukan banyaknya persegi panjang pada papan catur tersebut yang bukan merupakan persegi.
    \item Misalkan $ABC$ adalah segitiga dengan $BC = 9$ dan $AB = 7$. Titik $D$ terletak pada $BC$ dan $E$ pada $AC$ sehingga $AD$ adalah garis berat dan $BE$ adalah garis bagi dalam $\angle ABC$. Jika $EA = ED$, tentukan keliling segitiga $ABC$.

\end{enumerate}

\section{Kemampuan Lanjut Isian}
\textbf{JAWABAN AKHIR SAJA.} Caranya kalau ada, tidak akan dinilai. Setiap soal bernilai 4 jika benar, 0 jika salah atau kosong.
\begin{enumerate}[resume] 
    
    \item Tentukan banyaknya cara mengatur posisi duduk lima pasang anak kembar tidak identik pada meja bundar dengan 12 kursi identik di mana setiap pasang anak anak kembar harus selalu duduk bersebelahan.
    \item Jika $x^2 + 3x + 1 = 0$, tentukanlah nilai dari $x^7+\dfrac{1}{x^7}$.
    \item Empat buah dadu bermata empat berbentuk tetrahedron masing-masing berwarna merah, kuning, hijau, dan biru dilempar sekali secara bersamaan. Tentukan peluang kejadian bahwa jumlahan nilai yang muncul dari dadu merah dan kuning lebih besar daripada jumlahan nilai yang muncul dari dadu hijau dan biru.
    \item Diberikan segiempat talibusur $ABCD$ berdiameter $AC$. Titik $E$ terletak pada sinar $AC$ sehingga $A$ terletak di antara $C$ dan $E$, serta $\angle ABD = \angle ABE$. Diketahui bahwa $EB = 6$, $EA = 4$, dan $ED = 7$. Hitunglah panjang $AC$.
    \item Tentukan banyaknya tripel prima $(p, q, r)$ yang memenuhi $pqr = pq + 5qr + 21r$.
    \item Tentukan jumlah semua bilangan real $x$ yang memenuhi persamaan
        $$\sqrt[5]{\dfrac{x - 5}{23}} + \sqrt[5]{\dfrac{x - 4}{24}} + \sqrt[5]{\dfrac{x - 3}{25}} + \sqrt[5]{\dfrac{x - 2}{26}} = \sqrt[5]{\dfrac{x - 26}{2}} + \sqrt[5]{\dfrac{x - 25}{3}} + \sqrt[5]{\dfrac{x - 24}{4}} + \sqrt[5]{\dfrac{x - 23}{5}}$$
    \item Diberikan segitiga $ABC$ dengan panjang $BC = 36$. Misalkan $D$ adalah titik tengah $BC$ dan $E$ adalah titik tengah $AD$. Misalkan pula bahwa $F$ adalah perpotongan $BE$ dengan $AC$. Jika diketahui bahwa $AB$ menyinggung lingkaran luar segitiga $BFC$, hitunglah panjang $BF$.
    \item Diberikan fungsi yang memenuhi
        $$f(x) = \frac{x}{x^2 + 203}$$
        Jika $k$ adalah bilangan rasional positif takbulat sehingga $f(k) = f(\lfloor k \rfloor)$, tentukanlah nilai $k$ yang mungkin.
    \item Suatu kawasan memiliki enam buah pulau kecil. Pemerintah setempat memutuskan untuk membangun 18 jembatan identik di mana setiap jembatan menghubungkan tepat dua pulau kecil. Selain itu, setiap dua pulau kecil harus terkoneksi dengan satu atau dua jembatan. Tentukanlah banyaknya cara membangun jembatan-jembatan tersebut.
    \item Suatu fungsi $f(x)$ yang berdomain bilangan asli menyatakan banyaknya digit suatu bilangan yang dituliskan dalam basis 10. Sebagai contoh $f(423) = 3$, $f(8576) = 4$, dan $f(2) = 1$. Tentukanlah nilai dari $f(32^{2018}) + f(3125^{2018})$
\end{enumerate}

\section{Kemampuan Lanjut Esai}
\textbf{DENGAN CARANYA}, silakan tulis argumentasi beserta jawaban akhirnya. Setiap soal bernilai 6 jika benar, 0 jika salah atau kosong.
\begin{enumerate}[resume]
    \item Sebuah segitujuh beraturan tiap titik sudutnya diwarnai merah atau biru. Buktikan bahwa dapat dipilih tiga titik sudut dengan warna yang sama sehingga membentuk segitiga samakaki.
    \item Buktikan bahwa tidak terdapat pasangan bilangan bulat positif $(a, b)$ sehingga $a + b$, $a + 2b$, dan $2a + b$ ketiganya kuadrat sempurna.
    \item Sebuah polinomial $P(x)$ memenuhi $P(a) = P(b) = P(c) = 1$ untuk suatu bilangan bulat berbeda $a$, $b$, dan $c$. Buktikan bahwa $P(x)$ tidak mempunyai akar bilangan bulat.
    \item Tentukan semua fungsi kontinu yang memenuhi
        $$x^2f(y) + y^2f(x) = (x + y)f(x)f(y)$$
        untuk setiap bilangan real $x$ dan $y$.
    \item Diberikan segitiga samakaki $ABC$ dengan $AB = BC$. Titik $D$ dan $E$ berturut-turut pada sisi $AB$ dan $BC$ sehingga $AD + CE = DE$. Sebuah garis sejajar $BC$ melewati titik tengah segmen $DE$ memotong $AC$ di $F$. Tunjukkan bahwa $\angle DFE = 90^\circ$.
\end{enumerate}

\end{document}