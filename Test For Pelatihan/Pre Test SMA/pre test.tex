\documentclass{article}
\usepackage{amsmath}
\usepackage{amssymb}
\usepackage{graphicx}
\graphicspath{{./}}

\usepackage{enumitem}
\renewcommand{\baselinestretch}{1.5}
\addtolength{\oddsidemargin}{-1in}
\addtolength{\evensidemargin}{-1.5in}
\addtolength{\textwidth}{1.9in}

\addtolength{\topmargin}{-1in}
\addtolength{\textheight}{1.5in} 

\title{Pre-Test Pelatihan Olimpiade Matematika SMA}
\date{}

\setlength{\parindent}{0em}
\begin{document}
	\maketitle
	Karena tes ini bertujuan untuk mengetahui sejauh mana kemampuan anda, maka
	\textbf{diwajibkan untuk mengerjakan sendiri tanpa bantuan orang lain dan tanpa melihat pembahasan soal atau pembuktian teorema.}

\section{Benar/Salah}
Jawablah dengan "benar" atau "salah" yang sesuai dengan kebenaran pernyataan di soal. Masing-masing soal bernilai 1 poin.
\begin{enumerate}
	\item 0 adalah bilangan genap.
	
	\item $\sqrt{2}$ adalah bilangan irasional.
	
	\item $\sqrt{3}$ adalah bilangan real.
	
	\item Jika $A = \{x,y\}$ dan $B = \{x,y,z\}$, maka $A$ adalah himpunan bagian dari $B$.
	
	\item Pasti ada atau terdapat suatu segitiga dengan panjang sisi-sisi $1$, $2$, dan $3$.
	
	\item $\dfrac{1}{0}$ bernilai tak tentu.
	
	\item $\dfrac{0}{0}$ bernilai tak tentu.
	
	\item Bilangan berbentuk $\dfrac{a}{b}$ adalah bilangan rasional untuk sembarang bilangan real $a$ dan $b$.
	
	\item $\sqrt{x^2} = x$ untuk sembarang bilangan real $x$.
	
	\item $x^2-1=0$ mempunyai tepat dua solusi.
\end{enumerate}


\section{Isian Singkat}
Jawablah soal berikut dengan hasil akhirnya saja. Masing-masing soal bernilai 3 poin.
\begin{enumerate}[resume]
	\item Nilai paling sederhana dari $\dfrac{1}{2}+\dfrac{1}{6}+\dfrac{1}{12}+\dfrac{1}{20}+\dots +\dfrac{1}{9900}$ adalah ...
	
	\item Tentukan semua bilangan bulat $n$ sehingga $\dfrac{n+3}{n-1}$ merupakan bilangan bulat.
	
	\item Sebanyak 2015 koin dibagi menjadi 10 tumpukan. Tentukan banyak koin minimum pada tumpukan yang paling besar.
	
	\item Diketahui bilangan bulat positif $n$ merupakan hasil jumlah 2021 bilangan bulat berurutan. Tentukan nilai terkecil yang mungkin untuk $n$.
	
	\item Diberikan segi-8 beraturan $ABCDEFGH$. Tentukan besar sudut $\angle BCH$.
\end{enumerate}

\section{Esai Level 1}
Jawablah soal-soal berikut dengan menyertakan argumentasi atau cara Anda mendapatkannya. Setiap soal bernilai bilangan bulat positif dengan poin maksimum 7.

\begin{enumerate}[resume]
	\item Pada persamaan kuadrat $ax^2+bx+c=0$ dimana $a \neq 0$, tunjukkan bahwa $$x = \frac{-b \pm \sqrt{b^2-4ac}}{2a}.$$
	
	\item Diberikan segitiga $ABC$ dengan titik $K$, $L$ dan $M$ berturut-turut merupakan titik tengah sisi $BC$, $CA$, dan $AB$. Buktikan bahwa segitiga $KLM$ sebangun dengan segitiga $ABC$.
	
	\item Jika $m \equiv n \mod r$, buktikan bahwa $m^t \equiv n^t \mod r$ untuk sembarang bilangan asli $t$.
	
	\item Misalkan ada 5 orang siswa yang akan duduk mengelilingi sebuah meja bundar. Lalu, Kenichi dan So-Yeon datang dan akan duduk mengelilingi meja bundar tersebut. Namun, karena So-Yeon pemalu, ia hanya ingin duduk tepat disebelah Kenichi saja. Berapa banyak kemungkinan tempat duduk yang dapat terjadi pada ketujuh orang tersebut?
	
\end{enumerate}

\section{Esai Level 2}
Jawablah soal-soal berikut dengan menyertakan argumentasi atau cara Anda mendapatkannya. Setiap soal bernilai bilangan bulat positif dengan poin maksimal sesuai yang tertulis di soal.

\begin{enumerate}[resume]
	\item Untuk sembarang bilangan real positif $a,b,c$ kita akan membuktikan beberapa ketaksamaan berikut ini.
	\begin{enumerate}
		\item (2 poin) Tunjukkan bahwa $\dfrac{a+b}{2} \ge \sqrt{ab}$.
		
		\item (2 poin) Tunjukkan bahwa $\sqrt{ab} \ge \dfrac{2}{\frac{1}{a}+\frac{1}{b}}$.
		
		\item (2 poin) Tunjukkan bahwa $\sqrt{\dfrac{a^2+b^2}{2}} \ge \dfrac{a+b}{2}$.
		
		\item (4 poin) Tunjukkan bahwa $\dfrac{a+b+c}{3} \ge \sqrt[3]{abc}$\\
	
	\end{enumerate}

	\item Misalkan kita mempunyai ruangan berukuran $2 \times n$ dengan $n$ adalah bilangan asli (dalam meter). Kita akan memasang ubin pada ruangan tersebut dengan ubin-ubin yang hanya berukuran $1 \times 2$ dan $2 \times 1$.
	\begin{enumerate}
		\item (1 poin) Banyaknya cara pengubinan jika $n = 5$?
		\item (6 poin) Banyaknya cara pengubinan untuk sembarang bilangan asli $n$?
	\end{enumerate}

	\item (6 poin) Jelaskan dengan argumentasi kombinatorika atau aljabar kenapa $$(x+y)^n = \sum_{k=0}^{n} {n \choose k} x^{n-k}y^{k}$$ bisa berlaku?
	
	\item Misalkan segitiga $ABC$ dengan titik tinggi $H$ (titik perpotongan garis-garis tinggi segitiga $ABC$), mempunyai lingkaran luar $\Gamma$ dengan titik pusat $O$. Jika $AH$ memotong $BC$ di $D$, $BH$ memotong $CA$ di $E$, dan $CH$ memotong $AB$ di $F$. Misalkan $AD$ memotong lingkaran $\Gamma$ sekali lagi di $K$. Misalkan pula $M$ adalah titik tengah $BC$ sehingga sinar $HM$ memotong lingkaran $\Gamma$ di $T$ maka:
	\begin{enumerate}
		\item (2 poin) Tunjukkan bahwa segitiga $BEA$ dan $CFA$ sebangun.
		
		\item (2 poin) Buktikan bahwa $EFBC$ siklis (merupakan segiempat tali busur, dimana keempat titik itu berada pada satu lingkaran).
		
		\item (2 poin) Buktikan bahwa $AFHE$ siklis.
		
		\item (2 poin) Tunjukkan bahwa $HBKC$ merupakan sebuah layang-layang.
		
		\item (2 poin) Tunjukkan bahwa $HBTC$ merupakan sebuah jajar genjang.
		
		\item (2 poin) Tunjukkan bahwa $HD=DK$ dan $HM=MT$.
		
		\item (1 poin) Tunjukkan bahwa $AT$ adalah diameter lingkaran $\Gamma$.
	\end{enumerate}
	
	\item Diberikan dua bilangan asli $a$ dan $b$ dengan $a \ge b$. Misalkan $a$ dapat difaktorisasi prima menjadi\\ $a = p_1^{a_1}p_2^{a_2}\dots p_k^{a_k}$ dengan $k$ adalah bilangan asli, $p_1,p_2,\dots p_k$ adalah bilangan prima yang berbeda-beda, $a_1,a_2,\dots a_k$ adalah bilangan bulat non-negatif.
	\begin{enumerate}
		\item (3 poin) Tunjukkan bahwa $FPB(a,b) = FPB(a-b,b)$.
		\item (4 poin) Tunjukkan bahwa $FPB(a,b) \cdot KPK(a,b) = ab$.
		\item (2 poin) Tunjukkan bahwa banyak pembagi positif dari $a$ adalah $(a_1+1)(a_2+1)\dots(a_k+1)$
		\item (2 poin) Tunjukkan bahwa jumlah seluruh pembagi positif dari $a$ adalah $(1+p_1+p_1^2+\dots+p_1^{a_1})(1+p_2+p_2^2+\dots+p_2^{a_2})\dots (1+p_k+p_k^2+\dots+p_k^{a_k})$
	\end{enumerate}
	
\end{enumerate}
\end{document}

