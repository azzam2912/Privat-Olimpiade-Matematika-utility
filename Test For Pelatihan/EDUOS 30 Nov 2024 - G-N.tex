\documentclass[11pt]{scrartcl}
\usepackage[sexy]{evan}

%\addtolength{\textheight}{1.5in} 
\renewcommand{\baselinestretch}{1.35}



\begin{document}
	\title{Exam 4} % Beginner
	\date{Sabtu, 30 November 2024}
	\author{Compiled by Azzam}
	\maketitle

\textbf{Setiap soal bernilai bilangan bulat 0-20}\\
\textbf{Sertakan argumentasi / cara anda mendapatkan jawabannya}
	
\section{Soal Esai} 
		\begin{soalbaru}
		%2016 amc 12a probs 18
		Untuk bilangan bulat positif $n,$ bilangan $110n^3$ punya $110$ bilangan pembagi positif termasuk $1$ dan bilangan $110n^3.$ Berapa banyak pembagi positif dari $81n^4$?
		\end{soalbaru}
        
	\begin{soalbaru}
	% buku pelatos sma no 31
	Tentukan semua pasangan bilangan bulat non negatif $(x,y,z)$ dimana $x \le y$ yang memenuhi $x^2+y^2=3\times 2016^z+77$
	\end{soalbaru}

    	\begin{soalbaru}
	%ahsme 1989
	Misalkan $x$ adalah bilangan real yang dipilih secara acak diantara 100 dan 200. Jika $\floor{\sqrt{x}}=12$, carilah peluang terjadinya $\floor{\sqrt{100x}}=120$.
	\end{soalbaru}
	
	\begin{soalbaru}
	% geometri hayattir
	Diberikan segitiga $ABC$ siku-siku di $B$.Misalkan $D$ dan $E$ berturut-turut di segmen $AC$ dan $BC$ sedemikian sehingga $\angle EDC= 90^\circ$ dan $\angle CED = \angle AEB$. Jika $AE = 24$ dan $CE=EB$, tentukan panjang $DE$.
	\end{soalbaru}

    	\begin{soalbaru}
	% pelatos no 33 hal 70
	Bapak Presiden mempunyai sebuah segitiga $JKW$ di istana negara yang memiliki titik tengah $S$ pada segmen $KW$. Saat melihat huruf $S$, Pak Presiden terpikir untuk memberi para SJW di Twotter sebuah kuis: Jika $\angle SJW = \angle JKW$ dan $\angle KJS = (36+69)^\circ$, carilah besar dari sudut $\angle JKW$.
	\end{soalbaru}
\end{document}
