\documentclass[11pt]{scrartcl}
\usepackage{graphicx}
\graphicspath{{./}}
\usepackage[sexy]{evan}
\usepackage[normalem]{ulem}
\usepackage{hyperref}
\usepackage{mathtools}
\hypersetup{
    colorlinks=true,
    linkcolor=blue,
    filecolor=magenta,      
    urlcolor=cyan,
    pdfpagemode=FullScreen,
    }

\renewcommand{\dangle}{\measuredangle}

\renewcommand{\baselinestretch}{1.5}

\addtolength{\oddsidemargin}{-0.4in}
\addtolength{\evensidemargin}{-0.4in}
\addtolength{\textwidth}{0.8in}
% \addtolength{\topmargin}{-0.2in}
% \addtolength{\textheight}{1in} 


\setlength{\parindent}{0pt}

\usepackage{pgfplots}
\pgfplotsset{compat=1.15}
\usepackage{mathrsfs}
\usetikzlibrary{arrows}

\title{Exam 1 - Geometri dan Aljabar} % Beginner
\date{\today}
\author{Compiled by Azzam}

\begin{document}

\maketitle
\begin{remark*}
    \begin{itemize}
    \item Kerjakan tanpa meminta bantuan orang lain.
    \item \textbf{SOAL URAIAN/ESAI: KERJAKAN DENGAN CARA MENDAPATKANNYA}.
    \item Cara boleh ditulis tangan atau diketik. 
    \item \textbf{Tidak boleh melihat resources selain ppt camp / modul basic yang disediakan oleh tutor.}
    \item Ujian terdiri dari 10 soal uraian/esai. Setiap soal bernilai 0-10 poin
    \end{itemize}
\end{remark*}
\section{Aljabar}
\begin{enumerate}
      \item
		Diketahui $ab+cd=1$ dan $abc=d+3$. Nilai dari $a^2b^2+d^2$ jika diketahui $|c|=1$ adalah...


	\item
		Diketahui $x+y=\frac34 xy$ , $y+z=\frac{5}{12}yz$, dan $x+z=\frac23xz$ untuk $x,y,$ dan $z$ real positif. Tentukan nilai $x^3+y^3-z^3$ adalah...


		\item Nilai dari $\left(0,2 ^{\left(0,5 ^{\left(0,8 ^{...} \right)} \right)} \right)^{-1}$ adalah $\dots$ (perhatikan bahwa pangkatnya membentuk barisan aritmatika dengan beda 0,3)

    \item
		Misalkan $x$ dan $y$ adalah dua bilangan real berbeda yang memenuhi persamaan $y+6 = (x-6)^2$ dan $x+6 = (y-6)^2$ secara bersamaan. Tentukan nilai $x^3+y^3$.

		\item Tentukan jumlah seluruh bilangan real $x$ yang memenuhi $\sqrt{x}+\sqrt{x-\sqrt{1-x}}=1$.
\end{enumerate}

\
\section{Geometri}
\begin{enumerate}[resume]
		\item
		Diberikan segitiga $ABC$. Garis bagi sudut $A$ memotong $BC$ di titik $D$. Garis bagi
		$\angle ADB$ memotong $AB$ di titik $E$. Jika $BE = 7, AE = 14$, dan $DE$ sejajar $AC$,
		tentukan nilai $AD^2$.


		\item
		Pada persegi panjang $ABCD$, $AB = 12$ dan $BC=10$. Titik $E$ dan $F$ berada di dalam persegi panjang sehingga $BE=9$, $DF=8$, $BE$ sejajar $DF$, $EF$ sejajar $AB$, dan perpanjangan $BE$ memotong segmen $AD$. Bila panjang $EF$ dapat direpresentasikan dalam $m\sqrt n - p$ dimana $m,n,p$ adalah bilangan bulat positif dalam bentuk paling sederhana, tentukan nilai $m+n+p$.

	\item
	Pada segitiga $ABC$, diketahui $AB=125$, $AC=117$, dan $BC=120$. Garis bagi $\angle A$ memotong $BC$ di $L$ dan garis bagi $\angle B$ memotong $AC$ di $K$. Misalkan $M$ dan $N$ berturut-turut adalah kaki tegak lurus dari $C$ ke $BK$ dan $AL$. Carilah panjang $MN$.


		\item Diberikan persegi panjang $ABCD$ dengan $AB = 3$ dan $BC = 6$. $M$ merupakan titik pada $BC$ sedemikian sehingga $\angle AMB = \angle AMD$. Besar sudut $\angle BAM$ adalah...

  		\item
		Diberikan trapesium $ABCD$ dengan $AD$ sejajar $BC$. Diketahui $BD=1$, $\angle DBA = 23^\circ$, dan $\angle BDC = 46^\circ$. Jika perbandingan $BC:AD=9:5$, maka panjang sisi $CD$ adalah...
\end{enumerate}
	
	
\end{document}