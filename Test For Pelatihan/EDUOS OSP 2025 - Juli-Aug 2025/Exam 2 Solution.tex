\documentclass[11pt]{scrartcl}
\usepackage{graphicx}
\graphicspath{{./}}
\usepackage[sexy]{evan}
\usepackage[normalem]{ulem}
\usepackage{hyperref}
\usepackage{mathtools}
\usepackage{geometry}
\usepackage{asymptote}
\hypersetup{
    colorlinks=true,
    linkcolor=blue,
    filecolor=magenta,      
    urlcolor=cyan,
    pdfpagemode=FullScreen,
    }

\renewcommand{\dangle}{\measuredangle}

\renewcommand{\baselinestretch}{1.5}

\addtolength{\oddsidemargin}{-0.4in}
\addtolength{\evensidemargin}{-0.4in}
\addtolength{\textwidth}{0.8in}
% \addtolength{\topmargin}{-0.2in}
% \addtolength{\textheight}{1in} 


\setlength{\parindent}{0pt}

\usepackage{pgfplots}
\pgfplotsset{compat=1.15}
\usepackage{mathrsfs}
\usetikzlibrary{arrows}

\usepackage[most]{tcolorbox}



\title{Exam 2 - Solutions}
\author{Compiled by Azzam}

\date{\today}
\begin{document}
\maketitle

\renewcommand*\contentsname{Daftar Isi}
\tableofcontents

\newpage
\section{Soal 1} Diberikan persegi panjang $ABCD$. Titik $P$ adalah perpotongan garis $BC$ dengan garis yang melalui $A$ dan tegak lurus $AC$. Titik $Q$ pada segmen $CD$. Garis $PQ$ memotong garis $AD$ di $R$. Garis $BR$ memotong garis $AQ$ di $X$. Buktikan bahwa jika titik $Q$ bergerak sepanjang segmen $CD$, maka besar $\angle BXC$ konstan. 
\newline
\textbf{(Proposed by Wildan Bagus Wicaksono)}
\begin{center}
    \begin{asy}
        import olympiad;
        import geometry;
        size(200);

        pair A,B,C,D,P,Q,R,X,S;
        A = (-4,3);
        B = (4,3);
        C = (4,-3);
        D = (-4,-3);
        line l = perpendicular(A, line(A,C));
        P = intersectionpoint(l, line(B,C));
        Q = C/3+2*D/3;
        R = intersectionpoint(line(P,Q),line(A,D));
        X = intersectionpoint(line(B,R),line(A,Q));
        S = foot(A,P,R);
        dot("$A$", A, NW);
        dot("$B$", B, NE);
        dot("$C$", C, SE);
        dot("$D$", D, SW);
        dot("$P$", P, N);
        dot("$Q$", Q, SE);
        dot("$Q'$", Q, NW);
        dot("$R$", R, NW);
        dot("$X$", X, SW);
        dot("$X'$", X, SE);
        dot("$S$", S, SE);
        draw(A--C);
        draw(A--P);
        draw(A--S--B);
        draw(B--P--R--A);
        draw(A--X--C);
        draw(rightanglemark(A,X,C,10), red);
        draw(rightanglemark(A,S,Q,10), red);
        draw(rightanglemark(A,D,C,10), red);
        draw(rightanglemark(A,B,P,10), red);
        draw(B--R);
        draw(A--B--C--D--A);
        draw(circumcircle(triangle(A,B,C)), red);
        draw(circumcircle(triangle(A,B,S)), blue);
        draw(circumcircle(triangle(A,D,Q)));
        draw(circumcircle(triangle(B,S,Q)), dashed+blue);
        draw(B--S--Q--X--B, orange);
    \end{asy}
\end{center}
\subsection{Solusi}
\begin{proof}[\textbf{Solusi}]
Akan dibuktikan bahwa $ABCXD$ siklis. 

Misalkan $X'$ adalah perpotongan $BR$ dengan lingkaran $(ABCD)$. Misalkan pula $Q'$ perpotongan $AX'$ dengan $CD$.

Misalkan $S$ adalah proyeksi $A$ ke $PQ$. Karena $\angle ASQ = \angle ADQ = 90^\circ$ maka $ASQD$ siklis.

Berarti dengan \textit{power of a point}
$$RX'\cdot RB = RD\cdot DA = RQ\cdot QS.$$
yang menunjukkan bahwa $X'BSQ$ siklis. Berarti $\angle QX'B = \angle PSB$.

Sadari bahwa karena $\angle ABP = \angle PAC$ dan $\angle BCA = \angle PCA$ maka $\triangle BAP \sim \triangle BCA$ yang langsung mengakibatkan $\angle PAB = \angle ACB$.
Sekarang karena $\angle ASP = \angle ABP = 90^\circ$ maka $ASBP$ siklis, sehingga didapat $$\angle PSB = \angle PAB = \angle ACB = \angle AX'B = \angle Q'X'B.$$

Dari kedua fakta tersebut didapat $\angle QX'B = \angle Q'X'B$ yang menunjukkan $Q = Q'$ yang mengakibatkan $X = X'$. Soal terbukti.
\end{proof}

\newpage
\section{Soal 2} Misalkan $a,b,c > 0$ dan memenuhi $a+b+c=1$. Tunjukkan bahwa
\begin{align*}
\dfrac{ab}{1+c}+\dfrac{bc}{1+a}+\dfrac{ca}{1+b} \le \dfrac{1}{4}.
\end{align*}
\newline
\textbf{Source: entah ngambil dari Math Stack Exchange tapi lupa}
\subsection{Solusi}
\begin{proof}[\textbf{Solusi. }]
    Perhatikan dengan Cauchy-Schwarz kita punya
    \begin{align*}
        \cycsum \dfrac{ab}{1+c} &= \cycsum \dfrac{ab}{a+b+2c}\\
        &= \cycsum \dfrac{ab}{(c+a)+(b+c)}\\
        &\le \cycsum \dfrac{ab}{(1+1)^2}\left(\dfrac{1}{a+c} + \dfrac{1}{b+c}\right)\\
        &= \dfrac{1}{4} \cycsum \left(\dfrac{ab}{a+c} + \dfrac{ab}{b+c}\right)\\
        &= \dfrac{1}{4}\cycsum \left(\dfrac{ab}{b+c}+\dfrac{ca}{b+c}\right)\\
        &= \dfrac{1}{4}\cycsum \dfrac{ab+ca}{b+c}\\
        &= \dfrac{1}{4}\cycsum \dfrac{a(b+c)}{b+c}\\
        &= \dfrac{1}{4}\cycsum a\\
        &= \dfrac{1}{4}.
    \end{align*}
    dengan kesamaan saat $a=b=c$. Terbukti.
\end{proof}

\newpage
\section{Soal 3} Pada segitiga $ABC$, titik $D$ adalah kaki garis tinggi dari $C$. Titik $E$ dan $F$ pada $AC$ dan $BC$, berturut-turut $AE = AD$ dan $BF = BD$. Titik $S$ adalah refleksi $C$ pada titik pusat lingkaran luar $\triangle ABC$. Buktikan bahwa $SE=SF$.
\newline
\textbf{Source: Pertama dapet dari Pelatda Jakarta untuk OSK 2024}
    \begin{center}
\begin{asy}
import olympiad;
import geometry;
unitsize(1.3cm);
pair A,B,C,D,S,O,E,F;
A = (-3,0);
B = (3,0);
C = (-1,5);
D = foot(C,A,B);
O = circumcenter(A,B,C);
S = rotate(180,O)*C;
E = intersectionpoints(line(A,C),Circle(A,abs(A-D)))[0];
F = intersectionpoints(line(B,C),Circle(B,abs(B-D)))[0];
draw(A--B--C--cycle);
draw(C--D);
draw(C--S);
draw(E--D--F);
draw(A--S--B);
draw(E--S--F, dashed+red);
draw(Circle(O,0.05),red);
draw(circumcircle(triangle(A,B,C)));
draw(rightanglemark(B,D,C,12), purple);
draw(rightanglemark(S,A,C,12), blue);
draw(rightanglemark(C,B,S,12), blue);

label("$A$", A, SW);
label("$B$", B, SE);
label("$C$", C, N);
label("$D$", D, S);
label("$O$", O, NE);
label("$E$", E, NW);
label("$F$", F, NE);
label("$S$", S, SE);
\end{asy}
\end{center}
\subsection{Solusi 1}
    \begin{proof}[\textbf{Solusi 1. }]
Perhatikan bahwa $CS$ adalah diameter lingkaran luar $\triangle ABC$ sehingga $\angle CAS = \angle SBC = 90^\circ$. Oleh karena itu, dengan teorema Pythagoras di segitiga $EAS, SBF, CAS, CSB$ akan didapat
\begin{align*}
    CS^2 &= CS^2\\
    AC^2 + AS^2 &= BC^2+BS^2\\
    AC^2 - CD^2 + AS^2 &= BC^2 - CD^2 + BS^2\\
    AD^2 + AS^2 &= BD^2 + BS^2\\
    AE^2 + AS^2 &= BF^2 + BS^2\\
    SE^2 &= SF^2\\
    SE &= SF.
\end{align*}
Terbukti.
\end{proof}

\subsection{Solusi 2}
\begin{proof}[\textbf{Solusi 2. }] 
Perhatikan bahwa $\angle SAC = \angle CBS = 90^\circ$ karena $CS$ diameter lingkaran $(ACBS)$. Dari sini didapat 
$$\cos \angle SCF = \cos \angle SCB = \frac{BC}{CS}$$ 
dan 
$$\cos \angle ECS = \cos \angle ACS = \frac{AC}{CS}.$$

Dengan dalil cosinus pada $\triangle FCS$ serta teorema Pythagoras di $\triangle CDB$ didapat
\begin{align*}
    SF^2 &= CF^2 + CS^2 - 2\cdot CF \cdot CS \cos \angle SCF\\
    SF^2 &= CF^2 + CS^2 - 2\cdot CF \cdot BC\\
    SF^2 &= (BC-CF)^2 - BC^2 + CS^2\\
    SF^2 &= BF^2 - BC^2 + CS^2\\
    SF^2 &= BD^2 - BC^2 + CS^2\\
    SF^2 &= -CD^2 + CS^2
\end{align*}
Selanjutnya, dengan dalil cosinus pada $\triangle ECS$, serta teorema Pythagoras di $\triangle ADC$ didapat
\begin{align*}
    SE^2 &= CE^2 + CS^2 - 2\cdot CE \cdot CS \cos \angle ECS \\
    SE^2 &= CE^2 + CS^2 - 2\cdot CE \cdot AC\\
    SE^2 &= (AC-CE)^2 - AC^2 + CS^2\\
    SE^2 &= AE^2 - AC^2 + CS^2\\
    SE^2 &= AD^2 - AC^2 + CS^2\\
    SE^2 &= -CD^2 + CS^2\\
    SE^2 &= SF^2\\
    SE &= SF.
\end{align*}
Terbukti.
\end{proof}

\subsection{Solusi 3}
\begin{proof}[\textbf{Solusi 3. }] 
Perhatikan bahwa $\angle SAC = \angle CBS = 90^\circ$ karena $CS$ diameter lingkaran $(ACBS)$. Lalu, didapat $\angle BSC = \angle BAC = A$ dan $\angle CSA = \angle CBA = B$ sehingga
\begin{align*}
    SB &= \dfrac{BC}{\tan \angle BSC} = \dfrac{BC}{\tan A}\\
    SA &= \dfrac{AC}{\tan \angle CSA} = \dfrac{AC}{\tan B}
\end{align*}
Selanjutnya, observasi $\triangle BDC$ dan $\triangle ADC$ akan didapat
\begin{align*}
    BF &= BD = BC \cos B\\
    AE &= AD = AC \cos A
\end{align*}
Oleh karena itu dengan teorema Pythagoras
\begin{align*}
    SF^2 &= SB^2 + BF^2 = \dfrac{BC^2}{\tan^2 A} + BC^2 \cos^2 B\\
    SE^2 &= SA^2 + AE^2 = \dfrac{AC^2}{\tan^2 B} + AC^2 \cos^2 A
\end{align*}
Sekarang dengan properti trigonometri dan dalil sinus pada $\triangle ABC$ kita punya
\begin{align*}
    1 &= 1\\
    \dfrac{\cos^2 A}{\cos^2 A} &= \dfrac{\cos^2 B}{\cos^2 B}\\
    \dfrac{1-\sin^2 A}{\cos^2 A} &= \dfrac{1-\sin^2 B}{\cos^2 B}\\
    \dfrac{1}{\cos^2 A}+\dfrac{\sin^2 B}{\cos^2 B} &= \dfrac{1}{\cos^2 B}+\dfrac{\sin^2 A}{\cos^2 A}\\
    \dfrac{1}{\cos^2 A}+\tan^2 B &= \dfrac{1}{\cos^2 B}+\tan^2 A\\
    1 &= \dfrac{\frac{1}{\cos^2 A}+\tan^2 B}{\frac{1}{\cos^2 B}+\tan^2 A}\\
    \left(\dfrac{\sin A}{\sin B}\right)^2 &= \dfrac{\sin^2 A\left(\frac{1}{\cos^2 A}+\tan^2 B\right)}{\sin^2 B\left(\frac{1}{\cos^2 B}+\tan^2 A\right)}\\
    \left(\dfrac{BC}{AC}\right)^2 &= \dfrac{\tan^2 A + \sin^2 A \tan^2 B}{\tan^2 B + \sin^2 B \tan^2 A}\\
    BC^2(\tan^2 B + \sin^2 B \tan^2 A) &= AC^2(\tan^2 A + \sin^2 A \tan^2 B)\\
    \dfrac{BC^2(\tan^2 B + \sin^2 B \tan^2 A)}{\tan^2 A \tan^2 B} &= \dfrac{AC^2(\tan^2 A + \sin^2 A \tan^2 B)}{\tan^2 A \tan^2 B}\\
    \dfrac{BC^2}{\tan^2 A} + BC^2 \cos^2 B &= \dfrac{AC^2}{\tan^2 B} + AC^2 \cos^2 A\\
    SF^2 &= SE^2\\
    SF &= SE.
\end{align*}
Terbukti.
\end{proof}

\newpage
\section{Soal 4} Misalkan $Q(x) = a_0 + a_1x + \dots + a_nx^n$ adalah polinomial dengan koefisien-koefisien bilangan bulat, dan $0 \le a_i < 3$ untuk semua $0 \le i \le n$.

Diketahui bahwa $Q(\sqrt{3}) = 20 + 17\sqrt{3}$, hitunglah nilai $Q(2)$.
\newline
\textbf{HMMT 2017}

\subsection{Solusi}
\begin{proof}[\textbf{Solusi. }] 
Observasi
\[ Q(\sqrt{3}) = (a_0 + 3a_2 + 3^2a_4 + \dots) + (a_1 + 3a_3 + 3^2a_5 + \dots)\sqrt{3}. \]
Oleh karena itu, kita peroleh bahwa
\[ (a_0 + 3a_2 + 3^2a_4 + \dots) = 20 \quad \text{dan} \quad (a_1 + 3a_3 + 3^2a_5 + \dots) = 17. \]
Ini bersesuaian dengan ekspansi basis-3 dari 20 dan 17. Dari sini kita dapatkan bahwa $Q(x) = 2+2x+2x^3+2x^4+x^5$, maka $Q(2) = 86$.
\end{proof}

\end{document}