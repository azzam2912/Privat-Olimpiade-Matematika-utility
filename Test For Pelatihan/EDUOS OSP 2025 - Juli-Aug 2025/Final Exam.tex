\documentclass[11pt]{scrartcl}
\usepackage{graphicx}
\graphicspath{{./}}
\usepackage[sexy]{evan}
\usepackage[normalem]{ulem}
\usepackage{hyperref}
\usepackage{mathtools}
\hypersetup{
    colorlinks=true,
    linkcolor=blue,
    filecolor=magenta,      
    urlcolor=cyan,
    pdfpagemode=FullScreen,
    }

\renewcommand{\dangle}{\measuredangle}

\renewcommand{\baselinestretch}{1.5}

\addtolength{\oddsidemargin}{-0.4in}
\addtolength{\evensidemargin}{-0.4in}
\addtolength{\textwidth}{0.8in}
% \addtolength{\topmargin}{-0.2in}
% \addtolength{\textheight}{1in} 


\setlength{\parindent}{0pt}

\usepackage{pgfplots}
\pgfplotsset{compat=1.15}
\usepackage{mathrsfs}
\usetikzlibrary{arrows}

\usepackage[most]{tcolorbox}

\title{Final Exam}
\author{Compiled by Azzam}

\date{\today}
\begin{document}
\maketitle
\textbf{Waktu: 105 menit}

\section{Isian}
Silakan tulis \textbf{jawaban akhirnya saja} (esai atau coretan isian tidak akan dinilai). Setiap soal bernilai 5 jika benar dan 0 jika kosong atau salah.
\begin{enumerate}[resume]
    \item Giaa memilih secara acak satu bilangan dari himpunan $\{1,-1\}$, lalu menulis bilangan yang dipilihnya di papan tulis. Proses tersebut dilakukan 2021 kali. Jika $p$ adalah peluang di mana hasil kali ke-2021 angka di papan tulis adalah positif, tentukan nilai dari $100p$.
	
	
    \item Misalkan $a_n$ dan $b_n$ berturut-turut menyatakan banyaknya digit dari $4^n$ dan $25^n$. Tentukan nilai dari $a_1+a_2+\dots +a_{50}+b_{50}+\dots+b_1$.
	
	
    \item Diberikan segitiga $ABC$ dengan panjang sisi $BC,CA,$ dan $AB$ berturut-turut adalah $11,13,$ dan $20$. Misalkan $I$ adalah titik pusat lingkaran dalam segitiga $ABC$. Luas segitiga yang ketiga titik sudutnya adalah titik berat dari segitiga $ABI,BCI$ dan $ACI$ dapat dinyatakan dalam bentuk $\frac{a}{b}$, dengan $a$ dan $b$ adalah bilangan asli yang relatif prima. Tentukan nilai dari $a+b$.
	
    \item Sebuah papan berukuran $5 \times 5$ dibagi menjadi 25 kotak satuan. Bilangan-bilangan $1,2,3,4,$ dan $5$ dituliskan pada kotak satuan tersebut sedemikian sehingga setiap baris, setiap kolom, dan setiap diagonal yang mengandung lima bilangan hanya diisi oleh masing-masing dari bilangan tersebut sekali saja. Misalkan $S$ adalah jumlah bilangan pada empat kotak yang terletak di bawah kedua diagonal. Tentukan nilai terbesar $S$.
	
    \item Misalkan $X = 3 + 33 + 333 + \dots + \underbrace{333 \dots 333}_{2016 \text{ angka } 3}$ dan $Y$ adalah jumlah digit dari $X$. Tentukanlah nilai $Y$.
	
\end{enumerate}
\section{Esai}
Silakan \textbf{tulis argumentasi beserta jawaban akhirnya}. Setiap soal bernilai bilangan bulat dari 0 sampai 25 jika benar.
\begin{enumerate}[resume]
    \item Diberikan $\triangle ABC$ dimana $A',B',C'$ berturut-turut adalah pencerminan $A,B,C$ terhadap $BC,CA,AB$. Perpotongan lingkaran luar $\triangle ABB'$ dan $\triangle ACC'$ adalah $A_1$. Definisikan $B_1$ dan $C_1$ secara serupa. Buktikan bahwa $AA_1,BB_1,$ dan $CC_1$ konkuren (bertemu di satu titik).

    \item Diberikan polinomial $f(x)=x^n+a_1x^{n-1}+\dots+a_{n-1}x+a_n$ dengan koefisien bilangan bulat. Lalu, diketahui bahwa ada empat bilangan bulat berbeda $a,b,c,$ dan $d$ sehingga $f(a)=f(b)=f(c)=f(d)=5$. Tunjukkan bahwa tak ada bilangan bulat $k$ sehingga $f(k)=8$.

    \item Setelah show di teater TGR48, ketiga member: Intan, Giaa, dan Fritzy tiba-tiba terpikir untuk memainkan permainan "Dialog Dengan Kenari".
    
    Ada tiga ember kosong di atas meja. Intan, Giaa, dan Fritzy meletakkan kenari satu per satu ke dalam ember secara bergantian, dengan urutan yang ditentukan oleh Giaa di awal permainan. Dengan demikian, Intan meletakkan kenari di ember pertama atau kedua, Giaa meletakkan di ember kedua atau ketiga, dan Fritzy meletakkan di ember pertama atau ketiga. Pemain yang setelah gilirannya membuat ada tepat 2023 kenari di salah satu ember dinyatakan sebagai pemain yang kalah. Tunjukkan bahwa Intan dan Fritzy dapat bekerja sama sehingga membuat Giaa kalah.
\end{enumerate}

\end{document}