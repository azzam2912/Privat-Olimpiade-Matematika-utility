\documentclass[11pt]{scrartcl}
\usepackage{graphicx}
\graphicspath{{./}}
\usepackage[sexy]{evan}
\usepackage[normalem]{ulem}
\usepackage{hyperref}
\usepackage{mathtools}
\hypersetup{
    colorlinks=true,
    linkcolor=blue,
    filecolor=magenta,      
    urlcolor=cyan,
    pdfpagemode=FullScreen,
    }

\renewcommand{\dangle}{\measuredangle}

\renewcommand{\baselinestretch}{1.5}

\addtolength{\oddsidemargin}{-0.4in}
\addtolength{\evensidemargin}{-0.4in}
\addtolength{\textwidth}{0.8in}
% \addtolength{\topmargin}{-0.2in}
% \addtolength{\textheight}{1in} 


\setlength{\parindent}{0pt}

\usepackage{pgfplots}
\pgfplotsset{compat=1.15}
\usepackage{mathrsfs}
\usetikzlibrary{arrows}

\usepackage[most]{tcolorbox}

\title{Exam 2}
\author{Compiled by Azzam}

\date{\today}
\begin{document}
\maketitle
\textbf{Waktu: 105 menit}\\
Silakan tulis argumentasi beserta jawaban akhirnya. Setiap soal bernilai bilangan bulat dari 0 sampai 25 jika benar.
\begin{enumerate}[resume]
\item Diberikan persegi panjang $ABCD$. Titik $P$ adalah perpotongan garis $BC$ dengan garis yang melalui $A$ dan tegak lurus $AC$. Titik $Q$ pada segmen $CD$. Garis $PQ$ memotong garis $AD$ di $R$. Garis $BR$ memotong garis $AQ$ di $X$. Buktikan bahwa jika titik $Q$ bergerak sepanjang segmen $CD$, maka besar $\angle BXC$ konstan.

\item Misalkan $a,b,c > 0$ dan memenuhi $a+b+c=1$. Tunjukkan bahwa
\begin{align*}
\dfrac{ab}{1+c}+\dfrac{bc}{1+a}+\dfrac{ca}{1+b} \le \dfrac{1}{4}.
\end{align*}

\item Pada segitiga $ABC$, titik $D$ adalah kaki garis tinggi dari $C$. Titik $E$ dan $F$ pada $AC$ dan $BC$, berturut-turut $AE = AD$ dan $BF = BD$. Titik $S$ adalah refleksi $C$ pada titik pusat lingkaran luar $\triangle ABC$. Buktikan bahwa $SE=SF$.

\item Misalkan $Q(x) = a_0 + a_1x + \dots + a_nx^n$ adalah polinomial dengan koefisien-koefisien bilangan bulat, dan $0 \le a_i < 3$ untuk semua $0 \le i \le n$.

Diketahui bahwa $Q(\sqrt{3}) = 20 + 17\sqrt{3}$, hitunglah nilai $Q(2)$.
\end{enumerate}

\end{document}