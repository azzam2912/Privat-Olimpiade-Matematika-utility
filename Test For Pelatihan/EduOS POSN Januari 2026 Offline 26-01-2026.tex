\documentclass[12pt]{scrartcl}

\usepackage[hagavi]{azzam}
\title{}

\begin{document}

\begin{enumerate}
\section{Isian Singkat}
    % key : 9
   	\item
	Carilah digit terakhir dari $$\sum_{k=1}^{2018} \left \lfloor \sqrt{2k} \right \rfloor $$

    % key : 0
	\item
	Carilah banyak solusi bulat nonnegatif $(a,b)$ yang memenuhi $$a^3+b^2+1=7ab.$$

    % 103 trigonometry problem no 16
    % key : 9
    \item Tentukan nilai $k$ dimana $0< k < 180$ dan memenuhi
    $$(4 \cos ^2 9^\circ - 3)(4 \cos^2 27^\circ-3)=\tan k^\circ.$$

    % 103 trigonometry problem no 42
    % key : 2+67= 69
    \item Jika $x_1,x_2,y_1,y_2$ adalah bilangan real yang memenuhi $x_1^2+x_2^2 = y_1^2+y_2^2 = 67^2$ untuk suatu $c\in \RR$. Misalkan
    $$S=(1-x_1)(1-y_1)+(1-x_2)(1-y_2).$$
    Jika nilai maksimum $S$ dapat dinyatakan sebagai $(\sqrt{a}+b)^2$, carilah nilai $a+b$.

    % 103 trigonometry problem no 9 adv
    % key : 8 - 1 = 7
    \item Misalkan
    $$S=|\sin x+\cos x + \tan x + \cot x + \sec x + \csc x|$$
    untuk bilangan real $x$ yang membuat ekspresi tersebut terdefinisi. Jika nilai minimum $S$ dapat dinyatakan sebagai $\sqrt{a}+b$ dengan $a$ adalah bilangan bulat nonnegatif dan $b$ bilangan bulat. Carilah nilai $a + b$.

    % key : 5
    \item Jika $p$ dan $q$ adalah bilangan prima yang memenuhi $p^2|q^3+1$ dan $q^2|p^3+1$, tentukan jumlah semua nilai $p$ yang memenuhi.

    % key : 1 x 128 = 128
    \item Jika nilai eksak dari $$\prod_{k=1}^{7}\cos \left ( \frac{k\pi}{15} \right ).$$
    dapat dinyatakan sebagai $\dfrac{a}{b}$ dengan $a,b$ bilangan bulat positif yang relatif prima. Carilah nilai $a \times b$

    % key : 3
    \item Tentukan banyaknya solusi real $x$ yang memenuhi $(x-1)^8+(x^2-3x-1)^4=(x+2)^4.$

    % key : 3
    \item Tentukan banyaknya akar real dari polinomial $p(x)=x^5+x^4-x^3-x^2-2x-2.$.

    % key : 1 x 9 = 9
    \item Misalkan terdapat bilangan real $y$ sehingga untuk suatu bilangan real $x$ terpenuhi
    $$20\sin x-21\cos x=81y^2-18y+30.$$
    Jika jumlah seluruh nilai $y$ yang mungkin dapat dinyatakan sebagai $\dfrac{a}{b}$ dengan $a,b$ adalah dua bilangan bulat yang saling relatif prima, tentukan nilai $a \times b$.

\section{Esai}
    \item Tentukan semua fungsi kontinu yang memenuhi
    $$x^2f(y) + y^2f(x) = (x + y)f(x)f(y)$$
    untuk setiap bilangan real $x$ dan $y$.
\end{enumerate}

\end{document}