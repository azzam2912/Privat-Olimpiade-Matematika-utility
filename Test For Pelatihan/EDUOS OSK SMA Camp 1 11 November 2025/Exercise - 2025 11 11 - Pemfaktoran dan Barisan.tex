\documentclass[a4paper]{article}
\usepackage[hagavi]{azzam}

\title{Exercise}
\begin{document}
\text{\Huge Exercise Pemfaktoran, Sistem Persamaan,}\\
\text{\Huge Barisan, dan Deret}

\begin{enumerate}
    \item (Canadian MO 1998) Tentukan penyelesaian x real yang memenuhi persamaan:
    \[ x = \sqrt{x-\dfrac{1}{x}} + \sqrt{1-\dfrac{1}{x}} \]

    \item (AIME 1990) Bilangan real $a, b, x$ dan $y$ memenuhi $ax+by=3, \ ax^2+by^2=7, \ ax^3+by^3=16$ dan $ax^4+by^4=42$. Tentukan nilai dari $ax^5+by^5$.

    \item (OSN 2003 SMP) Diketahui $a+b+c=0$. Tunjukkan bahwa $a^3+b^3+c^3=3abc$.

        % olimpiade.org
    \item Jika $a \neq b \neq c \neq a$ dan
    \begin{align*}
        \dfrac{a}{b+c}+\dfrac{b}{c+a}+\dfrac{c}{a+b} = 1,
    \end{align*}
	berapakah nilai
    \begin{align*}
        \dfrac{a^2}{b+c}+\dfrac{b^2}{c+a}+\dfrac{c^2}{a+b}?
    \end{align*}


     % problem 13
    \item Hitunglah
    \[ \frac{3}{1! + 2! + 3!} + \frac{4}{2! + 3! + 4!} + \dots + \frac{2025}{2023! + 2024! + 2025!} \]

    % problem 27
    \item (China 1992) Buktikan bahwa
    \[ 16 < \sum_{k=1}^{80} \frac{1}{\sqrt{k}} < 17. \]

    % problem 43
    \item Buktikan bahwa
    \[ \frac{1}{2} \cdot \frac{3}{4} \cdot \ldots \cdot\frac{67}{68} < \frac{1}{\sqrt{102}}\]

    \item (AIME 1985) Barisan bilangan bulat $a_1, a_2, a_3, \dots$ memenuhi $a_{n+2} = a_{n+1} - a_n$ untuk $n > 0$. Jumlah 1492 bilangan pertama adalah 1985 dan jumlah 1985 bilangan pertama adalah 1492. Tentukan jumlah 2001 bilangan pertama.

        
    \item (MATNC 2001) Tentukan jumlah 100 bilangan asli pertama yang bukan bilangan kuadrat sempurna.
    \item Nilai $x$ yang memenuhi persamaan:
    \[ \sqrt{x\sqrt{x\sqrt{x\dots}}} = \sqrt{4x+\sqrt{4x+\sqrt{4x+\dots}}} \]
    adalah \dots
    
    \item Tentukan jumlah dari:
    \[ \frac{1}{1+\sqrt{2}} + \frac{1}{\sqrt{2}+\sqrt{3}} + \frac{1}{\sqrt{3}+\sqrt{4}} + \dots + \frac{1}{\sqrt{99}+\sqrt{100}} \]
    % problem 3
    \item (AHSME 1999) Misalkan $x_1, x_2, \dots, x_n$ adalah barisan bilangan bulat sedemikian sehingga
    \begin{itemize}
        \item[(i)] $-1 \le x_i \le 2$, untuk $i = 1, 2, \dots, n$;
        \item[(ii)] $x_1 + x_2 + \dots + x_n = 19$;
        \item[(iii)] $x_1^2 + x_2^2 + \dots + x_n^2 = 99$.
    \end{itemize}
    Tentukan nilai minimum dan maksimum yang mungkin dari
    \[ x_1^3 + x_2^3 + \dots + x_n^3. \]
    
\end{enumerate}

\end{document}