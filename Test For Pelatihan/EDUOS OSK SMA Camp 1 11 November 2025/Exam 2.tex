\documentclass[a4paper, 12pt]{article}
\usepackage[hagavi]{azzam}

\title{Exam 1}
\begin{document}

\maketitle
\begin{enumerate}
    %% paket soal osp
    \item Diberikan segitiga $ABC$ lancip. Garis tinggi terpanjang adalah dari titik sudut $A$ tegak lurus pada $BC$, dan panjangnya sama dengan panjang median (garis berat) dari titik sudut $B$. Nilai terbesar $\angle ABC$ adalah $\dots^\circ$
    \\
    %\textbf{Answer: 60}
	
	\item Pada segitiga $ABC$ terdapat titik $P$ di dalamnya sehingga $\angle PAB = 10^\circ, \angle PBA = 20^\circ , \angle PAC = 40^\circ, \angle PCA = 30^\circ$. Besar sudut $\angle ABC$ adalah $\dots^\circ$ 
    \\
    %\textbf{Answer: 80}
	
	\item Pada sembarang segitiga $ABC$, titik $D,E,F$ berturut-turut pada $BC,CA,AB$ dimana $AD,BE,CF$ bertemu di $M$. Nilai eksak dari $\dfrac{AM}{AD}+\dfrac{BM}{BE}+\dfrac{CM}{CF}$ adalah $\dots$
    \\
    %\textbf{Answer: 2}
	
	\item Misalkan $ABCD$ adalah segiempat konveks dengan $\angle DAC=\angle BDC = 36^\circ$, $\angle CBD = 18^\circ$, dan $\angle BAC = 72^\circ$. Diagonal $AC$ dan $BD$ berpotongan di titik $P$. Tentukan besar sudut $\angle APD$ dalam derajat. %ko ss spring camp 30 maret 2019 
    \\
    %\textbf{Answer: 100}
	
\end{enumerate}

\end{document}