\documentclass{article}
\usepackage[hagavi]{azzam}

\newtheorem{contoh}{Contoh}
\newcommand{\bracket}[1]{\left( #1 \right)}
\usepackage[makeroom]{cancel}

\title{Barisan dan Deret}

\begin{document}
\text{\Huge Barisan dan Deret}

\section{Barisan dan Deret Aritmatika}
\subsection{Barisan Aritmatika (\textit{Arithmetic Sequence}}
Barisan dengan suku (\textit{term}) pertama $a$ dan beda barisan antar suku (\textit{common difference}) $b$. Rumus umum suku ke-$n$ barisan ini adalah $U_n = a + (n-1)b$.
\begin{align*}
    a, \quad a+b, \quad a+2b, \quad \ldots, \quad a+(n-1)b
\end{align*}

\subsection{Deret Aritmatika (\textit{Arithmetic Series})}
Penjumlahan $n$ suku pertama dari deret artimatika dengan rumus 
\begin{align*}
    S_n = a + (a+b) + (a+2b) + \ldots + (a+(n-1)b) = \dfrac{n}{2}\left(2a+(n-1)b\right) =\dfrac{n}{2}\left(U_1+U_n\right)
\end{align*}

\section{Barisan dan Deret Geometri}
\subsection{Barisan Geometri (\textit{Geometric Sequence})}
Barisan dengan suku (\textit{term}) pertama $a$ dan rasio antar suku (\textit{common ratio}) $r$. Rumus umum suku ke-$n$ barisan ini adalah $U_n = ar^{n-1}$.
\begin{align*}
    a, \quad ar, \quad ar^2, \quad \ldots, \quad ar^{n-1}
\end{align*}

\subsection{Deret Geometri (\textit{Geometric Series})}
Penjumlahan $n$ suku pertama dari deret artimatika dengan rumus
\begin{align*}
    S_n = a + ar+ar^2+\ldots+ar^{n-1} = \dfrac{a(1-r^n)}{1-r}
\end{align*}

\subsection{Deret Geometri Tak Hingga}
Penjumlahan suku pertama sampai tak hingga dari deret artimatika dengan rumus berikut:\\
\textbf{Syarat Wajib: } $-1 < r < 1$.
\begin{align*}
    S_\infty = a + ar+ar^2+\ldots = \dfrac{a}{1-r}
\end{align*}

\section{Bentuk Barisan dan Deret Lainnya}

\subsection{Bentuk Tak Hingga Lainnya}
\begin{contoh}
Jika $x=\sqrt{2\sqrt{2\sqrt{2\sqrt{2\cdots}}}}$ adalah sebuah bilangan bulat, hitunglah nilai $x$.
\end{contoh}
\begin{solusi}
    \begin{align*}
        x &= \sqrt{2\sqrt{2\sqrt{2\sqrt{2\cdots}}}}\\
        x &= \sqrt{2x}\\
        x^2 &= 2x\\
        x^2 - 2x &= 0\\
        x(x - 2) &= 0\\
        \text{karena } x &\neq 0 \text{ maka } x = 2
    \end{align*}
\end{solusi}

\subsection{Rumus Cepat Yang Berguna}
\begin{itemize}
\item $1 + 2 + 3 + \cdots + n = \dfrac{n(n+1)}{2}$
\item $1^2 + 2^2 + 3^2 + \cdots + n^2 = \dfrac{n(n+1)(2n+1)}{6}$
\item $1^3 + 2^3 + 3^3 + \cdots + n^3 = \left(\dfrac{n(n+1)}{2}\right)^2$
\end{itemize}

\subsection{Fibonacci Sequence / Barisan Fibonacci}
Barisan dengan setiap suku adalah jumlah dari dua suku sebelumnya.
\[1, 1, 2, 3, 5, 8, 13, 21, 34, \ldots\]
Rumus rekursifnya: $F_n = F_{n-1} + F_{n-2}$, with $F_1 = F_2 = 1$

\subsection{Telescoping Series}
Examples speak louder than theory:
\begin{contoh}
Nilai sederhana dari $\bracket{\dfrac{1}{2}}\times\bracket{\dfrac{2}{3}}\times\ldots\times\bracket{\dfrac{99}{100}}$ adalah \ldots
\end{contoh}
\begin{solusi}
    \begin{align*}
        \bracket{\dfrac{1}{\cancel{2}}}\times\bracket{\dfrac{\cancel{2}}{\cancel{3}}}\times\bracket{\dfrac{\cancel{3}}{\cancel{4}}}\times\ldots\times\bracket{\dfrac{\cancel{98}}{\cancel{99}}}\times\bracket{\dfrac{\cancel{99}}{100}}=\dfrac{1}{100}
    \end{align*}
\end{solusi}
\begin{contoh}
Nilai sederhana dari $\dfrac{1}{1\times 2} + \dfrac{1}{2 \times 3} + \dfrac{1}{3 \times 4} + \dots + \dfrac{1}{99 \times 100}$ adalah \ldots
\end{contoh}
\begin{solusi}
    \begin{align*}
        \bracket{\dfrac{1}{1} - \dfrac{1}{2}} + \bracket{\dfrac{1}{2} - \dfrac{1}{3}} + \bracket{\dfrac{1}{3} - \dfrac{1}{4}} + \ldots + \bracket{\dfrac{1}{98} - \dfrac{1}{99}} + \bracket{\dfrac{1}{99} - \dfrac{1}{100}}=\dfrac{1}{1}-\dfrac{1}{100}=\dfrac{99}{100}
    \end{align*}
\end{solusi}

\subsection{Sigma dan Pi}
\subsubsection{Sigma}
\begin{contoh}
$$\sum_{k=3}^{100} 2k = 2(3)+2(4)+2(5)+\ldots+2(100)$$
\end{contoh}

\subsubsection{Pi}
\begin{contoh}
$$\prod_{k=5}^{67} k^3 = 5^3 \times 6^3 \times 7^3 \times \ldots \times 67^3$$
\end{contoh}

\section{Latihan Soal Dasar Banget \textit{(Sanity Test)}}
\begin{enumerate}
    % Soal 1: Barisan Aritmatika (Mencari U_n)
    \item Diketahui suatu barisan aritmatika dengan suku pertama adalah 5 dan beda adalah 3. Tentukan nilai dari suku ke-20 ($U_{20}$) barisan tersebut.

    % Soal 2: Deret Aritmatika (Mencari S_n)
    \item Hitunglah jumlah 15 suku pertama ($S_{15}$) dari deret aritmatika berikut:
    $ 2 + 6 + 10 + 14 + \dots $

    % Soal 3: Barisan Geometri (Mencari U_n)
    \item Suku pertama dari sebuah barisan geometri adalah 3 dan rasionya adalah 2. Tentukan suku ke-8 ($U_8$).

    % Soal 4: Deret Geometri (Mencari S_n)
    \item Tentukan jumlah 6 suku pertama ($S_6$) dari deret geometri:
    $ 1 + 3 + 9 + 27 + \dots $

    % Soal 5: Deret Geometri Tak Hingga (Mencari S_tak_hingga)
    \item Tentukan jumlah tak hingga ($S_\infty$) dari deret geometri berikut:
    $ 16 + 8 + 4 + 2 + \dots $
    
    % Soal 6: Barisan Aritmatika (Mencari n)
    \item Pada suatu barisan aritmatika, diketahui suku pertamanya adalah 7 dan bedanya adalah 4. Jika suku ke-n ($U_n$) barisan tersebut adalah 83, tentukan nilai $n$.
\end{enumerate}

\section{Latihan Soal}
\begin{enumerate}
    \item Perhatikan barisan bilangan 500, 465, 430, 395, ... Suku negatifnya yang pertama adalah \dots
    
    \item Pada suatu deret aritmatika berlaku $u_2+u_5+u_6+u_9 = 40$. Maka $S_{10} = \dots$
    
    \item (OSK SMA 2006) Diketahui $a+(a+1)+(a+2)+\dots+50 = 1139$. Jika $a$ bilangan positif, maka $a = \dots$
    
    \item (OSK SMA 2011 Tipe 3) Bilangan bulat positif terkecil $a$ sehingga $2a+4a+6a+\dots+200a$ merupakan kuadrat sempurna adalah \dots
    
    \item (AIME 1984) Barisan $a_1, a_2, a_3, \dots, a_{98}$ memenuhi $a_{n+1} = a_n + 1$ untuk $n=1, 2, 3, \dots, 97$ dan mempunyai jumlah sama dengan 137. Tentukan nilai dari $a_2+a_4+a_6+\dots+a_{98}$.
    
    \item Misalkan $u_n$ adalah suku ke-n dari suatu barisan aritmatika. Jika $u_k = t$ dan $u_t = k$ maka tentukan nilai dari suku ke-$(k+t)$.
    
    \item (OSK SMA 2004) Agar bilangan $2^0+2^1+2^2+\dots+2^n$ sedekat mungkin kepada 2004, haruslah $n = \dots$
    
    \item Jika $9-7x$, $5x-6$ dan $x-1$ adalah tiga suku pertama deret geometri tak hingga, maka jumlah suku-sukunya adalah \dots
    
    \item Pada suatu deret tak hingga, suku-suku yang bernomor ganjil berjumlah $9/4$ sedangkan suku-suku yang bernomor genap berjumlah $3/4$, maka suku pertamanya adalah \dots
    
    \item (OSP SMA 2006) Afkar memilih suku-suku barisan geometri takhingga $1, \frac{1}{2}, \frac{1}{4}, \frac{1}{8}, \dots$ untuk membuat barisan geometri takhingga baru yang jumlahnya $\frac{1}{7}$. Tiga suku pertama pilihan Afkar adalah \dots
    
    \item Tiga buah bilangan merupakan barisan aritmatika. Bila suku tengahnya dikurangi 5, maka terbentuk suatu barisan geometri dengan rasio sama dengan 2. Jumlah barisan aritmatika itu adalah \dots
    
    \item Tentukan rumus jumlah $n$ suku pertama dari barisan 4, 10, 20, 35, 56, \dots
    
    \item (AIME 1992) Misalkan $A$ adalah barisan $a_1, a_2, a_3, \dots$ dengan $a_{19}=a_{92}=0$ dan $\Delta A$ didenisikan dengan barisan $a_2-a_1, a_3-a_2, a_4-a_3, \dots$ Jika semua suku-suku barisan $\Delta(\Delta A)$ sama dengan 1, maka nilai $a_1$ adalah \dots
    
    \item (MATNC 2001) Tentukan jumlah 100 bilangan asli pertama yang bukan bilangan kuadrat sempurna.
    
    \item (OSK SMA 2009) Bilangan bulat positif terkecil $n$ dengan $n>2009$ sehingga $\sqrt{\frac{1^3+2^3+3^3+\dots+n^3}{n}}$ merupakan bilangan bulat adalah \dots
    
    \item (AIME 1985) Barisan bilangan bulat $a_1, a_2, a_3, \dots$ memenuhi $a_{n+2} = a_{n+1} - a_n$ untuk $n > 0$. Jumlah 1492 bilangan pertama adalah 1985 dan jumlah 1985 bilangan pertama adalah 1492. Tentukan jumlah 2001 bilangan pertama.
    
    \item Nilai $x$ yang memenuhi persamaan:
    \[ \sqrt{x\sqrt{x\sqrt{x\dots}}} = \sqrt{4x+\sqrt{4x+\sqrt{4x+\dots}}} \]
    adalah \dots
    
    \item Tentukan jumlah dari:
    \[ \frac{1}{1+\sqrt{2}} + \frac{1}{\sqrt{2}+\sqrt{3}} + \frac{1}{\sqrt{3}+\sqrt{4}} + \dots + \frac{1}{\sqrt{99}+\sqrt{100}} \]

    % problem 11
    \item (Romania 1990) Tentukan bilangan bulat positif $m$ terkecil sedemikian sehingga
    \[ \binom{2n}{n} < m^n \]
    untuk semua bilangan bulat positif $n$.

    % problem 3
    \item (AHSME 1999) Misalkan $x_1, x_2, \dots, x_n$ adalah barisan bilangan bulat sedemikian sehingga
    \begin{itemize}
        \item[(i)] $-1 \le x_i \le 2$, untuk $i = 1, 2, \dots, n$;
        \item[(ii)] $x_1 + x_2 + \dots + x_n = 19$;
        \item[(iii)] $x_1^2 + x_2^2 + \dots + x_n^2 = 99$.
    \end{itemize}
    Tentukan nilai minimum dan maksimum yang mungkin dari
    \[ x_1^3 + x_2^3 + \dots + x_n^3. \]

    % problem 2
    \item Tentukan suku umum dari barisan yang didefinisikan oleh $x_0 = 3, x_1 = 4$ dan
    \[ x_{n+1} = x_{n-1}^2 - n x_n \]
    untuk semua $n \in \mathbb{N}$.

    % problem 13
    \item Hitunglah
    \[ \frac{3}{1! + 2! + 3!} + \frac{4}{2! + 3! + 4!} + \dots + \frac{2025}{2023! + 2024! + 2025!} \]

    % problem 19
    \item (Modifikasi Korea MO 2001) Misalkan $f(x) = \frac{2}{4^x + 2}$ untuk bilangan real $x$.
    Hitunglah
    \[ f\left(\frac{1}{2025}\right) + f\left(\frac{2}{2025}\right) + \dots + f\left(\frac{2024}{2025}\right). \]

    % problem 27
    \item (China 1992) Buktikan bahwa
    \[ 16 < \sum_{k=1}^{80} \frac{1}{\sqrt{k}} < 17. \]

    % problem 30
    \item Hitunglah
    \[ \sum_{k=0}^{n} \frac{1}{(n-k)!(n+k)!} \]

    % problem 43
    \item Buktikan bahwa
    \[ \frac{1}{2} \cdot \frac{3}{4} \cdot \ldots \cdot\frac{67}{68} < \frac{1}{\sqrt{102}}\]
\end{enumerate}

\end{document}