\documentclass[11pt]{scrartcl}
\usepackage{graphicx}
\graphicspath{{./}}
\usepackage[sexy]{evan}
\usepackage[normalem]{ulem}
\usepackage{hyperref}
\usepackage{mathtools}
\hypersetup{
    colorlinks=true,
    linkcolor=blue,
    filecolor=magenta,      
    urlcolor=cyan,
    pdftitle={Overleaf Example},
    pdfpagemode=FullScreen,
    }

\renewcommand{\dangle}{\measuredangle}

\renewcommand{\baselinestretch}{1.5}

\addtolength{\oddsidemargin}{-0.4in}
\addtolength{\evensidemargin}{-0.4in}
\addtolength{\textwidth}{0.8in}
% \addtolength{\topmargin}{-0.2in}
% \addtolength{\textheight}{1in} 


\setlength{\parindent}{0pt}

\usepackage{pgfplots}
\pgfplotsset{compat=1.15}
\usepackage{mathrsfs}
\usetikzlibrary{arrows}

\title{Post Test SMAN 5 Bekasi}
\author{compiled by Azzam}
\date{Sabtu, 4 Februari 2023}

\begin{document}

\maketitle
\begin{enumerate}
    \section{Aljabar + Geometri}
    \item Jika $\dfrac{(2^{2024})^2-(2^{2022})^2}{(2^{2025})^2-(2^{2023})^2}=\dfrac{a}{b}$ dimana $a$ dan $b$ masing-masing adalah bilangan asli yang saling relatif prima, nilai $a+b$ adalah \dots

    \item Diberikan tiga bilangan bulat positif berurutan. Jika bilangan pertama tetap, bilangan kedua ditambah 10 dan bilangan ketiga ditambah bilangan prima, maka ketiga bilangan ini membentuk deret geometri. Bilangan ketiga dari bilangan bulat berurutan adalah \dots

    \item Misalkan $x,y$ adalah bilangan real sedemikian sehingga $x+y=\dfrac{1}{x}+\dfrac{1}{y}=6$. Nilai dari $\dfrac{a}{b}+\dfrac{b}{a}+2023=\dots$

    \item Segitiga $ABC$ mempunyai panjang sisi $AB = 20, AC = 21$ dan $BC = 29$. Titik $D$ dan $E$ terletak pada segmen garis $BC$, dengan $BD = 8$ dan $EC = 9$. Besar $\angle DAE$ adalah \dots

    \item Diberikan segitiga lancip $ABC$. Misalkan lingkaran dengan diameter $BC$ memotong segmen $AB$ dan $AC$ berturut-turut di $F$ dan $E$. Jika $BC=25$, $BF=15$, dan $CE=7$, maka nilai perbandingan $\dfrac{BE}{CF}=\dots$

    \newpage
    \section{Kombinatorika + Teori Bilangan}
    \item Sekelompok orang akan berjabat tangan. Setiap orang hanya dapat melakukan jabat tangan sekali. Tidak boleh melakukan jabat tangan dengan dirinya sendiri. Jika dalam sekelompok orang tejadi 190 jabat tangan, maka banyaknya orang dalam kelompok tersebut adalah ...

    \item Misalkan $s$ adalah himpunan bilangan asli yang digitnya tidak berulang dan dipilih dari 1, 3, 5, 7. Jumlah digit satuan dari semua anggota $S$ adalah ...

    \item Ada sebanyak $6!$ permutasi dari huruf-huruf $OSNMAT$. Jika semua permutasi tersebut diurutkan secara abjad dari $A$ ke $Z$, maka $OSNMAT$ pada urutan ke \dots

    \item Jika bilangan $m$ dibagi 2022 memberikan sisa 41, dan bilangan $n$ dibagi 2022 memberikan sisa 50, maka bilangan $mn$  bila dibagi 1011 akan memberikan sisa ...

    \item Bilangan bulat $x$ jika dikalikan 11 terletak diantara 1500 dan 2000. Jika $x$ dikalikan 7 terletak antara 970 dan 1275. Jika $x$ dikalikan 5 terletak antara 960 dan 900. Banyaknya bilangan $x$ sedemikian yang habis dibagi 3 sekaligus habis dibagi 5 ada sebanyak ….
\end{enumerate}


\end{document}
