\documentclass[11pt]{scrartcl}
\usepackage{graphicx}
\graphicspath{{./}}
\usepackage[sexy]{evan}
\usepackage[normalem]{ulem}
\usepackage{hyperref}
\usepackage{mathtools}
\hypersetup{
    colorlinks=true,
    linkcolor=blue,
    filecolor=magenta,      
    urlcolor=cyan,
    pdfpagemode=FullScreen,
    }

\renewcommand{\dangle}{\measuredangle}

\renewcommand{\baselinestretch}{1.5}

\addtolength{\oddsidemargin}{-0.4in}
\addtolength{\evensidemargin}{-0.4in}
\addtolength{\textwidth}{0.8in}
% \addtolength{\topmargin}{-0.2in}
% \addtolength{\textheight}{1in} 


\setlength{\parindent}{0pt}

\usepackage{pgfplots}
\pgfplotsset{compat=1.15}
\usepackage{mathrsfs}
\usetikzlibrary{arrows}

\usepackage[most]{tcolorbox}

\title{Exam 1}
\author{Compiled by Azzam and Pak Fendy}

\date{Selasa, 14 Januari 2025}
\begin{document}
\maketitle
\textbf{Waktu: 2 jam (termasuk upload)}

\begin{enumerate}

% modifikasi USAMO 2015 #2
\item Segiempat \( APBQ \) terletak di dalam lingkaran \( \omega \) dengan \( \angle P = \angle Q = 90^{\circ} \) dan \( AP = AQ < BP \). Misalkan \( X \) adalah titik variabel pada ruas garis \( \overline{PQ} \). Garis \( AX \) memotong \( \omega \) lagi di titik \( S \) (selain \( A \)). Titik \( T \) terletak pada busur \( AQB \) dari \( \omega \) sedemikian sehingga \( \overline{XT} \) tegak lurus terhadap \( \overline{AX} \). Misalkan \( M \) adalah titik tengah dari tali busur \( \overline{ST} \). Misalkan pula \( L \) sebagai proyeksi titik \(A\) pada \(\overline{ST}\). Buktikan bahwa \( P, Q, M, L\) berada di satu lingkaran yang sama.

\item Carilah semua solusi $(a,p,k)$, dengan $a,k$ adalah bilangan bulat dan $p$ adalah bilangan prima, dari persamaan $$a^p-1=p^k$$

\item Tentukan semua fungsi \( f: \mathbb{N} \to \mathbb{N} \) dengan sifat
   \[
   (f(m))^2 + f(n) \text{ habis membagi } (m^2 + n)^2
   \]
   untuk semua \( m, n \in \mathbb{N} \).

\item Papan catur \( 8 \times 8 \) diwarnai hitam putih seperti biasa. Tentukan banyaknya cara meletakkan 16 koin identik pada kotak-kotak putih sehingga setiap kotak berisi paling banyak satu koin dan tidak ada dua koin yang bersebelahan secara diagonal.

\end{enumerate}
\end{document}