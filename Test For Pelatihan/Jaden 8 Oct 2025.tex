\documentclass[11pt]{article}
\usepackage[hagavi]{azzam}
% %japanese
\usepackage{CJKutf8}
\begin{CJK*}{UTF8}{min}
\title{Jaden の事前テスト}
\end{CJK*}


\begin{document}
\begin{CJK*}{UTF8}{min}
	
	\date{Rabu, 8 Oktober 2025}
	\author{Compiled by Azzam}
	\maketitle
	
	\begin{soalbaru}
	%ahsme 1989
	Misalkan $x$ adalah bilangan real yang dipilih secara acak diantara 100 dan 200. Jika $\floor{\sqrt{x}}=12$, carilah peluang terjadinya $\floor{\sqrt{100x}}=120$.
	\end{soalbaru}

    \begin{soalbaru}
        Sebuah bilangan palindrom 6 digit dengan digit terakhir 4 merupakan hasil perkalian antara dua atau lebih bilangan asli berurutan. Hitunglah hasil penjumlahan digit-digit palindrom tersebut.
    \end{soalbaru}

    \begin{soalbaru}
        % smo 2024 junior
        Misalkan $a$, $b$ dan $c$ adalah bilangan real yang memenuhi $a + b + c = 8$ dan $ab + bc + ca = 0$.
Tentukan nilai maksimum dari $3(a + b)$.
    \end{soalbaru}

    \begin{soalbaru}
    Misalkan $A$ adalah himpunan 9 sembarang bilangan asli yang faktor primanya adalah anggota dari himpunan $\{3,7,11\}$. Buktikan bahwa terdapat setidaknya dua anggota $A$ sehingga hasil kalinya merupakan bilangan kuadrat sempurna.
\end{soalbaru}

\begin{soalbaru}
    Misalkan himpunan $A \subset \{1,2,3,\dots,4040\}$ dengan $|A| = 2021$. Tunjukkan bahwa ada dua elemen berbeda $a,b \in A$ sehingga $a \mid b$.
\end{soalbaru}
	
	\begin{soalbaru}
	% geometri hayattir
	Diberikan segitiga $ABC$ siku-siku di $B$.Misalkan $D$ dan $E$ berturut-turut di segmen $AC$ dan $BC$ sedemikian sehingga $\angle EDC= 90^\circ$ dan $\angle CED = \angle AEB$. Jika $AE = 24$ dan $CE=EB$, tentukan panjang $DE$.
	\end{soalbaru}
	
	\begin{soalbaru}
	%ahsme 1988 prob 22
	Berapa banyak bilangan bulat $n$ sehingga terdapat segitiga lancip yang mempunyai sisi $10,24$, dan $n$?
	\end{soalbaru}

        \begin{soalbaru}
        Carilah semua solusi real dari sistem persamaan berikut
	\begin{align*}
		x+y &= \sqrt{4z-1} \\
		y+z &= \sqrt{4x-1} \\
		z+x &= \sqrt{4y-1}
	\end{align*}
    \end{soalbaru}

    \begin{soalbaru}
        % ksr 2021
        Diberikan segitiga sama kaki $ABC$ dengan $AB=AC$. Titik $D$ dan $E$ berturut-turut merupakan titik tengah $AC$ dan $BC$, dan $M$ titik tengah $DE$. Jika $AM$ memotong $BC$ di $F$, nilai $\dfrac{AF}{DF} = \dots$\\

\usetikzlibrary{decorations.markings, arrows.meta} % Untuk tanda segmen yang sama panjang

% Definisi untuk tanda segmen yang sama panjang (seperti di sketsa)
\tikzset{
    seg mark/.style={
        decoration={markings, mark=at position 0.5 with {\draw[#1] (-2pt,-2pt) -- (2pt,2pt) (2pt,-2pt) -- (-2pt,2pt);}},
        postaction={decorate}
    },
    seg mark half/.style={
        decoration={markings, mark=at position 0.5 with {\draw[#1] (0pt,-2pt) -- (0pt,2pt);}},
        postaction={decorate}
    },
    seg mark double/.style={
        decoration={markings, mark=at position 0.5 with {\draw[#1] (-2pt,-2pt) -- (-2pt,2pt) (2pt,-2pt) -- (2pt,2pt);}},
        postaction={decorate}
    },
    seg mark triple/.style={
        decoration={markings, mark=at position 0.5 with {\draw[#1] (-4pt,-2pt) -- (-2pt,2pt) (0pt,-2pt) -- (0pt,2pt) (2pt,-2pt) -- (4pt,2pt);}},
        postaction={decorate}
    },
}

\begin{tikzpicture}[scale=1]
    % 1. Definisikan segitiga sama kaki ABC (AB=AC)
    \tkzDefPoint(0,0){B}
    \tkzDefPoint(4,0){C}
    \tkzDefPoint(2,4){A} % Titik A di tengah-tengah atas BC
    
    % Gambar segitiga ABC
    \tkzDrawPolygon(A,B,C)

    % 2. D adalah titik tengah AC, E adalah titik tengah BC, K titik tengah AB
    \tkzDefMidPoint(A,C)\tkzGetPoint{D}
    \tkzDefMidPoint(B,C)\tkzGetPoint{E}
    
    % 3. M adalah titik tengah DE
    \tkzDefMidPoint(D,E)\tkzGetPoint{M}
    
    % 4. F adalah perpotongan AM dan BC
    \tkzInterLL(A,M)(B,C)\tkzGetPoint{F}
    
    % Garis-garis yang menghubungkan titik-titik konstruksi
    \tkzDrawSegments(A,D D,E A,M D,F M,F)
    
    % Gambar lingkaran luar ADM (yang akan disentuh oleh DF)
    \tkzDefCircumCenter(A,D,M)\tkzGetPoint{O_ADM}
    \tkzDrawCircle[blue](O_ADM,A) % Gambar lingkaran luar ADM
    
    % --- TANDA SEGMEN SAMA PANJANG ---
    %\tkzDrawSegments[seg mark half=black](A,B A,C) % AB=AC
    \tkzDrawSegments[seg mark=black](A,D D,C) % D midpoint AC
    \tkzDrawSegments[seg mark half=black](B,E E,C) % E midpoint BC
    \tkzDrawSegments[seg mark triple=black](D,M M,E) % M midpoint DE


    % --- LABEL TITIK ---
    \tkzDrawPoints[size=4, fill=black](A,B,C,D,E,M,F)
    \tkzLabelPoints[above](A)
    \tkzLabelPoints[below left](B)
    \tkzLabelPoints[below right](C)
    \tkzLabelPoints[right](D)
    \tkzLabelPoints[below](E,F)
    \tkzLabelPoints[right](M)
        
    % --- ANGKA DARI SKETSA (jika perlu) ---
    % \tkzLabelSegment[below](B,E){3} % Contoh angka
    % \tkzLabelSegment[below](E,F){1}
    % \tkzLabelSegment[below](F,C){2}
    % \tkzLabelSegment[left](N,M){2}
    % \tkzLabelSegment[right](D,M){2}
    
\end{tikzpicture}
    \end{soalbaru}

\end{CJK*}
\end{document}



