\documentclass[11pt]{scrartcl}
\usepackage[sexy]{evan}

%\addtolength{\textheight}{1.5in} 
\renewcommand{\baselinestretch}{1.35}



\begin{document}
	\title{Paket Soal Untuk KSP SMA 11} % Beginner
	\date{\today}
	\author{Compiled by Azzam}
	\maketitle
	
	\section{Isian Singkat}
	Dianjurkan mengerjakan setiap soal dalam waktu maksimal 5 menit.
	
	\begin{soalbaru}
	%ahsme 1989
	Misalkan $x$ adalah bilangan real yang dipilih secara acak diantara 100 dan 200. Jika $\floor{\sqrt{x}}=12$, carilah peluang terjadinya $\floor{\sqrt{100x}}=120$.
	\end{soalbaru}
	
	\begin{soalbaru}
	% geometri hayattir
	Diberikan segitiga $ABC$ siku-siku di $B$.Misalkan $D$ dan $E$ berturut-turut di segmen $AC$ dan $BC$ sedemikian sehingga $\angle EDC= 90^\circ$ dan $\angle CED = \angle AEB$. Jika $AE = 24$ dan $CE=EB$, tentukan panjang $DE$.
	\end{soalbaru}
	
	\begin{soalbaru}
	%ahsme 1988 prob 22
	Berapa banyak bilangan bulat $n$ sehingga terdapat segitiga lancip yang mempunyai sisi $10,24$, dan $n$?
	\end{soalbaru}

	\begin{soalbaru}
	%aime 1986
	Polinomial $1-x+x^2-x^3+\cdots+x^{16}-x^{17}$ dapat ditulis dalam bentuk $a_0+a_1y+a_2y^2+\cdots +a_{16}y^{16}+a_{17}y^{17}$, dimana $y=x+1$ dan semua $a_i$ adalah konstanta. Nilai dari $a_2$ adalah$\dots$
	\end{soalbaru}

	\section{Esai}
	Dianjurkan untuk mengerjakannya dalam waktu 120 menit.
	
	Soal \textbf{tidak} diurutkan berdasarkan kesulitan. 
	
	\begin{soalbaru}
	%bmc bary
	Segitiga $ABC$ mempunyai lingkaran luar $\Omega$. Garis bagi dalam $\angle A$ memotong sisi $BC$ dan $\Omega$ secara beruturut-turut di $D$ dan $L$ (selain $A$). Misalkan $M$ adalah titik tengah sisi $BC$. Lingkaran luar $\triangle  ADM$ memotong sisi $AB$ dan $AC$ di $Q$ dan $P$ (selain $A$) secara berutrut-turut. Jika $N$ adalah titik tengah segmen $PQ$, tunjukkan bahwa $MN \parallel AD$. 
	\end{soalbaru}
	
	\begin{soalbaru}
	%imo 1964
	Misalkan $a$,$b$, dan $c$ adalah sisi-sisi suatu segitiga. Buktikan bahwa$$\sum_{cyc} a^2(b+c-a) \le 3abc.$$
	\end{soalbaru}
	
	\begin{soalbaru}
	%imo 1980
	Carilah bilangan asli $x$, $y$ dan bilangan asli $n \ge 2$ dengan $x$ relatif prima dengan $n+1$, yang memenuhi $$x^n+1=y^{n+1}.$$
	\end{soalbaru}
	
	\begin{soalbaru}
	%imo 1964
	Tujuh belas orang dipilih oleh Bapak Presiden Yang Terhormat untuk menghadiri suatu permainan 17-an yang akan ditampilkan di Channel WeTube Sekretariat Negara. Mereka diperintahkan untuk saling mengirim pesan WhutsUpp. Mereka hanya diperbolehkan membicarakan tiga topik berbeda di pesan tersebut, yaitu Revolusi Industri 69.42.1, Covid365, dan Radikalisme Anime Attack on Titan terhadap keberlangsungan budaya Bangsa. Setiap pasang orang yang saling berkirim pesan hanya boleh mendiskusikan tepat satu topik diantara tiga topik tersebut. Buktikan bahwa ada setidaknya tiga orang yang saling menulis satu topik yang sama ke satu sama lain.
	\end{soalbaru}
\end{document}


