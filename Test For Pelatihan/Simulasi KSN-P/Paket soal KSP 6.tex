\documentclass[11pt]{scrartcl}
\usepackage[sexy]{evan}

%\addtolength{\textheight}{1.5in} 
\renewcommand{\baselinestretch}{1.5}



\begin{document}
	\title{Paket Soal Untuk KSP SMA 6} % Beginner
	\date{Sabtu, 24 Juli 2021}
	\author{Compiled by Azzam}
	\maketitle
	
	
	\section{Soal Esai}
	
	\begin{soalbaru} 
			Mungkinkah menempatkan masing-masing angka-angka $1,2,\dots,9$ tepat sekali ke dalam kotak-kotak satuan pada papan catur berukuran $3\times 3$, sehingga untuk setiap dua persegi yang bertetangga, baik secara vertikal ataupun horizontal, jumlah dari dua bilangan yang ada di dalamnya selalu prima?
		\end{soalbaru}
		
		
		\begin{soalbaru}
				Sebuah barisan bilangan asli $a_1,a_2,a_3,\dots$ memenuhi $a_k+a_l=a_m+a_n$ untuk setiap bilangan asli $k,l,m,n$ dengan $kl=mn$. Jika $m\mid n$, buktikan bahwa $a_m \le a_n$.
			\end{soalbaru}

	\begin{soalbaru}
		Misalkan $P$ dan $Q$ di segmen $BC$ dari segitiga lancip $ABC$ sedemikian sehingga $\angle PAB = \angle BCA$ dan $\angle CAQ = \angle ABC$. Misalkan $M$ dan $N$ adalah titik di garis $AP$ dan $AQ$ sehingga $P$ adalah titik tengah $AM$ dan $Q$ adalah titik tengah $AN$. Buktikan bahwa perpotongan $BM$ dan $CN$ berada di lingkaran luar segitiga $ABC$.
	\end{soalbaru}

	\begin{soalbaru}
		Misalkan $a_0 < a_1 < a_2 < \dots$ adalah barisan tak hingga bilangan bulat positif. Buktikan bahwa ada sebuah bilangan bulat $n \ge 1$ yang unik sehingga $$a_n < \dfrac{a_0+a_1+a_2+\dots+a_n}{n} \le a_{n+1}.$$
	\end{soalbaru}

	
	\begin{soalbaru}
	Misalkan ada 200 peserta dalam lomba olimpiade. Terdapat 6 soal di lomba tersebut. Ternyata masing-masing soal dapat diselesaikan setidaknya oleh 120 peserta.
	Buktikan bahwa terdapat pasangan peserta (2 orang) yang kalau bekerjasama bisa menyelesaikan ke-6 soal (masing-masing orang dari pasangan tersebut tidak perlu mengerjakan semua soal asalkan soal-soal yang dikerjakan dua orang tersebut jika digabung totalnya adalah 6 soal di lomba tersebut).
	\end{soalbaru}

	
\end{document}
