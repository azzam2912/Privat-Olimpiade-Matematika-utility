\documentclass{article}
\usepackage{amsmath}
\usepackage{amssymb}
\usepackage{graphicx}
\graphicspath{{./Farhan/}}

\usepackage{enumitem}
\renewcommand{\baselinestretch}{1.6}
\addtolength{\oddsidemargin}{-1in}
\addtolength{\evensidemargin}{-1.5in}
\addtolength{\textwidth}{1.9in}

\addtolength{\topmargin}{-1in}
\addtolength{\textheight}{1.7in} 

\title{Paket Soal Untuk KSP SMA 3}
\date{Senin, 12 Juli 2021}

\setlength{\parindent}{0em}
\begin{document}
	\maketitle

\section{Esai}
Jawablah soal-soal berikut dengan menyertakan argumentasi atau cara Anda mendapatkannya. Setiap soal bernilai bilangan bulat positif dengan poin maksimum 7.

Alokasi waktu : 15-30 menit/soal

\begin{enumerate}[resume]
	\item Jika $p$ adalah suatu bilangan prima dan persamaan $$p^k+p^l+p^m=n^2$$ memiliki penyelesaian bulat positif $(k,l,m,n)$, buktikan bahwa $4\mid p+1.$
	
	\item Carilah seluruh fungsi $f:\mathbb{R}\rightarrow\mathbb{R}$ yang memenuhi $$f(xf(x)+f(y))=f(x)^2+y$$ untuk $\forall x,y\in\mathbb{R}.$
	
	\item Misalkan $ABC$ adalah segitiga lancip. $M$ dan $N$ adalah titik pada segmen $AB$ dan $AC$ secara berturut-turut sedemikian sehingga lingkaran berdiameter $BN$ dan lingkaran berdiameter $CM$ berpotongan di $P$ dan $Q$. Jika $H$ adalah titik tinggi, buktikan bahwa $P,Q,H$ segaris.
	
	\item Kata sandi tanpa perulangan karakter dibentuk dengan menggunakan huruf kapital. Sebuah kata sandi dikatakan sempurna bila tidak memuat untaian karakter $XYZ$ maupun $ZYX$. Hitunglah peluang untuk membentuk kata sandi sempurna yang terdiri dari atas 8 huruf.
\end{enumerate}
\newpage
\section{Esai Bonus}
\textbf{This part is intended for challenge only, since the problems difficulty is higher than previous part.}

Alokasi waktu mengerjakan per soal: 45-75 menit/soal

\begin{enumerate}[resume]
	\item Carilah seluruh fungsi $f:\mathbb{R}\rightarrow\mathbb{R}$ yang memenuhi
	$$f(y+f(x))=2x+f(f(y)-x)$$ untuk ssebarang bilangan real $x$ dan $y$.
	
	\item Diberikan segitiga lancip $ABC$ yang tidak sama kaki dengan $\angle BAC = 45^\circ$. Garis-garis tinggi $AD,BE,CF$ berpotongan di $H$. Garis $EF$ meotong $BC$ di $P$. Titik $I$ merupakan titik tengah segmen $BC$ dan $IF$ memotong $PH$ di $Q$. Tunjukkan bahwa $\angle IQH = \angle AIE$.
	
	\item Misalkan $n > 1$ adalah bilangan asli yang memenuhi $n \mid 3^n+4^n$. Buktikan bahwa $7 \mid n$.
	
	\item Misalkan $N$ adalah suatu bilangan bulat positif dan $j$ adalah suatu bilangan irasional. Buktikan bahwa terdapat bilangan rasional $\frac{a}{b}$ dengan $1 \le b \le N$ memenuhi $$\left | j-\frac{a}{b} \right | < \frac{1}{b(N+1)}$$
\end{enumerate}
\end{document}

