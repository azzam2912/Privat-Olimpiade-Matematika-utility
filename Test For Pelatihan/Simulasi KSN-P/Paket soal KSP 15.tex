\documentclass[11pt]{scrartcl}
\usepackage[sexy]{evan}

\renewcommand{\baselinestretch}{1.5}
\addtolength{\oddsidemargin}{-0.3in}
\addtolength{\evensidemargin}{-0.3in}
\addtolength{\textwidth}{0.65in}

\addtolength{\topmargin}{-1in}
\addtolength{\textheight}{1.5in} 



\begin{document}
	\title{Paket Soal Untuk KSP SMA 15} % Beginner
	\date{\today}
	\author{Compiled by Azzam}
	\maketitle
	

	\section{Esai A}
	Dianjurkan untuk mengerjakannya dalam waktu \textbf{25 menit} \textit{termasuk penulisan solusi}. Setiap soal bernilai maksimal 4 poin. Soal \textbf{tidak} diurutkan berdasarkan kesulitan. 
	
	\begin{soalbaru}
	% buku pelatos sma no 31
	Tentukan semua pasangan bilangan bulat non negatif $(x,y,z)$ dimana $x \le y$ yang memenuhi $x^2+y^2=3\times 2016^z+77$
	\end{soalbaru}
	
	\begin{soalbaru}
	%https://arxiv.org/pdf/1110.1556.pdf#:~:text=Any%20student%20who%20failed%20to,%E2%80%9D%20problems%20or%20%E2%80%9Ccoffins%E2%80%9D.
	Carilah bilangan real $y$ yang memenuhi $2\sqrt[3]{2y-1}=y^3+1.$
	\end{soalbaru}
	
	\section{Esai B}
	Dianjurkan untuk mengerjakannya dalam waktu \textbf{60 menit} \textit{termasuk penulisan solusi}. Setiap soal bernilai maksimal 7 poin. Soal \textbf{tidak} diurutkan berdasarkan kesulitan.
	
	\begin{soalbaru}
		%https://nivotko.wordpress.com/2016/04/18/jajar-genjang-dan-bujur-sangkar/#more-3485
		Diberikan jajargenjang $ABCD$. Pada sisi-sisinya digambarkan bujur sangkar di luar jajar genjang. Buktikan bahwa keempat titik pusat bujur sangkar ini membentuk bujur sangkar.
	\end{soalbaru} 
	
	\begin{soalbaru}
	%https://nivotko.wordpress.com/2016/03/09/tahap-2-imo-tes-3/
	Tiap-tiap dari $n >1$ orang membawa sebuah bingkisan ke suatu acara. Setiap dua orang bertukar bingkisan tepat satu kali dengan urutan pertukaran tidak ditentukan. Tunjukkan bahwa setiap bingkisan dapat kembali ke pemilik awalnya jika $4\mid n$.
	
	
	\end{soalbaru}
	
	\section{Bonus}
	Untuk bersenang-senang :)
	\begin{soalbaru}
	%https://nivotko.wordpress.com/2020/04/11/cevian-that-intersects-median/
	
	%Dutch iMO TST 2014
	 Let $\triangle ABC$ be a triangle. Let $M$ be the midpoint of $BC$ and let $D$ be a point on the interior of side $AB$. The intersection of $AM$ and $CD$ is called E. Suppose that $|AD|=|DE|.$ Prove that $|AB|=|CE|.$
	
	
	\end{soalbaru}
\end{document}


