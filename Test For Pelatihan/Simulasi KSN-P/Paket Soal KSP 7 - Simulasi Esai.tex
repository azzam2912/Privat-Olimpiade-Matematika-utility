\documentclass[11pt]{scrartcl}
\usepackage[sexy]{evan}

%\addtolength{\textheight}{1.5in} 
\renewcommand{\baselinestretch}{1.35}



\begin{document}
	\title{Paket Soal campur 7} % Beginner
	\date{}
	\author{Compiled by Azzam}
	\maketitle
	
	
	\section{Soal Esai}
	Dianjurkan untuk mengerjakannya dalam waktu 120 menit.
	\begin{soalbaru} 
			Misalkan $p$ adalah bilangan prima. Buktikan bahwa $p \mid ab^p - ba^p$ untuk sembarang bilangan bulat positif $a$ dan $b$.
		\end{soalbaru}
		
		
		\begin{soalbaru}
				Misalkan $X,Y,Z$ berturut-turut pada segmen $BC,CA,AB$ dari $\triangle ABC$ sehingga $AX, BY, CZ$ konkuren di $P$. Selanjutnya, misalkan $D,E,F$ berturut-turut pada segmen $YZ,ZX,XY$ sehingga $XD,YE,ZF$ konkuren di $Q$. Buktikan bahwa $AD,BE,CF$ konkuren.
			\end{soalbaru}

	\begin{soalbaru}
		Tentukan seluruh bilangan asli $n$ sehingga $n\cdot 2^{n+1}+1$ merupakan bilangan kuadrat sempurna.
	\end{soalbaru}

	\begin{soalbaru}
		Sebuah polinomial $P(x)=x^3+ax^2+bx+c$ mempunyai tiga akar real yang berbeda. Polinomial $P(Q(x))$ tidak mempunyai akar real dimana $Q(x)=x^2+x+2001$. Buktikan bahwa $P(2001)>\frac{1}{64}$.
	\end{soalbaru}

	
	\begin{soalbaru}
	Di kamar Siskacolhe (seorang influencer tajir melintir tujuh turunan) yang sangat besar ada $n$ kotak $B_1,B_2,\dots,B_n$ yang disusun secara sejajar dalam satu baris. $n$ bola emas 24 karat didistibusikan ke $n$ kotak tersebut (tidak mesti dengan pembagian yang merata). Jika terdapat setidaknya satu bola di $B_1$, maka Siskacolhe dapat memindahkan satu bola dari $B_1$ ke $B_2$. Jika setidaknya ada satu bola di $B_n$, Siskacolhe dapat memindahkan satu bola dari $B_n$ ke $B_{n-1}$. Untuk $2 \le k \le n-1$, jika ada setidaknya dua bola di $B_k$, Siskacolhe dapat memindahkan dua bola dari $B_k$ dan menempatkan masing-masing satu bola ke $B_{k+1}$ dan $B_{k-1}$. Tunjukkan bahwa berapapun banyaknya bola emas pada awalnya, Siskacolhe selalu dapat membuat setiap kotak mempunyai tepat satu bola.
	\end{soalbaru}

	
\end{document}
