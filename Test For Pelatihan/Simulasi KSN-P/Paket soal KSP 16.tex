\documentclass[11pt]{scrartcl}
\usepackage[sexy]{evan}
\usepackage[normalem]{ulem}

\renewcommand{\baselinestretch}{1.5}



\begin{document}
	\title{Paket Soal Untuk KSP SMA 16} % Beginner
	\date{\today}
	\author{Compiled by Azzam}
	\maketitle
	

	\section{Esai A}
	Dianjurkan untuk mengerjakannya dalam waktu \textbf{25 menit} \textit{termasuk penulisan solusi}. Setiap soal bernilai maksimal 4 poin. Soal \textbf{tidak} diurutkan berdasarkan kesulitan. 
	
	\begin{soalbaru}
	% buku pelatos sma no 16 hal 118
	Carilah nilai eksak dari $$\tan^2 \frac{\pi}{11}+\tan^2 \frac{2\pi}{11}+\tan^2 \frac{3\pi}{11}+\tan^2 \frac{4\pi}{11}+\tan^2 \frac{5\pi}{11}$$
	\end{soalbaru}
	
	\begin{soalbaru}
	%pelatos no 28 hal 119
	Suatu segiempat $ABCD$ memiliki luas $L$ yang memenuhi persamaan $AB+BD+DC=16$. Tentukan panjang sisi $AC$ agar $L$ maksimum.
	\end{soalbaru}
	
	\section{Esai B}
	Dianjurkan untuk mengerjakannya dalam waktu \textbf{60 menit} \textit{termasuk penulisan solusi}. Setiap soal bernilai maksimal 7 poin. Soal \textbf{tidak} diurutkan berdasarkan kesulitan.
	
	\begin{soalbaru}
		%pelatos no. 45 hal. 12
		Tentukan semua pasangan bilangan bulat positif $(n,t)$ yang memenuhi $6^n+1=n^2t$ dan $FPB(n, 29\times 197)=1$.
	\end{soalbaru} 
	
	\begin{soalbaru}
	%pelatos no. 16 hal 95
	Pada tahun 9999 sekolah di seluruh alam semesta telah mendaftarkan siswa terbaiknya untuk mengikuti jambore matematika tingkat jagat raya yang jumlah pesertanya mencapai 10001. Beberapa peserta tergabung dalam beberapa klub studi (seorang peserta mungkin terdaftar pada beberapa klub studi yang berbeda). Beberapa klub tadi tergabung dalam beberapa komunitas (sebuah klub mungkin tergabung dalam beberapa komunitas yang berbeda). Tercatat bahwa terdapat $k$ komunitas, yang memenuhi syarat:
	\begin{itemize}
	\item Setiap satu pasang peserta terdaftar tepat pada satu klub.
	\item Untuk setiap peserta dan setiap komunitas, peserta-peserta terdaftar pada tepat satu klub di sebuah komunitas.
	\item Setiap klub memiliki murid sebanyak bilangan ganjil, dan sebuah klub dnegan $2m+1$ murid ($m \in \ZZ^+$) terdaftar pada tepat $m$ komunitas.
	\end{itemize}
	Tentukan nilai $k$ yang mungkin.
	
	
	\end{soalbaru}
	
	
	\section{Soal Bonus}
	
	For Fun :D !
	\begin{soalbaru}
	% pelatos no 33 hal 70
	Bapak Presiden mempunyai sebuah segitiga $JKW$ di istana negara yang memiliki titik tengah $S$ pada segmen $KW$ (\sout{segmennya ternyata ngga asli tapi kw super}). Saat melihat huruf $S$, Pak Presiden terpikir untuk memberi para SJW di Twotter tebak-tebakan: Jika $\angle SJW = \angle JKW$ dan $\angle KJS = (36+69)^\circ$, carilah besar dari sudut $\angle JKW$.
	\end{soalbaru}
\end{document}


