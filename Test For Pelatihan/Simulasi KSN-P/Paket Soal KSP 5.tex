\documentclass[11pt]{scrartcl}
\usepackage[sexy]{evan}

%\addtolength{\textheight}{1.5in} 
\renewcommand{\baselinestretch}{1.5}



\begin{document}
	\title{Paket Soal Untuk KSP SMA 5} % Beginner
	\date{Rabu, 21 Juli 2021}
	\author{Compiled by Azzam}
	\maketitle
	
	
	\section{Soal Esai}
	Nyoba \LaTeX   class nya Evan Chen =). Ternyata bagus juga 0.0
	\begin{soalbaru}
		Carilah semua fungsi $f:\ZZ^+ \to \ZZ^+$ sehingga $$f(m+n)=f(m)+f(n)$$ untuk sembarang $m,n \in \ZZ^+$
	\end{soalbaru}

	\begin{soalbaru}
		Misalkan $ABCD$ adalah segiempat konveks. Misalkan lingkaran dalam segitiga $ABD$ menyinggung segmen $AB,AD,BD$ di titik $W,Z,K$ secara berturut-turut. Misalkan pula lingkaran dalam segitiga $CBD$ menyinggung segmen $CB,CD,BD$ di $X,Y,K$ secara berturut-berturut. Buktikan bahwa $WXYZ$ siklis.
	\end{soalbaru}

	\begin{soalbaru}
		Bilangan bulat positif $x_1,x_2,\dots, x_n$ $(n>4)$ disusun di sebuah lingkaran sehingga setiap $x_i$ membagi jumlah dari tetangga-tetangganya; yaitu 
		$$\dfrac{x_{i-1}+x_{i+1}}{x_{i}}=k_i$$ adalah bilangan bulat untuk setiap $i$, dimana $x_0=x_n$,$x_{n+1}=x_1$. Tunjukkan bahwa 
		$$ 2 \le \dfrac{k_1+\dots+k_n}{n} < 3.$$
	\end{soalbaru}

	\begin{soalbaru}
		Misalkan $m$ dan $s$ adalah bilangan asli dengan $2 \le s \le 3m^2.$ Definisikan barisan $a_1,a_2,\dots$ secara rekursif dengan $a_1=s$ dan $$a_{n+1} = 2n + a_n \text{    untuk   }  n=1,2,\dots.$$
		Buktikan bahwa jika $a_1,a_2,\dots,a_m$ adalah bilangan-bilangan prima, maka $a_{s-1}$ juga prima. 
	\end{soalbaru}
	
	\begin{soalbaru}
	Misalkan $C$ adalah titik pada setengah lingkaran dengan diameter $AB$ (terletak di keliling lingkarannya, bukan di diameternya). Misalkan pula $D$ adalah titik tengah busur $AC$. Notasikan $E$ sebagai proyeksi $D$ pada garis $BC$ dan $F$ adalah perpotongan $AE$ dengan setengah lingkaran. Buktikan bahwa $BF$ membagi garis $DE$ sama panjang.
	\end{soalbaru}
	
	\begin{soalbaru}
	Diberikan $\dfrac{\pi}{4} = 1 - \dfrac{1}{3}+\dfrac{1}{5}-\dfrac{1}{7}+\dots$. Carilah nilai dari $\dfrac{1}{1 \times 3 \times 5}+\dfrac{3}{5 \times 7 \times 9}+\dfrac{5}{9 \times 11 \times 13}+\dots$.
	\end{soalbaru}
\end{document}
