\documentclass[11pt]{scrartcl}
\usepackage[sexy]{evan}

%\addtolength{\textheight}{1.5in} 
\renewcommand{\baselinestretch}{1.35}



\begin{document}
	\title{Paket Soal Untuk KSP SMA 14} % Beginner
	\date{\today}
	\author{Compiled by Azzam}
	\maketitle
	

	\section{Esai A}
	Dianjurkan untuk mengerjakannya dalam waktu \textbf{25 menit} \textit{termasuk penulisan solusi}.
	
	Setiap soal bernilai maksimal 4 poin.
	
	Soal \textbf{tidak} diurutkan berdasarkan kesulitan. 
	
	\begin{soalbaru}
	%nivotko, 29 mar 20
	% dutch mo 2015
	%https://nivotko.wordpress.com/2020/03/29/two-aligned-equilateral-triangles/
	Titik-titik $A,B,C$ berada pada satu garis dengan urutan tersebut. Titik $D$ dan $E$ berada di sisi yang sama terhadap garis $AB$ sedemikian sehingga $\triangle ABD$ dan $\triangle BCE$ adalah segitiga sama sisi. Jika segmen $AE$ dan $CD$ berpotongan di $S$, besar sudut $\angle ASD = \dots$
	\end{soalbaru}
	
	\begin{soalbaru}
	%brilliant.org
	%from self book
	%much troubles in combinatorics no. 1
	Pada sebuah bangun segi-$n$, akan dipilih $k \le \frac{n}{2}$ titik dari $n$ titik sudut tersebut sehingga terbentuk bangun segi-$k$ dimana tidak ada sisi segi-$k$ tersebut yang merupakan sisi segi-$n$. Berapa banyak kemungkinan segi-$k$ yang mungkin?
	\end{soalbaru}
	
	\section{Esai B}
	Dianjurkan untuk mengerjakannya dalam waktu \textbf{60 menit} \textit{termasuk penulisan solusi}.
	
	Setiap soal bernilai maksimal 7 poin.
		
	Soal \textbf{tidak} diurutkan berdasarkan kesulitan.
	
	\begin{soalbaru}
		%nivotko 13 apr 20
		%south african 2010
		%https://nivotko.wordpress.com/2020/04/13/bizarre-inequality/
		Diberikan bilangan asli $n$ yang memenuhi $x_1 \ge x_2 \ge \cdots \ge x_n \ge 0$ dan $x_1^2+x_2^2+\cdots+x_n^2=1$, 
		buktikan bahwa
		
		$$\dfrac{x_1}{\sqrt{1}}+\dfrac{x_2}{\sqrt{2}}+\cdots+\dfrac{x_n}{\sqrt{n}}\ge 1.$$
	\end{soalbaru} 
	
	\begin{soalbaru}
	%nivotko, 17 apr 20
	%baltic way 2019
	%https://nivotko.wordpress.com/2020/04/17/the-number-is-a-perfect-square/
	Untuk setiap bilangan asli $n$, definisikan $f(n)$ sebagai banyaknya pasangan bilangan asli $(a,b)$ sehingga $\dfrac{ab}{a+b}$ adalah bilangan asli yang membagi $n$. Tunjukkan bahwa $f(n)$ adalah kuadrat sempurna.
	\end{soalbaru}
	
	\section{Bonus}
	Untuk bersenang-senang :)
	\begin{soalbaru}
	%arief anbiya
	%https://www.youtube.com/watch?v=600X-ZGNBbk&pp=ugMICgJpZBABGAE%3D
	Misalkan $a,b,c > 0$ dengan $a+b+c=3$. Buktikan bahwa 
	$$\dfrac{a}{b^2+1}+\dfrac{b}{c^2+1}+\dfrac{c}{a^2+1} \ge \dfrac{3}{2}.$$
	\end{soalbaru}
\end{document}


