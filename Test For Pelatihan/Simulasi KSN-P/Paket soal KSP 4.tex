\documentclass{article}
\usepackage{amsmath}
\usepackage{amssymb}
\usepackage{graphicx}
\graphicspath{{./Farhan/}}

\usepackage{enumitem}
\renewcommand{\baselinestretch}{1.6}
\addtolength{\oddsidemargin}{-1in}
\addtolength{\evensidemargin}{-1.5in}
\addtolength{\textwidth}{1.9in}

\addtolength{\topmargin}{-1in}
\addtolength{\textheight}{1.7in} 

\title{Paket Soal Untuk KSP SMA 4}
\date{Kamis, 15 Juli 2021}

\setlength{\parindent}{0em}
\begin{document}
	\maketitle

\section{Esai}
Jawablah soal-soal berikut dengan menyertakan argumentasi atau cara Anda mendapatkannya. Setiap soal bernilai bilangan bulat positif dengan poin maksimum 7.

Alokasi waktu : 15-30 menit/soal

\begin{enumerate}[resume]
	\item Diberikan polinom monik $P(x)$ yang berderajat $2021$ dan merupakan polinomial real serta diberikan
	$Q(x)=x^2+2x+2021$. Misalkan persamaan $P(x)=0$ mempunyai 2021 solusi real
	dan $P(2021) \le 1$. Tunjukkan bahwa persamaan $P(Q(x))=0$ mempunyai solusi real. 
	
	\item Diberikan segitiga $ABC$, dengan garis berat $AD, BE, CF$. Misalkan $m=AD+BE+CF$ dan misalkan $s=AB+BC+CA$. Buktikan bahwa $$\dfrac{3s}{2} > m > \dfrac{3s}{4}$$
	
	\item Misalkan $n > 1$ adalah bilangan ganjil. Buktikan bahwa $n \nmid 3^n + 1$.
	
	\item Misalkan himpunan $A \subset \{1,2,3,\dots,4040\}$ dengan $|A| = 2021$. Tunjukkan bahwa ada dua elemen berbeda $a,b \in A$ sehingga $a \mid b$.
\end{enumerate}
\end{document}

