\documentclass[11pt]{scrartcl}
\usepackage[sexy]{evan}

%\addtolength{\textheight}{1.5in} 
\renewcommand{\baselinestretch}{1.35}



\begin{document}
	\title{Paket Soal Untuk KSP SMA 8} % Beginner
	\date{Rabu, 4 Agustus 20211}
	\author{Compiled by Azzam}
	\maketitle
	
	
	\section{Soal Esai}
	Dianjurkan untuk mengerjakannya dalam waktu 120 menit.
	
		
		
		\begin{soalbaru}
				Diberikan sebuah polinomial berkoefisien real positif $p(x)$ yang memenuhi $p(1)=1$. Untuk $x > 0$, buktikan bahwa $$p\left(\dfrac{1}{x}\right) \ge \dfrac{1}{p(x)}.$$
			\end{soalbaru}

	\begin{soalbaru}
		Diberikan $2009$ titik berbeda di sebuah bidang datar yang diwarnai hanya dengan warna merah atau biru. Di setiap titik berwarna biru, dibuat sebuah lingkaran berjari-jari 1 sehingga titik pusatnya adalahh titik biru tersebut dan di lingkarannya terdapat tepat dua titik berwarna merah. Carilah nilai terbesar banyaknya titik biru yang mungkin.
		
		%jbmo shortlist 2009 c1
	\end{soalbaru}

	\begin{soalbaru}
		Tentukan seluruh bilangan bulat positif $k$ sehingga $k+9$ adalah kuadrat sempurna dan faktor prima dari $k$ hanyalah 2 atau 3.
		
		%jbmo 2009 shortlist n1
	\end{soalbaru}

	
	\begin{soalbaru}
	%Geometri hayattir
	
	Diberikan sebuah segitiga $ABC$ dimana titik $D$ pada segmen $AC$ sehingga $AB=CD$. Jika $\angle BAC = 40^\circ$ dan $\angle BCA = 30^\circ$, hitunglah besar $\angle CBD$.
	\end{soalbaru}

\begin{soalbaru} 
			Diberikan $\triangle ABC$ dengan titik $E$ di luar $\triangle ABC$ sehingga $BE$ memotong segmen $AC$ di $D$. Diketahui $\angle EBC = 36^\circ$, $\angle ABE = 12^\circ$, $\angle ECA = 24^\circ$, dan $\angle ACB = 48^\circ$. Tentukan besar sudut $\angle AEB$.
		\end{soalbaru}
	
\end{document}


