\documentclass[11pt]{scrartcl}
\usepackage[sexy]{evan}

%\addtolength{\textheight}{1.5in} 
\renewcommand{\baselinestretch}{1.35}



\begin{document}
	\title{Paket Soal Untuk KSP SMA 12} % Beginner
	\date{\today}
	\author{Compiled by Azzam}
	\maketitle
	

	\section{Esai}
	Dianjurkan untuk mengerjakannya dalam waktu 120 menit.
	
	Soal \textbf{tidak} diurutkan berdasarkan kesulitan. 
	
	\begin{soalbaru}
	Dua lingkaran berbeda berpotongan di titik $A$ dan $B$. Garis $l$ meninggung kedua lingkaran di $E$ dan $F$. Jika $B$ berada di dalam $\triangle AEF$, dan $H$ adalah titik tinggi $AEF$, buktikan bahwa $HB \perp AB$.
	\end{soalbaru}
	
	\begin{soalbaru}
	Garis singgung lingkaran luar $\triangle ABC$ di titik $A$ memotong $BC$ di $D$. Selanjutnya, titik $M$ dan $N$ berturut-turut pada segmen $AB$ dan $AC$. Jika garis $MN$ memotong lingkaran $(ABC)$  di $P$ dan $Q$, dan $AB < AC$, buktikan bahwa $\angle ADP = \angle QDC$.
	\end{soalbaru}
	
	\begin{soalbaru}
	% aops x=ka, y=kb
	Misalkan $a,b>0$ dan $a^2+b^2\geq 2.$ Buktikan bahwa $$\frac{a}{a^2+b}+\frac{b}{b^2+a}\leq1$$
	\end{soalbaru}
	
	\begin{soalbaru}
	%st petersburg 1997
	%pranav game
	Bilangan asli $N$ dihasilkan dari perkalian $k \ge 3$ bilangan prima berbeda. Arumi dan Bachsin bergantian menulis pembagi komposit dari $N$ di papan tulis dengan aturan sebagai berikut:
	\begin{itemize}
	\item Tidak boleh menulis $N$.
	\item Tidak boleh ada dua bilangan yang saling relatif prima di papan tulis.
	\item Tidak boleh ada dua bilangan $x$ dan $y$ di papan tulis sehingga $x \mid y$ atau $y \mid x$.
	\item Pemain yang tidak bisa menulis bilangan dinyatakan kalah.
	\end{itemize}
	
	Jika Arumi bermain duluan, siapa yang pasti mempunyai strategi menang?
	\end{soalbaru}
	
	\newpage
	\begin{soalbaru}
	Sebut bilangan asli $n$ sebagai $berguna$ jika untuk setiap pembagi $a$ dari $n$, $a+1$ juga merupakan pembagi $n+1$. Tentukan semua bilangan asli $berguna$ yang memenuhi.
	\end{soalbaru}
	
	\begin{soalbaru}
	(Modifikasi IMO 2021/4) Misalkan $\Gamma$ adalah lingkaran dengan pusat $I$ dan $ABCD$ adalah segiempat konveks sehingga setiap segmen $AB$,$BC$,$CA$,dan $DA$ menyinggung $\Gamma$. Misalkan $\Omega$ adalah lingkaran luar segitiga $AIC$. Perpanjangan sisi $BA$ melewati $A$ memotong $\Omega$ di $X$, dan perpanjangan sisi $BC$ melewati $C$ memotong $\Omega $ di $Z$. Jika perpanjangan sisi $AD$ dan $CD$ melewati $D$ memotong $\Omega$ di $Y$ dan $T$, buktikan bahwa $XT=YZ$.
	\end{soalbaru}
\end{document}


